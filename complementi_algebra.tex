\documentclass[11pt]{scrartcl}
\usepackage[italian]{babel}
\usepackage{wrapfig}
\usepackage[sexy]{evan} %evan.sty



%%%%%%%%%%%%%%%%%%%%%%%%%%%%%%%%%%%%%%%%%%%%%%%%%%%%%%%%%%%%%%%%%%%%%%%%%%%%%%
%
% BOOST SOFTWARE LICENSE - VERSION 1.0 - 17 AUGUST 2003
%
% Copyright (c) 2022 Evan Chen [evan at evanchen.cc]
% https://web.evanchen.cc/ || github.com/vEnhance
%
% Available for download at:
% https://github.com/vEnhance/dotfiles/blob/main/texmf/tex/latex/evan/evan.sty
%
% Permission is hereby granted, free of charge, to any person or organization
% obtaining a copy of the software and accompanying documentation covered by
% this license (the "Software") to use, reproduce, display, distribute,
% execute, and transmit the Software, and to prepare derivative works of the
% Software, and to permit third-parties to whom the Software is furnished to
% do so, all subject to the following:
%
% The copyright notices in the Software and this entire statement, including
% the above license grant, this restriction and the following disclaimer,
% must be included in all copies of the Software, in whole or in part, and
% all derivative works of the Software, unless such copies or derivative
% works are solely in the form of machine-executable object code generated by
% a source language processor.
%
% THE SOFTWARE IS PROVIDED "AS IS", WITHOUT WARRANTY OF ANY KIND, EXPRESS OR
% IMPLIED, INCLUDING BUT NOT LIMITED TO THE WARRANTIES OF MERCHANTABILITY,
% FITNESS FOR A PARTICULAR PURPOSE, TITLE AND NON-INFRINGEMENT. IN NO EVENT
% SHALL THE COPYRIGHT HOLDERS OR ANYONE DISTRIBUTING THE SOFTWARE BE LIABLE
% FOR ANY DAMAGES OR OTHER LIABILITY, WHETHER IN CONTRACT, TORT OR OTHERWISE,
% ARISING FROM, OUT OF OR IN CONNECTION WITH THE SOFTWARE OR THE USE OR OTHER
% DEALINGS IN THE SOFTWARE.
%%%%%%%%%%%%%%%%%%%%%%%%%%%%%%%%%%%%%%%%%%%%%%%%%%%%%%%%%%%%%%%%%%%%%%%%%%%%%%


\begin{document}
\title{Complementi di Algebra 1}
\subtitle{\large\normalfont\rmfamily\scshape APPUNTI DEL CORSO DI ALGEBRA 1 TENUTO\\ DALLA PROF. DEL CORSO E DAL PROF. LOMBARDO}
\author{Leonardo Migliorini \\ \textnormal{\href{l.migliorini@studenti.unipi.it}{l.migliorini@studenti.unipi.it}} \\ Università di Pisa}
\date{Anno Accademico 2022-23}
\maketitle
\newpage

\tableofcontents

\newpage

\section*{Premessa}
Il seguente scritto è una mia rielaborazione delle note del corso di Algebra 1 tenuto dalla professoressa Del Corso e dal professor Lombardo nell'anno accademico 
2022-23, \textbf{le note non sono state revisionate dai suddetti prof}. In questo file ci sono soltanto gli appunti delle lezioni del professor Lombardo (per
le note delle lezioni della professoressa Del Corso vi rimando agli \href{https://github.com/diego-unipi/Algebra-1}{\textcolor{purple}{Appunti di Algebra 1}}), l'ordine degli argomenti è lo stesso di quello seguito a lezione se non 
in qualche caso.
Chiunque volesse aiutarmi a migliorare questi appunti può farlo segnalando eventuali errori e/o imprecisioni alla mia mail.

\section*{Ringraziamenti}

Diego Monaco, Niccolò Nannicini, Pietro Crovetto, Leonardo Alfani, Daniele
Lapadula, Francesco Sorce, Alessandro Moretti, Matteo Gori.

\mbox{}
\vfill
\begin{wrapfigure}{R}{0.2\textwidth}
	\centering
	\href{https://creativecommons.org/licenses/by-nc/4.0/deed.it}{\includegraphics[width=0.2\textwidth]{licenza.png}}
\end{wrapfigure}

Quest'opera è stata rilasciata con licenza Creative Commons Attribuzione - Condividi allo stesso modo 4.0 Internazionale. Per leggere
una copia della licenza visita il sito web \href{http://creativecommons.org/licenses/by-sa/4.0/deed.it}{\textcolor{blue}{https://creativecommons.org/licenses/by-nc/4.0/deed.it}}.\\

\newpage

\section{Gruppi}

\subsection{Insiemi di generatori}

\begin{definition}
    Dati un gruppo $G$ e $x_1, \ldots, x_n$ elementi di $G$, chiamiamo \vocab{sottogruppo 
    generato} da $x_1, \ldots, x_n$ il più piccolo sottogruppo $\langle x_1, \ldots x_n
    \rangle$ di $G$ contenente $x_1, \ldots, x_n$, cioè \[\langle x_1, \ldots, x_n\rangle =
    \bigcap_{\substack{H\leqslant G\\ \{x_1, \ldots, x_n\} \subseteq H}} H\] 
\end{definition}

\begin{remark}
    La definizione è ben posta, infatti l'intersezione avviene su una 
    famiglia non vuota di insiemi dal momento che $G$ è un sottogruppo di 
    se stesso contenente $x_1, \ldots, x_n$. Inoltre l'intersezione non è vuota in 
    quanto contiene almeno l'identità e gli elementi $x_1, \ldots, x_n$.
\end{remark}

La definizione data non dà informazioni su come sono fatti gli elementi di 
$\langle x_1, \ldots, x_n\rangle$, cerchiamo quindi di caratterizzare in modo
diverso tale sottogruppo. Poiché chiuso per l'operazione indotta da $G$, $\langle x_1, \ldots, x_n\rangle$
deve contenere tutti i prodotti finiti, in qualsiasi ordine, delle potenze di
$x_1, \ldots, x_n$, cioè deve contenere l'insieme 
\[\{g_1^{\pm 1} \ldots g_r^{\pm 1}\mid r \in \NN, g_i \in \{x_1, \ldots, x_n\}
~\forall i \in \{1, \ldots, r\}\}\]

\begin{proposition}
Dati un gruppo $G$ e $x_1, \ldots, x_n$ elementi di $G$, allora \[
    \langle x_1 \ldots x_n\rangle = \{g_1^{\pm 1} \ldots g_r^{\pm 1}\mid r 
    \in \NN, g_i \in \{x_1, \ldots, x_n\}~\forall i \in \{1, \ldots, r\}\}
    \]
\end{proposition}

\begin{proof}
Poniamo $S = \{g_1^{\pm 1} \ldots g_r^{\pm 1}\mid r \in \NN, g_i \in \{x_1, \ldots, x_n\}
~\forall i \in \{1, \ldots, r\}\}$, mostriamo che $S$ è un sottogruppo di $G$. 
Effettivamente $e \in S$ in quanto è prodotto di nessuna potenza di $x_1, \ldots, x_n$, 
il prodotto di due elementi di $S$ è ancora un elemento di $S$ in quanto
prodotto finito di potenze di $x_1, \ldots, x_n$ e l'inverso di un elemento
$g_1^{\pm 1}\ldots g_r^{\pm 1} \in\nolinebreak S$ è $(g_1^{\pm 1}\ldots 
g_r^{\pm 1})^{-1} = g_r^{\mp 1}\ldots g_1^{\mp 1}$, che è un elemento di $S$.
Abbiamo quindi che $S$ è un sottogruppo di $G$ contenente $x_1, \ldots, x_n$,
pertanto $\langle x_1, \ldots, x_n\rangle\subseteq S$ per minimalità di $\langle x_1,
\ldots, x_n\rangle$. D'altra parte, per quanto osservato sopra abbiamo che
tutti gli elementi della forma $g_1^{\pm 1}\ldots g_r^{\pm 1}$ con $r \in \NN$, 
$g_i \in \{x_1, \ldots, x_n\}$ per ogni $i \in \{1, \ldots, r\}$ devono essere
contenuti in $\langle x_1, \ldots, x_n\rangle$, pertanto i due sottogruppi
coincidono.
\end{proof}

\begin{remark}
    Se $G$ è un gruppo ciclico abbiamo che esiste $x \in G$ tale che 
    $\langle x\rangle = G$, cioè tutti gli elementi di $G$ sono potenze di $x$.
\end{remark}

Diciamo che $x_1, \ldots, x_n \in G$ sono \vocab{generatori} per $G$, o che 
l'insieme $\{x_1, \ldots, x_n\}$ \vocab{genera} $G$ se $\langle x_1, \ldots, x_n\rangle = G$.

\newpage

\subsection{Automorfismi di $(\Zp)^n$}
\label{sez1.2}

Dato $p$ un primo, vogliamo determinare quanti sono gli automorfismi di 
$(\Zp)^n$, per fare ciò è conveniente definire una struttura di spazio vettoriale,
quindi un prodotto per scalari\[
    \cdot:\Zp\times (\Zp)^n \longrightarrow (\Zp)^n : 
    (\overline{\lambda}, v)\longmapsto \overline{\lambda}v
\]
con $\overline{\lambda}v = \underset{\tilde{\lambda}\text{ volte}}{\underbrace{v + \ldots + v}}$
e $\tilde{\lambda}$ un qualsiasi rappresentante di $\overline{\lambda}$.
Tale prodotto è ben definito, infatti se $\lambda, \lambda' \in \ZZ$ sono tali che
$\overline{\lambda} = \overline{\lambda'}$, cioè esiste $k \in \ZZ$ tale che
$\lambda = \lambda' + kp$, allora \[
    \overline{\lambda'} v = \underset{\lambda'\text{ volte}}{\underbrace{v + \ldots + v}} = 
    \underset{\lambda + kp\text{ volte}}{\underbrace{v + \ldots + v}} = 
    \underset{\lambda\text{ volte}}{\underbrace{v + \ldots + v}}
\]in quanto $\underset{kp\text{ volte}}{\underbrace{v + \ldots + v}} = 0$. 
Si verifica che $((\Zp)^n, +, \cdot)$ è effettivamente uno spazio vettoriale
sul campo $\FF_p = \Zp$ (dove $\cdot$ è il prodotto per scalari appena definito).
Per come abbiamo definito il prodotto per scalari, abbiamo che per ogni
$\varphi \in \Aut((\Zp)^n)$ vale $\varphi(\lambda v) = \lambda\varphi(v)$ 
per ogni $\lambda \in \FF_p$, pertanto
\[
    \Aut((\Zp)^n) = GL((\FF_p)^n) = \{\varphi: (\FF_p)^n\longrightarrow(\FF_p)^n
    \mid\varphi\text{ isomorfismo di spazi vettoriali}\}
\]

Poiché $GL((\FF_p)^n) \cong GL_n(\FF_p) = \{M \in M_{n\times n}(\FF_p)\mid \det M \neq 0\}$
possiamo rappresentare ogni automorfismo di $(\Zp)^n$ con una matrice invertibile
di taglia $n\times n$ a coefficienti in $\FF_p$.

\begin{proposition}
Dato $p$ un primo, allora \[
    |\Aut((\Zp)^n)| = \prod_{i = 0}^{n - 1} (p^n - p^i)
    \]
\end{proposition}

\begin{proof}
    Osserviamo che un elemento di $\Aut((\Zp)^n)$ deve necessariamente mandare 
    una base di $(\Zp)^n$ in un'altra base, e si dermina univocamente in questo 
    modo. Sia $\{v_1, \ldots, v_n\}$ una base di $(\Zp)^n$ e $\varphi \in 
    \Aut((\Zp)^n)$, consideriamo $\varphi(v_1)$: $\varphi(v_1)$ può assumere
    qualsiasi valore non nullo, pertanto abbiamo $(p^n - 1)$ possibilità per 
    l'immagine del primo vettore. Per quanto riguarda $v_2$, $\varphi(v_2)$
    può assumere qualsiasi valore non nullo che non sia multiplo di $\varphi(v_1)$,
    che sono $p^n - p$, analogamente $\varphi(v_3)$ può assumere qualsiasi
    valore non nullo che non sia combinazione lineare di $v_1$ e $v_2$, che sono
    $p^n - p^2$, e così via. Reiteriamo questo ragionamento fino a $\varphi(v_n)$,
    che può essere scelto in $p^n - p^{n - 1}$ modi, da cui \[
        |\Aut((\Zp)^n)| = \prod_{i = 0}^{n - 1}(p^n - p^i)
    \]
\end{proof}

\newpage

\subsection{Gruppo diedrale}

\subsubsection{Elementi del gruppo}

\begin{definition}
    Dato $n \geqslant 2$ un numero naturale consideriamo un poligono regolare di $n$ vertici
    centrato nell'origine del piano $\RR^2$,
    chiamiamo \vocab{gruppo diedrale} su $n$ vertici l'insieme $D_n$
    delle isometrie di $\RR^2$ che fissano il poligono, cioè che mandano i 
    vertici in se stessi (per $n = 2$ consideriamo le isometrie che mandano un 
    segmento in se stesso).
\end{definition}

\begin{remark}
    $D_n$ è un gruppo, in quanto l'applicazione identità che 
    fissa tutti i vertici è un'isometria dal poligono in se stesso, la 
    composizione di isometrie è un'isometria e un'isometria ammette sempre 
    un'inversa, che è anch'essa un'isometria.
\end{remark}

\begin{remark}
    Una rotazione di angolo $\displaystyle\frac{2\pi}{n}$ è un elemento di $D_n$,
    così come una simmetria rispetto a un asse.
\end{remark}

Proseguendo con questa intuizione geometrica, indicheremo con $r$ una rotazione
di angolo $\displaystyle \frac{2\pi}{n}$ e con $s$ una simmetria rispetto a
un qualsiasi asse. Notiamo che $\ord(r) = n$ e $\ord(s) = 2$ (per convenzione, 
indichiamo con un angolo positivo una rotazione in senso antiorario e con un 
angolo negativo una rotazione in senso orario).

\begin{definition}
    Data $r \in D_n$ una rotazione di ordine $n$, indichiamo con $\mathcal{R}$ il
    \vocab{sottogruppo delle rotazioni} $\langle r\rangle$.
\end{definition}

\begin{remark}
    Il sottogruppo $\mathcal{R}$ contiene tutte le rotazioni di $D_n$, infatti
    se $r'$ è una rotazione di angolo $\displaystyle\frac{2k\pi}{n}$, $k \in \ZZ$,
    allora $r^k = r'$ in quanto anche $r^k$ è una rotazione di angolo 
    $\displaystyle\frac{2k\pi}{n}$.
\end{remark}

Per determinare come sono fatti gli elementi di $D_n$, studiamo il sottogruppo
$\langle r, s\rangle$. Sicuramente $\langle r, s\rangle$ contiene il sottogruppo $\mathcal{R}$
e tutti gli elementi della forma $sr^k$, $sr^ks$, $sr^ksr^h$ e così via, vogliamo
mostrare che in effetti $D_n$ è generato da $r$ e $s$.

\begin{remark}
    \label{obs1.11}
    Gli elementi della forma $r^k$ e $sr^h$ sono distinti per ogni $h, k \in \ZZ$. 
    Infatti sappiamo dall'algebra lineare che il determinante di una simmetria
    è $-1$ e che il determinante di una rotazione è $1$, per la moltiplicatività
    del determinante quindi $\det (r^k) = (\det r)^k = 1$ e
    $\det (sr^h) = (\det s)(\det r)^h = -1$, da cui $r^k \neq sr^h$.
\end{remark}

\begin{lemma}
    Per ogni rotazione $r \in D_n$ e per ogni simmetria $s \in D_n$ vale
    \[srs^{-1} = r^{-1}\]
\end{lemma}

\begin{proof}
    Senza perdita di generalità possiamo supporre che $r$ sia la rotazione 
    di angolo $\displaystyle\frac{2\pi}{n}$ e che $s$ sia la simmetria
    (rispetto all'asse $y$) che 
    a ogni punto $x$ del piano associa il punto $-x$. Possiamo rappresentare
    rispettivamente $r$ e $s$ tramite le matrici
    \begingroup
    \renewcommand{\arraystretch}{1.2}
    \[
        \begin{pmatrix}
            \cos\left(\frac{2\pi}{n}\right) & -\sin\left(\frac{2\pi}{n}\right)\\
            \sin\left(\frac{2\pi}{n}\right) & \cos\left(\frac{2\pi}{n}\right)
        \end{pmatrix},
        \begin{pmatrix}
            -1 & 0\\
            0 & 1
        \end{pmatrix}
    \]
    \endgroup

    svolgendo esplicitamente il prodotto quindi
    
    \begingroup
    \renewcommand{\arraystretch}{1.2}
    \begin{multline*}
        \begin{pmatrix}
            -1 & 0\\
            0 & 1
        \end{pmatrix}
        \begin{pmatrix}
            \cos\left(\frac{2\pi}{n}\right) & -\sin\left(\frac{2\pi}{n}\right)\\
            \sin\left(\frac{2\pi}{n}\right) & \cos\left(\frac{2\pi}{n}\right)
        \end{pmatrix}
        \begin{pmatrix}
            -1 & 0\\
            0 & 1
        \end{pmatrix} = 
        \begin{pmatrix}
            -1 & 0\\
            0 & 1
        \end{pmatrix}
        \begin{pmatrix}
            -\cos\left(\frac{2\pi}{n}\right) & -\sin\left(\frac{2\pi}{n}\right)\\
            -\sin\left(\frac{2\pi}{n}\right) & \cos\left(\frac{2\pi}{n}\right)
        \end{pmatrix} = \\ =
        \begin{pmatrix}
            \cos\left(\frac{2\pi}{n}\right) & \sin\left(\frac{2\pi}{n}\right)\\
            -\sin\left(\frac{2\pi}{n}\right) & \cos\left(\frac{2\pi}{n}\right)
        \end{pmatrix} = 
        \begin{pmatrix}
            \cos\left(-\frac{2\pi}{n}\right) & -\sin\left(-\frac{2\pi}{n}\right)\\
            \sin\left(-\frac{2\pi}{n}\right) & \cos\left(-\frac{2\pi}{n}\right)
        \end{pmatrix}
    \end{multline*}
    \endgroup
    
    che è la matrice associata alla rotazione di angolo $-\displaystyle
    \frac{2\pi}{n}$, cioè $r^{-1}$.
\end{proof}

\begin{proposition}
    Se $n \geqslant 3$ allora $|D_n| = 2n$.
\end{proposition}

\begin{proof}
    Indicando con $1, \ldots, n$ gli $n$ vertici di un poligono regolare di $n$ lati, notiamo
    che un elemento $g \in D_n$ è univocamente determinato da $g(1), \ldots, g(n)$.
    In particolare, fissato $g(1)$, per il quale abbiamo $n$ possibili scelte,
    abbiamo al massimo due valori per $g(2)$, cioè $g(2) \in \{g(1) + 1, g(1) - 1\}$
    (a meno di sommare $n$ se uno dei due elementi è negativo). Poiché $g(1)$
    e $g(2)$ individuano due vettori nel piano non allineati, cioè
    linearmente indipendenti, ne costituiscono una base: fissati i valori di 
    $g(1)$ e $g(2)$ abbiamo quindi determinato ogni
    elemento di $D_n$ in modo unico e, poiché possiamo farlo in al più $2n$ modi, 
    $|D_n| \leq 2n$. Ricordiamo adesso che $D_n$ contiene gli elementi
    della forma $r^k$, $sr^h$ al variare di $h, k \in \ZZ$, mostriamo che questi sono 
    infatti $2n$. Gli elementi $r^k$ appartengono al gruppo ciclico $\mathcal{R}$
    di ordine $n$, pertanto sono $n$ elementi distinti, inoltre 
    \[
        sr^i = sr^j \iff r^i = r^j\iff i \equiv j \pmod n
    \] pertanto anche questi sono $n$
    elementi distinti. Poiché gli insiemi $\mathcal{R}$ e 
    $\{sr^h\mid h \in \ZZ\}$ sono disgiunti (\hyperref[obs1.11]{Osservazione 1.11}) 
    abbiamo $|D_n| = 2n$.
\end{proof}

\begin{remark}
    Abbiamo mostrato che effettivamente $D_n = \langle r, s\rangle$, quindi i
    suoi elementi sono tutti della forma $r^k$, $sr^h$ al variare di $h, k \in \ZZ$. 
\end{remark}

\begin{remark}
    Il risultato è valido anche per $D_2$, ma con motivazioni diverse. 
    Se consideriamo un segmento nel piano $\RR^2$ giacente sulla retta $y = 0$, 
    le isometrie che possiamo applicare sono l'identità, la rotazione di 
    angolo $\pi$, la simmetria lungo la retta $y = 0$ e la simmetria lungo l'asse
    passante per il suo punto medio. $D_2$ contiene quindi quattro elementi,
    l'identità e tre elementi di ordine 2, pertanto è isomorfo a $\Z2\times\Z2$.
\end{remark}


\subsubsection{Sottogruppi}

Consideriamo un sottogruppo $H\leqslant D_n$, distinguiamo due possibilità: 
$H \subseteq \mathcal{R}$ oppure $H \nsubseteq \mathcal{R}$. Nel primo caso
abbiamo che $|H|\mid n$, ed è l'unico sottogruppo di $\mathcal{R}$ con questa 
proprietà in quanto $\mathcal{R}$ è ciclico, in particolare $H$ è ciclico 
della forma $\langle r^{\frac n d}\rangle$, con $\mathcal{R} = 
\langle r\rangle$ e $d \mid n$. \newline
Studiamo quindi il caso $H \nsubseteq \mathcal{R}$: notiamo che 
$\mathcal{R}\trianglelefteqslant D_n$ in quanto $[D_n : \mathcal{R}] = 2$,
pertanto $\faktor{D_n}{\mathcal{R}}$ è un gruppo con l'operazione indotta da $D_n$
e risulta essere isomorfo a $\Z2$. \newline
Consideriamo la proiezione al quoziente 
\[
    \pi_{\mathcal{R}}: D_n \longrightarrow \faktor{D_n}{\mathcal{R}} : g \mapsto g\mathcal{R}
\]
poiché $H \nsubseteq \mathcal{R}$ abbiamo che esiste $h \in H$ tale che 
$h \notin \mathcal{R}$, pertanto $\pi_{\mathcal{R}}(h) \neq \mathcal{R}$ e
in particolare $\pi_{\mathcal{R}}(H) \nsubseteq \{\mathcal{R}\}$. Dato che i 
sottogruppi di $\faktor{D_n}{\mathcal{R}}$ sono solo $\{\mathcal{R}\}$ e
$\faktor{D_n}{\mathcal{R}}$ abbiamo $\pi_{\mathcal{R}}(H) = 
\faktor{D_n}{\mathcal{R}}$. Osserviamo inoltre che $\ker \pi_{\mathcal{R}\mid H} = 
\ker \pi_{\mathcal{R}} \cap H = \mathcal{R}\cap H$, per il Primo Teorema di Omomorfismo
allora $\displaystyle\frac{H}{H\cap \mathcal{R}} \cong \Z2$, quindi 
$|H\cap\mathcal{R}| = \displaystyle\frac 1 2 |H|$. Dato che $\mathcal{R}\cap H \subseteq
\mathcal{R}$, esiste $k \in \ZZ$ tale che $H\cap\mathcal{R} = \langle r^k\rangle$
in particolare $\langle r^k\rangle$ e $\langle sr^h\rangle$, $h \in \ZZ$, sono
contenuti in $H$. 

\begin{proposition}
    Dati $H\leqslant D_n$ un sottogruppo tale che $H\nsubseteq \mathcal{R}$, se
    $r$ è un generatore di $\mathcal{R}$ tale che $H\cap\mathcal{R} = \langle r^k\rangle$ 
    e $s$ è una simmetria allora \[
    H = \langle r^k\rangle\cdot\langle sr^h\rangle = \{xy \mid x \in \langle r^k\rangle,
    y \in \langle sr^h\rangle\}\qquad h, k \in \ZZ
    \]
\end{proposition}

\begin{proof}
    Per quanto visto sopra vale $|\langle r^k\rangle| = 
    \displaystyle\frac 1 2|H|$, inoltre $\ord(sr^h) = 2$:
    \[
        (sr^h)^2 = sr^hsr^h = (srs)^hr^h = (srs^{-1})^hr^h = r^{-h}r^h = e
    \]
    pertanto $\langle sr^h\rangle \cong \Z2$. Da questo ricaviamo $\langle sr^h\rangle
    \subseteq N_{D_n}(\langle r^k\rangle)$, infatti per ogni $m \in \ZZ$ 
    \[
        (sr^h)r^{mk}(sr^h)^{-1} = sr^{h + mk} sr^h = r^{-h-mk}r^h = r^{-mk}
        \in \langle r^k \rangle
    \]
    cioè $\langle sr^h\rangle \subseteq N_{D_n}(\langle r^k\rangle)$ e quindi
    $\langle r^k\rangle\cdot\langle sr^h\rangle$ è un sottogruppo di $D_n$\footnote
    {Dati $K, N$ sottogruppi
    di un gruppo $G$, se vale almeno una delle inclusioni $K \subseteq N_G(N)$,
    $N \subseteq N_G(K)$ allora $HK = KH$, quindi $HK$ è un sottogruppo di $G$.}.
    Poiché $\langle r^k\rangle$ e $\langle sr^h\rangle$ sono contenuti in $H$
    abbiamo che $\langle r^k\rangle\cdot\langle sr^h\rangle \subseteq H$, inoltre
    \[|\langle r^k\rangle\cdot\langle sr^h\rangle| = \displaystyle\frac 1 2 |H|\cdot 2 = |H|\]
    in quanto $\langle r^k\rangle\cap\langle sr^h\rangle = \{e\}$\footnote{
        Se $H, K$ sono sottogruppi finiti di un gruppo $G$ e $HK\leqslant G$ allora
        vale $|HK| = \displaystyle\frac{|H|\cdot|K|}{|H\cap K|}$.
    }, pertanto 
    i due sottogruppi coincidono.
\end{proof}

\begin{remark}
    Per $k \mid n$ e $0\leq h < k$, i sottogruppi $H_{k, h} = \langle r^k, sr^h\rangle$
    e $H = \langle r^k\rangle\cdot\langle sr^h\rangle$ coincidono. Infatti 
    $H_{k, h}\subseteq H$ in quanto $r^k, sr^h$ sono elementi di $H$, 
    d'altra parte $H \subseteq H_{k, h}$ in quanto $H_{h, k}$ contiene tutti i 
    prodotti finiti delle potenze di $r^k$ e $sr^h$, in particolare gli elementi di $H$.
\end{remark}

\begin{remark}
    Per $k \mid n$ e $0\leq h < k$, $\langle r^k, sr^h\rangle = 
    \langle r^k, sr^{h + k}\rangle$. Infatti $\langle r^k, sr^h\rangle \subseteq
    \langle r^k, sr^{h + k}\rangle$ in quanto $sr^h = (sr^{h + k})r^{-k}$ è
    un elemento del secondo gruppo, simmetricamente $\langle r^k, sr^{h + k}\rangle
    \subseteq \langle r^k, sr^h\rangle$ in quanto $sr^{h + k} = (sr^h)r^k$ è un
    elemento del primo gruppo.
\end{remark}

\begin{theorem}
    [Classificazione dei sottogruppi di $D_n$]
I sottogruppi di $D_n$ sono della forma \begin{enumerate}[(1)]
    \item $\langle r^k\rangle$ con $k\mid n$;
    \item $\langle r^k, sr^h\rangle$ con $k \mid n$, $0\leq h < k$, 
\end{enumerate}
con $r \in \mathcal{R}$ e $s$ una simmetria. Inoltre tali sottogruppi sono
tutti distinti.
\end{theorem}

\begin{proof}
    Abbiamo già visto che i sottogruppi di $D_n$ sono di questo tipo, 
    mostriamo quindi che sono tutti distinti. A meno di cambiare $k$, possiamo
    supporre $\mathcal{R} = \langle r\rangle$, cioè $\ord(r) = n$. 
    Consideriamo $H, K\leqslant D_n$ due sottogruppi, abbiamo tre casi:
    \begin{itemize}
        \item se $H = \langle r^k\rangle$ e $K = \langle r^m\rangle$, $m \in \ZZ$,
        allora $H = K\iff k = m$ in quanto entrambi sottogruppi di $\mathcal{R}$, 
        pertanto esiste un unico sottogruppo della forma $\langle r^k\rangle$
        per $k \mid n$;
        \item se $H = \langle r^k\rangle$ e $K = \langle r^m, sr^h\rangle$, $m \mid n$,
        allora $H \neq K$ in quanto $H$ è ciclico e $K$ no;
        \item se $H = \langle r^k, sr^h\rangle$ e $K = \langle r^m, sr^l\rangle$, 
        con $m \mid n$ e $0\leq l < m$, considerando le intersezioni
        $H \cap \mathcal{R} = \langle r^k\rangle$ e $K \cap \mathcal{R} = \langle r^m\rangle$ 
        abbiamo \[
        H \cap \mathcal{R} = K\cap\mathcal{R} \iff \langle r^k\rangle = \langle r^m\rangle
        \iff k = m
        \] Inoltre, se $sr^h \in \langle r^m, sr^l\rangle = \langle r^m\rangle
        \cdot \langle sr^l\rangle$, allora esiste $t \in \ZZ$ tale che \[
        sr^h = (r^m)^t sr^l \iff sr^h = s^2r^{mt}sr^l \iff r^h = r^{-mt + l}
        \iff h \equiv l - mt \pmod n
        \]da cui ricaviamo $h \equiv l \pmod m$ in quanto $m \mid n$. Ma allora 
        $h =l$ dato che $0 \leq h < k$, $0\leq l < m$ e $k = l$.
    \end{itemize}
\end{proof}

\begin{lemma}
    \label{lemma1.20}
    Dati un gruppo $G$ e $A, B$ due sottogruppi tali che $A \leqslant B \leqslant G$,
    se $B\trianglelefteqslant G$ e $A$ è caratteristico in $B$ allora 
    $A \trianglelefteqslant G$.
\end{lemma}

\begin{proof}
    Fissato $g \in G$, consideriamo l'omomorfismo di coniugio 
    \[
        \varphi_g : G\longrightarrow G : x\longmapsto gxg^{-1}
    \] poiché 
    $B\trianglelefteqslant G$ è ben definita la restrizione $\varphi_{g\mid B} \in \Aut(B)$\footnote{
        Notiamo che $\varphi_{g\mid B}$ in generale non è un 
        coniugio di $B$, poiché $g$ non appartiene necessariamente a $B$.
    }. 
    Dal momento che $A$ è
    un sottogruppo caratteristico di $B$ abbiamo che $\varphi_{g\mid B}(A) =
    \varphi_g(A) = A$,
    pertanto $A \trianglelefteqslant G$.
\end{proof}


\begin{corollary}
    Ogni sottogruppo di $\mathcal{R}$ è normale in $D_n$.
\end{corollary}

\begin{proof}
    Siano $\langle r^k\rangle$ un sottogruppo di $\mathcal{R}$ e $\varphi
    \in \Aut(\mathcal{R})$, allora $\varphi(\langle r^k\rangle) = \langle r^k\rangle$
    in quanto $\varphi$ preserva l'ordine del sottogruppo e $\langle r^k\rangle$
    è l'unico sottogruppo di $\mathcal{R}$ di tale ordine ($\mathcal{R}$ è ciclico),
    pertanto $\langle r^k\rangle$
    è caratteristico in $\mathcal{R}$. Poiché $\mathcal{R}$ è un sottogruppo
    normale di $D_n$, per il \hyperref[lemma1.20]{Lemma 1.20}
    abbiamo $\langle r^k\rangle\trianglelefteqslant D_n$.
\end{proof}

\begin{remark}
    $\mathcal{R} = \langle r \rangle$ è caratteristico in $D_n$ per $n \geqslant 3$.
    Infatti per ogni $\varphi \in \Aut(D_n)$ allora
    $\ord(r) = \ord(\varphi(r))$, da cui $|\langle\varphi(r)\rangle| = n$.
    Se fosse $\varphi(r) \notin \mathcal{R}$ avremmo $\ord(\varphi(r)) = 2$, 
    quindi $|\langle \varphi(r)\rangle| = n = 2$, che è assurdo in quanto $|D_n| \geqslant 6$.
    Questo non è vero per $D_2$, che contiene una rotazione e due
    simmetrie: poiché $\Aut(D_2) \cong S_3$ esiste un $\psi \in \Aut(D_2)$ che manda 
    la rotazione in una riflessione.
\end{remark}

\begin{corollary}
    Per $k\mid n$ e $0\leq h < k$, il sottogruppo $H_{k, h} = \langle r^k, sr^h\rangle$
    è normale in $D_n$ se e solo se $r, s \in N_{D_n}(H_{k, h})$.
\end{corollary}

\begin{proof}~
    \begin{itemize}
        \item Se $H_{k, h}\trianglelefteqslant D_n$ allora $N_{D_n}(H_{k, h}) = D_n$, 
        in particolare $r, s \in N_{D_n}(H_{k, h})$;
        \item se $r, s \in N_{D_n}(H_{k, h})$, poiché il normalizzatore è un
        sottogruppo di $D_n$ abbiamo che $D_n = \langle r, s\rangle \subseteq
        N_{D_n}(H_{k, h})$, pertanto $H_{k, h} \trianglelefteqslant D_n$.
    \end{itemize}
\end{proof}

Vediamo quali sono i sottogruppi normali della forma 
$\langle r^k, sr^h\rangle$, consideriamo i coniugi \[
    \varphi_s: D_n \longrightarrow D_n :x \longmapsto sxs^{-1}\qquad
    \varphi_r: D_n \longrightarrow D_n :x \longmapsto rxr^{-1}
\]e sia $x_1^{\pm 1}\ldots x_m^{\pm 1} \in H_{k, h} = \langle r^k, sr^h\rangle$, allora
\[
    \varphi_s(x_1^{\pm 1}\ldots x_m^{\pm 1}) = \varphi_s(x_1)^{\pm 1}\ldots \varphi_s(x_m)^{\pm 1}
    \in \langle srs, r^hs^{-1}\rangle = \langle sr^ks, r^hs^{-1}\rangle = \langle
    r^k, sr^{-h}\rangle
\]
\[
    \varphi_r(x_1^{\pm 1}\ldots x_m^{\pm 1}) = \varphi_r(x_1)^{\pm 1}\ldots \varphi_r(x_m)^{\pm 1}
    \in \langle r^k, rsr^{h - 1}\rangle = \langle r^k, sr^{h - 2}\rangle
\]

Pertanto $H_{k, h}\trianglelefteqslant D_n$ se e solo se $\langle r^k, sr^{h - 2}\rangle
= \langle r^k, sr^{-h}\rangle = \langle r^k, sr^h\rangle$, se e solo se 
$h \equiv h - 2 \pmod k$, cioè $k \in \{1, 2\}$.\begin{itemize}
    \item Se $k = 1$ allora $H_{k, h} = \langle r, s\rangle = D_n$;
    \item se $k = 2$ (e $n$ pari) allora $H_{k, h} = \langle r^2, sr\rangle$ oppure 
    $H_{k, h} = \langle r^2, s\rangle$.
\end{itemize}

\begin{remark}
    Il secondo caso si presenta solo se $n$ è pari, questo corrisponde al fatto 
    che in un poligono con un numero pari di lati gli assi di simmetria sono 
    per metà passanti per i lati e metà passanti per i vertici opposti. In un poligono con un numero dispari
    di lati gli assi di simmetria sono tutti passanti per i lati.
\end{remark}


\subsubsection{Classi di coniugio}

Abbiamo visto che possiamo scrivere ogni elemento di $D_n$ nella forma
 $s^hr^k$, dove $s$ è una simmetria e $r$ è una rotazione che genera 
 $\mathcal{R}$, con $h \in \{0, 1\}$ e $k \in \{0, \ldots, n - 1\}$
in quanto $\ord(s)= 2$ e $\ord(r) = n$. Inoltre tutti gli elementi della
forma $sr^k$ hanno ordine $2$.

Consideriamo la classe di coniugio di $r$, $\Cl(r) = \{grg^{-1}\mid g \in D_n\}$,
fissato $g \in D_n$ abbiamo due possibili valori per $grg^{-1}$:
\begin{itemize}
    \item se $g \in \mathcal{R}$ allora $g$ è una potenza di $r$, pertanto i
    due elementi commutano e si ha $grg^{-1} = r$;
    \item se $g\notin\mathcal{R}$ allora $g = sr^h$ con $h \in \ZZ$, quindi
    \[
        (sr^h)r(sr^h)^{-1} = (sr^h)r(sr^h) = sr^{h + 1}sr^h = s^2r^{-1-h}r^h = r^{-1}
    \]
\end{itemize}
cioè $\Cl(r) = \{r, r^{-1}\}$. In modo analogo si mostra che $\Cl(r^k) = \{r^k, r^{-k}\}$
per ogni $k \in \ZZ$.

\begin{remark}
    Se $n$ è pari, scriviamo $n = 2m$ e consideriamo la classe di coniugio
    di $r^m$. Poiché $r^m \neq e$ e $r^{2m} = (r^m)^2 = e$ abbiamo che
    $\ord(r^m) = 2$, cioè $(r^m)^{-1} = r^m$. Allora $\Cl(r^m) = \{r^m\}$,
    pertanto abbiamo trovato un elemento del centro di $D_n$ (infatti se $G$
    è un gruppo e $x \in G$, allora $x \in Z(G)$ se e solo se $\Cl(x) = \{x\}$).
\end{remark}

Consideriamo adesso la classe di coniugio di $sr^h$, $\Cl(sr^h) = 
\{g(sr^h)g^{-1}\mid g\in D_n\}$, fissato $g \in D_n$ abbiamo due possibili
valori per $g(sr^h)g^{-1}$:
\begin{itemize}
    \item se $g \in \mathcal{R}$ allora $g = r^k$ con $k \in \ZZ$, pertanto
    \[
        r^k(sr^h)r^{-k} = sr^{-k}r^h r^{-k} = sr^{h - 2k}
    \]
    \item se $g \notin \mathcal{R}$ allora $g = sr^k$ con $k \in \ZZ$, pertanto
    \[
        (sr^k)(sr^h)(sr^k)^{-1} = (sr^k)(sr^h)(sr^k) = sr^{2k - h}
    \]
\end{itemize}
cioè $\Cl(sr^h) = \{sr^{h - 2k}, sr^{2k - h}\mid k \in \ZZ\}$. 

\begin{remark}
    La classe di coniugio di $sr^h$ contiene tutte le simmetrie in cui 
    l'esponente di $r$ ha la stessa parità di $h$. Se $n$ è 
    dispari tutte le simmetrie appartengono alla stessa classe,
    mentre se $n$ è pari abbiamo due classi distinte: quella 
    delle simmetrie rispetto agli assi passanti per i vertici opposti e quella 
    delle simmetrie rispetto agli assi passanti per i lati.
\end{remark}


\subsubsection{Legge di gruppo e omomorfismi}

Se $g$ è un elemento di $D_n$ possiamo scrivere $g$ in modo unico come $s^ar^b$ 
con $a \in \{0, 1\}$ e $b \in \{0, \ldots, n - 1\}$, utilizziamo questa 
proprietà per esplicitare la legge di gruppo di $D_n$. \newline
Fissati $g_1, g_2 \in D_n$, scriviamo $g_1 = s^{a_1}r^{b_1}$ e $g_2 = s^{a_2}r^{b_2}$
con $a_1, a_2 \in \{0, 1\}$ e $b \in \{0, \ldots, n - 1\}$, 
\[
    g_1g_2 = (s^{a_1}r^{b_1})(s^{a_2}r^{b_2}) = s^{a_1}s^{a_2}(s^{a_2}r^{b_1}s^{-a_2})r^{b_2} = 
    s^{a_1}s^{a_2}\varphi_{s^{a_2}}(r^{b_1})r^{b_2}
\]
dove $\varphi_{s^{a_2}}$ è l'automorfismo di coniugio per $s^{a_2}$
(ricordiamo che $s^{a_2} = s^{-a_2}$). Poiché
$\varphi_{s^{a_2}}$ è un omomorfismo e $\varphi_x\circ\varphi_y = \varphi_{xy}$ per ogni $x, y \in G$,
abbiamo $(\varphi_{s^{a_2}}(r^{b_1})) = (\varphi_s^{a_2}(r))^{b_1}$, quindi
\[
    g_1g_2 = s^{a_1}s^{a_2}(\varphi_s^{a_2}(r))^{b_1}r^{b_2} = 
    s^{a_1 + a_2}r^{(-1)^{a_2}b_1 + b_2}
\]
Per l'unicità della scrittura che stiamo usando (scegliendo 
$a \in \{0, 1\}$ e $b \in \{0, \ldots, n - 1\}$)\footnote{
    Ricordiamo che $\varphi_s^m = \underset{m\text{ volte}}{\underbrace{\varphi_s\circ\ldots\circ\varphi_s}}$
    in quanto l'operazione del gruppo degli automorfismi è la composizione di 
    funzioni.} possiamo identificare 
ogni elemento $g = s^ar^b \in D_n$ con la coppia $(a, b)$, la legge di gruppo
è quindi tale che \[
    (a_1, b_1)(a_2, b_2) = (a_1 + a_2, (-1)^{a_2}b_1 + b_2)
\]
\newline 
Usiamo il risultato appena ottenuto per descrivere gli omomorfismi da $D_n$ in 
un qualsiasi gruppo $G$. Poiché ogni elemento $g \in D_n$ si scrive come
$s^ar^b$, con $a, b \in \ZZ$, un omomorfismo $\varphi \in \Hom(D_n, G)$ è univocamente
determinato da $\varphi(r)$ e $\varphi(s)$: infatti \[
    \varphi(g) = \varphi(s^ar^b) = \varphi(s)^a\varphi(r)^b
\]Poniamo $x = \varphi(s)$, $y = \varphi(r)$, necessariamente $\ord(x) \mid 2$
e $\ord(y)\mid n$, cioè $x^2 = e_G$ e $y^n = e_G$, inoltre \[
    xyx^{-1} = \varphi(s)\varphi(r)\varphi(s)^{-1} = \varphi(srs^{-1}) = 
    \varphi(r^{-1}) = \varphi(r)^{-1} = y^{-1}
\]Mostriamo che effettivamente queste condizioni sono anche sufficienti:

\begin{proposition}
    Dati un gruppo $G$ e un'applicazione
    \[
        \varphi:D_n\longrightarrow G :s^ar^b \longmapsto x^ay^b
    \]dove $x = \varphi(s)$ e $y = \varphi(r)$, allora $\varphi$ è un omomorfismo
    se e solo se $x^2 = e_G$, $y^n = e_G$ e $xyx^{-1} = y^{-1}$.
\end{proposition}

\begin{proof}
    Mostriamo che tali condizioni sono sufficienti affinché $\varphi$ sia un 
    omomorfismo. Poiché $x^m = x^{-m}$ per ogni $m \in \ZZ$, fissati 
    $a_1, a_2, b_1, b_2 \in \ZZ$ abbiamo 
    \begin{multline*}
        (x^{a_1}y^{b_1})(x^{a_2}y^{b_2}) = x^{a_1}x^{a_2}(x^{a_2}y^{b_1}x^{-a_2})y^{b_2} = 
        x^{a_1 + a_2}\varphi_{x^{a_2}}(y^{b_1})y^{b_2} = \\
        = x^{a_1 + a_2} (\varphi_x^{a_2}(y))^{b_1}y^{b_2} =
        x^{a_1 + a_2}y^{(-1)^{a_2}b_1}y^{b_2} = x^{a_1 + a_2}y^{(-1)^{a_2}b_1 + b_2}
    \end{multline*}dove $\varphi_g$ è l'automorfismo di coniugio per $g \in G$.
    Allora abbiamo che $\varphi$ è un omomorfismo, infatti per ogni $h_1, h_2, k_1, k_2 \in \ZZ$
    \begin{multline*}
        \varphi((s^{h_1}r^{k_1})(s^{h_2}r^{k_2})) = \varphi(s^{h_1 + h_2}r^{(-1)^{h_2}k_1 + k_2}) =\\
        = x^{h_1 + h_2}y^{(-1)^{h_2}k_1 + k_2} = (x^{h_1}y^{k_1})(x^{h_2}y^{k_2}) = 
        \varphi(s^{h_1}r^{h_2})\varphi(s^{h_2}r^{h_2})
    \end{multline*}
\end{proof}

\begin{remark}
    Abbiamo visto che le condizioni $D_n = \langle r, s\rangle$ con $\ord(r) = n$,
    $\ord(s) = 2$ e $srs^{-1} = r^{-1}$ determinano in modo univoco 
    la struttura astratta di $D_n$, racchiudiamo queste proprietà fondamentali
    nella scrittura
    \[
        \langle r, s\mid r^n = s^2 = e, srs^{-1} = r^{-1}\rangle
    \]
    Tale scrittura si chiama \vocab{presentazione di un gruppo} e ne determina 
    in modo univoco la classe di isomorfismo. Senza scendere troppo nei dettagli,
    nella presentazione indichiamo un insieme di generatori minimale e il 
    minor numero di proprietà che i generatori devono rispettare affinché il 
    gruppo abbia la struttura desiderata. Altri esempi di presentazioni sono
    \[
        \langle x \mid x^n = e\rangle
    \]
    \[
        \langle x\rangle
    \]
    \[
        \langle x, y \mid x^2 = y^2 = e, xy = yx\rangle
    \]
    rispettivamente dei gruppi $\Zn$, $\ZZ$, $\Z2\times\Z2$
    (notiamo che $\Z2\times\Z2$ e $D_2$ hanno la stessa presentazione,
    e questo ha senso in quanto i due gruppi sono isomorfi).
\end{remark}


\subsubsection{Automorfismi}

Studiamo separatamente gli automorfismi di $D_n$ per $n \geqslant 3$ e di $D_2$.\newline
Per $n\geqslant 3$ consideriamo $\varphi \in \Aut(D_n)$, poiché $D_n = \langle
r, s\rangle$ è sufficiente studiare le immagini di $r, s$ per determinare $\varphi$.
Osserviamo che necessariamente $\varphi(r) = r^k$ con $(n, k) = 1$, infatti 
$\varphi$ deve preservare l'ordine di $r$ e la sua immagine deve essere un 
generatore di $\mathcal{R}$, in quanto $\mathcal{R}$ è caratteristico in $D_n$
e isomorfo a $\Zn$. Per quanto riguarda $\varphi(s)$, se $n$ è dispari allora le simmetrie
sono gli unici elementi di ordine 2, pertanto $\varphi(s) = sr^h$ con 
$0\leq h < n$. Se $n$ è pari abbiamo apparentemente due possibilità:
\begin{enumerate}[(1)]
    \item $\varphi(s) = sr^h$, con $0\leq h < n$;
    \item $\varphi(s) = r^{\frac n 2}$, se $n$ è pari.
\end{enumerate}

D'altra parte, se fosse $\varphi(s) = r^{\frac n 2}$ allora $\varphi$ non
sarebbe né iniettiva né surgettiva, pertanto $\varphi(s) = sr^h$ con 
$0\leq h \leq n$. Verifichiamo che $\varphi$ è un omomorfismo, per la 
caratterizzazione che abbiamo dato sopra è sufficiente verificare che
$\varphi(s)\varphi(r)\varphi(s)^{-1} = \nolinebreak\varphi(r)^{-1}$:
\[
    \varphi(s)\varphi(r)\varphi(s)^{-1} = (sr^h)r^k(sr^h)^{-1} = sr^{h + k}r^{-h}s =
    sr^k s^{-1} = r^{-k} = \varphi(r)^{-1}
\]
Inoltre $\varphi$ è surgettiva, infatti $r^k, sr^h \in \mathrm{Im}\varphi$,
cioè 
\[
    \langle r^k, sr^h\rangle = \langle r, sr^h\rangle = \langle s, r\rangle =
    D_n\subseteq \mathrm{Im}\varphi
\]da cui $\mathrm{Im}\varphi = D_n$. Poiché $D_n$ è finito abbiamo che $\varphi$
è un automorfismo. Gli automorfismi di $D_n = \langle r, s\rangle$ quindi sono
tutti e soli gli omomorfismi da $D_n$ in $D_n$ che mandano $r$ in un generatore
di $\mathcal{R}$, che sono $\phi(n)$, e $s$ in un'altra simmetria, che sono 
$n$, pertanto $|\Aut(D_n)| = n\phi(n)$.\newline

Per $n = 2$, sappiamo che $D_2 \cong (\Z2)^2$, pertanto 
\[
    \Aut(D_2) \cong \Aut((\Z2)^2) \cong S_3
\]
Alternativamente possiamo considerare $(\Z2)^2$ come spazio vettoriale su $\FF_2$,
pertanto abbiamo 
\[
    \Aut(D_2) \cong GL_2(\FF_2)
\]Per quanto visto nella sezione \hyperref[sez 1.2]{(1.2)}, $GL_2(\FF_2)$ 
contiene $(4 - 1)(4 - 2) = 6$ elementi, inoltre $GL_2$ non è un gruppo 
commutativo (con l'operazione di prodotto tra matrici), pertanto $GL_2(\FF_2) \cong S_3$.
In particolare, gli elementi di $GL_2(\FF_2)$ sono:
\begin{itemize}
    \item $\begin{pmatrix}
    1 & 0\\
    0 & 1
    \end{pmatrix}$, che è l'identità del gruppo;
    \item $\begin{pmatrix}
    0 & 1\\
    1 & 0
    \end{pmatrix}, \begin{pmatrix}
        1 & 0\\
        1 & 1
    \end{pmatrix}, \begin{pmatrix}
        1 & 1\\
        0 & 1
    \end{pmatrix}$, che sono gli elementi di ordine 2 corrispondenti alle 
    trasposizioni;
    \item $\begin{pmatrix}
    1 & 1\\
    1 & 0
    \end{pmatrix}, \begin{pmatrix}
        0 & 1\\
        1 & 1
    \end{pmatrix}$ che sono gli elementi di ordine $3$ corrispondenti ai $3$-cicli.
\end{itemize}

\newpage

\subsection{Automorfismi di un prodotto diretto}

Consideriamo due gruppi finiti $H, K$, studiamo il gruppo degli automorfismi 
di $H\times K$. Chiaramente esiste un'inclusione di $\Aut(H)\times \Aut(K)$ in 
$\Aut(H\times K)$ data dall'omomorfismo 
\[
    \iota: \Aut(H)\times \Aut(K)\longhookrightarrow \Aut(H\times K) :
    (\varphi_1, \varphi_2)\longmapsto \varphi_1\times \varphi_2
\]con 
\[
    \varphi_1\times\varphi_2: H\times K \longrightarrow H\times K:
    (g_1, g_2)\longmapsto (\varphi_1(g_1), \varphi_2(g_2))
\]
Mostriamo che $\iota$ è ben definita e che è effettivamente un omomorfismo iniettivo:
\begin{itemize}
    \item per ogni $(\varphi_1, \varphi_2)\in \Aut(H)\times \Aut(K)$, per ogni
     $(g_1, g_2), (h_1, h_2)\in H\times K$ abbiamo 
     \begin{multline*}
        (\varphi_1\times\varphi_2)((g_1, g_2)(h_1, h_2)) = 
        (\varphi_1(g_1h_1), \varphi_2(g_2h_2)) = 
        (\varphi_1(g_1)\varphi_1(h_1), \varphi_2(g_2)\varphi_2(h_2)) = \\
        =(\varphi_1(g_1), \varphi_2(g_2))(\varphi_1(h_1),\varphi_2(h_2)) = 
        ((\varphi_1\times\varphi_2)(g_1, g_2))((\varphi_1\times\varphi_2)(h_1,h_2))
     \end{multline*}
     cioè $\varphi_1\times\varphi_2$ è un omomorfismo. Inoltre 
    \[
        \ker (\varphi_1\times\varphi_2) = \{(g_1, g_2) \in H\times K\mid 
        (\varphi_1(g_1), \varphi_2(g_2)) = (e_H, e_K)\} = \{(e_H, e_K)\}
    \]
    quindi $\varphi_1\times \varphi_2 \in \Aut(H\times K)$
    in quanto $H\times K$ è finito, pertanto $\iota$ è ben definita;
    \item per ogni $(\varphi_1, \varphi_2), (\psi_1, \psi_2) \in \Aut(H)\times \Aut(K)$,
    per ogni $(g_1, g_2) \in H\times K$ abbiamo 
    \begin{multline*}
        \iota((\varphi_1, \varphi_2)(\psi_1, \psi_2))(g_1, g_2) = 
        \iota(\varphi_1\psi_1, \varphi_2\psi_2)(g_1, g_2) = 
        (\varphi_1\psi_1\times\varphi_2\psi_2)(g_1, g_2) =\\
        = (\varphi_1(\psi_1(g_1)), \varphi_2(\psi_2(g_2))) = 
        (\varphi_1\times\varphi_2)(\psi_1(g_1), \psi_2(g_2)) = \\
        = ((\varphi_1\times\varphi_2)(\psi_1\times\psi_2))(g_1, g_2) = 
        (\iota(\varphi_1, \varphi_2)\iota(\psi_1,\psi_2))(g_1, g_2)
    \end{multline*}cioè $\iota((\varphi_1, \varphi_2)(\psi_1, \psi_2)) = 
    \iota(\varphi_1, \varphi_2)\iota(\psi_1, \psi_2)$, quindi $\iota$ è un 
    omomorfismo;
    \item$\iota$ è iniettiva, infatti \begin{multline*}
        \ker \iota = \{(\varphi_1, \varphi_2) \in \Aut(H)\times \Aut(K) \mid
        \iota(\varphi_1, \varphi_2) = id_{\Aut(H\times K)}\} = \\
        = \{(\varphi_1, \varphi_2) \in \Aut(H)\times \Aut(K)\mid 
        (\varphi_1(g_1), \varphi_2(g_2)) = (g_1, g_2)~\forall 
        (g_1, g_2) \in H\times K\} = \\
        = \{(id_{\Aut(H)}, id_{\Aut(K)})\} = \{id_{\Aut(H)\times\Aut(K)}\}
    \end{multline*}

\end{itemize}

\begin{proposition}
    Dati due gruppi finiti $H, K$, $\Aut(H)\times \Aut(K)\cong \Aut(H\times K)$
    se e solo se $H\times \{e_K\}$ e $\{e_H\}\times K$ sono sottogruppi 
    caratteristici di $H\times K$.
\end{proposition}

\begin{proof}
    Sia $\iota$ l'immersione da $\Aut(H)\times \Aut(K)$ in $\Aut(H\times K)$ 
    definita come sopra, se $\iota$ è surgettiva allora ogni elemento di 
    $\Aut(H\times K)$ può essere scritto come $\varphi_1\times\varphi_2$ con
    $\varphi_1 \in \Aut(H)$ e $\varphi_2 \in \Aut(K)$. Allora abbiamo 
    \[
        (\varphi_1\times\varphi_2)(H\times\{e_K\}) = 
        (\varphi_1(H), \varphi_2(\{e_K\})) = H\times\{e_K\}
    \]
    \[
        (\varphi_1\times \varphi_2)(\{e_H\}\times K) = 
        (\varphi_1(\{e_H\}), \varphi_2(K)) = \{e_H\}\times K
    \]cioè $H\times\{e_K\}$ e $\{e_H\}\times K$ sono caratteristici in
    $H\times K$. Viceversa, se i due sottogruppi sono caratteristici, dato
    $\varphi \in \Aut(H\times K)$ poniamo $\varphi_1 \in \Aut(H)$ tale che 
    $\varphi(g_1, e_K) = (\varphi_1(g_1), e_K)$ e $\varphi_2 \in \Aut(K)$ 
    tale che $\varphi(e_H, g_2) = (e_H, \varphi_2(g_2))$ per ogni $g_1 \in H$,
    per ogni $g_2 \in K$ (questo possiamo farlo in quanto $H\times\{e_K\}$ 
    e $\{e_H\}\times K$ sono caratteristici). Allora abbiamo 
    \begin{multline*}
        \varphi(g_1, g_2) = \varphi((g_1, e_K)(e_H, g_2)) = 
        \varphi(g_1, e_K)\varphi(e_H, g_2) = \\
        = (\varphi_1(g_1), e_K)(e_H, \varphi_2(g_2)) = 
        (\varphi_1(g_1), \varphi_2(g_2)) = (\varphi_1\times\varphi_2)(g_1, g_2)
    \end{multline*}cioè $\iota$ è surgettiva e quindi un isomorfismo tra
    $\Aut(H)\times \Aut(K)$ e $\Aut(H\times K)$.
\end{proof}

\begin{example}
    Consideriamo il gruppo $G = \ZZ \times \Zn$, osserviamo che il sottogruppo 
    $\{0\}\times \Zn$ è caratteristico in quanto un automorfismo $\varphi$ di $G$ deve
    preservare gli ordini degli elementi, in particolare quello di un generatore,
    quindi l'immagine di un generatore è un altro generatore del sottogruppo.
    Poiché gli elementi di $G$ di ordine finito sono tutti della forma $(0, d)$
    abbiamo che $\varphi(\{0\}\times\Zn) = \{0\}\times\Zn$. Viceversa, l'immagine
    di $\varphi$ su un generatore di $\ZZ\times\{0\}$, ad esempio 
    $\varphi(1, 0)$, è della forma $(a, b)$,
    e questo implica che $\ZZ\times \{0\}$ non è caratteristico. Se $\varphi$ è
    surgettivo, necessariamente esiste $(x, y) \in \ZZ\times \Zn$ tale che
    $\varphi(x, y) = (\pm 1, 0)$, da cui, posti $\varphi(1, 0) = (a, b)$ e
    $\varphi(0, 1) = (0, d)$ con $n$ e $d$ coprimi, abbiamo
    \begin{multline*}
        \varphi(x, y) = \varphi(x(1, 0) + y(0, 1)) = x\varphi(1, 0) + y\varphi(0, 1) = \\
        = x(a, b) + y(0, d) = (xa, xb + yd) = (\pm 1, 0) \iff a = \pm 1
    \end{multline*}
    Viceversa, se $a = \pm 1$ allora $\varphi$ è surgettiva, infatti 
    per ogni $(x_0, y_0) \in \ZZ\times\Zn$, scegliendo
    $x = x_0a$ e $y \equiv d^{-1}(y_0 - x_0ab)\pmod n$ abbiamo \[
        \varphi(x, y) = (x_0a^2, x_0ab + d d^{-1}(y_0 - x_0ab)) = (x_0, y_0)
    \]e questo ci permette di concludere che $\ZZ\times\{0\}$ non è un sottogruppo
    caratteristico. In questo caso abbiamo solo un'immersione del gruppo
    $\Aut(\ZZ) \times \Aut(\Zn)$ dentro a $\Aut(\ZZ\times \Zn)$, in quanto 
    gli automorfismi che mandano $(\pm 1, 0)$ in $(a, b)$ con $a = \pm 1$ e 
    $b \neq 0$ non possono essere ristretti ad automorfismi di $\ZZ\times \{0\}$.
\end{example}

È utile riuscire a determinare se i sottogruppi $H\times\{e_K\}$, $\{e_H\}\times K$
sono caratteristici in $H\times K$, da cui il seguente risultato:

\begin{proposition}
    Dati due gruppi finiti $H, K$, se $(|H|, |K|) = 1$ allora $H\times\{e_K\}$
    e $\{e_H\}\times K$ sono sottogruppi caratteristici di $H\times K$.
\end{proposition}

\begin{proof}
    Posti $n = |H|$, $m = |K|$, consideriamo l'insieme
    \[
        S = \{(g_1, g_2) \in H\times K\mid (g_1, g_2)^n = (e_H, e_K)\}\]
    Osserviamo che $H\times \{e_K\} = S$, infatti 
    $H\times \{e_K\} \subseteq S$ in quanto tutti gli elementi di $H\times\{e_K\}$
    hanno ordine che divide $n$. D'altra parte dato $(g_1, g_2) \in S$, se
    $\ord(g_1, g_2) \mid n$ allora $\ord(g_1)\mid n$ e $\ord(g_2)\mid n$, ma 
    $\ord(g_2) \mid m$ per il Teorema di Lagrange, quindi $\ord(g_2) = 1$ e
    $S \subseteq H\times\{e_K\}$, da cui l'uguaglianza. Con un ragionamento
    analogo possiamo caratterizzare $\{e_H\} \times K$ come 
    \[
        \{e_H\} \times K = \{(g_1, g_2) \in H\times K\mid (g_1, g_2)^m = (e_H, e_K)\}
    \] Poiché un automorfismo di $H\times K$ deve preservare gli ordini degli
    elementi, per la caratterizzazione data abbiamo che i due sottogruppi sono
    caratteristici.
\end{proof}

\begin{corollary}
    Se $m, n \geqslant 2$ sono interi coprimi allora
    \[
        \Aut(\Zn\times\Zm) \cong \Aut(\Zn)\times \Aut(\Zm)
    \]
\end{corollary}

\newpage

\subsection{Gruppo derivato}

\begin{definition}
    Dati un gruppo $G$ e $x, y$ elementi di $G$, chiamiamo \vocab{commutatore}
    di $x$ e $y$ l'elemento $[x, y] = xyx^{-1}y^{-1}$. Chiamiamo \vocab{sottogruppo
    derivato} di $G$, oppure \vocab{sottogruppo dei commutatori} di $G$
     il sottogruppo 
    \[
        G' = \langle\{[x, y]\mid x, y \in G\}\rangle
    \]
\end{definition}

\begin{remark}
    $[x, y] = e$ se e solo se $x$ e $y$ commutano.
\end{remark}

\begin{proposition}
    \label{prop1.35}
    Dato un gruppo $G$, valgono i seguenti fatti:
    \begin{enumerate}[(1)]
        \item $G'$ è un sottogruppo caratteristico di $G$;
        \item $\faktor{G}{G'}$ è un gruppo abeliano;
        \item dato $A$ un gruppo abeliano e $\varphi \in \Hom(G, A)$,
        allora $G' \subseteq \ker\varphi$.
    \end{enumerate}
\end{proposition}

\begin{proof}
    Mostriamo le affermazioni singolarmente:
    \begin{enumerate}[(1)]
        \item consideriamo $\varphi \in \Aut(G)$, poiché $\varphi$ preserva la struttura
        di gruppo è sufficiente descrivere come $\varphi$ agisce sui 
        generatori di $G'$ per determinare $\varphi(G')$. 
        Fissati $x, y \in\nolinebreak G$ abbiamo 
        \[
            \varphi([x, y]) = \varphi(xyx^{-1}y^{-1}) = \varphi(x)\varphi(y)
            \varphi(x)^{-1}\varphi(y)^{-1} = [\varphi(x), \varphi(y)]\in G'
        \]pertanto $\varphi(G') \subseteq G'$, da cui l'uguaglianza in quanto 
        $\varphi$ è bigettiva;
        \item dati $x, y \in G$, $xG'\cdot yG' = yG'\cdot xG'$ se e solo se 
        $xyG' = yxG'$, che è equivalente a richiedere $xyx^{-1}y^{-1} \in G'$. 
        Dato che effettivamente $xyx^{-1}y^{-1} = [x, y]$ è un elemento di $G'$
        abbiamo che $\faktor{G}{G'}$ è abeliano;
        \item dati $x, y \in G$, abbiamo 
        \[
            \varphi([x, y]) = \varphi(xyx^{-1}y^{-1}) = 
        \varphi(x)\varphi(y)\varphi(x)^{-1}\varphi(y)^{-1}
        \]
        e questo coincide con
        l'identità di $A$ in quanto $A$ è abeliano. Poiché l'immagine di $\varphi$
        è un sottogruppo di $A$ allora $G' \subseteq \ker\varphi$, in quanto
        il commutatore di ogni coppia di elementi di $G$ è contenuto in $\ker \varphi$.
    \end{enumerate}
\end{proof}

\begin{remark}
    Come conseguenza del Primo Teorema di Omomorfismo abbiamo che $\faktor{G}{G'}$ è 
    il "più grande" quoziente abeliano di $G$, o analogamente che 
    $G'$ è il "più piccolo" sottogruppo di $G$ che produce un quoziente abeliano.
    In questo senso, $G'$ misura quanto è abeliano il gruppo $G$.
\end{remark}

\begin{remark}
    Dato $A$ un gruppo abeliano, il Primo Teorema di Omomorfismo produce una bigezione naturale tra 
    $\Hom(G, A)$ e $\Hom\left(\faktor{G}{G'}, A\right)$. Consideriamo infatti $\varphi \in \Hom(G, A)$,
    $\pi_{G'}: G \longrightarrow \faktor{G}{G'}$ la proiezione al quoziente e 
    $\overline{\varphi} : \faktor{G}{G'}\longrightarrow A$, il Teorema
    fornisce un'unico omomorfismo $\overline{\varphi}: \faktor{G}{G'}\longrightarrow A$
    che rende commutativo il diagramma
    \begin{center}
        \begin{tikzcd}[column sep = small, row sep = small]
            G\arrow[rrr, "\varphi"]\arrow[ddd, "\pi_{G'}"', two heads]& & &A \\
            {}\arrow[rr, "\circlearrowleft", phantom]& & {}& \\
            & & & \\
            \faktor{G}{G'}\arrow[uuurrr, "\overline{\varphi}"', hook]& & &
        \end{tikzcd}
    \end{center}
    Viceversa, dato un omomorfismo $\overline{\varphi}:\faktor{G}{G'}\longrightarrow A$
    otteniamo un'unico omomorfismo $\varphi:G\longrightarrow A$ con la 
    composizione $\overline{\varphi}\circ\pi_{G'}$.
\end{remark}

\begin{example}
    Consideriamo il gruppo $S_3$, chiaramente $(S_3)' \neq \{id\}$ in quanto
    $\faktor{S_3}{\langle id\rangle} \cong S_3$ che non è abeliano, pertanto 
    abbiamo due possibilità: $(S_3)' = S_3$ oppure $(S_3)' = \langle\cycle{1, 2, 3}\rangle$\footnote{
        Gli unici sottogruppi normali di $S_3$ sono $\{id\}$, 
        $\langle\cycle{1, 2, 3}\rangle$, $S_3$.}. D'altra parte 
        $\faktor{S_3}{\langle\cycle{1, 2, 3}\rangle}$ è isomorfo a $\Z2$, che
        è abeliano, pertanto $(S_3)'$ è contenuto in $\langle\cycle{1, 2, 3}\rangle$,
        da cui necessariamente $(S_3)' = \langle\cycle{1, 2, 3}\rangle$.
        Più in generale vedremo che $(S_n)' = \mathcal{A}_n$, dove $\mathcal{A}_n$ è il sottogruppo
        di $S_n$ delle permutazioni pari (sappiamo già che $(S_n)' \subseteq
        \mathcal{A}_n$ in quanto $\faktor{S_n}{\mathcal{A}_n}\cong\Z2$).
\end{example}

\newpage

\subsection{Azioni di gruppo}

\subsubsection{Azioni transitive}

\begin{definition}
    Siano $G$ un gruppo e $X$ un insieme, un'azione \[
        \varphi:G\longrightarrow S(X) :g \longmapsto \varphi_g
    \]si dice \vocab{transitiva} se per ogni $x, y \in X$ esiste $g \in G$
    tale che $\varphi_g(x) = y$, equivalentemente se $\Orb(x) = X$ per ogni 
    $x \in X$. Diciamo anche che G \vocab{agisce transitivamente} su $X$ 
    tramite $\varphi$.
\end{definition}

\begin{lemma}
    \label{lemma1.40}
    Dato $G$ un gruppo finito e $H \lneq G$ un suo sottogruppo proprio, allora \[
        G \neq \bigcup_{g \in G}gHg^{-1}
    \]
\end{lemma}

\begin{proof}
    Poniamo $K = \displaystyle\bigcup_{g \in G}gHg^{-1}$, osserviamo che gli
    elementi della forma $xHx^{-1}$ con $x \in N_G(H)$ contribuiscono una
    sola volta all'unione, in quanto $xHx^{-1} = H$, pertanto $K$
    è unione di $[G:N_G(H)] = \displaystyle\frac{|G|}{|N_G(H)|}$ elementi distinti\footnote
    {Infatti, se $X = \{N\mid N\leqslant G\}$ e $\varphi$ è l'azione di coniugio
    su $X$, per ogni $N \in X$ abbiamo $\St(N) = N_G(N)$ e $\Orb(N) = \Cl(N) =
    \{gNg^{-1}\mid g \in G\}$. Vale quindi la relazione $|G| = |\Cl(N)|\cdot|N_G(N)|$.}. Poiché
    $H \subseteq N_G(H)$ e $|gHg^{-1}| = |H|$ per ogni $g \in G$, possiamo stimare 
    la cardinalità di $K$ nel seguente modo 
    \[
        |K| \leq\frac{|G|}{|N_G(H)|}|H| \leq\frac{|G|}{|H|}|H| = |G|.
    \]D'altra parte, per il Principio di Inclusione-Esclusione abbiamo che $|K|$ 
    è somma delle cardinalità dei singoli termini dell'unione se e solo se 
    l'unione è disgiunta, ma questo è falso in quanto ogni classe di coniugio
    di $H$ contiene l'identità del gruppo, quindi $|K| < |G|$, cioè $G \neq K$.
\end{proof}

\begin{proposition}
    \label{prop1.41}
    Dati un gruppo $G$ e un insieme $X$, se 
    \[
        \varphi:G\longrightarrow S(X) :g\longmapsto \varphi_g
    \]è un'azione transitiva valgono i seguenti fatti:
    \begin{enumerate}[(1)]
        \item per ogni $x, y \in X$ esiste $g \in G$ tale che $g\St(x)g^{-1} = \St(y)$;
        \item se $|X|\geqslant 2$ e $G$ è finito allora esiste $g \in G$ che agisce su $X$ senza
        punti fissi, cioè tale che $\varphi_g(x) \neq x$ per ogni $x \in X$.
    \end{enumerate}
\end{proposition}

\begin{proof}
    Mostriamo i due fatti singolarmente:
    \begin{enumerate}[(1)]
        \item sia $g \in G$ tale che $\varphi_g(x) = y$, dato 
        $h \in g\St(x)g^{-1}$ esiste $w \in \St(x)$ tale che $h = gwg^{-1}$. 
        Allora
        \[
            \varphi_h(y) = \varphi_{gwg^{-1}}(y) = 
            \varphi_g(\varphi_w(\varphi_g^{-1}(y))) = \varphi_g(\varphi_w(x)) =
            \varphi_g(x) = y
        \]pertanto $g\St(x)g^{-1} \subseteq \St(y)$. Osservando che 
        $\varphi_{g^{-1}}(y) = x$ e ragionando in modo simmetrico otteniamo
        l'inclusione $g^{-1}\St(y)g \subseteq \St(x)$, da cui $g\St(x)g^{-1} = \St(y)$;
        \item un elemento $g \in G$ con tali proprietà non può essere contenuto 
        nello stabilizzatore di nessun elemento di $X$, cioè cerchiamo $g \in G$
        tale che
        \[
            g \in \bigcap_{x \in X}\St(x)^{\mathcal{C}}
        \]
        che è equivalente a
        \[
            g \notin \bigcup_{x \in X} \St(x) = \bigcup_{h \in G}h\St(x_0)h^{-1}
        \]per il fatto precedente, fissato $x_0 \in G$. Osserviamo che 
        $\St(x_0) \neq G$, infatti se fosse $\St(x_0) = G$ avremmo 
        \[
            |\Orb(x_0)| = \frac{|G|}{|\St(x_0)|} = 1
        \]ma questo è assurdo in quanto $\Orb(x_0) = X$ per la transitività di 
        $\varphi$ e $|X|\geqslant 2$. Allora per il \hyperref[lemma1.40]{Lemma 1.40}
        abbiamo 
        \[
            G \neq \bigcup_{h \in G}h\St(x_0)h^{-1}
        \]pertanto esiste almeno un elemento $g\in G$ con la proprietà voluta.
    \end{enumerate}
\end{proof}

\begin{remark}
    Se $\varphi$ è l'azione di un gruppo $G$ su un insieme $X$, restringendo
    $\varphi$ all'orbita di un elemento $x \in X$ otteniamo per definizione
    un'azione transitiva su $\Orb(x)$. Pertanto gli stabilizzatori degli elementi 
    di $\Orb(x)$ sono tra loro coniugati.
\end{remark}

\begin{proposition}
    Dato $G$ un gruppo finito e $H \lneq G$ un sottogruppo proprio, se $[G:H] = p$
    con $p$ il più piccolo primo che divide l'ordine di $G$ allora $H$ è normale
    in $G$.
\end{proposition}

\begin{proof}
    Consideriamo l'azione di $G$ sull'insieme quoziente $\faktor{G}{H}$ 
    \[
        \psi: G\longrightarrow S\left(\faktor{G}{H}\right) : g \longmapsto \psi_g
    \]con 
    \[
        \psi_g : \faktor{G}{H}\longrightarrow\faktor{G}{H} : g'H\longmapsto gg'H
    \]Poiché l'immagine di $\psi$ è un sottogruppo di $S\left(\faktor{G}{H}\right)$,
    che è isomorfo a $S_p$, abbiamo che $|\mathrm{Im}\psi| \mid p!$, inoltre 
    $|\mathrm{Im}\psi| = \displaystyle\frac{|G|}{|\ker \psi|}$ come conseguenza
    del Primo Teorema di Omomorfismo. Pertanto $|\mathrm{Im}\psi| \mid (p!, |G|) = p$,
    in quanto $p$ è il più piccolo primo che divide $|G|$, quindi $|\mathrm{Im}\psi| \in \{1, p\}$.
    D'altra parte osserviamo che $\psi$ è un'azione transitiva, infatti per 
    ogni $g_1, g_2 \in G$ abbiamo $\psi_{g_2^{}g_1^{-1}}(g_1H) = g_2g_1^{-1}g_1H = g_2H$,
    pertanto non è possibile $\mathrm{Im}\psi = \{id\}$, da cui $|\mathrm{Im}\psi| = p$
    e $[G:\ker\psi] = p$. Consideriamo il nucleo di $\psi$
    \[
        \ker\psi = \{g\in G\mid gg'H = g'H~\forall g' \in G\}
    \]
    nel caso particolare $g' = e$ otteniamo l'inclusione
    \[
        \ker\psi \subseteq \{g \in G\mid gH = H\} = H
    \]
    in quanto stiamo indebolendo la condizione di appartenenza all'insieme.
    Poiché $[G:\ker \psi] = [G : H] = p$ e $G$ è un gruppo finito abbiamo
    che effettivamente $\ker\psi = H$, cioè $H$ è normale in $G$.
\end{proof}


\subsubsection{Teorema di Cauchy e Piccolo Teorema di Fermat}

Vediamo una dimostrazione alternativa del Teorema di Cauchy e del Piccolo
Teorema di Fermat, di cui ricordiamo gli enunciati, che fa uso del concetto
 di azione. 

\begin{theorem}
    [Teorema di Cauchy]
    \label{teorema1.44}
    Dato un gruppo $G$ e un numero primo $p$, se $p\mid |G|$ allora esiste 
    $g \in G$ tale che $\ord(g) = p$.
\end{theorem}

\begin{theorem}
    [Piccolo Teorema di Fermat]
    \label{teorema1.45}
    Dato un numero primo $p$, se $n \in \ZZ$ è coprimo con $p$ allora 
    $n^{p - 1} \equiv 1 \pmod p$.
\end{theorem}

Dati un gruppo $G$ e un numero primo $p$, consideriamo l'insieme 
\[
    X = \{(g_1, \ldots, g_p) \in G^p\mid g_1\ldots g_p = e\}
\]osserviamo che $|X| = |G|^{p - 1}$, possiamo infatti scegliere liberamente
i primi $p - 1$ elementi di ogni $p$-upla, che ne determinano l'ultimo in 
modo univoco (per unicità dell'inverso). Definiamo un'azione di $\Zp$ su $X$
nel seguente modo:
\[
    \psi: \Zp \longrightarrow S(X) : a \longmapsto \psi_a
\]
con
\[
    \psi_a:X\longrightarrow X : (g_1, \ldots, g_p)\longmapsto (g_{1 + a}, \ldots, g_p, g_1, \ldots, g_a)
\]

Fissato $x \in X$, poiché la cardinalità di $\Orb(x)$ divide l'ordine di $\Zp$
abbiamo che $|\Orb(x)| \in \{1, p\}$, in particolare le orbite di cardinalità
1 sono date dalle $p$-uple della forma $(g, \ldots, g)$ con $g^p = e$.
Poniamo $S = \{g\in G \mid \ord(g) = p\}$ e $\mathcal{R}$ un insieme di 
rappresentanti per la relazione di equivalenza indotta da $\psi$, poiché 
le orbite degli elementi di $X$ formano una partizione dell'insieme abbiamo
\[
    |G|^{p - 1} = |X| = \sum_{x \in \mathcal{R}} |\Orb(x)| = 1 + |S| + \sum_{x \in \mathcal{R}\setminus S}|\Orb(x)|
\]dove l'ultimo termine della somma è divisibile per $p$. Distinguiamo 
quindi due casi:
\begin{itemize}
    \item se $p\mid |G|$, riducendo modulo $p$ la formula sopra otteniamo
    $|S| \equiv -1 \pmod p$, in particolare esiste almeno un elemento di
    ordine $p$ (\hyperref[teorema1.44]{Teorema di Cauchy});
    \item se $G = \Zn$ con $p$ e $n$ coprimi, $\Zn$ non contiene elementi
    di ordine $p$, pertanto riducendo modulo $p$ la formula sopra otteniamo
    $n^{p - 1} \equiv 1 \pmod p$ (\hyperref[teorema1.45]{Piccolo Teorema di Fermat}).
\end{itemize}

\begin{exercise}
    Mostrare che i gruppi di ordine $15$ sono ciclici.
\end{exercise}

\begin{soln}
    Sia $G$ un gruppo di ordine 15, poiché 5 è un primo che divide $|G|$
    esiste $h \in G$ tale che $\ord(h) = 5$ per il \hyperref[teorema1.44]{Teorema di Cauchy}.
    Inoltre, posto $H = \langle h\rangle$, abbiamo che $[G:H] = 3$ e quindi,
    dato che 3 è il più piccolo primo che divide $|G|$, $H$ è un sottogruppo 
    normale di $G$. Mostriamo che $H \subseteq Z(G)$, questo è equivalente a 
    richiedere che l'omomorfismo \[
        \varphi: G\longrightarrow \Aut(H) :g\longmapsto \varphi_{g\mid H}
    \]
    dove $\varphi_g$ è il coniugio per $g$, abbia come unico elemento dell'immagine
    l'applicazione
    \[
        id_H:H\longrightarrow H: h \longmapsto h
    \]
    Poiché $H \cong \Z5$, abbiamo $\Aut(H)\cong (\Z5)^* \cong \Z4$, d'altra 
    parte $|\mathrm{Im}\varphi_{\mid H}|$ divide $(|G|, |\Aut(H)|) = 1$, pertanto
    $|\mathrm{Im}\varphi| = 1$ e l'omomorfismo è banale, cioè $H \subseteq Z(G)$.
    Diamo adesso due modi per concludere l'esercizio:
    \begin{enumerate}[(1)]
        \item osserviamo che se $G$ è un gruppo abeliano, cioè se $Z(G) = G$,
        allora abbiamo che $G$ è ciclico. Infatti posto $k \in G$ un elemento di 
        ordine 3 (che esiste in virtù del \hyperref[teorema1.44]{Teorema di Cauchy}),
        abbiamo che $\ord(hk) = \ord(h)\ord(k) = 15$ in quanto i due elementi hanno
        ordine coprimo. D'altra parte, se $G$ non fosse abeliano allora avremmo 
        necessariamente $Z(G) = H$, quindi $\faktor{G}{Z(G)}$ sarebbe ciclico 
        in quanto di ordine 3, pertanto $G$ sarebbe un gruppo abeliano, da cui 
        la tesi per quanto appena detto;
        \item sia $k \in G$ un elemento di ordine 3, consideriamo il centralizzatore
        di $k$
        \[
            Z_G(k) = \{x \in G\mid xk = kx\}
        \]Osserviamo che $k \in Z_G(k)$ e $Z(G) \subseteq Z_G(k)$, pertanto $h$ è un elemento 
        di $Z_G(k)$. Abbiamo quindi che $\ord(h)\mid |Z_G(k)|$ e $\ord(k)\mid |Z_G(k)|$, 
        da cui $|Z_G(k)| = 15$. Abbiamo che tutti gli elementi di ordine 3
        sono contenuti nel centro di $G$, che quindi coincide con $G$. Allora $G$
        è ciclico in quanto abeliano e contenente un elemento di ordine 3 e uno
        di ordine 5, quindi uno di ordine 15.
    \end{enumerate}
\end{soln}

\begin{remark}
    In generale dati $x, y\in G$, se $x$ e $y$ commutano allora 
    $\ord(xy) = [\ord(x), \ord(y)]$ anche se $G$ non è un gruppo abeliano.
\end{remark}

\begin{exercise}
    \label{ex1.48}
    Dato $d$ un numero dispari, mostrare che ogni gruppo di ordine $2d$ ammette
    un sottogruppo normale di indice 2.
\end{exercise}

\begin{soln}
    Consideriamo la rappresentazione regolare a sinistra di $G$
    \[
        \lambda: G \longrightarrow S(G) : g\longmapsto \lambda_g
    \]
    con
    \[
        \lambda_g : G\longrightarrow G : x\longmapsto gx
    \]
    Fissato un isomorfismo $\psi: S(G) \longrightarrow S_{2d}$ poniamo
    $\varphi = \psi\circ\lambda :G\longrightarrow S_{2d}$, $\varphi$ è 
    un omomorfismo iniettivo (infatti nella dimostrazione del Teorema di Cayley
    abbiamo visto che $\lambda$ è un omomorfismo iniettivo). Consideriamo 
    il sottogruppo $\varphi^{-1}(\mathcal{A}_{2d})$, mostriamo che il suo 
    indice in $G$ è al più 2:
    posta $\pi_{\mathcal{A}_{2d}}$ la proiezione al quoziente
    \[
        \pi_{\mathcal{A}_{2d}}:G\longrightarrow \faktor{S_{2d}}{\mathcal{A}_{2d}}\cong \Z2
    \]
    possiamo caratterizzare $\varphi^{-1}(\mathcal{A}_{2d})$ come
    \[
        \varphi^{-1}(\mathcal{A}_{2d}) = \{g \in G \mid \varphi(g) \in \mathcal{A}_{2d}\}
        = \ker (\pi_{\mathcal{A}_{2d}}\circ\varphi)
    \]
    pertanto $\varphi^{-1}(\mathcal{A}_{2d})\trianglelefteqslant G$. 
    Per il Primo Teorema di Omomorfismo abbiamo che esiste un omomorfismo
    iniettivo da $\faktor{G}{\ker(\pi_{\mathcal{A}_{2d}}\circ\varphi)}$ in
    $\Z2$, da cui $[G:\ker(\pi_{\mathcal{A}_{2d}})] \leq 2$. Tale 
    sottogruppo ha indice 1 se e solo se $G = \ker(\pi_{\mathcal{A}_{2d}}\circ\varphi)$,
    cioè $\varphi(G) \subseteq \mathcal{A}_{2d}$, mostriamo che in effetti esiste 
    un elemento di $G$ la cui immagine tramite $\varphi$ è una permutazione 
    dispari. Consideriamo $g \in G$ un elemento di ordine 2, poiché $\varphi$
    è un omomorfismo iniettivo abbiamo che $\ord(\varphi(g)) = \ord(g) = 2$,
    pertanto la permutazione $\varphi(g)$ ha una decomposizione in $d$ 2-cicli,
    cioè è dispari. Pertanto $G \neq \varphi^{-1}(\mathcal{A}_{2d})$,
    da cui $[G: \varphi^{-1}(\mathcal{A}_{2d})] = 2$,
\end{soln}

Possiamo generalizzare il ragionamento appena usato nel seguente risultato

\begin{proposition}
    \label{prop1.49}
    Dato un gruppo $G$ e $H\lneq G$ un sottogruppo tale che $[G:H] = 2$, se
    $K$ è un sottogruppo di $G$ allora $H\cap K$ ha indice 1 o 2 in $K$,
    cioè $[K:H\cap K] \in \{1, 2\}$.
\end{proposition}

\begin{proof}
    Distinguiamo due casi:
    \begin{itemize}
        \item se $K \subseteq H$ allora $H \cap K = K$, da cui $[K:H\cap K] = 1$;
        \item se $K \nsubseteq H$ consideriamo la proiezione 
        \[
            \pi_H: G\longrightarrow \faktor{G}{H} :g \longmapsto gH
        \]
        Poiché $\faktor G H \cong \Z2$ abbiamo che gli unici sottogruppi 
        del quoziente sono $\{H\}$ e $\faktor G H$, pertanto 
        $\pi_H(K) = \faktor G H$. Osserviamo che $\ker\pi_{H\mid K}
        = \ker \pi_H \cap K$, per il Primo Teorema di Omomorfismo allora 
        $\faktor{K}{H\cap K} \cong \Z2$, cioè $[K:H\cap K] = 2$.
    \end{itemize}
\end{proof}

\subsubsection{Teorema di Poincaré}

Vediamo un risultato che sarà utile nel futuro, che permette di esibire,
se esistono, sottogruppi normali non banali di un gruppo finito.

\begin{theorem}
    [Teorema di Poincaré]
    \label{teorema1.50}
    Dato un gruppo $G$ finito e $H\leqslant G$ un suo sottogruppo, 
    se $[G:H] = n$ allora esiste un sottogruppo normale $N\triangleleft G$ 
    tale che:
    \begin{enumerate}[(1)]
        \item $N\leqslant H \leqslant G$;
        \item $n \mid [G:N] \mid n!$.
    \end{enumerate}
\end{theorem}

\begin{proof}
    Consideriamo l'azione di $G$ su $\faktor{G}{H}$
    \[
        \psi: G\longrightarrow S\left(\faktor{G}{H}\right):g\longmapsto \psi_g
    \]
    con
    \[
        \psi_g:\faktor{G}{H}\longrightarrow\faktor{G}{H} : g'H\longmapsto gg'H
    \]
    \begin{enumerate}[(1)]
        \item Consideriamo il nucleo di $\psi$
        \[
            \ker\psi = \{g \in G\mid gg'H = g'H~\forall g' \in G\}
        \]
        nel caso particolare $g' = e$ otteniamo l'inclusione
        \[
            \ker\psi \subseteq \{g \in G \mid gH = H\} = H
        \]
        in quanto stiamo indebolendo la condizione di appartenenza all'insieme,
        pertanto $\ker\psi \leqslant H$;
        \item poiché $\ker\psi \leqslant H$ abbiamo $[G:H]\mid [G:\ker\psi]$, cioè
        $n \mid [G:\ker\psi]$. Dal Primo Teorema di Omomorfismo abbiamo che
        $\faktor{G}{\ker\psi} \cong \mathrm{Im\psi}$, che è un sottogruppo
        di $S\left(\faktor{G}{H}\right)\cong S_n$, pertanto $[G:\ker\psi]\mid n!$.
    \end{enumerate}
    Poiché $\ker\psi$ è normale in $G$ abbiamo che $N = \ker\psi$ è un sottogruppo
    con le proprietà cercate.
\end{proof}

\begin{remark}
    In particolare, se $G$ ha un sottogruppo di indice $n$ e $n! < |G|$
    allora $G$ ammette sottogruppi normali non banali.
\end{remark}

\newpage

\subsection{Gruppo simmetrico}

\subsubsection{Generatori di $S_n$}

Esibiamo alcuni insiemi di generatori per $S_n$:

\begin{itemize}
    \item $\{\cycle{i, j} \mid i, j \in\{1, \ldots, n\}, i < j\}$, abbiamo visto 
    che ogni permutazione può essere scritta come prodotto di trasposizioni;
    \item $\{\cycle{1, j}\mid j \in \{2, \ldots, n\}\}$, infatti per ogni $i<j$ abbiamo
    \[
        \cycle{i, j} = \cycle{1, i}\cycle{1, j}\cycle{1, i}
    \]
    \item $\{\cycle{i, i + 1}\mid i \in\{1, \ldots, n - 1\}\}$,
    infatti per ogni $j$ abbiamo 
    \[
        \cycle{1, j} = \cycle{j - 1, j}\cycle{1, j - 1}\cycle{j - 1, j}
    \]
    \item $\{\cycle{1, 2}, \cycle{1, 2, \ldots, n}\}$, infatti per ogni
    $i$ abbiamo 
    \[
        \cycle{i, i + 1} = \cycle{1, \ldots, n}^{i - 1}\cycle{1, 2}\cycle{1, \ldots, n}^{1 - i}
    \]
\end{itemize}

\begin{remark}
    Non è vero in generale che una trasposizione e un $n$-ciclo generano $S_n$,
    consideriamo ad esempio $\langle\sigma, \rho\rangle\leqslant S_4$ con
    $\sigma = \cycle{1, 2, 3, 4}$, $\rho = \cycle{2, 4}$. Abbiamo
    $\sigma^4 = \rho^2 = 1$ e $\rho\sigma\rho^{-1} = \cycle{1, 4, 3, 2} =
    \sigma^{-1}$, pertanto $\langle\sigma, \rho\rangle$ è isomorfo a un 
    quoziente di $D_4$. D'altra parte $\langle\sigma\rangle\cap \langle\rho\rangle = \{id\}$
    e $\rho \in N_{S_4}(\sigma)$, pertanto $\langle\sigma, \rho\rangle =
    \langle\sigma\rangle\langle\rho\rangle$ e $|\langle\sigma, \rho\rangle| = 8$,
    pertanto è isomorfo a $D_4$. 
\end{remark}

\subsubsection{Sottogruppi abeliani massimali di $S_n$}

Vogliamo studiare i sottogruppi abeliani di $S_n$, caratterizzando in particolare
i suoi sottogruppi abeliani massimali.

\begin{definition}
    Un sottogruppo $G\leqslant S_n$ si dice \vocab{transitivo} se l'azione
    \[
        \varphi: G\longrightarrow S_n :\sigma \longmapsto \sigma
    \]indotta da $G$ su $\{1, \ldots, n\}$ è transitiva, cioè se per ogni
    $i, j \in \{1, \ldots, n\}$ esiste $\sigma \in G$ tale che $\sigma(i) = j$.
\end{definition}

\begin{lemma}
    \label{lemma1.54}
    Dato $G$ un sottogruppo abeliano di $S_n$, se $G$ è transitivo allora $|G| = n$.
\end{lemma}

\begin{proof}
    Consideriamo l'azione di $G$ su $\{1, \ldots, n\}$
    \[
        \psi : G\longrightarrow S_n :\sigma \longmapsto \sigma
    \]
    poiché $G$ è transitivo, per la \hyperref[prop1.41]{Proposizione 1.41}
    gli stabilizzatori degli elementi di $\{1, \ldots, n\}$ sono tra loro coniugati.
    D'altra parte, poiché lo stabilizzatore è un sottogruppo di $G$, che
    è un gruppo abeliano, la restrizione del coniugio agli
    stabilizzatori coincide con l'applicazione identità, da cui $\St(i) = \St(j)$
    per ogni $i, j \in \{1, \ldots, n\}$. Osserviamo infine che 
    \[
        \bigcap_{i = 1}^n \St(i) = \{id_{S_n}\}
    \]in quanto $id_{S_n}$ è l'unica permutazione che fissa tutti gli elementi
    di $\{1, \ldots, n\}$, pertanto $\St(i) = \{id_{S_n}\}$ per ogni $i \in \{1, \ldots, n\}$.
    Fissato $i \in \{1, \ldots, n\}$, abbiamo 
    \[
        |G| = |\Orb(i)|\cdot|\St(i)| = |\Orb(i)| = n
    \]
    in quanto $G$ è transitivo.
\end{proof}

\begin{lemma}
    \label{lemma1.55}
    Se $a_1, \ldots, a_k$ sono interi positivi tali che $\displaystyle
    \sum_{i = 1}^k a_i = 3m$,
    con $m \geqslant k$ intero, allora $\displaystyle
        \prod_{i = 1}^k a_i \leq 3^m$, 
        e vale l'uguaglianza se e solo se $k = m$ e $a_i = 3$ per ogni $i \in \{1, \ldots, k\}$.
\end{lemma}

\begin{proof}
    Senza perdita di generalità, a meno di aumentare $k$ possiamo supporre
    $a_i \in \{1, 2, 3\}$ per ogni $i \in \{1, \ldots, k\}$, infatti se 
    uno degli $a_i$ è uguale a 4 possiamo sostituirlo con $2 + 2$, se uno degli
    $a_i$ è uguale a 5 possiamo sostituirlo con $2 + (a_i - 2)$ e così via
    (queste sostituzioni mantengono inalterato il valore della somma).
    In particolare abbiamo che $a_i \leqslant 3$ per ogni $i \in \{1, \ldots, n\}$,
    pertanto 
    \[
        \prod_{i = 1}^k a_i \leq 3^k \leq 3^m
    \]
    inoltre se $k = m$ e tutti gli $a_i$ sono uguali a 3 abbiamo chiaramente
    \[
        \prod_{i = 1}^k a_i = 3^k = 3^m
    \]
    Viceversa, se il prodotto degli $a_i$ è uguale a $3^m$ allora necessariamente
    $k = m$ e $a_i = 3$ per ogni $i \in \{1, \ldots, k\}$ in quanto possiamo
    supporre $a_i \in \{1, 2, 3\}$ senza perdita di generalità.
\end{proof}

\begin{lemma}
    \label{lemma1.56}
    Dati $\sigma, \tau \in S_n$, se $\sigma = \cycle{x_1, \ldots, x_k}$ è un 
    $k$-ciclo allora 
    \[
        \tau\sigma\tau^{-1} = \cycle{\tau(x_1),\ldots, \tau(x_k)}
    \]
\end{lemma}

\begin{proof}
    \[
        (\tau\sigma\tau^{-1})(\tau(x_i)) = (\tau\sigma)(x_i) = \tau(x_{i + 1})
    \]per ogni $i \in \{1, \ldots, k\}$, pertanto
    \[
        \tau\sigma\tau^{-1} = \cycle{\tau(x_1), \ldots, \tau(x_k)}
    \]
\end{proof}


\begin{exercise}
    Posto $n = 3m$, mostrare che la massima cardinalità di un sottogruppo
    abeliano di $S_n$ è $3^m$ e caratterizzare la sua classe di isomorfismo.
\end{exercise}

\begin{soln}
    Per prima cosa, osserviamo che $S_n$ contiene sottogruppi abeliani di
    cardinalità $3m$, ad esempio
    \[
        \langle\cycle{1, 2, 3}\rangle\cdot \langle\cycle{4, 5, 6}\rangle
        \cdot \ldots \cdot \langle\cycle{n - 2, n - 1, n}\rangle
    \]
    è un sottogruppo abeliano di $S_n$ di cardinalità $3^m$, essendo 
    isomorfo a
    \[
        \langle\cycle{1, 2, 3}\rangle\times\langle\cycle{4, 5, 6}\rangle
        \times\ldots\times\langle\cycle{n - 2, n - 1, n}\rangle
    \]
    Sia $G$ un sottogruppo abeliano di $S_n$ di ordine massimo, data
    \[
        \psi: G\longrightarrow S_n : \sigma \longmapsto \sigma
    \]
    l'azione naturale di $G$ su $\{1, \ldots, n\}$ chiamiamo $\Omega_1, \ldots, \Omega_k$
    le orbite. Consideriamo le mappe $\varphi_i : G \longrightarrow S(\Omega_i)$
    tali che, data $\sigma \in G$ e fissata $\rho_1\ldots\rho_k$ una sua decomposizione
    in cicli disgiunti, $\varphi_i(\sigma) = \rho_i$, poniamo $G_i = \mathrm{Im}\varphi_i=
    \mathrm{Im}\psi \cap S(\Omega_i)$. Possiamo quindi costruire l'omomorfismo
    \[
        \varphi: G \longrightarrow G_1\times \ldots \times G_k : g \longmapsto (\varphi_1(g), \ldots, \varphi_k(g))
    \]
    che è iniettivo in quanto 
    \[
        \varphi(\sigma) = id \iff \varphi_i(\sigma) = id_{S(\Omega_i)} \iff 
        \sigma_{\mid \Omega_i} = id_{S(\Omega_i)}
    \]
    per ogni $i \in \{1, \ldots, k\}$, che è equivalente a $\sigma = id_{S_n}$
    dato che le orbite ricoprono $\{1, \ldots, n\}$, da cui $\ker\varphi = 
    \{id_{S_n}\}$. Osserviamo adesso che ogni $G_i$ è un gruppo abeliano poiché
    immagine omomorfa di $G$, che è un gruppo abeliano, inoltre è transitivo
    sull'orbita $\Omega_i$ per costruzione, pertanto per il 
    \hyperref[lemma1.54]{Lemma 1.54} abbiamo $|G_i| = |\Omega_i|$ per ogni 
     $i \in \{1, \ldots, k\}$. Vale quindi la seguente disuguaglianza, data
    dall'iniettività di $\varphi$
    \[
        |G| \leqslant \prod_{i = 1}^k|G_i| = \prod_{i = 1}^k |\Omega_i|
    \]
    D'altra parte
    \[
        3m = \sum_{i = 1}^k|\Omega_i|
    \]pertanto per il \hyperref[lemma1.55]{Lemma 1.55} abbiamo $|G| \leq 3^m$, 
    ma questa è effettivamente un'uguaglianza in quanto $S_n$ contiene
    almeno un sottogruppo abeliano di ordine $3^m$ e $G$ ha ordine massimo.
    Sempre per il \hyperref[lemma1.55]{Lemma 1.55} allora $k = m$ e $|\Omega_i| = 3$
    per ogni $i \in \{1, \ldots, k\}$. Abbiamo quindi che $\varphi$ è un isomorfismo
    e che $G_1\times\ldots\times G_k$ è isomorfo a $(\Z3)^k$, pertanto
    $G$ è isomorfo a $(\Z3)^k$.
\end{soln}

\begin{remark}
    Se $a_1, \ldots, a_k$ sono interi tali che 
    \[
        3m + 2 = \sum_{i = 1}^k a_i
    \]
    ragionando come nella dimostrazione del \hyperref[lemma1.55]{Lemma 1.55}
    possiamo scrivere
    \[
        3m + 2 = 2 + \sum_{i = 1}^{k - 1}a_i
    \]
    da cui ricaviamo
    \[
        \prod_{i = 1}^k a_i \leqslant 2\cdot3^m
    \]
    Inoltre questa è un'uguaglianza se e solo se esiste $j \in \{1, \ldots, k\}$
    tale che $a_j = 2$, $a_i = 3$ per ogni $i \in \{1, \ldots, k\}\setminus\{j\}$
    e $k = m$. Ragionando come sopra otteniamo $|G| \leqslant 2\cdot 3^m$, 
    d'altra parte osserviamo che $S_n$ contiene un sottogruppo abeliano
    \[
        \langle\cycle{1, 2, 3}\rangle\cdot\ldots\cdot\langle\cycle{3m - 2, 3m - 1, 3m}\rangle\cdot\langle\cycle{3m + 1, 3m + 2}\rangle
    \]
    di ordine $2\cdot3^m$ poiché isomorfo a 
    \[
        \langle\cycle{1, 2, 3}\rangle\times\ldots\times\langle\cycle{3m - 2, 3m - 1, 3m}\rangle\times\langle\cycle{3m + 1, 3m + 2}\rangle
    \]
    pertanto $|G| = 2\cdot3^m$ e $G \cong (\Z3)^m \times \Z2$.
    Se $n = 3m + 1$, ragionando in modo simile abbiamo che la somma delle
    cardinalità delle orbite $\Omega_1, \ldots, \Omega_k$ è $3m + 1$ e il 
    loro prodotto è minore o uguale a $4\times 3^{m - 1}$, da cui $|G| \leq
    4\cdot3^{m - 1}$. D'altra parte $S_n$ contiene almeno due tipi di 
    sottogruppi abeliani di ordine $3m + 1$, uno isomorfo a 
    $(\Z3)^{m - 1} \times \Z4$ e uno isomorfo a $(\Z3)^{m - 1}\times V_4$,
    dove 
    \[
        V_4 = \{\cycle{1, 2}\cycle{3, 4}, \cycle{1, 3}\cycle{2, 4}, 
    \cycle{1, 4}\cycle{2, 3}, id\}
    \]
    è un sottogruppo abeliano non ciclico di $S_4$, chiamato 
    \vocab{gruppo di Klein} o \vocab{Klein 4-group}. Pertanto un sottogruppo
    abeliano di ordine massimo deve avere una di queste due forme.
\end{remark}

\begin{remark}
    I sottogruppi di $S_n$ di questo tipo sono tutti coniugati tra loro, infatti
    se 
    \[
        G = \langle\cycle{x_1, x_2, x_3}\rangle\cdot\ldots\cdot\langle\cycle{x_{n - 2}, x_{n - 1}, x_n}\rangle
    \]
    \[
        G' = \langle\cycle{y_1, y_2, y_3}\rangle\cdot\ldots\cdot\langle\cycle{y_{n - 2}, y_{n - 1}, y_n}\rangle
    \]
    sono due sottogruppi abeliani di $S_n$ di ordine massimo (per semplicità
    supponiamo $n = 3m$, gli altri due casi si studiano in modo analogo)
    consideriamo $\sigma \in S_n$ tale che $\sigma(y_i) = x_i$ per ogni 
    $i \in \{1, \ldots, n\}$, è sufficiente mostrare che i generatori delle componenti del
    prodotto sono tra loro coniugate. Infatti, per il \hyperref[lemma1.56]{Lemma 1.56}
    abbiamo 
    \[
        \sigma\cycle{x_i, x_{i + 1}, x_{i + 2}}\sigma^{-1} = 
        \cycle{\sigma(x_i), \sigma(x_{i + 1}), \sigma(x_{i + 2})} = 
        \cycle{y_i, y_{i + 1}, y_{i + 2}}
    \]
    per ogni $i \in \{1, \ldots, n - 2\}$, pertanto $G$ e $G'$ sono coniugati.
\end{remark}

\subsubsection{Classi di coniugio in $\mathcal{A}_n$}

Studiamo le classi di coniugio in $\mathcal{A}_n$. In particolare,
fissato $\sigma \in \mathcal{A}_n$, vogliamo determinare una relazione tra
$\Cl_{\mathcal{A}_n}(\sigma)$ e $\Cl_{S_n}(\sigma)$.
Poiché valgono $|\mathcal{A}_n| = |\Cl_{\mathcal{A}_n}(\sigma)|\cdot |Z_{\mathcal{A}_n}(\sigma)|$
e $Z_{\mathcal{A}_n}(\sigma) = Z_{S_n}(\sigma) \cap \mathcal{A}_n$, abbiamo
\[
    |\Cl_{\mathcal{A}_n}(\sigma)| = \frac{|\mathcal{A}_n|}{|Z_{\mathcal{A}_n}(\sigma)|} =
    \frac 1 2 \frac{|S_n|}{|Z_{S_n}(\sigma) \cap \mathcal{A}_n|}
\]
Dato che $[S_n:\mathcal{A}_n] = 2$, per la \hyperref[prop1.49]{Proposizione 1.49}
abbiamo $[Z_{S_n}(\sigma):Z_{S_n}(\sigma) \cap \mathcal{A}_n] \in \{1, 2\}$,
distinguiamo quindi due casi:
\begin{itemize}
    \item $|Z_{S_n}(\sigma) \cap \mathcal{A}_n| = \displaystyle\frac 1 2 |Z_{S_n}(\sigma)|$;
    \item $|Z_{S_n}(\sigma) \cap \mathcal{A}_n| = |Z_{S_n}(\sigma)|$.
\end{itemize}

Nel primo caso otteniamo 
\[
    |\Cl_{\mathcal{A}_n}| = \frac 1 2 \frac{|S_n|}{|Z_{S_n}(\sigma) \cap \mathcal{A}_n|} =
    \frac{|S_n|}{|Z_{S_n}(\sigma)|} = |\Cl_{S_n}(\sigma)|
\]
poiché $\Cl_{\mathcal{A}_n}(\sigma)\subseteq \Cl_{S_n}(\sigma)$ abbiamo che le
due classi coincidono. In particolare questo succede se $Z_{S_n}(\sigma)
\nsubseteq \mathcal{A}_n$.

Nel secondo caso invece, che si verifica se $Z_{S_n}(\sigma)\subseteq 
\mathcal{A}_n$,
\[
    |\Cl_{\mathcal{A}_n}(\sigma)| = \frac 1 2 \frac{|S_n|}{|Z_{S_n}(\sigma)\cap\mathcal{A}_n|}
    = \frac 1 2 \frac{|S_n|}{|Z_{S_n}(\sigma)|} = \frac 1 2 |\Cl_{S_n}(\sigma)|
\]
Più precisamente, abbiamo $\Cl_{S_n}(\sigma) = \Cl_{\mathcal{A}_n}(\sigma) \cup
\Cl_{\mathcal{A}_n}(\tau\sigma\tau^{-1})$ per ogni $\tau$ permutazione dispari. 
Infatti $\Cl_{\mathcal{A}_n}(\sigma) \cup \Cl_{\mathcal{A}_n}(\tau\sigma\tau^{-1})
\subseteq \Cl_{S_n}(\sigma)$ (i coniugati di $\tau\sigma\tau^{-1}$ sono anche
coniugati di $\sigma$), d'altra parte per ogni $\rho \in S_n$ abbiamo
$\rho\sigma\rho^{-1} \in \Cl_{\mathcal{A}_n}(\sigma)$ se $\rho$ è pari,
$\rho\sigma\rho^{-1} = (\rho\tau^{-1})(\tau\sigma\tau^{-1})(\rho\tau^{-1})^{-1}
\in \Cl_{\mathcal{A}_n}(\tau\sigma\tau^{-1})$ se $\rho$ è dispari, da cui l'uguaglianza.
Abbiamo altri due casi:
\begin{itemize}
    \item $|\Cl_{\mathcal{A}_n}(\tau\sigma\tau^{-1})| = |\Cl_{S_n}(\tau\sigma\tau^{-1})|$;
    \item $|\Cl_{\mathcal{A}_n}(\tau\sigma\tau^{-1})| = \displaystyle\frac 1 2
    |\Cl_{S_n}(\tau\sigma\tau^{-1})|$.
\end{itemize}
Tuttavia se fosse $|\Cl_{\mathcal{A}_n}(\tau\sigma\tau^{-1})| = |\Cl_{S_n}(\tau\sigma\tau^{-1})|$
avremmo $\Cl_{\mathcal{A}_n}(\sigma) = \Cl_{\mathcal{A}_n}(\tau\sigma\tau^{-1})$,
che è assurdo in quanto $\tau\sigma\tau^{-1} \notin \Cl_{\mathcal{A}_n}(\sigma)$,
pertanto \[
    |\Cl_{\mathcal{A}_n}(\tau\sigma\tau^{-1})| = \displaystyle\frac 1 2
    |\Cl_{S_n}(\tau\sigma\tau^{-1})| = \frac 1 2|\Cl_{S_n}(\sigma)| =
    |\Cl_{\mathcal{A}_n}(\sigma)|
\]
Poiché 
$|\Cl_{S_n}(\sigma)| = |\Cl_{\mathcal{A}_n}(\sigma)| + |\Cl_{\mathcal{A}_n}(\tau\sigma\tau^{-1})|$,
per il Principio di Inclusione-Esclusione abbiamo che l'unione è disgiunta,
cioè
\[
    \Cl_{S_n}(\sigma) = \Cl_{\mathcal{A}_n}(\sigma) \cupdot \Cl_{\mathcal{A}_n}(\tau\sigma\tau^{-1})
\]

\subsubsection{Studio di $S_5$}

%trovare un modo per incastrare meglio questo pezzo
Consideriamo gli elementi di $S_5$ $\sigma = \cycle{1, 2, 3, 4, 5}, 
\tau = \cycle{2, 5}\cycle{3, 4}$, studiamo il sottogruppo $H = \langle\sigma, \tau\rangle$,
in particolare siamo interessati a determinare una regola di commutazione
per $\sigma$ e $\tau$. Osserviamo che 
\[
    \tau\sigma\tau^{-1} = \cycle{\tau(1), \tau(2), \tau(3), \tau(4), \tau(5)} = 
    \cycle{1, 5, 4, 3, 2}
\]
e che questo coincide con $\sigma^{-1}$. Abbiamo quindi che $H$ è generato 
da un elemento $\tau$ di ordine 2 e da un elemento $\sigma$ di ordine 5 
che soddisfano la relazione $\tau\sigma\tau^{-1} = \sigma^{-1}$, pertanto 
$H$ è isomorfo a un sottogruppo del gruppo diedrale $D_5$. D'altra parte, 
da questa relazione ricaviamo che $\langle \tau\rangle \subseteq N_{S_5}(\langle\sigma\rangle)$,
pertanto possiamo scrivere $H = \langle\sigma\rangle\cdot\langle\tau\rangle$
in quanto $\langle\sigma\rangle\cdot\langle\tau\rangle$ è un sottogruppo di $H$
che ha la sua stessa cardinalità. In particolare otteniamo che $|H| = 10 = |D_5|$,
quindi $H \cong D_5$.\\

%%%

Abbiamo visto che le classi di coniugio in un gruppo simmetrico su $n$ elementi
sono parametrizzate dalle partizioni di $n$
\begingroup
\renewcommand{\arraystretch}{2}
\begin{center}
    \begin{tabular}{c|c}
        Partizioni di 5 & Cardinalità della classe di coniugio associata\\
        \hline
        5 & $\displaystyle\binom{5}{5}4! = 4! = 24$\\
        4 + 1 & $\displaystyle\binom{5}{4}3! = 30$\\
        3 + 2 & $\displaystyle\binom{5}{3}2!\binom{2}{2}1! = 20$\\
        3 + 1 + 1 & $\displaystyle\binom{5}{3}2! = 20$\\
        2 + 2 + 1 & $\displaystyle\frac 1 2\binom{5}{2}1!\binom{3}{2}1! = 15$\\
        2 + 1 + 1 + 1 & $\displaystyle\binom{5}{2}1! = 10$\\
        1 + 1 + 1 + 1 + 1 & $1$
    \end{tabular}
\end{center}
\endgroup
(Nel calcolo della cardinalità della classe associata alla partizione 2 + 2 + 1
dividiamo per 2 in quanto contiamo i cicli a meno dell'ordine, e
le coppie di trasposizioni che stiamo considerando commutano).
Di queste, le permutazioni che appartengono a $\mathcal{A}_5$ sono quelle
la cui classe di coniugio è associata alle partizioni 5, 3 + 1 + 1, 
2 + 2 + 1, 1 + 1 + 1 + 1 + 1, cioè le permutazioni $\sigma, \tau, \rho$
aventi una decomposizione in cicli disgiunti della forma
\[
    \sigma = \cycle{a_1, a_2, a_3, a_4, a_5}
\]
\[
    \tau = \cycle{b_1, b_2, b_3}
\]
\[
    \rho = \cycle{c_1, c_2}\cycle{d_1, d_2}
\]
e l'identità. Vediamo come sono fatte le loro classi di coniugio in $\mathcal{A}_5$.
Chiaramente $\Cl_{\mathcal{A}_n}(id) = \Cl_{S_n}(id) = \{id\}$, studiamo quindi
le classi di $\sigma, \tau, \rho$ fissate come sopra.
\begin{itemize}
    \item $Z_{S_5}(\sigma) = \langle\cycle{a_1, a_2, a_3, a_4, a_5}\rangle$,
    infatti 
    \[
        |Z_{S_5}(\sigma)| = \frac{|S_5|}{|\Cl_{S_5}(\sigma)|} = \frac{5!}{4!} = 5
    \]
    Allora $Z_{S_5}(\sigma)$ contiene solo permutazioni pari, fissata $\psi$
    una permutazione dispari la sua
    classe di coniugio in $S_5$ si scrive come
    \[
        \Cl_{S_5}(\sigma) = \Cl_{\mathcal{A}_5}(\sigma) \cupdot \Cl_{\mathcal{A}_5}(\psi\sigma\psi^{-1})
    \]
    \item $Z_{S_5}(\tau)$ non è contenuto in $\mathcal{A}_5$, infatti una
    trasposizione $\psi$ disgiunta da $\tau$ è una permutazione dispari che
    appartiene al centralizzatore. Pertanto
    \[
        \Cl_{S_5}(\tau) = \Cl_{\mathcal{A}_5}(\tau)
    \]
    \item $Z_{S_5}(\rho)$ non è contenuto in $\mathcal{A}_5$, infatti la 
    trasposizione $\cycle{c_1, c_2}$ è una permutazione dispari che commuta
    con $\rho$ (infatti $\cycle{c_1, c_2}$ e $\cycle{d_1, d_2}$ commutano in 
    quanto cicli disgiunti e $\cycle{c_1, c_2}$ commuta con se stessa).
    Pertanto 
    \[
        \Cl_{S_5}(\rho) = \Cl_{\mathcal{A}_5}(\rho)
    \]
\end{itemize}

\subsubsection{Sottogruppi normali di $\mathcal{A}_n$}

Esibiamo alcuni insiemi di generatori per $\mathcal{A}_n$:
\begin{itemize}
    \item $\{\cycle{i, j}\cycle{k, l}\mid i\neq j, k \neq l\}$, infatti
    ogni elemento di $\mathcal{A}_n$ può essere scritto come prodotto di 
    coppie di trasposizioni in quanto permutazione pari;
    \item $\{\cycle{i, j, k}\mid i, j, k\text{ distinti}\}$. Infatti se 
    $\{i, j\} = \{k, l\}$ allora $\cycle{i, j}\cycle{k, l} = id$ è un elemento
    generato dall'insieme, se invece $|\{i, j\}\cap\{k, l\}| = 1$, ad esempio $j = k$,
    abbiamo $\cycle{i, j}\cycle{k, l} = \cycle{i, j}\cycle{j, l} = \cycle{i, j, l}$,
    che è un elemento generato dall'insieme. Nel caso $\{i, j\}\cap \{k, l\} = \emptyset$ abbiamo
    $\cycle{i, j}\cycle{k, l} = \cycle{i, j}\cycle{j, k}\cycle{j, k}\cycle{k, l}=
    \cycle{i, j, k}\cycle{j, k, l}$, che è un elemento generato dall'insieme.
    Possiamo quindi ottenere il precedente insieme di generatori a partire
    da questo;
    %\item $\{\cycle{1, 2, k}\mid k \in \{3, \ldots, n\}\}$, infatti 
\end{itemize}

\begin{definition}
    Un gruppo non banale $G$ si dice \vocab{semplice} se i suoi unici sottogruppi
    normali sono $\{e\}$ e $G$.
\end{definition}


\begin{proposition}
    \label{prop1.61}
    $\mathcal{A}_5$ è un gruppo semplice.
\end{proposition}

\begin{proof}
    Ricordiamo le cardinalità delle classi di coniugio in $\mathcal{A}_5$:


    \begingroup
    \renewcommand{\arraystretch}{1.5}
    \begin{center}
    \begin{tabular}{c|c}
        Rappresentante della classe & Cardinalità della classe\\
        \hline
        $\cycle{1, 2, 3, 4, 5}$ & 12\\
        $\cycle{2, 1, 3, 4, 5}$ & 12\\
        $\cycle{1, 2}\cycle{3, 4}$ & 15\\
        $\cycle{1, 2, 3}$ & 20\\
        $id$ & 1 
    \end{tabular}
    \end{center}
    \endgroup
    In generale, un sottogruppo è normale se e solo se è unione disgiunta
    delle classi di coniugio dei suoi elementi, quindi la cardinalità
    di $N \trianglelefteqslant \mathcal{A}_5$ deve essere somma di alcuni 
    termini nella seconda colonna, compreso 1. D'altra parte $|N| \mid |\mathcal{A}_5| = 60$,
    da cui $|N| = 1$ oppure $|N| = 60$. Pertanto $\mathcal{A}_5$ è semplice.
\end{proof}

\begin{lemma}
    \label{lemma1.62}
    Dati un gruppo $G$ e $N\trianglelefteqslant G$ un sottogruppo normale di 
    indice finito, $N$ contiene ogni elemento di $G$ il cui ordine è coprimo
    con $[G:N]$.
\end{lemma}

\begin{proof}
    Sia $g \in G$ tale che $(\ord(g), [G:N]) = 1$, consideriamo la proiezione
    \[
        \pi_N:G \longrightarrow \faktor{G}{N} (x \longmapsto xN)
    \]
    Poiché $\pi_N$ è un omomorfismo abbiamo $\ord(\pi_N(g)) \mid (\ord(g), [G:N]) = 1$,
    pertanto $\pi_N(g) = N$, cioè $g \in N$.
\end{proof}

Diamo adesso una dimostrazione alternativa della semplicità di $\mathcal{A}_5$.

\begin{proof}
    Consideriamo un sottogruppo normale $N\trianglelefteqslant \mathcal{A}_5$.
    Distinguiamo tre casi:
    \begin{itemize}
        \item se $2 \nmid [\mathcal{A}_5:N]$, per il \hyperref[lemma1.62]{Lemma 1.62}
        $N$ contiene tutti gli elementi di 
        $\mathcal{A}_5$ di ordine $2$, cioè le permutazioni della forma $\cycle{a, b}\cycle{c, d}$
        con $a\neq b$ e $c \neq d$, da cui $N = \mathcal{A}_5$ in quanto contiene 
        un suo insieme di generatori;
        \item se $3\nmid [\mathcal{A}_5:N]$, per il \hyperref[lemma1.62]{Lemma 1.62}
        $N$ contiene tutti gli elementi di 
        $\mathcal{A}_5$ di ordine 3, cioè i 3-cicli, da cui $N = \mathcal{A}_5$
        in quanto contiene un suo insieme di generatori;
        \item se $6 \mid [\mathcal{A}_5:N]$ allora $|N| \mid 10$, ma l'unica
        classe di coniugio di $\mathcal{A}_5$ di cardinalità minore di 10 è
        $\{id\}$, pertanto $N = \{id\}$.
    \end{itemize}
    Quindi $\mathcal{A}_5$ è semplice.
\end{proof}

In effetti vale un risultato più generale

\begin{proposition}
    $\mathcal{A}_n$ è un gruppo semplice per $n \geq 5$.
\end{proposition}

\begin{proof}
    Procediamo per induzione su $n$, per $n = 5$ la tesi è garantita dalla 
    \hyperref[prop1.61]{Proposizione 1.61}, supponiamo quindi che $\mathcal{A}_n$
    sia un gruppo semplice e mostriamo che anche $\mathcal{A}_{n + 1}$ lo è.
    Consideriamo un sottogruppo normale $N \trianglelefteqslant \mathcal{A}_{n + 1}$
    e i sottogruppi 
    \[
        H_i = \{\sigma \in \mathcal{A}_{n + 1}\mid \sigma(i) = i\},~
        i \in \{1, \ldots, n + 1\}
    \]
    questi sono tutti isomorfi a $\mathcal{A}_n$ (infatti
    gli elementi di $H_i$ sono tutte e sole le permutazioni pari su $n + 1$
    elementi che fissano l'$i$-esimo, cioè sono permutazioni pari su $n$ elementi).
    Notiamo che l'azione naturale di $\mathcal{A}_{n + 1}$ su $\{1, \ldots, n + 1\}$
    \[
        \psi:\mathcal{A}_{n + 1} \longrightarrow S_{n + 1} :\sigma \longmapsto \sigma
    \]
    è transitiva, infatti per $i, j \in \{1, \ldots, n + 1\}$ distinti 
    la permutazione pari $\rho =\nolinebreak \cycle{i, j}\cycle{h, k}$, con $\cycle{i, j}$
    disgiunta da $\cycle{h, k}$, è tale che $\rho(i) = j$. Per costruzione 
    vale $\St(i) = H_i$ per ogni $i \in \{1, \ldots, n + 1\}$, pertanto
    per la \hyperref[prop1.41]{Proposizione 1.41} abbiamo che gli $H_i$ sono 
    tutti coniugati.\newline
    Fissato $i \in \{1, \ldots, n + 1\}$, consideriamo $N \cap H_i$: questo
    è un sottogruppo normale di $H_i$, infatti per ogni $h \in H_i$ si ha 
    $h(N\cap H_i)h^{-1} = N\cap H_i$ in quanto $N$ è normale in $\mathcal{A}_{n + 1}$
    e $h \in H_i$, d'altra parte $H_i \cong \mathcal{A}_n$ è un gruppo semplice
    per ipotesi induttiva, pertanto $N\cap H_i$ coincide con $\{id\}$ oppure 
    con $H_i$. \newline
    Se $N \cap H_i = H_i$ allora $H_i \subseteq N$, pertanto $N$ 
    contiene almeno un 3-ciclo $\cycle{i, j, k}$ e tutti i suoi coniugati in 
    $\mathcal{A}_{n + 1}$. Notiamo che una trasposizione $\cycle{a, b}$
    disgiunta da $\cycle{i, j, k}$ (che esiste in quanto $n \geq 5$) è una
    permutazione dispari in $Z_{S_{n + 1}}(\cycle{i, j, k})$, pertanto 
    $\Cl_{\mathcal{A}_{n + 1}}(\cycle{i, j, k}) = \Cl_{S_{n + 1}}(\cycle{i, j, k})$
    e $N$ contiene l'insieme dei 3-cicli di $S_{n + 1}$, quindi $N = 
    \mathcal{A}_{n + 1}$ dal momento che contiene un suo insieme di generatori.
    \newline
    Altrimenti $N\cap H_i = \{id\}$ per ogni $i \in \{1, \ldots, n + 1\}$,
    cioè l'unico elemento di $N$ avente almeno un punto fisso è l'identità,
    vogliamo mostrare che in effetti $N = \{id\}$. Osserviamo che se
    $\sigma \in N$ ha una decomposizione in cicli disgiunti
    della forma
    \[
        \sigma = \cycle{x_1^{(1)}, \ldots, x_{l_1}^{(1)}}\ldots\cycle{x_1^{(k)}, \ldots, x_{l_k}^{(k)}}
    \]
    con $l_1\leqslant l_2\leqslant \ldots\leqslant l_k$, allora i suoi cicli 
    hanno tutti la stessa lunghezza, cioè $l_i = l_j$ per ogni $i, j \in \{1, \ldots, n + 1\}$.
    Infatti, posto $r = \min\{l_i\mid 1\leqslant i \leqslant k\} = l_1$, 
    abbiamo
    \[
        \sigma^{l_1} = id \cdot \cycle{x_1^{(2)}, \ldots, x_{l_2}^{(2)}}^{l_1}
        \ldots\cycle{x_1^{(k)}, \ldots, x_{l_k}^{(k)}}^{l_1}
    \]
    Poiché $\sigma^{l_1}$ ha almeno un punto fisso, esiste $i \in \{1, \ldots, n + 1\}$
    tale che $\sigma^{l_1} \in N\cap H_i = \{id\}$, cioè $\sigma^{l_1} = id$. Pertanto 
    $l_1 = l_2 = \ldots = l_k$. Fissata $\sigma \in N$ possiamo quindi scrivere
    $\sigma = \sigma_1\ldots\sigma_k$,
    dove $\sigma_i$ sono $l$-cicli disgiunti con $l = \displaystyle\frac{n + 1}{k}$.
    Supponiamo per assurdo $N \neq \{id\}$, distinguiamo tre casi:
    \begin{itemize}
        \item se $k = 1$ abbiamo $l = n + 1$, cioè $\sigma$ è un $n + 1$-ciclo.
        Scriviamo $\sigma = \cycle{a_1, \ldots, a_l}$ e consideriamo la permutazione
        pari $\tau = \cycle{a_1, a_2}\cycle{a_3, a_4}$, poiché $N$ è normale in 
        $\mathcal{A}_{n + 1}$ contiene
        \[
            \tau\sigma\tau^{-1} = \cycle{a_2, a_1, a_4, a_3, a_5, a_6, \ldots, a_l}
        \]
        Consideriamo $\rho = (\tau\sigma\tau^{-1})\sigma \in N$, notiamo che
        $\rho \neq id$ in quanto
        \[
            \rho(a_4) = (\tau\sigma\tau^{-1})(\sigma(a_4)) = (\tau\sigma\tau^{-1})(a_5) = 
            a_6 \neq a_4
        \]
        d'altra parte $a_1$ è un punto fisso per $\rho$, che è assurdo;
        \item se $k > 1$ e $l > 2$, poiché $\sigma_1^{-1}$ è un $l$-ciclo
        disgiunto da $\sigma_2, \ldots, \sigma_k$ la permutazione 
        $\rho = \sigma_1^{-1}\sigma_2\ldots\sigma_k$ è un elemento di $N$. 
        Consideriamo $\alpha = \rho\sigma \in N$, osserviamo che 
        \[
            \alpha = \sigma_2^2\ldots\sigma_k^2 \neq id
        \]
        in quanto $\ord(\sigma_i) = l > 2$ per ogni $i \in \{1, \ldots, k\}$,
        tuttavia $a_1$ è un punto fisso per $\alpha$, che è assurdo;
        \item se $k > 1$ e $l = 2$, scriviamo $\sigma$ come prodotto di $k$
        trasposizioni disgiunte
        \[
            \sigma = \cycle{a_1, b_1}\ldots\cycle{a_k, b_k}
        \]
        Consideriamo la permutazione pari $\tau = \cycle{a_1, a_2, b_1}$,
        poiché $N$ è normale in $\mathcal{A}_{n + 1}$ contiene
        \[
            \rho = \tau\sigma\tau^{-1} = \cycle{a_2, a_1}\cycle{b_1, b_2}
            \cycle{a_3, b_3}\ldots\cycle{a_k, b_k}
        \]
        e anche la permutazione 
        \[
            \alpha = \rho\sigma = (\cycle{a_2, a_1}\cycle{b_1, b_2})
            (\cycle{a_1, b_1}\cycle{a_2, b_2}) = \cycle{a_1, b_2}\cycle{a_2, b_1} \neq id
        \]
        ma $a_3$ è un punto fisso per $\alpha$, che è assurdo.
    \end{itemize}
    Pertanto $N = \{id\}$, cioè $\mathcal{A}_{n + 1}$ è un gruppo semplice.
\end{proof}

\begin{corollary}
    L'insieme $X = \{\sigma \in S_n\mid \sigma \text{ è un 5-ciclo}\}$ genera
    $\mathcal{A}_n$ per $n \geqslant 5$.
\end{corollary}

\begin{proof}
    Sia $\sigma \in X$ un 5-ciclo, per ogni $\tau \in \mathcal{A}_n$
    abbiamo che $\tau\sigma\tau^{-1}$ è ancora un elemento di $X$, pertanto
    $\langle X\rangle$ è un sottogruppo normale di $\mathcal{A}_n$, da cui 
    $\langle X \rangle = \mathcal{A}_n$ in quanto diverso da $\{id\}$.
\end{proof}


\subsubsection{Sottogruppi normali di $S_n$}

\begin{lemma}
    \label{lemma1.65}
    Per $n \geq 3$ il centro di $S_n$ è banale, cioè $Z(S_n) = \{id\}$.
\end{lemma}

\begin{proof}
    Sia $\sigma \in Z(S_n) \setminus \{id\}$, allora esistono distinti 
    $x, y \in \{1, \ldots, n\}$ tali che $\sigma(x) = y$. Fissiamo 
    $z \in \{1, \ldots, n\}\setminus \{x, y\}$ e consideriamo la permutazione
    $\tau = \cycle{y, z}$, abbiamo
    \[
        (\tau\sigma)(x) = z \quad (\sigma\tau)(x) = y
    \]
    che è assurdo in quanto $y \neq z$. Pertanto $Z(S_n) = \{id\}$.
\end{proof}

\begin{proposition}
    Per $n\geq 5$, gli unici sottogruppi normali di $S_n$ sono $\{id\}$,
    $\mathcal{A}_n$ e $S_n$.
\end{proposition}

\begin{proof}
    Sia $N$ un sottogruppo normale di $S_n$, consideriamo $K= N \cap \mathcal{A}_n$.
    $K$ è normale in $\mathcal{A}_n$, pertanto $K = \{id\}$ oppure $K = \mathcal{A}_n$,
    distinguiamo 2 casi:
    \begin{itemize}
        \item se $K = \mathcal{A}_n$ allora $\mathcal{A}_n \leqslant N$: per il
        Teorema di Corrispondenza i sottogruppi di $S_n$ contententi $\mathcal{A}_n$
        sono in bigezione con i sottogruppi di $\faktor{S_n}{\mathcal{A}_n}\cong \Z2$,
        pertanto $N = \mathcal{A}_n$ oppure $N = S_n$;
        \item se $K = \{id\}$, poiché $[S_n:\mathcal{A}_n] = 2$ per la 
        \hyperref[prop1.49]{Proposizione 1.49} vale $[N:K] \in \{1, 2\}$, 
        da cui $|N| \leq 2$. Se $|N| = 1$ allora $N = \{id\}$, se $|N| = 2$
        consideriamo l'azione di coniugio di $S_n = N_{S_n}(N)$ su $N$
        \[
            \varphi: N_{S_n}(N)\longrightarrow Aut(N): g \longmapsto \varphi_g
        \]
        dove $\varphi_g$ è la mappa
        \[
            \varphi_g:N \longrightarrow N :h \longmapsto ghg^{-1}
        \]
        il nucleo di $\varphi$ coincide con $Z_{S_n}(N)$. Per il Primo Teorema
        di Omomorfismo allora abbiamo un omomorfismo iniettivo
        \[
            \psi :\frac{N_{S_n}(N)}{Z_{S_n}(N)}\longhookrightarrow \Aut(N)
        \]
        Poiché $|N| = 2$ abbiamo $N \cong \Z2$, pertanto $\Aut(N) = \{id\}$. 
        Dato che $N_{S_n}(N) = S_n$ per la normalità di $N$ questo implica che
        sia $Z_{S_n}(N) = S_n$, cioè che $N \subseteq Z(S_n)$, ma questo è assurdo
        in quanto $Z(S_n) = \{id\}$ per il \hyperref[lemma1.65]{Lemma 1.65}.
    \end{itemize}
\end{proof}

\begin{remark}
    L'enunciato è vero anche per $n = 3$ con la stessa dimostrazione, infatti
    $\mathcal{A}_3 \cong \Z3$ è un gruppo semplice, mentre per $n = 4$ è falso.
    Infatti $\mathcal{A}_4$ non è semplice, e il sottogruppo di Klein
    \[
        V_4 = \{id, \cycle{1, 2}\cycle{3, 4}, \cycle{1, 3}\cycle{2, 4}, \cycle{1, 4}\cycle{2, 3}\}
    \]
    è un suo sottogruppo normale non banale. Notiamo che $V_4$ è normale in 
    $S_4$ in quanto unione delle classi di coniugio di tutti i suoi elementi.
\end{remark}


\subsubsection{Sottogruppi isomorfi a $S_{n - 1}$}\footnote{
    Non sono sicuro di essere stato chiarissimo in questa sezione, se ci 
    sono dei passi che ritenete poco comprensibili fatemelo sapere :)
}

Abbiamo osservato più volte che $S_{n - 1}$ si immerge naturalmente in $S_n$,
vediamo adesso un risultato che generalizza questo fatto ad alcuni sottogruppi di 
$S_n$.

\begin{proposition}
    \label{prop1.68}
    Dato un sottogruppo $H\leqslant S_n$ con $n \geq 5$, se $[S_n : H] = n$ allora $H$ è 
    isomorfo a $S_{n - 1}$.
\end{proposition}

\begin{proof}
    Consideriamo l'azione di moltiplicazione a sinistra di $S_n$ sull'insieme
    quoziente $\faktor{S_n}{H}$:
    \[
        \varphi: S_n\longrightarrow S\left(\faktor{S_n}{H}\right) \cong S_n
    \]
    tale azione è transitiva in quanto per ogni $\sigma, \rho \in S_n$ vale
    \[
        \varphi(\sigma\rho^{-1})(\rho H) = \sigma \rho\rho^{-1}H = \sigma H
    \]
    in particolare $\ker \varphi \neq S_n$. Poiché $\ker\varphi \trianglelefteqslant S_n$
    allora il nucleo di $\varphi$ è banale oppure è $\mathcal{A}_n$. D'altra 
    parte se fosse $\ker \varphi = \mathcal{A}_n$ avremmo $|\mathrm{Im}\varphi| = 2$,
    pertanto l'orbita di ogni elemento di $\faktor{S_n}{H}$ contiene al più 
    due elementi, ma questo è assurdo in quanto per la transitività di $\varphi$
    si ha $\Orb(\rho H) = \faktor{S_n}{H}$ per ogni $\rho \in S_n$, che contiene almeno $5$ elementi.
    Pertanto $\ker\varphi = \{id\}$, cioè $\varphi$ è un omomorfismo iniettivo e
    in particolare un isomorfismo.
    Notiamo che $H$ è lo stabilizzatore della classe $H$, infatti
    \[
        \St(H) = \{\sigma \in S_n \mid \sigma H = H\} = \{\sigma \in H\} = H
    \]
    pertanto $\varphi(H)$ è lo stabilizzatore di un elemento di $\faktor{S_n}{H}$
    per l'azione naturale di $S\left(\faktor{S_n}{H}\right)$ su $\faktor{S_n}{H}$.
    Tramite la corrispondenza tra $\faktor{S_n}{H}$ e $\{1, \ldots, n\}$
    possiamo identificare $\varphi(H)$ con le permutazioni di $S_n$ che fissano
    un elemento di $\{1, \ldots, n\}$, che a loro volta costituiscono un 
    gruppo isomorfo a $S_{n - 1}$, pertanto $H \cong S_{n - 1}$.
\end{proof}

Utilizzando il seguente teorema (di cui non diamo la dimostrazione) possiamo
dire qualcosa di più forte nei casi $n \neq 2$ e $n \neq 6$.

\begin{theorem}
    Per $n \notin \{2, 6\}$ i gruppi $S_n$ e $\Aut(S_n)$ sono isomorfi, e 
    l'isomorfismo è dato dall'azione di coniugio
    \[
        \varphi: S_n \longrightarrow \Aut(S_n): \sigma \longmapsto \varphi_{\sigma}
    \]
\end{theorem}

\begin{remark}
    In particolare gli automorfismo di $S_n$ sono tutti interni nei casi $n \notin \{2, 6\}$,
    cioè sono coniugi per elementi di $S_n$
\end{remark}

Con le stesse notazioni di sopra chiamiamo $\varphi'$ l'isomorfismo tra
$S\left(\faktor{S_n}{H}\right)$ e $S_n$, componendo $\varphi'$ con $\varphi$ 
otteniamo un isomorfismo
\[
    \psi :S_n \longrightarrow S_n
\]
che, per $n \notin \{2, 6\}$, è il coniugio per un elemento di $S_n$. Abbiamo
quindi che $\psi(H)$ è lo stabilizzatore di un elemento per l'azione naturale
di $S_n$ su $\{1, \ldots, n\}$, ma allora anche $H$ è uno stabilizzatore per tale
azione in quanto coniugato a $\psi(H)$\footnote{
    Notiamo che l'azione naturale di $S_n$ su $\{1, \ldots, n\}$ è transitiva,
    pertanto gli stabilizzatori sono tra loro coniugati.
}.Pertanto i sottogruppi di $S_n$ isomorfi a $S_{n - 1}$ sono tra loro coniugati
e ognuno è lo stabilizzatore di un elemento per l'azione naturale di $S_n$
su $\{1, \ldots, n\}$.

\subsubsection{Costruzione di un automorfismo esterno di $S_6$}

Abbiamo visto che i casi $n = 2$ e $n = 6$ sono gli unici per cui non vale 
che $S_n \cong \Aut(S_n)$. Per $n = 2$ il motivo è semplice, infatti essendo
$S_2$ isomorfo a $\Z2$ il suo gruppo di automorfismi è banale, per $n = 6$
invece abbiamo che gli automorfismi di $S_6$ non sono tutti elementi di $\Inn(S_n)$,
vogliamo quindi esibire un automorfismo di $S_6$ che non sia interno.\newline

Iniziamo osservando che $S_5$ contiene 6 5-Sylow, infatti tali sottogruppi 
sono isomorfi a $\Z5$ e, essendo i 5-cicli gli unici elementi di ordine 5,
$S_5$ ne contiene esattamente 
\[
    \frac{1}{\phi(5)} \binom{5}{4}3! = 6
\]
Posto $X = \{P_1, \ldots, P_6\}$ l'insieme dei 5-Sylow di $S_5$, consideriamo l'azione di coniugio 
di $S_5$ su $X$
\[
    \varphi: S_5 \longrightarrow S(X) \cong S_6
\]
dove l'isomorfismo tra $S(X)$ e $S_6$ è dato dall'associare $P_i$ a $i$, 
poniamo $\Phi$ la composizione di $\varphi$ con tale isomorfismo, notiamo
che $\Phi$ è un'immersione di $S_5$ in $S_6$. L'azione 
$\varphi$ è transitiva in quanto i 5-Sylow di $S_5$ sono tutti coniugati,
pertanto $\ker\varphi = \{id\}$ oppure $\ker\varphi = \mathcal{A}_5$. D'altra parte
se fosse $\ker\varphi = \mathcal{A}_5$ si avrebbe che $|\mathrm{Im}\varphi| = 2$,
pertanto l'orbita di ogni elemento ha cardinalità 2, che è assurdo in quanto
$\Orb(P_i) = X$ per ogni $P_i \in X$ per transitività di $\varphi$, quindi l'azione
è inettiva.\newline
La transitività di $\varphi$ implica che $\Phi$ sia un'azione transitiva
di $S_5$ sull'insieme $\{1, \ldots, 6\}$, notiamo quindi che $\mathrm{Im}\Phi$
non può essere lo stabilizzatore di un elemento di $\{1, \ldots, 6\}$
per l'azione naturale di $S_6$ su tale insieme. Infatti se lo fosse esisterebbe
$k \in \{1, \ldots, n\}$ tale che $\Phi(\sigma)(i) = i$ per ogni $\sigma \in S_5$,
ma questo è assurdo in quanto per la \hyperref[prop1.41]{Proposizione 1.41} 
$S_5$ contiene una permutazione che agisce su $\{1, \ldots, 6\}$ senza punti
fissi.

Abbiamo che $H = \mathrm{Im}\Phi$ è un sottogruppo di $S_6$ di indice $6$ e possiamo
considerare l'azione transitiva e iniettiva di moltiplicazione a sinistra di $S_6$ su 
$\faktor{S_6}{H}$
\[
    \alpha: S_6\longrightarrow S\left(\faktor{S_6}{H}\right) \cong S_6
\]
chiamiamo $\psi: S_6\longrightarrow S_6$ l'isomorfismo risultante dalla
composizione di $\alpha$ con l'isomorfismo tra $S\left(\faktor{S_6}{H}\right)$
e $S_6$. Sia $i \in \{1, \ldots, 6\}$ l'elemento associato alla classe $H$,
abbiamo visto nella dimostrazione della \hyperref[prop1.68]{Proposizione 1.68}
che $\psi(H) = \St(i)$ per l'azione naturale di $S_6$ sull'insieme $\{1, \ldots, 6\}$.
Concludiamo osservando che se $\psi$ fosse un automorfismo interno di $S_6$,
allora anche $\psi^{-1}$ sarebbe un automorfismo interno, cioè $\psi^{-1}$
sarebbe il coniugio per un qualche $\sigma \in S_6$ fissato, da cui
\[
    H = \psi^{-1}(\St(i)) = \sigma\St(i)\sigma^{-1} = \St(\sigma(i))
\]
che è assurdo in quanto $H$ non può essere uno stabilizzatore per tale azione,
pertanto $\psi \notin \Inn(S_6)$.

\newpage

\subsection{Prodotti semidiretti}

\subsubsection{Descrizione di $S_4$ come prodotto semidiretto}

Per ogni $n \geq 2$ vale in generale la relazione 
\[
    S_n \cong \mathcal{A}_n \rtimes \langle\cycle{a, b}\rangle
\]
dove $\cycle{a, b}$ è una trasposizione di $S_n$, vogliamo però dare una
decomposizione di $S_4$ più specifica. \newline
Consideriamo il sottogruppo di Klein $V_4 = \{id, \cycle{1, 2}\cycle{3, 4}, \cycle{1, 3}\cycle{2, 4},
\cycle{1, 4}\cycle{2, 3}\}$ e $H = \{\sigma \in S_4\mid \sigma(4) = 4\}$ lo 
stabilizzatore di $4$ secondo l'azione naturale di $S_4$ su $\{1, 2, 3, 4\}$,
osserviamo che $V_4$ è normale in $S_4$ in quanto
unione delle classi di coniugio di ogni suo elemento\footnote{
    La classe di coniugio in $S_4$ di $\cycle{1, 2}\cycle{3, 4}$ è 
    $\{\cycle{1, 2}\cycle{3, 4}, \cycle{1, 3}\cycle{2, 4}, \cycle{1, 4}\cycle{2, 3}\}$.
}
e che $H$ è isomorfo a $S_3$ (in effetti gli elementi di $H$ sono tutte e 
sole le permutazioni di 3 elementi). Dato che l'unica permutazione di $V_4$
che fissa 4 è l'identità abbiamo $V_4 \cap H = \{id\}$, inoltre $V_4H = S_4$
in quanto i due insiemi hanno la stessa cardinalità. Possiamo quindi scrivere
\[
    S_4 \cong V_4 \rtimes H \cong (\Z2 \times \Z2) \rtimes_{\varphi} S_3
\]

con

\[
    \varphi: S_3 \longrightarrow \Aut(\Z2\times\Z2)
\]

Specifichiamo come agisce la mappa $\varphi$\footnote{Se descriviamo $S_4$ 
    come prodotto semidiretto di due sottogruppi questo non è 
    necessario, in quanto tale mappa è sempre il coniugio.
}:
consideriamo gli isomorfismi
\[
    \alpha: V_4\longrightarrow \Z2\times\Z2: \cycle{1, 2}\cycle{3, 4}\longmapsto (1, 0),
    \cycle{1, 3}\cycle{2, 4} \longmapsto (0, 1)
\]
\[
    \beta: H \longrightarrow S_3: \sigma \longmapsto \sigma_{\mid\{1, 2, 3\}}
\]
le immagini di $\varphi$ in $\Aut(\Z2\times\Z2)$ corrispondono tramite $\alpha$
e $\beta$ ai coniugi su $V_4$ per elementi di $H$. Vediamo quindi come
i generatori $\cycle{1, 2, 3}$, $\cycle{1, 2}$ di $H$ agiscono per coniugio
sui generatori $\cycle{1, 2}\cycle{3, 4}$, $\cycle{1, 3}\cycle{2, 4}$ di $V_4$:
\[
    \cycle{1, 2, 3}(\cycle{1, 2}\cycle{3, 4})\cycle{1, 3, 2} = \cycle{1, 4}\cycle{2, 3}
\]
\[
    \cycle{1, 2, 3}(\cycle{1, 3}\cycle{2, 4})\cycle{1, 3, 2} = \cycle{1, 2}\cycle{3, 4}
\]
\[
    \cycle{1, 2}(\cycle{1, 2}\cycle{3, 4})\cycle{1, 2} = \cycle{1, 2}\cycle{3, 4}
\]
\[
    \cycle{1, 2}(\cycle{1, 3}\cycle{2, 4})\cycle{1, 2} = \cycle{1, 4}\cycle{2, 3}
\]
Pertanto $\varphi(\cycle{1, 2, 3}) = f$ e $\varphi(\cycle{1, 2}) = g$,
dove $f$ e $g$ sono gli automorfismi di $\Z2\times\Z2$ tali che
\[
    f: (1, 0)\longmapsto (1, 1), (0, 1) \longmapsto (1, 0)
\]
\[
    g: (1, 0)\longmapsto (1, 0), (0, 1) \longmapsto (1, 1)
\]

\subsubsection{Automorfismi di $D_n$}

Consideriamo il gruppo 
\[
    G = \{f: \Zn\longrightarrow\Zn\mid \exists a \in (\Zn)^*, b \in \Zn\text{ per cui }
    f(x) = ax + b~\forall x \in \Zn\}
\]
delle sostituzioni lineari in $\Zn$, effettivamente $G$ è un gruppo con 
l'operazione di composizione. Infatti fissati $a \in (\Zn)^*$, $b \in \Zn$ e
$f \in G$ tali che $f(x) = ax + b$, abbiamo che $f^{-1}$ è tale che 
$f^{-1}(x) = a^{-1}(x - b)$ (chiaramente $G$ contiene l'applicazione nulla 
ed è chiuso per composizione). Notiamo che un elemento di $G$ è univocamente
determinato dalla coppia $(b, a) \in \Zn \times (\Zn)^*$\footnote{
    Consideriamo qua solo l'insieme prodotto cartesiano, non la struttura di 
    gruppo data dal prodotto diretto.
}, pertanto $G$
contiene $n\phi(n)$ elementi. In realtà possiamo essere più precisi:

\begin{proposition}
    Il gruppo $G$ definito come sopra è isomorfo a un prodotto semidiretto 
    \[
        \Zn \rtimes (\Zn)^*
    \]
\end{proposition}

\begin{proof}
    Consideriamo i sottogruppi di $G$
    \[
        N = \{f \in G\mid f(x) = x + b,~b \in \Zn\}
    \]
    \[
        H = \{f \in G\mid f(x) = ax,~a \in (\Zn)^*\}
    \]
    osserviamo che $N$ e $H$ sono naturalmente isomorfi a $\Zn$, $(\Zn)^*$ 
    rispettivamente e che $N \cap H = \{id\}$, pertanto $NH = G$ in quanto 
    \[
        |NH| = \frac{|N|\cdot|H|}{|N\cap|H|} = |N|\cdot|H| = n\phi(n) = |G|
    \]
    Mostriamo quindi che $N$ è un sottogruppo normale di $G$: fissati $f \in N$
    e $g \in G$ tali che $f(x) = x + t$ e $g(x) = ax + b$, con $b, t \in \Zn$ e 
    $a \in (\Zn)^*$, abbiamo
    \[
        (g^{-1}\circ f \circ g)(x) = (g^{-1}\circ f)(ax + b) = g^{-1}(ax + b + t) = 
        x + a^{-1}t
    \]
        pertanto $g^{-1}\circ f\circ g \in N$, cioè $N\trianglelefteqslant G$.
    Possiamo quindi decomporre $G$ come prodotto semidiretto:
    \[
        G \cong N\rtimes H
    \]
    poiché $N \cong \Zn$ e $H \cong (\Zn)^*$ abbiamo che $G$ è isomorfo a un
    prodotto semidiretto 
    \[
        \Zn \rtimes (\Zn)^*
    \]
\end{proof}

Rappresentiamo gli elementi di $G$ tramite le coppie $(b, a) \in \Zn\times (\Zn)^*$
(come insieme, non come gruppo),
la composizione in $G$ produce la seguente operazione sulle coppie:
\[
    (b_1, a_1)(b_2, a_2) = (b_1 + a_1b_2, a_1a_2)
\]
pertanto l'omomorfismo che definisce il prodotto semidiretto $\Zn \rtimes_{\varphi}(\Zn)^*$
a cui è isomorfo $G$ è 
\[
    \varphi: (\Zn)^* \longrightarrow \Aut(\Zn): a \longmapsto \varphi_a
\]
dove $\varphi_a$ è l'omomorfismo di moltiplicazione per $a$
\[
    \varphi_a :\Zn \longrightarrow \Zn: x \longmapsto ax
\]

\begin{proposition}
    Il gruppo $G$ delle sostituzioni lineari in $\Zn$ è isomorfo a $\Aut(D_n)$
    per $n \geq 3$.
\end{proposition}

\begin{proof}
    Siano $r, s \in D_n$ tali che $\ord(r) = n$, $\ord(s) = 2$, $D_n = \langle r, s\rangle$,
    consideriamo $\varphi \in Aut(D_n)$. Poiché $\langle r\rangle \cong \Zn$ è un 
    sottogruppo caratteristico di $D_n$ abbiamo che esistono unici $a \in (\Zn)^*$,
    $b \in \Zn$ tali che 
    \[
        \varphi(r) = r^a\qquad \varphi(s) = sr^b
    \]
    Consideriamo $\varphi_1$, $\varphi_2 \in \Aut(D_n)$ tali che
    \[
        \varphi_i(r) = r^{a_i} \qquad \varphi_i(s) = sr^{b_i}
    \]
    con $a_i \in (\Zn)^*$, $b_i \in \Zn$ per $i \in \{1, 2\}$, componendo
    $\varphi_1$ con $\varphi_2$ otteniamo
    \[
        \varphi_1(\varphi_2(r)) = \varphi_1(r^{a_2}) = r^{a_1a_2}
    \]
    \[
        \varphi_1(\varphi_2(s)) = \varphi_1(sr^{b_2}) = sr^{b_1 + a_1b_2}
    \]
    Pertanto $\Aut(D_n)$ è isomorfo a un quoziente di $G$ in quanto i suoi
    elementi rispettano la stessa legge di gruppo, d'altra parte $|\Aut(D_n)| = |G|$,
    pertanto i due gruppi sono proprio isomorfi.
\end{proof}

\subsubsection{Prodotti semidiretti isomorfi}

Dati due gruppi, può succedere che il loro prodotto diretto sia isomorfo a 
un loro prodotto semidiretto non banale.\newline
Consideriamo il gruppo $GL_3(\RR)$ e $N = SL_3(\RR) = \{M \in GL_3(\RR)
\mid \det M = 1\}$, $N$ è un sottogruppo normale di $GL_3(\RR)$ in quanto è il nucleo
dell'omomorfismo
\[
    \det: GL_3(\RR) \longrightarrow \RR^*
\]
mostriamo che $GL_3(\RR) \cong SL_3(\RR)\times\RR^*$. Consideriamo il sottogruppo
\[
    H = \left\{\begin{pmatrix}
    \lambda & 0 & 0\\
    0 & \lambda & 0\\
    0 & 0 & \lambda
    \end{pmatrix}\Bigm| \lambda \in \RR^* \right\}
\]isomorfo a $\RR^*$, abbiamo che:

\begin{itemize}
    \item $N \cap H = \{Id\}$ in quanto $M = \lambda Id \in N\cap H$ è tale
    che $\det M = \lambda^3 = 1$, cioè $\lambda = 1$ e quindi $M = Id$;
    \item $H$ è un sottogruppo normale di $GL_3(\RR)$, in quanto tutti i suoi elementi
    sono multipli scalari della matrice identità e quindi commutano con gli
    elementi di $GL_3(\RR)$;
    \item $GL_3(\RR) = NH$, infatti per ogni $M \in GL_3(\RR)$ possiamo scrivere $M = S(\lambda Id)$,
    dove $\lambda = \nolinebreak(\det M)^{\frac 1 3}$ e $S = (\det M)^{-\frac 1 3} M \in N$.
\end{itemize}

Possiamo quindi scrivere 
\[
    GL_3(\RR) \cong SL_3(\RR) \times H \cong SL_3(\RR)\times \RR^*
\]
Consideriamo adesso il sottogruppo di $GL_3(\RR)$

\[
    K = \left\{\begin{pmatrix}
        \lambda & 0 & 0\\
        0 & 1 & 0\\
        0 & 0 & 1
    \end{pmatrix}\Bigm| \lambda \in \RR^* \right\}
\]
anch'esso isomorfo a $\RR^*$. Ragionando in modo analogo abbiamo $N \cap H = \{Id\}$, 
inoltre $GL_3(\RR) = NK$ in quanto per ogni $M \in GL_3(\RR)$ possiamo scrivere
$M = (MA^{-1})A$ con 
\[
    A = \begin{pmatrix}
        \det M & 0 & 0\\
        0 & 1 & 0\\
        0 & 0 & 1
    \end{pmatrix} \in K, \quad MA^{-1} \in N
\]
Possiamo quindi scrivere 
\[
    GL_3(\RR) \cong SL_3(\RR)\rtimes K
\]
Notiamo che l'azione di coniugio di $K$ su $SL_3(\RR)$ non è banale, in 
quanto
\[
    \begin{pmatrix}
        \lambda & 0 & 0\\
        0 & 1 & 1\\
        0 & 0 & 1
    \end{pmatrix}
    \begin{pmatrix}
        1 & 1 & 0\\
        0 & 1 & 0\\
        0 & 0 & 1
    \end{pmatrix}
    \begin{pmatrix}
        \lambda^{-1}& 0 & 0\\
        0 & 1 & 0\\
        0 & 0 & 1
    \end{pmatrix} = 
    \begin{pmatrix}
        1 & \lambda & 0\\
        0 & 1 & 0\\
        0 & 0 & 1
    \end{pmatrix} \neq \begin{pmatrix}
        1 & 1 & 0\\
        0 & 1 & 0\\
        0 & 0 & 1
    \end{pmatrix},\quad \lambda \neq 0, 1
\]
quindi il prodotto non è diretto.\newline


È in realtà relativamente semplice costruire prodotti diretti e prodotti semidiretti
isomorfi a partire da un gruppo non abeliano, diamo l'esempio di una possibile
procedura nella seguente dimostrazione.

\begin{proposition}
    Dato un gruppo $G$ non abeliano, esiste un omomorfismo
    \[
        \varphi: G\longrightarrow \Aut(G)
    \]
    non banale tale che $G\times G \cong G\rtimes_{\varphi} G$.
\end{proposition}

\begin{proof}
    Consideriamo i sottogruppi $N = G\times\{e\}$, $H = \{(g, g) \mid g \in G\}$,
    notiamo che $N$ è un sottogruppo normale di $G\times G$. Inoltre 
    $N\cap H = \{e, e\}$ e $NH = G\times G$, in quanto per 
    ogni elemento $(g_1, g_2) \in G\times G$ abbiamo
    \[
        (g_1, g_2) = (g_1g_2^{-1}, e)(g_2, g_2)
    \]
    con $(g_1g_2^{-1}, e) \in N$ e $(g_2, g_2) \in H$, pertanto possiamo 
    scrivere $G \times G = N \rtimes_{\varphi} H$, dove $\varphi$ è un omomorfismo
    \[
        \varphi: H \longrightarrow \Aut(N)
    \]
    Tale $\varphi$ è banale se e solo se $\varphi(h) = id$ per ogni $h \in H$,
    se e solo se $hnh^{-1} = n$ per ogni $h \in H$, per ogni $n \in N$.
    Questo è equivalente a richiedere 
    \[
        (g, g)(h, e)(g^{-1}, g^{-1}) = (ghg^{-1}, e)= (h, e)~\forall g, h \in G
    \]
    cioè $g \in Z(G)$ per ogni $g \in G$,
    ma questo è assurdo in quanto $G$ non è abeliano, pertanto $\varphi$
    non è l'omomorfismo banale. Poiché $N \cong H \cong G$ abbiamo quindi
    \[
        G \times G \cong G\rtimes_{\varphi'}G
    \]
    dove
    \[
        \varphi':G \longrightarrow \Aut(G)
    \]
    è l'omomorfismo non banale corrispondente a $\varphi$.
\end{proof}

Vediamo adesso un criterio che stabilisce una condizione sufficiente affinché 
i prodotti semidiretti di due gruppi siano isomorfi.

\begin{proposition}
    [Criterio di isomorfismo tra prodotti semidiretti]
    \label{prop1.74}
    Siano $H, N$ due gruppi e $\varphi: H \longrightarrow \Aut(N)$ un omomorfismo.
    Dato $f \in \Aut(H)$ allora $N\rtimes_{\varphi} H \cong N\rtimes_{\varphi\circ f} H$.
\end{proposition}

\begin{proof}
    Consideriamo l'applicazione
    \[
        \psi: N\rtimes_{\varphi}H \longrightarrow N\rtimes_{\varphi\circ f}H:
        (n, h) \longmapsto (n, f^{-1}(h))
    \]
    $\psi$ è una bigezione tra i due insiemi in quanto $f$ è bigettiva,
    mostriamo che è anche un omomorfismo di gruppi. Per ogni $(n_1, h_1)$, $(n_2, h_2)
    \in N\rtimes_{\varphi}H$ abbiamo
    \begin{multline*}
        \psi((n_1, h_1)(n_2, h_2)) = \psi(n_1\cdot\varphi(h_1)(n_2), h_1h_2) = \\
        = (n_1\cdot\varphi(h_1)(n_2), f^{-1}(h_1h_2)) = (n_1\cdot\varphi(h_1)(n_2), f^{-1}(h_1)f^{-1}(h_2))
    \end{multline*}
    d'altra parte
    \begin{multline*}
        \psi(n_1, h_1)\psi(n_2, h_2) = (n_1, f^{-1}(h_1))(n_2, f^{-1}(h_2)) =\\
        = (n_1\cdot(\varphi\circ f)(f^{-1}(h_1))(n_2), f^{-1}(h_1)f^{-1}(h_2)) = 
        (n_1\cdot \varphi(h_1)(n_2), f^{-1}(h_1)f^{-1}(h_2))
    \end{multline*}
    cioè $\psi$ è un omomorfismo, quindi i due gruppi sono isomorfi.
\end{proof}

\begin{example}
    Abbiamo visto che i prodotti semidiretti della forma $\Zp \rtimes\Z{q}$
    con $p$, $q$ primi tali che $q \mid p - 1$ si suddividono in esattamente
    due classi di isomorfismo, utilizziamo il risultato appena mostrato 
    per verificare che tutti i prodotti semidiretti non banali sono tra loro 
    isomorfi. Consideriamo un omomorfismo
    \[
        \varphi_a: \Z{q}\longrightarrow \Z{(p - 1)} : 1 \longmapsto a
    \]
    con $\ord(a) = q$ (poiché $\Aut(\Zp) \cong \Z{(p - 1)}$
    questo è equivalente a fissare un omomorfismo tra $\Z{q}$ e $\Aut(\Zp))$,
    possiamo scrivere 
    \[
        a = k\frac{p - 1}{q}\quad k \in \{1, \ldots, q - 1\}
    \]
    Posto $f_k \in \Aut(\Z{q})$ tale che
    \[
        f_k :\Z{q}\longrightarrow\Z{q}: x \longmapsto kx
    \]
    con $(k, q) = 1$, possiamo scrivere 
    $\varphi_a = \varphi_{\frac{p - 1}{q}}\circ f_k$. Allora i prodotti
    semidiretti non banali $\Zp\rtimes_{\varphi_a}\Z{q}$ sono tutti isomorfi
    a tra loro per la \hyperref[prop1.74]{Proposizione 1.74}.
\end{example}

Vediamo adesso un criterio che fornisce una condizione sufficiente affinché
due prodotti semidiretti di $p$-gruppi non siano isomorfi.

\begin{proposition}
    Siano $p$, $q$ due primi distinti, $G$ un $p$-gruppo e $H$ un $q$-gruppo,
    consideriamo i prodotti semidiretti 
    \[
        X_1 = G\rtimes_{\varphi_1}H \qquad X_2 = G\rtimes_{\varphi_2}H
    \]
    con 
    \[
        \varphi_1, \varphi_2 :H \longrightarrow \Aut(G)
    \]
    Se $\ker\varphi_1$ e $\ker\varphi_2$ non sono isomorfi allora $X_1$ e 
    $X_2$ non sono isomorfi.
\end{proposition}

\begin{proof}
    Dimostriamo la contronominale, cioè che se $X_1$ e $X_2$ sono isomorfi
    allora $\ker\varphi_1 \cong \ker\varphi_2$.\newline
    Sia $f: X_1\longrightarrow X_2$ un isomorfismo, poniamo $\mathcal{G}_1 = 
    G\rtimes_{\varphi_1}\{e_H\}$, $\mathcal{G}_2 = G\rtimes_{\varphi_2}\{e_H\}$,
    $\mathcal{H}_1 = \{e_G\}\rtimes_{\varphi_1}H$, $\mathcal{H}_2 = \{e_G\}\rtimes_{\varphi_2}H$.
    Osserviamo che $f(\mathcal{G}_1) = \mathcal{G}_2$ in quanto $\mathcal{G}_1$
    è l'unico $p$-Sylow di $X_1$ e $\mathcal{G}_2$ è l'unico $p$-Sylow di $X_2$
    (infatti $\mathcal{G}_1 \trianglelefteqslant X_1$ e $\mathcal{G}_2 
    \trianglelefteqslant X_2$), mentre $f(\mathcal{H}_1)$ è un $q$-Sylow di $X_2$
    coniugato a $\mathcal{H}_2$. In particolare esiste $\psi \in \Inn(X_2)$
    tale che
    \[
        (\psi\circ f)(\mathcal{G}_1) = \mathcal{G}_2 \qquad 
        (\psi\circ f)(\mathcal{H}_1) = \mathcal{H}_2
    \]
    pertanto, a meno di coniugio, possiamo supporre $f(\mathcal{G}_1) = \mathcal{G}_2$ e 
    $f(\mathcal{H}_1) = \mathcal{H}_2$. Caratterizziamo i nuclei di $\varphi_1$, 
    $\varphi_2$ in termini di centralizzatori, in particolare scriviamo
    \begin{multline*}
        Z_{\mathcal{H}_1}(\mathcal{G}_1) = \{(e_G, h) \in \mathcal{H}_1\mid
        (e_G, h)(g, e_H)(e_G, h)^{-1} = (g, e_H)~\forall g \in G\} = \\
        = \{(e_G, h)\in \mathcal{H}_1 \mid (\varphi_1(h)(g), h)(e_G, h^{-1})
        = (g, e_H)~\forall g \in G\} = \\
        = \{(e_G, h) \in \mathcal{H}_1\mid (\varphi_1(h)(g), e_H) = (g, e_H)~
        \forall g \in G\} = \\
        = \{(e_G, h) \in \mathcal{H}_1\mid \varphi_1(h) = id\} =
        \{e_G\}\rtimes_{\varphi_1}\ker\varphi_1\\
    \end{multline*}
    e ragionando in modo analogo
    \[
        Z_{\mathcal{H}_2}(\mathcal{G}_2) = \{e_G\}\rtimes_{\varphi_2}\ker\varphi_2
    \]
    Poniamo $\chi = \psi\circ f$, chiaramente $\chi: X_1\longrightarrow X_2$
    è un isomorfismo e $\chi(\mathcal{G}_1) = \mathcal{G}_2$, $\chi(\mathcal{H}_1)
    = \mathcal{H}_2$ per quanto detto sopra, pertanto
    \begin{multline*}
        \{e_G\}\rtimes_{\varphi_2}\ker\varphi_2 = Z_{\mathcal{H}_2}(\mathcal{G}_2) =
        Z_{\chi(\mathcal{H}_1)}(\chi(\mathcal{G}_1)) = \\
        = \{\chi(h_1) \mid h_1 \in \mathcal{H}_1, \chi(h_1)\chi(g_1) = 
        \chi(g_1)\chi(h_1)~\forall g_1 \in \mathcal{G}_1\} = \\
        = \{\chi(h_1)\mid h_1 \in \mathcal{H}_1, \chi(h_1g_1) = \chi(g_1h_1)
        ~\forall g_1 \in \mathcal{G}_1\} = \\
        = \{\chi(h_1) \mid h_1 \in Z_{\mathcal{H}_1}(\mathcal{G}_1)\} = 
        \chi(\{e_G\}\rtimes_{\varphi_1}\ker\varphi_1)\\
    \end{multline*}
    In particolare quindi $\chi$ induce un isomorfismo tra $\ker\varphi_2$ e 
    $\ker\varphi_1$.
\end{proof}

\newpage

\subsection{Classificazione dei gruppi semplici di ordine al più 100}

In questa sezione vogliamo determinare quali sono i sottogruppi semplici di 
ordine minore o uguale a 100. Facciamo prima una serie di osservazioni che 
ci permetterà di ridurre lo studio a pochi casi interessanti.

\begin{itemize}
    \item Gli unici gruppi abeliani semplici sono i gruppi $\Zp$ con $p$ primo,
    in quanto i loro sottogruppi sono solo quelli banali e tutti i sottogruppi
    di un gruppo abeliano sono normali;
    \item i gruppi $G$ di ordine $p^k$ con $p$ primo e $k > 1$ non sono semplici
    in quanto hanno centro non banale e il centro è un sottogruppo caratteristico,
    in particolare normale (alternativamente, dal Teorema di Sylow abbiamo che 
    $G$ contiene un sottogruppo proprio di ordine $p^{k - 1}$, che è normale in quanto
    il suo indice è $p$, il più piccolo primo che divide $|G|$);
    \item i gruppi di ordine $2d$ con $d$ dispari non sono semplici in quanto
    contengono un sottogruppo di indice $2$, che è normale e non banale, per 
    l'\hyperref[ex1.48]{Esercizio 1.48};
    \item i gruppi di ordine $pq$ con $q > p$ primi non sono semplici, in
    quanto possiamo scriverli come prodotto semidiretto dei loro sottogruppi
    di Sylow, pertanto almeno uno di questi è normale e non banale;
    \item $\mathcal{A}_5$ è un gruppo semplice di ordine $60$.
\end{itemize}

Ci riduciamo quindi a studiare i gruppi di ordine 56, 60, 72, 80, 96.
\newline\newline
\underline{$|G| = 56 = 2^3\cdot 7$}: poiché $n_7 \equiv 1 \pmod 7$ e $n_7 \mid 56$ abbiamo
$n_7 \in \{1, 8\}$. Se $n_7 = 1$ allora $G$ contiene un unico 7-Sylow, chè 
è quindi un sottogruppo proprio normale di $G$. Se $n_7 = 8$ allora $G$ contiene
$6\cdot8 = 48$ elementi di ordine 7 (dato che i 7-Sylow di $G$ sono isomorfi a $\Z7$)
pertanto i restanti $7$ elementi non banali devono essere contenuti in un 
unico 2-Sylow, che è quindi normale. In entrambi i casi $G$ non è semplice.
\newline\newline
\underline{$|G| = 96 = 2^5\cdot 3$}: sia $P_2$ un 2-Sylow di $G$, poiché 
$[G:P_2] = 3$ per il \hyperref[teorema1.50]{Teorema 1.50} esiste un sottogruppo
$N \triangleleft G$ tale che $N \subseteq P_2$ e $[G:N] \mid 3!$, da cui
$N \neq G$ e $N \neq \{e\}$ in quanto $[G:\{e\}] = |G|$. Pertanto $G$ non 
è semplice.
\newline\newline
\underline{$|G| = 72 = 2^3 \cdot 3^2$}: dalle condizioni 
\[
    \begin{cases}
        n_2 \equiv 1 \pmod 2\\
        n_2 \mid 72
    \end{cases}\quad
    \begin{cases}
        n_3 \equiv 1 \pmod 3\\
        n_3 \mid 72
    \end{cases}
\]
otteniamo $n_2 \in \{1, 3, 9\}$ e $n_3 \in \{1, 4\}$, distinguiamo quindi
due casi.
\begin{itemize}
    \item Se $n_3 = 1$ allora $G$ contiene un unico 3-Sylow, che è quindi un
    sottogruppo normale non banale di $G$, cioè $G$ non è semplice;
    \item se $n_3= 4$, siano $Q_1$, $Q_2$, $Q_3$, $Q_4$ i 3-Sylow di $G$ e 
    $X = \{Q_1, Q_2, Q_3, Q_4\}$, consideriamo l'azione di coniugio di $G$ su $X$
    \[
        \varphi: G\longrightarrow S(X) \cong S_4
    \]
    poiché i 3-Sylow di $G$ sono tutti coniugati tale azione è transitiva.
    Mostriamo che $\ker \varphi$ è un sottogruppo di $G$ non banale. Se $\ker\varphi
    = \{e\}$ allora $\varphi$ sarebbe un omomorfismo iniettivo, che è assurdo
    in quanto l'ordine di $G$ non divide l'ordine di $S(X) \cong S_4$. D'altra
    parte se fosse $\ker\varphi = G$ allora $\varphi$ sarebbe l'azione banale,
    che è assurdo in quanto $\varphi$ è transitiva e $|X| > 1$ (alternativamente,
    se $\varphi$ fosse l'azione banale allora i 3-Sylow di $G$ sarebbero tutti
    normali). Pertanto $\ker\varphi$ è un sottogruppo normale non banale di $G$,
    cioè $G$ non è semplice.
\end{itemize}
\underline{$|G| = 80 = 2^4\cdot 5$}: dalle condizioni
\[
    \begin{cases}
        n_2 \equiv 1 \pmod 2\\
        n_2 \mid 80
    \end{cases}\quad
    \begin{cases}
        n_5 \equiv 1 \pmod 5\\
        n_5 \mid 80
    \end{cases}
\]
otteniamo $n_2 \in \{1, 5\}$ e $n_5 \in \{1, 16\}$, distinguiamo quindi due
casi.
\begin{itemize}
    \item Se $n_5 = 1$ allora $G$ contiene un unico 5-Sylow, che è quindi un
    sottogruppo normale non banale di $G$, cioè $G$ non è semplice;
    \item se $n_5 = 16$ allora $G$ contiene $4\cdot 16 = 64$ elementi di ordine
    $5$ (dato che i 5-Sylow di $G$ sono isomorfi a $\Z5$), pertanto i restanti
    15 elementi devono esserre contenuti in un unico 2-Sylow, che è quindi normale.
    Allora $G$ non è semplice. \newline Alternativamente, consideriamo $P_2$ un 2-Sylow
    e l'azione di moltiplicazione a sinistra di $G$ sull'insieme quoziente $\faktor{G}{P_2}$
    \[
        \varphi: G\longrightarrow S\left(\faktor{G}{P_2}\right) \cong S_5
    \]
    Poiché $|G| \nmid |S_5|$ abbiamo $\ker\varphi\neq\{e\}$, d'altra parte
    $\ker\varphi \neq G$ in quanto $\varphi$ è un'azione transitiva (per 
    ogni $x, y \in G$ vale $\varphi(xy^{-1})(yP_2) = xy^{-1}yP_2 = xP_2)$.
    Quindi $\ker\varphi$ è un sottogruppo normale di $G$ non banale, cioè $G$
    non è semplice. 
\end{itemize}

Rimangono da studiare i gruppi di ordine $60$, vogliamo dimostrare che 
$\mathcal{A}_5$ è l'unico sottogruppo semplice di tale ordine (a meno di isomorfismo).

\begin{lemma}
    $\mathcal{A}_5$ contiene esattamente 5 2-Sylow.
\end{lemma}

\begin{proof}
    Sia $X$ l'insieme dei 2-Sylow di $\mathcal{A}_5$, consideriamo l'azione
    di coniugio di $\mathcal{A}_5$ su $X$
    \[
        \varphi: \mathcal{A}_5\longrightarrow S(X)
    \]
    poiché i 2-Sylow di $\mathcal{A}_5$ sono tutti coniugati e $\mathcal{A}_5$
    è semplice tale azione è transitiva, in particolare $X$ è composto da 
    un'unica orbita. Fissato $P$ un 2-Sylow abbiamo
    \[
        n_2 = |\Orb(P)| = \frac{|\mathcal{A}_5|}{|N_{\mathcal{A}_5}(P)|}
    \]
    Scegliamo $P = \{id, \cycle{1, 2}\cycle{3, 4}, \cycle{1, 3}\cycle{2, 4},
    \cycle{1, 4}\cycle{2, 3}\}$ una copia di $V_4$ in $\mathcal{A}_5$,
    il normalizzatore di $P$ in $\mathcal{A}_5$ contiene necessariamente 
    il sottogruppo 
    \[
        \St(5) = \{\sigma \in \mathcal{A}_5\mid \sigma(5) = 5\} \cong \mathcal{A}_4\footnote{
        Qua stiamo considerando l'azione naturale di $\mathcal{A}_5$ 
        sull'insieme $\{1, 2, 3, 4, 5\}$.}
    \]
    in quanto $V_4$ è un sottogruppo normale di $\mathcal{A}_4$, quindi 
    $|N_{\mathcal{A}_5}(P)| \in \{12, 60\}$. D'altra parte $|N_{\mathcal{A}_5}(P)| \neq 60$,
    altrimenti $\mathcal{A}_5$ conterrebbe un unico 2-Sylow, che sarebbe quindi
    un sottogruppo normale non banale, che è assurdo in quanto $\mathcal{A}_5$
    è semplice. Allora $|N_{\mathcal{A}_5}(P)| = 12$, cioè $n_2 = 5$.
\end{proof}

\begin{proposition}
    Se $G$ è un gruppo semplice di ordine $60$ allora è isomorfo a $\mathcal{A}_5$.
\end{proposition}

\begin{proof}
    Dalle condizioni
    \[
        \begin{cases}
            n_2 \equiv 1 \pmod 2\\
            n_2 \mid 60
        \end{cases}
        \quad
        \begin{cases}
            n_3 \equiv 1 \pmod 3\\
            n_3 \mid 60
        \end{cases}
        \quad
        \begin{cases}
            n_5 \equiv 1 \pmod 5\\
            n_5 \mid 60
        \end{cases}
    \]
    otteniamo $n_2 \in \{1, 3, 5, 15\}$, $n_3 = \{1, 4, 10\}$, $n_5 = \{1, 6\}$.
    Poiché $G$ è semplice, $n_2$, $n_3$ e $n_5$ sono tutti diversi da 1,
    altrimenti $G$ conterrebbe un sottogruppo caratteristico, quindi normale,
    non banale. Distinguiamo tre casi:
    \begin{itemize}
        \item supponiamo per assurdo $n_2 = 3$, posto $X$ l'insieme dei 2-Sylow di $G$
        consideriamo l'azione di coniugio di $G$ su $X$
        \[
            \varphi: G \longrightarrow S(X) \cong S_3
        \]
        poiché i 2-Sylow sono tutti coniugati e $G$ è semplice tale azione è 
        transitiva, pertanto $\ker\varphi \neq G$. Allora $\ker\varphi = \{e\}$ 
        in quanto $\ker\varphi\triangleleft G$,
        ma questo è assurdo dato che $|G| > |S_3|$;
        \item supponiamo $n_2 = 5$, posto $X$ l'insieme dei 2-Sylow di $G$
        consideriamo l'azione di coniugio di $G$ su $X$
        \[
            \varphi: G \longrightarrow S(X) \cong S_5
        \]
        argomentando come sopra si ha che tale azione è transitiva, pertanto
        $\ker\varphi \neq G$. Allora $\ker\varphi = \{e\}$ in quanto 
        $\ker\varphi \triangleleft G$, cioè $\varphi$ è un omomorfismo iniettivo
        e $G$ è isomorfo a un sottogruppo $H\leqslant S_5$ di indice 2. Consideriamo
        l'intersezione $H \cap \mathcal{A}_5$, per la \hyperref[prop1.49]{Proposizione 1.49}
        allora $[\mathcal{A}_5:H\cap\mathcal{A}_5] \in \{1, 2\}$. D'altra parte
        se fosse 2 allora $H\cap\mathcal{A}_5$ sarebbe un sottogruppo normale di $\mathcal{A}_5$
        non banale, che è assurdo, pertanto l'indice di $H$ è 1, cioè $H = \mathcal{A}_5$.
        Quindi $G$ è isomorfo a $\mathcal{A}_5$;
        \item supponiamo per assurdo $n_2 = 15$, notiamo che due 2-Sylow distinti
        di $G$ si intersecano banalmente 
        o in un sottogruppo isomorfo a $\Z2$\footnote{
            Questo perché la massima potenza di 2 che divide 60 è 4, pertanto
            un 2-Sylow di $G$ è isomorfo a $\Z4$ oppure a $\Z2\times \Z2$.
        }. Se tutti i 2-Sylow di $G$ si intersecassero banalmente allora la 
        loro unione conterrebbe $1 + 3\cdot 15 = 46$ elementi, poiché l'unione dei
        5-Sylow di $G$ contribuisce con $4\cdot 6 = 24$ elementi di ordine 5,
        ma allora $G$ non conterrebbe elementi di ordine 3, che è assurdo.
        Siano quindi $S_1$ e $S_2$ 2-Sylow distinti di $G$ tali che 
        $H = S_1\cap S_2 \cong \Z2$, consideriamo il normalizzatore $N_G(H)$.
        Osserviamo che $S_1$ e $S_2$ sono sottogruppi di $N_G(H)$ in quanto, essendo
        abeliani, $H$ è un sottogruppo normale di entrambi, pertanto $|N_G(H)| > 4$.
        D'altra parte poiché tale ordine deve dividere 60 abbiamo 
        $|N_G(H)| \in \{12, 20\}$, infatti se fosse uguale a 60 $H$ sarebbe un 
        sottogruppo normale non banale di $G$, che non è possibile in quanto $G$
        è semplice. Inoltre $|N_G(H)| \neq 20$ in quanto si avrebbe $[G:N_G(H)] = 3$,
        allora per il \hyperref[teorema1.50]{Teorema 1.50} $G$ conterrebbe
        un sottogruppo normale non banale di indice al più $3!$, che è assurdo.
        Abbiamo quindi $|N_G(H)| = 12$, consideriamo l'azione di 
        moltiplicazione a sinistra di $G$ sull'insieme quoziente $\faktor G H$
        \[
            \varphi : G\longrightarrow S\left(\faktor{G}{N_G(H)}\right)\cong S_5
        \]
        argomentando come sopra si ha che tale azione è transitiva, pertanto
        $\ker\varphi \neq G$. Allora $\ker\varphi = \{e\}$ in quanto 
        $\ker\varphi\triangleleft G$, cioè $\varphi$ è un omomorfismo iniettivo
        e si mostra come sopra che $G \cong \mathcal{A}_5$, ma questo è assurdo 
        in quanto $\mathcal{A}_5$ contiene 5 2-Sylow.
    \end{itemize}
\end{proof}

\newpage

\subsection{Studio di $SL_2(\FF_3)$}

Consideriamo il gruppo $GL_2(\FF_3)$, ricordiamo che il determinante è 
un omomorfismo di gruppi surgettivo
\[
    \det: GL_2(\FF_3) \longrightarrow \FF_3^*
\]
e che il suo nucleo è il gruppo $SL_2(\FF_3) = \{M \in GL_2(\FF_3)\mid \det M = 1\}$,
che è quindi un sottogruppo normale di $GL_2(\FF_3)$. Inoltre, poiché
$\FF_3^* \cong \Z2$ abbiamo che $SL_2(\FF_3)$ ha indice 2 in $GL_2(\FF_3)$, pertanto 
$|SL_2(\FF_3)| = 24$ in quanto $|GL_2(\FF_3)| = (3^2 - 1)(3^2 - 3) = 48$.
\\\\
Consideriamo quindi il gruppo $S = SL_2(\FF_3)$, dalle condizioni 
\[
    \begin{cases}
        n_3 \equiv 1 \pmod 3\\
        n_3 \mid 24
    \end{cases}
\]
otteniamo $n_3 \in \{1, 4\}$, notiamo però che $S$ non può contenere un unico 
3-Sylow in quanto questi sono isomorfi a $\Z3$ e le matrici 
\[
    \begin{pmatrix}
        1 & 1\\
        0 & 1
    \end{pmatrix}\qquad
    \begin{pmatrix}
        1 & 0\\
        1 & 1
    \end{pmatrix}
\]
hanno ordine 3 e i gruppi che generano sono distinti. In particolare $S$
contiene almeno 2 3-Sylow, pertanto ne contiene esattamente 4.
Calcoliamo il centro di $S$ imponendo la commutazione sulle matrici appena 
esibite. Dall'equazione 
\[
    \begin{pmatrix}
        1 & 1\\
        0 & 1
    \end{pmatrix}
    \begin{pmatrix}
        a & b\\
        c & d
    \end{pmatrix} = 
    \begin{pmatrix}
        a & b\\
        c & d
    \end{pmatrix}
    \begin{pmatrix}
        1 & 1\\
        0 & 1
    \end{pmatrix}
\]
otteniamo 
\[
    \begin{pmatrix}
        a + c & b + d\\
        c & d
    \end{pmatrix} = 
    \begin{pmatrix}
        a & a + b\\
        c & c + d
    \end{pmatrix}
\]
da cui $c = 0$ e $a = d$. In modo analogo dall'equazione
\[
    \begin{pmatrix}
        1 & 0\\
        1 & 1
    \end{pmatrix}
    \begin{pmatrix}
        a & b\\
        0 & a
    \end{pmatrix} = 
    \begin{pmatrix}
        a & b\\
        0 & a
    \end{pmatrix}
    \begin{pmatrix}
        1 & 0\\
        1 & 1
    \end{pmatrix}
\]
otteniamo 
\[
    \begin{pmatrix}
        a & b\\
        a & a + b
    \end{pmatrix} = 
    \begin{pmatrix}
        a + b & b\\
        a & a
    \end{pmatrix}
\]
da cui $b = 0$, pertanto un generico elemento del centro è della forma
$
    \begin{pmatrix}
        a & 0\\
        0 & a
    \end{pmatrix}
$
d'altra parte il suo determinante deve essere uguale a 1, quindi $Z(S) = 
\left\{\begin{pmatrix}
    1 & 0\\
    0 & 1
\end{pmatrix}, \begin{pmatrix}
    -1 & 0\\
    0 & -1
\end{pmatrix}\right\}$. Utilizziamo questo fatto per determinare la classe
di isomorfismo del normalizzatore di un 3-Sylow di $S$. \newline 
Fissiamo $P$ un 3-Sylow, poiché $n_3 = [S: N_S(P)]$ abbiamo 
\[
    |N_S(P)| = \frac{|S|}{n_3} = 6
\]
inoltre $Z(S)$ e $P$ sono sottogruppi di $N_S(P)$. Notiamo che $N_S(P)$
contiene un elemento di ordine 3 e un elemento di ordine 2 che commutano, 
ad esempio il generatore di $P$ e il generatore di $Z(S)$,
pertanto contiene un elemento di ordine 6, il loro prodotto, da cui 
$N_S(P) = PZ(S) \cong \Z6$.
\\\\
Posto $X$ l'insieme dei 3-Sylow di $S$, consideriamo
l'azione transitiva di coniugio di $S$ su $X$
\[
    \Phi: S \longrightarrow S(X) \cong S_4
\]
il nucleo di $\Phi$ è
\begin{multline*}
    \ker\Phi = \{g \in S \mid gPg^{-1} = P~\forall P \in X\} = \\
    = \{g \in S\mid g \in N_S(P)~\forall P \in X\} = \\
    = \bigcap_{P \in X}N_S(P) = \bigcap_{P \in X}PZ(S) = Z(S)
\end{multline*}
dove l'ultima uguaglianza è giustificata dal fatto che i 3-Sylow di $S$
si intersecano banalmente. Per il Primo Teorema di Omomorfismo otteniamo
che $\mathrm{Im}\Phi \cong \faktor{S}{Z(S)}$, che ha cardinalità 12. D'altra
parte $\mathrm{A}_4$ è l'unico sottogruppo di $S_4$ con 12 elementi, pertanto
$\faktor{S}{Z(S)} \cong \mathcal{A}_4$, sfruttiamo questo fatto per studiare 
i 2-Sylow di $S$. Per il Teorema di Corrispondenza i sottogruppi di $S$
contenenti $Z(S)$ sono in bigezione con i sottogruppi di $\mathcal{A}_4$,
e tale bigezione preserva l'indice e la normalità dei sottogruppi. Poiché 
$V_4$ è l'unico 2-Sylow di $\mathcal{A}_4$ abbiamo che $S$ contiene un unico 2-Sylow 
di indice $3$, cioè di cardinalità 8, chiamiamo $J$ tale sottogruppo. 
$J$ contiene le matrici
\[
    i = \begin{pmatrix}
        0 & -1\\
        1 & 0
    \end{pmatrix}\qquad
    j = \begin{pmatrix}
        1 & 1\\
        1 & -1
    \end{pmatrix}\footnote{
        Il determinante di questa matrice è $-2$, che è uguale a 1 in $\FF_3$
    }
\]
entrambe di ordine 4, inoltre
\[
    ij = 
    \begin{pmatrix}
        0 & -1\\
        1 & 0
    \end{pmatrix}
    \begin{pmatrix}
        1 & 1\\
        1 & -1
    \end{pmatrix} = 
    \begin{pmatrix}
        -1 & 1\\
        1 & 1
    \end{pmatrix}
\]
\[
    ji = \begin{pmatrix}
        1 & 1\\
        1 & -1
    \end{pmatrix}
    \begin{pmatrix}
        0 & -1\\
        1 & 0
    \end{pmatrix} = 
    \begin{pmatrix}
        1 & -1\\
        -1 & -1
    \end{pmatrix}
\]
pertanto $ij = -ji$. Quindi $J$ è un gruppo di ordine $8$ che contiene due
elementi di ordine 4 che anticommutano, in particolare ha la seguente presentazione
\[
    J = \langle i, j\mid i^4 = j^4 = 1, i^2 = -1, ij = -ji\rangle
\]
quindi è isomorfo a $Q_8$. Osserviamo che il sottogruppo derivato $S'$ è 
contenuto in $J$ in quanto il quoziente $\faktor{S}{J}$ è abeliano (in particolare
è isomorfo a $\Z3$), mostriamo che effettivamente vale l'uguaglianza. 
Sicuramente $S'$ non è il sottogruppo formato dalla sola identità in quanto 
$S$ non è abeliano, inoltre $S'$ deve necessariamente contenere un elemento di 
ordine 2 in quanto sottogruppo non banale di $J$, quindi $Z(S) \subseteq S'$\footnote{
    Infatti l'unico elemento di ordine 2 di $S$ è $\begin{pmatrix}-1 & 0\\
    0 & -1\end{pmatrix}$.
}. Inoltre $Z(S) \neq S'$ in quanto il quoziente è isomorfo a $\mathcal{A}_4$,
pertanto $S'$ ha ordine 4 oppure 8, cioè $[S: S'] \in \{3, 6\}$.
Consideriamo l'omomorfismo surgettivo 
\[
    \varphi: S \longrightarrow \mathcal{A}_4
\]
dato dalla composizione della proiezione su $\faktor{S}{Z(S)}$ con l'isomorfismo
tra il quoziente e $\mathcal{A}_4$, per il Teorema di Corrispondenza $\varphi(S')$
è un sottogruppo normale di $\mathcal{A}_4$ con 
$[\mathcal{A}_4: \varphi(S')] = [S: S']$. D'altra parte un sottogruppo di 
indice 6 di $\mathcal{A}_4$ è della forma $\{id, \cycle{a, b}\cycle{c, d}$\}
con $\cycle{a, b}$ e $\cycle{c, d}$ trasposizioni disgiunte, che non è normale
in $\mathcal{A}_4$, pertanto $\varphi(S')$ ha indice 3 e quindi $S'$ ha ordine
8, da cui $S' = J$.


\newpage

\section{Anelli}

A meno di ulteriori specificazioni, gli anelli che tratteremo saranno sempre 
anelli commutativi con identità.

\subsection{Interpolazione polinomiale via TCR}

Mostriamo il seguente enunciato di interpolazione utilizzando il Teorema
Cinese del Resto.

\begin{proposition}
    Siano $a_1, \ldots, a_n \in \QQ$ valori distinti e $b_1, \ldots, b_n \in \QQ$,
    allora esiste un unico polinomio $p(x) \in \QQ[x]$ di grado al più $n - 1$
    tale che $p(a_i) = b_i$ per ogni $i \in \{1, \ldots, n\}$. 
\end{proposition}

\begin{proof}
    Posto $I_i = (x - a_i)$ per $i \in \{1, \ldots, n\}$, osserviamo che 
    $I_i + I_j = \QQ[x]$ per $i \neq j$, infatti $a_i - a_j \in I_i + I_j$, 
    che è un elemento invertibile di $\QQ[x]$. Per il Teorema Cinese del Resto
    abbiamo quindi
    \[
        \frac{\QQ[x]}{I_1\ldots I_n} \cong \faktor{\QQ[x]}{I_1}\times 
        \ldots \times \faktor{\QQ[x]}{I_n}
    \]
    tramite l'isomorfismo
    \[
        \Phi: \frac{\QQ[x]}{I_1\ldots I_n} \longrightarrow \faktor{\QQ[x]}{I_1}\times 
        \ldots \times \faktor{\QQ[x]}{I_n}: \overline{p(x)}\longmapsto
        (p(x) + I_1, \ldots, p(x) + I_n)
    \]
    abbiamo inoltre che $\faktor{\QQ[x]}{I_i} \cong \QQ$ per ogni $i \in \{1, \ldots, n\}$
    tramite l'isomorfismo
    \[
        \Psi: \faktor{\QQ[x]}{I_i} \longrightarrow \QQ: p(x) + I_i \longmapsto p(a_i)
    \]
    La composizione di questi due risulta in un isomorfismo
    \[
        \Psi\circ\Phi: \frac{\QQ[x]}{I_1\ldots I_n}\longrightarrow \QQ^n:
        \overline{p(x)}\longmapsto (p(a_1), \ldots, p(a_n))
    \]
    in particolare per ogni $n$-upla di razionali $(b_1, \ldots, b_n)$ esiste
    un unico\footnote{
        L'unicità deriva dal fatto che gli elementi di $\displaystyle\frac{\QQ[x]}{I_1\ldots I_n}$
        possono essere rappresentati in modo unico da polinomi a coefficienti 
        razionali di grado al più $n - 1$.
    } polinomio $p \in \QQ[x]$ con $\deg p \leq n - 1$ tale che 
    $p(a_i) = b_i$ per ogni $i \in \{1, \ldots, n\}$.
\end{proof}

\begin{remark}
    Con la dimostrazione data l'enunciato è vero su ogni campo con almeno 
    $n$ elementi distinti. Più in generale è vero in ogni anello con almeno $n$
    elementi distinti a patto di scegliere $a_1, \ldots, a_n$ tali che $a_i - a_j$
    sia invertibile per ogni $i \neq j$.
\end{remark}

\newpage

\subsection{Localizzazione di $\ZZ$ rispetto a un ideale primo}

Sia $P = (p)$ un ideale primo non nullo di $\ZZ$, consideriamo $S = \ZZ \setminus P$.
$S$ è una parte moltiplicativa di $\ZZ$, infatti
\begin{itemize}
    \item $0 \notin S$ in quanto $0 \in P$;
    \item $1 \in S$ in quanto $0 \notin P$;
    \item per ogni $x, y \in S$ vale $xy \in S$, infatti se $x, y \notin P$
    allora $xy \notin P$ poiché $P$ è un ideale primo.
\end{itemize}

Per quanto già visto, sappiamo che $S^{-1}\ZZ$ è un anello contenente
un unico ideale massimale, detto anche \vocab{anello locale}. Più precisamente,
tale ideale è $S^{-1}\ZZ \setminus (S^{-1}\ZZ)^*$ e il gruppo degli elementi
invertibili è
\[
    (S^{-1}\ZZ)^* = \left\{\frac a s \Bigm| a, s \in S\right\}
\] 
Inoltre, gli ideali di $S^{-1}\ZZ$
sono tutti della forma $S^{-1}(m)$ con $(m)\subseteq \ZZ$ un ideale, e vale 
\[
    S^{-1}(m) = \left\{\frac{mk}{s} \Bigm|k \in \ZZ, s \in S\right\} = 
    \left\{m\frac k s\Bigm| \frac k s \in S^{-1}\ZZ\right\} = (m)S^{-1}\ZZ
\]
Vediamo un esempio esplicito per $P = (2)$, descrivendo esplicitamente gli 
ideali di $S^{-1}\ZZ$. 
\[
    S^{-1}\ZZ = \left\{\frac m s \Bigm| s \text{ dispari}\right\}
\]
Per prima cosa osserviamo che la corrispondenza tra 
gli ideali di $\ZZ$ e quelli di $S^{-1}\ZZ$ non è biunivoca, ma solo surgettiva.
Infatti alcuni ideali di $\ZZ$ diventano uguali quando localizziamo l'anello
rispetto a $S$, in particolare
\[
    S^{-1}(m) = S^{-1}(n) \iff \exists u \in (S^{-1}\ZZ)^* \text{ tale che }
    m = un \iff u = \frac m n \in (S^{-1}\ZZ)^*
\]
dove l'ultima uguaglianza è giustificata dal fatto che $S^{-1}\ZZ$ è un 
sottoanello di $\QQ$, pertanto esiste $\displaystyle\frac m n$ come numero
razionale ed è l'unico valore per cui l'equazione è verificata. D'altra parte
abbiamo 
\[
    (S^{-1}\ZZ)^* = \left\{\frac m s \Bigm| m, s \in S\right\} = 
    \left\{\frac m n \Bigm| m, n\text{ entrambi dispari}\right\}
\]
pertanto $S^{-1}(m) = S^{-1}(n)$ se e solo se la massima potenza di due che 
divide $m$ e $n$ è la stessa\footnote{
    In tal caso infatti il razionale $\frac m n$, se ridotto ai 
    minimi termini, ha numeratore e denominatore entrambi dispari, quindi
    è un'unità dell'anello.
}.
Gli ideali di $S^{-1}\ZZ$ sono quindi tutti e soli quelli della forma $S^{-1}(2^k)$.
Consideriamo la bigezione
\[
    \{\text{Ideali primi di }S^{-1}\ZZ\}\longleftrightarrow 
    \{\text{Ideali primi } P\subseteq \ZZ \mid P\cap S = \emptyset\}
\]
poiché gli unici ideali primi di $\ZZ$ che non intersecano $S = \ZZ\setminus(2)$
sono $(0)$ e $(2)$, abbiamo che gli unici ideali primi di $S^{-1}\ZZ$ sono
$(0)$ e $S^{-1}(2)$.

\newpage

\subsection{Ideali massimali e primi di $\ZZ[x]$}

\begin{lemma}
    \label{lemma2.3}
    Se $A \subseteq R$ sono due anelli e $P \subseteq R$ è un ideale primo
    di $R$ allora $P\cap A$ è un ideale primo di $A$.
\end{lemma}

\begin{proof}
    $P\cap A$ è un ideale di $A$ in quanto controimmagine di $P$ tramite
    l'omomomorfismo di anelli
    \[
        \varphi: A \longhookrightarrow R: a \longmapsto a
    \]
    Poiché $P$ è un ideale primo di $R$, per ogni $a, b \in A$ tali che 
    $ab \in P\cap A$ si ha $a \in P$
    oppure $b \in P$, cioè $a \in P \cap A$ oppure $b \in P\cap A$, quindi
    $P \cap A$ è un ideale primo di $A$.
\end{proof}


Consideriamo $P \subseteq \ZZ[x]$ un ideale primo, studiamo l'intersezione
$P \cap \ZZ$. Questo è un ideale primo di $\ZZ$ per il 
\hyperref[lemma2.3]{Lemma 2.3}, pertanto $P\cap \ZZ = (0)$ oppure esiste un
primo $p \in \ZZ$ tale che $P\cap\ZZ = (p)$. Se non è l'ideale nullo allora
$(p)\ZZ[x]$ è un ideale contenuto in $P$, per il Teorema di 
Corrispondenza gli ideali primi di $\ZZ[x]$ contenenti $(p)\ZZ[x]$ sono in 
bigezione con gli ideali primi del quoziente $\faktor{\ZZ[x]}{(p)\ZZ[x]} \cong \FF_p[x]$
e vale la stessa cosa per gli ideali massimali. 
Poiché $\FF_p[x]$ è un dominio a ideali principali, i suoi ideali primi 
sono $(\overline{0})$ e quelli generati da un polinomio irriducibile, in particolare
tutti i suoi ideali primi non nulli sono anche massimali. Pertanto se $\overline{f(x)} \in \FF_p[x]$ è un polinomio
irriducibile allora $(\overline{f(x)})$ è un ideale primo di $\FF_p[x]$
e quindi $(p, f(x))$ è un ideale primo e massimale di $\ZZ[x]$. Abbiamo quindi che 
l'insieme degli ideali massimali di $\ZZ[x]$ contenenti $p$ è
\[
    \mathcal{M}_p = \{(p, f(x))\mid \overline{f(x)}\text{ è irriducibile in }
    \FF_p[x]\}
\]
mentre l'insieme degli ideali primi di $\ZZ[x]$ contenenti $p$ è
\[
    \mathcal{P}_p = \mathcal{M}_p \cup (p)\ZZ[x]
\]

Supponiamo adesso che $P$ sia un ideale primo di $\ZZ[x]$ tale che 
$P \cap \ZZ = 0$. $S = \ZZ\setminus \{0\}$ è una parte moltiplicativa di 
$\ZZ$ e l'ipotesi appena data su $P$ può essere espressa come $P \cap S = \emptyset$.
Consideriamo la bigezione 
\[
    \{\text{Ideali primi di }S^{-1}\ZZ[x]\}\longleftrightarrow 
    \{\text{Ideali primi } P\subseteq \ZZ[x] \mid P\cap S = \emptyset\}: 
    \mathfrak{P}\longmapsto \mathfrak{P}\cap\ZZ[x]
\]
poiché $S^{-1}\ZZ[x] = \QQ[x]$ abbiamo che $P$ corrisponde a un unico ideale
primo di $\QQ[x]$. Essendo $\QQ[x]$ un dominio a ideali principali, questi sono
l'ideale nullo e gli ideali generati da polinomi irriducibili. Se $f(x)\in \QQ[x]$
è un polinomio irriducibile il cui ideale corrisponde a $P$ allora, posto 
$m$ il minimo comune denominatore dei suoi coefficienti, abbiamo che 
$P = (f(x))\QQ[x] \cap \ZZ[x] = (mf(x))\QQ[x]\cap\ZZ[x] = (mf(x))\ZZ[x]$. 
In particolare $P$ è generato da un polinomio primitivo irriducibile.
Gli ideali primi di $\ZZ[x]$ possono quindi avere la seguente forma:
\begin{itemize}
    \item $(0)$;
    \item $(p)\ZZ[x]$ con $p \in \ZZ$ primo;
    \item $(p, f(x))$ con $p \in \ZZ$ primo e $\overline{f(x)}$ irriducibile
    in $\FF_p[x]$;
    \item $(f(x))$ con $f(x)$ primitivo e irriducibile in $\ZZ[x]$.
\end{itemize}

Mostriamo che gli ideali primi di quest'ultimo tipo non sono massimali.\newline
Siano $f(x) \in \ZZ[x]$ un polinomio primitivo, irriducibile, non costante,
$a \in \ZZ$ tale che $f(a) \notin\{-1, 0, 1\}$, $p \in \ZZ$ un primo che 
divide $f(a)$ e consideriamo le applicazioni
\[
    \varphi: \ZZ[x]\longrightarrow \faktor{\ZZ[x]}{(p)\ZZ[x]}: 
    g(x)\longmapsto \overline{g(x)}
\] 
\[
    \psi: \faktor{\ZZ[x]}{(p)\ZZ[x]}\longrightarrow \FF_p: 
    \overline{g(x)}\longmapsto g(a)
\]
Osserviamo che $(\psi\circ\varphi)(f(x)) = f(a) \equiv 0 \pmod{p}$ e che 
$(\psi\circ\varphi)(p) = p \equiv 0 \pmod{p}$, pertanto $p, f(x) \in 
\ker{\psi\circ\varphi}$ e quindi $(p, f(x)) \subseteq \ker(\psi\circ \varphi) \neq \ZZ[x]$.
Abbiamo quindi $(f(x)) \subseteq (p, f(x))$, se $(f(x))$ fosse massimale allora
conterrebbe $p$, che è assurdo in quanto $\deg f \geq 1$.

Poiché gli ideali di questo tipo non sono massimali, gli ideali massimali di
$\ZZ[x]$ sono tutti e soli quelli della forma
\[
    (p, f(x)) \text{ con }\overline{f(x)} \text{ irriducibile in }\FF_p[x]
\]

\newpage

\subsection{Criterio di Eisenstein}

Conosciamo il Criterio di Eisenstein per verificare che un polinomio a coefficienti
interi è irriducibile in $\ZZ[x]$. Lo stesso risultato vale in generale in 
ogni anello UFD con praticamente la stessa dimostrazione, che ricordiamo.

\begin{proposition}
    [Criterio di Eisenstein]
    Siano $A$ un UFD, $p \in A$ un elemento primo e $f(x) = \displaystyle\sum_{i = 0}^n a_ix^i$
    un polinomio a coefficienti in $A$ se sono verificate le ipotesi
    \begin{itemize}
        \item $p \nmid a_n$;
        \item $p \mid a_i$ per $i \in \{0, \ldots, n - 1\}$;
        \item $p^2 \nmid a_0$;
    \end{itemize}
    allora $f(x)$ è irriducibile in $A[x]$.
\end{proposition}

%\begin{proof}
%    Supponiamo per assurdo che $f(x)$ sia riducibile, allora possiamo scrivere $f(x) = g(x)h(x)$
%    con $g(x), h(x) \in A[x]$, $\deg g = m \geq 1$, $\deg h = n - m \geq 1$. 
%    Consideriamo l'omomorfismo di proiezione
%    \[
%        \pi: A[x] \longrightarrow \faktor{A}{(p)}[x]: 
%        \sum_{j = 0}^k\alpha_ix^i \longmapsto \sum_{j = 0}^k \overline{\alpha_i}x^i
%    \]
%    abbiamo che 
%    \[
%        \pi(f(x)) = \overline{a_n}x^n \neq \overline{0} \qquad\pi(f(x)) = \pi(g(x))\pi(h(x))
%    \]
%    Scriviamo 
%    \[
%        g(x) = \sum_{i = 0}^m b_ix^i\qquad h(x) = \sum_{i = 0}^{n - m}c_ix^i
%    \]
%    poiché $A$ è $UFD$ lo è anche $A[x]$, pertanto le proiezioni di $g(x)$
%    e di $h(x)$ sono necessariamente della forma
%    \[
%        \pi(g(x)) = \overline{b_m}x^m\qquad \pi(h(x)) = \overline{c_{n - m}}x^{n - m}
%    \]
%    da cui $\overline{b_0} = \overline{c_0} = \overline{0}$, cioè $p \mid b_0$ e
%    $p \mid c_0$. D'altra parte $a_0 = b_0c_0$ e quindi $p^2 \mid a_0$, che 
%    è assurdo.
%\end{proof}

\newpage

\subsection{Domini a ideali principali}

\begin{proposition}
    Sia $A$ un PID, ogni ideale primo diverso da $(0)$ di $A$ è un ideale 
    massimale
\end{proposition}

\begin{proof}
    Sia $P = (p)$ un ideale primo non nullo, supponiamo per assurdo che esista
    un ideale $M$ tale che 
    \[
        P \subsetneq M \subsetneq A
    \]
    Poiché $A$ è un dominio a ideali principali, esiste $x \in A$ tale che 
    $M = (x)$, quindi $x \mid p$ dato che $P \subsetneq M$. Poiché $P$ è 
    un ideale primo si ha che $p \in A$ è un elemento primo, quindi 
    irriducibile. Sia $q \in A$ tale che $p = xq$, dato che $x \notin A^*$ abbiamo che 
    $q \in A^*$, cioè $(x) = (p)$, che è assurdo.
\end{proof}

\begin{corollary}
    Siano $A$ un PID e $B$ un dominio di integrità e $\varphi: A \longrightarrow B$
    un omomorfismo di anelli surgettivo, allora $\varphi$ è un isomorfismo
    oppure $B$ è un campo. 
\end{corollary}

\begin{proof}
    Notiamo che $\ker \varphi$ è un ideale primo di $A$ in quanto $\faktor{A}{\ker\varphi}
    \cong B$ è un dominion di integrità. Se $\ker\varphi = (0)$, allora 
    $\varphi$ è un isomorfismo di anelli. Altrimenti $\ker\varphi$ è un ideale
    massimale, pertanto $\faktor{A}{\ker\varphi}\cong B$ è un campo.
\end{proof}

\begin{corollary}
    Se $C$ è un anello tale che $C[x]$ è un PID, allora $C$ è un campo.
\end{corollary}

\begin{proof}
    Dall'inclusione $C \subseteq C[x]$ abbiamo che $C$ è un dominio di integrità.
    L'ideale $(x)$ è quindi primo in $C[x]$ in quanto $\faktor{C[x]}{(x)} \cong C$
    è un dominio, quindi è massimale dato che $C[x]$ è un PID. Pertanto 
    $C$ è un campo.
\end{proof}

\newpage

\subsection{Operazioni tra ideali}

Ricordiamo che in un anello commutativo con identità $A$, sono ben definite
le seguenti operazioni su due ideali $I, J$ e danno luogo a un terzo ideale
(possibilmente uguale a uno dei due):
\begin{itemize}
    \item $I\cap J = \{k \in A \mid k \in I, k \in J\}$;
    \item $I + J = (I, J) = \{i + j \mid i \in I, j \in J\}$;
    \item $IJ = (\{ij\mid i \in I, j \in J\})$;
    \item $\sqrt{I} = \{x \in A \mid \exists n \in \NN \text{ per cui } x^n \in I\}$;
    \item $(I : J) = \{x \in A \mid xJ \subseteq I\}$.
\end{itemize}

\begin{proposition}
    Dati $A$ un anello commutativo con identità, $I, J \subseteq A$ due ideali,
    allora
    \[
        \sqrt{IJ} = \sqrt{I\cap J} = \sqrt{I}\cap \sqrt{J}
    \]
\end{proposition}

\begin{proof}
    Poiché vale l'inclusione $IJ \subseteq I\cap J$ abbiamo che $\sqrt{IJ}
    \subseteq \sqrt{I\cap J}$. Viceversa, se $a \in \sqrt{I\cap J}$ allora
    esiste $n \in \NN$ tale che $a^n \in I\cap J$, allora abbiamo 
    \[
        a^{2n} = \underset{\in I}{\underbrace{a^n}} \cdot \underset{\in J}{\underbrace{a^n}}
        \in IJ
    \]
    da cui $\sqrt{I\cap J} \subseteq \sqrt{IJ}$ e quindi l'uguaglianza.\newline
    Consideriamo adesso $b \in \sqrt{I}\cap \sqrt{J}$, allora esistono $m, n \in \NN$
    tali che $b^m \in I$ e $b^n \in J$, da cui
    \[
        b^{m + n} = \underset{\in I}{\underbrace{b^m}}\cdot
        \underset{\in J}{\underbrace{b^n}} \in I\cap J
    \]
    Pertanto $\sqrt{I}\cap \sqrt{J} \subseteq \sqrt{I\cap J}$. Viceversa, Se
    $c \in \sqrt{I\cap J}$ allora esiste $n \in \NN$ tale che $c^n \in I\cap J$, 
    in particolare $c^n \in I$ e $c^n \in J$, quindi $c \in \sqrt{I}\cap\sqrt{J}$,
    da cui l'uguaglianza.
\end{proof}

\begin{proposition}
    \label{prop2.9}
    Dato $A$ un anello commutativo con identità, allora
    \[
        \sqrt{(0)} = \bigcap_{\substack{P\subseteq A\\ P \text{ ideale primo}}}P
    \]
\end{proposition}

\begin{proof}
    Sia $X$ l'intersezione di tutti gli ideali primi di $A$. 
    Se $x \in \sqrt{(0)}$ allora esiste $n \in \NN$ tale che $x^n = 0$, procediamo
    per induzione su $n$. Se $n = 1$ allora $x = 0$, quindi $x$ è contenuto 
    in tutti gli ideali di $A$, in particolare in quelli primi. Se $n > 1$, 
    supponiamo che se $x^{n - 1} \in X$ allora $x \in X$.
    Per ogni ideale primo $P$, poiché $x^n = 0$ si ha che $x^n$ è contenuto nella
    loro intersezione, da cui almeno uno tra $x$ e $x^{n - 1}$ è un elemento
    di $X$. Se $x^{n - 1} \in X$ allora $x \in X$ per ipotesi induttiva, pertanto 
    $\sqrt{(0)} \subseteq X$. Viceversa, mostriamo che se $x \notin \sqrt{(0)}$
    allora esiste un ideale primo $P$ tale che $x \notin P$. Consideriamo 
    l'insieme
    \[
        \mathcal{F} = \{I \subseteq A \mid I \text{ ideale}, x^n \notin I
        ~\forall n \in \NN\}
    \]
    notiamo che $\mathcal{F}$ è non vuoto in quanto contiene l'ideale nullo.
    Posta $\mathscr{C} = \{I_i\}$ una catena di ideali tali che $I_i \subseteq I_{i + 1}$,
    sia
    \[
        \mathcal{I} = \bigcup I_i
    \]
    Per costruzione $\mathcal{I}$ è un maggiorante per $\mathscr{C}$ in quanto
    ogni $I_i$ è contenuto in $\mathcal{I}$, inoltre $\mathcal{I}$
    è un ideale di $A$ dato che $I_i \subseteq I_{i + 1}$. L'ideale $\mathcal{I}$
    è un elemento di $\mathcal{F}$, infatti se le potenze di $x$ non sono elementi
    degli ideali di $\mathcal{F}$, a maggior ragione non sono contenute in $\mathcal{I}$.
    Pertanto ogni catena $\mathscr{C}$ di $\mathcal{F}$ ammette un maggiorante 
    in $\mathscr{C}$, pertanto per il Lemma di Zorn esiste un ideale $M$ massimale
    in $\mathcal{F}$. Mostriamo che $M$ è un ideale primo di $A$. Siano $a, b \in A$
    tali che $ab \in M$, supponiamo per assurdo $a \notin M$ e $b \notin M$,
    allora $M$ è contenuto strettamente negli ideali $(M, a), (M, b)$. Poiché
    $M$ è massimale in $\mathcal{F}$, esistono $h, k \in \NN$ tali che $x^k \in (M, a)$
    e $x^h \in (M, b)$, da cui
    \[
        x^{h + k} \in (M, a)(M, b) \subseteq M
    \]
    che è assurdo in quanto $M$ è un elemento di $\mathcal{F}$. Pertanto 
    $M$ è un ideale primo di $A$ che non contiene nessuna potenza di $x$, 
    da cui segue la tesi.
\end{proof}

\begin{corollary}
    Dati $A$ un anello commutativo con identità e $I\subseteq A$ un ideale, 
    allora
    \[
        \sqrt{I} = \bigcap_{\substack{P\supseteq I\\ P \subseteq A\text{ ideale primo}}}P
    \]
\end{corollary}

\begin{proof}
    Consideriamo l'omomorfismo di proiezione
    \[
        \pi: A\longrightarrow \faktor A I
    \]
    osserviamo che $\sqrt{I} = \pi^{-1}(\sqrt{(0)})$, dove $\sqrt{(0)}$ è il radicale di 
    $0$ in $\faktor A I$. Per la \hyperref[prop2.9]{Proposizione 2.9}
    abbiamo
    \[
        \sqrt{I} = \pi^{-1}(\sqrt{(0)}) = \pi^{-1}\left(\bigcap_{\substack{
            P\subseteq \faktor A I\\ P \text{ ideale primo}}} P\right) = 
            \bigcap_{\substack{P\supseteq I\\P \subseteq A\text{ ideale primo}}}P
    \]
\end{proof}

Grazie a questo risultato, possiamo classificare gli elementi invertibili 
degli anelli di polinomi.

\begin{proposition}
    Se $A$ è un anello commutativo con identità allora 
    \[
        A[x]^* = \left\{p(x) = \sum_{i = 0}^n a_ix^i\mid n \in \NN,
        a_0 \in A^*, a_i \in \sqrt{0}~\forall i \in \{1, \ldots, n\}\right\}
    \]
\end{proposition}

\begin{proof}
    Sia $X$ l'insieme definito come sopra. Consideriamo 
    $p(x) = \displaystyle\sum_{i = 0}^n a_i x^i$ un elemento di $X$, poiché
    $a_0 \in A^*$ possiamo scrivere 
    \[
        a_0^{-1}p(x) = 1 + \sum_{i = 1}^n a_i'x^i \qquad 
        a_i' = \frac{a_i}{a_0}~\forall i \in \{1, \ldots,n\}
    \]
    poniamo $t = \displaystyle -\sum_{i = 1}^n a_i'x^i$. Notiamo che $t$ è
    nilpotente in quanto tutti i coefficienti $a_i'$ sono nilpotenti e l'insieme
    dei nilpotenti è un ideale. Fissiamo $n \in \NN$ tale che $t^n = 0$, 
    dalla fattorizzazione 
    \[
        1 - t^n = (1 - t)\left(\sum_{i = 0}^{n - 1}t^i\right)
    \]
    otteniamo 
    \[
        1 = a_0^{-1}p(x)\left(\sum_{i = 0}^{n - 1}t^i\right)
    \]
    in particolare $p(x) \in A[x]^*$ e quindi $X \subseteq A[x]^*$.\newline
    Viceversa, siano $f(x) = \displaystyle\sum_{i = 0}^n\alpha_ix^i$ un 
    elemento di $A[x]^*$ e $g(x) = f(x)^{-1}$, allora $f(x)g(x) = \nolinebreak 1$.
    Notiamo che $\alpha_0 \in A^*$, infatti valutando i due polinomi in 0
    abbiamo
    \[
        f(0)g(0) = a_0g(0) = 1
    \]
    Sia $P \subseteq A$ un ideale primo, $P[x]$ è un ideale primo di $A[x]$,
    riduciamo l'espressione $f(x)g(x)$ modulo $P[x]$ tramite l'omomorfismo 
    di proiezione
    \[
        \pi: A[x]\longrightarrow \faktor A P [x] \cong \faktor{A[x]}{P[x]}
    \]
    Abbiamo $\pi(f(x))\pi(g(x)) = \pi(1)$, cioè $\pi(f(x))$ è invertibile in 
    $\faktor A P [x]$, da cui otteniamo $\pi(f(x)) \in (\faktor A P)^*$ in quanto $\faktor A P$
    è un dominio di integrità. Allora abbiamo $a_i \in P$ per ogni $i \in \{1, \ldots, n\}$,
    in particolare tali coefficienti sono contenuti nell'intersezione di tutti
    gli ideali primi di $A$ per l'arbitrarietà di $P$, sono quindi nilpotenti
    per la \hyperref[prop2.9]{Proposizione 2.9}. Vale quindi l'inclusione 
    $A[x]^* \subseteq X$, da cui l'uguaglianza.
\end{proof}

\begin{proposition}
    Siano $A$ un anello commutativo con identità e $I, J, K \subseteq A$ ideali.
    Valgono i seguenti fatti:
    \begin{enumerate}[(1)]
        \item se $I + J + K = A$ allora $I^n + J^n + K^n = A$ per ogni $n \geq 1$;
        \item se $I + J = J + K = I + K = A$ allora $IJ + JK + IK = A$.
    \end{enumerate}
\end{proposition}

\begin{proof}
    Mostriamo i due fatti separatamente:
    \begin{enumerate}[(1)]
        \item poiché $I + J + K = A$ esistono $i \in I, j \in J, k \in K$ tali che $i + j + k = 1$.
        Consideriamo la potenza 
        \[
            (i + j + k)^N = \sum_{x + y + z = N}\binom{N}{x, y, z}i^xj^yk^z\footnote{
                Ricordiamo che $\displaystyle\binom{N}{x, y, z} = \frac{N!}{x!\ y!\ z!}$.
            }
        \]
        Se $N \geq 3n$ osserviamo che $\max{x, y, z} \geq n$ per ogni $x, y, z \in \NN$
        tali che $x + y + z = N$, pertanto scegliendo $N$ in questo modo abbiamo
        che $(i + j + k)^N = 1$ è un elemento di $I^n + J^n + K^n$, quindi l'ideale
        coincide con $A$;
        \item dalle ipotesi esistono $i_1, i_2 \in I, j_1, j_2 \in J, k_1, k_2 \in K$
        tali che 
        \[
            i_1 + j_1 = 1\qquad j_2 + k_1 = 1\qquad i_2 + k_2 = 1
        \]
        Per la proprietà di assorbimento degli ideali $IJ, JK, IK$, svolgendo 
        i calcoli si ha
        \[
            1 = (i_1 + j_1)(j_2 + k_1)(i_2 + k_2) \in IJ + JK + IK
        \]
    \end{enumerate}
\end{proof}

\newpage

\subsection{Interi di Gauss}

Consideriamo l'anello degli Interi di Gauss $\ZZ[i] = \{a + ib\mid a, b \in \ZZ\}$,
abbiamo visto che $\ZZ[i]$ è un dominio euclideo e la sua funzione grado è
\[
    N: \ZZ[i]\longrightarrow \NN: a + ib \longmapsto a^2 + b^2
\]
che chiamiamo \vocab{norma}. Notiamo che questa norma è il quadrato dell'usuale 
norma complessa, pertanto è una funzione moltiplicativa. 

\subsubsection{Elementi primi}

\begin{lemma}
    Il gruppo degli elementi invertibili di $\ZZ[i]$ è $\{1, -1, i, -i\}$.
\end{lemma}

\begin{proof}
    Chiaramente $\{1, -1, i, -i\} \subseteq \ZZ[i]^*$, mostriamo quindi l'altra
    inclusione. Sia $a + ib \in \ZZ[i]^*$, allora esistono $c, d \in \ZZ$
    tali che $(a + ib)(c + id) = 1$, passando alle norme otteniamo l'equazione
    \[
        (a^2 + b^2)(c^2 + d^2) = 1
    \]
    da cui ricaviamo $a^2 + b^2 = c^2 + d^2 = 1$, quindi $a + bi \in \{1, -1, i, -i\}$.
\end{proof}

\begin{lemma}
    \label{lemma2.13}
    Dato $p \in \ZZ$ un primo, se $p \equiv 3 \pmod 4$ allora $p$ è irriducibile
    in $\ZZ[i]$.
\end{lemma}

\begin{proof}
    Supponiamo per assurdo che $p$ non sia irriducibile in $\ZZ[i]$, scriviamo quindi
    la fattorizzazione
    \[
        p = (a + ib)(c + id)
    \]
    con entrambi i fattori non invertibili, passando alle norme otteniamo 
    l'equazione
    \[
        p^2 = (a^2 + b^2)(c^2 + d^2)
    \]
    Poiché gli elementi invertibili di $\ZZ[i]$ coincidono con gli elementi di 
    norma 1, abbiamo $a^2 + b^2 = c^2 + d^2 = p$. Quindi 
    \[
        a^2 + b^2 = p \equiv 3 \pmod 4
    \]
    ma questo è assurdo in quanto gli unici quadrati in $\Z4$ sono 0 e 1.
\end{proof}

\begin{lemma}
    \label{lemma2.14}
    Dato $a + ib \in \ZZ[i]$, se $N(a + ib)$ è primo in $\ZZ$ allora $a + ib$
    è irriducibile in $\ZZ[i]$.
\end{lemma}

\begin{proof}
    Fattorizziamo $a + ib$ come 
    \[
        a + ib = (c + id)(e + if)
    \]
    passando alle norme otteniamo l'equazione
    \[
        a^2 + b^2 = (c^2 + d^2)(e^2 + f^2)
    \]
    Dato che $a^2 + b^2$ è primo in $\ZZ$ (quindi irriducibile in $\ZZ$)
    abbiamo che almeno uno dei due fattori ha norma 1, cioè è invertibile e 
    quindi $a + ib$ è irriducibile in $\ZZ[i]$.
\end{proof}

\begin{lemma}
    \label{lemma2.15}
    Valgono i seguenti fatti:
    \begin{enumerate}[(1)]
        \item $1 + i$ è irriducibile in $\ZZ[i]$;
        \item $(2)\ZZ[i] = (1 + i)^2\ZZ[i] = (1 - i)^2\ZZ[i]$;
        \item $\displaystyle\frac{\ZZ[i]}{(1 + i)}\cong \FF_2$;
    \end{enumerate}
\end{lemma}

\begin{proof}Mostriamo i tre fatti separatamente:
    \begin{enumerate}[(1)]
        \item poiché $N(1 + i) = 2$, per il \hyperref[lemma2.14]{Lemma 2.14} abbiamo 
        che $1 + i$ è irriducibile in $\ZZ[i]$;
        \item Notiamo che $2 = -i(1 + i)^2 = i(1 - i)^2$, pertanto 
        \[
        (2)\ZZ[i] = (1 + i)^2\ZZ[i] = (1 - i)^2\ZZ[i]
        \]
        \item Consideriamo l'isomorfismo 
        \[
            \varphi: \ZZ[i]\longrightarrow \frac{\ZZ[x]}{(x^2 + 1)}:
            a + bi \longmapsto \overline{a + bx}
        \]
        tramite $\varphi$ abbiamo
        \[
            \frac{\ZZ[i]}{(1 + i)} \cong \frac{\ZZ[x]}{(x^2 + 1, 1 + x)}
        \]
        Notiamo che 2 è un elemento dell'ideale $(x^2 + 1, 1 + x)$, in quanto 
        possiamo scrivere
        \[
            2 = x^2 + 1 - x(x + 1) + x + 1
        \]
        Pertanto
        \[
            \frac{\ZZ[i]}{(i + 1)} \cong \frac{\ZZ[x]}{(2, 1 + x)} \cong 
            \frac{\faktor{\ZZ[x]}{(2)}}{\faktor{(2, 1 + x)}{(2)}} \cong 
            \frac{\FF_2[x]}{(1 + x)} \cong \FF_2
        \]
    \end{enumerate}
\end{proof}

\begin{lemma}
    \label{lemma2.16}
    Sia $p \in \ZZ$ un primo, se $p \equiv 1 \pmod 4$ allora $p = (a + bi)(a - bi)$
    con $a + bi, a - bi \in \ZZ[i]$ primi e non associati.
\end{lemma}

\begin{proof}
    Poiché $p \equiv 1 \pmod 4$ esiste $x \in \ZZ$ tale che $x^2 \equiv -1 \pmod p$,
    da cui $p \mid x^2 + 1$. Fattorizziamo $x^2 + 1$ in $\ZZ[i]$ come 
    \[
        x^2 + 1 = (x + i)(x - i)
    \]
    notiamo che $p\nmid x + i$ e $p \nmid x - i$, pertanto $p$ non è primo in $\ZZ[i]$.
    In particolare possiamo scrivere 
    \[
        p = (a + bi)(c + di)
    \]
    con $a + bi, c + di \in \ZZ[i] \setminus \ZZ[i]^*$. Passando alle norme 
    abbiamo
    \[
        p^2 = (a^2 + b^2)(c^2 + d^2)
    \]
    da cui $a^2 + b^2 = c^2 + d^2 = p$ in quanto nessuno dei due fattori è 
    invertibile. Abbiamo quindi 
    \[
        p = a^2 + b^2 = (a + bi)(a - bi)
    \]
    notiamo che $a + bi$ e $a - bi$ sono primi in $\ZZ[i]$ in quanto la loro 
    norma è un primo di $\ZZ$. Supponiamo per assurdo che esista $u \in \ZZ[i]^*$
    tale che $a + bi = u(a - bi)$, distinguiamo i vari casi:
    \begin{itemize}
        \item se $u = 1$ allora $a + bi = a - bi$, da cui $b = 0$ e quindi $p = a^2$,
        che è assurdo in quanto $p$ è irriducibile in $\ZZ$;
        \item se $u = -1$ allora $a + bi = -a + bi$, da cui $a = 0$ e quindi
        $p = b^2$, che è assurdo in quanto $p$ è irriducibile in $\ZZ$;
        \item se $u = i$ allora $a + bi = ai + b$, da cui $a = b$ e quindi $p = 
        2a^2$, che è assurdo in quanto $p$ è dispari;
        \item se $u = -i$ allora $a + bi = -ai - b$, da cui $a = -b$ e quindi
        $p = 2a^2$, che è assurdo in quanto $p$ è dispari.
    \end{itemize}
    Pertanto $a + bi$ e $a - bi$ sono primi di $\ZZ[i]$ non associati.
\end{proof}

\begin{proposition}
    Gli elementi primi di $\ZZ[i]$ sono, a meno di associati, tutti e soli
    gli elementi della forma
    \begin{itemize}
        \item $1 + i$;
        \item i primi $p$ di $\ZZ$ tali che $p \equiv 3 \pmod 4$;
        \item $a + bi, a - bi \in \ZZ[i]$ tali che $a^2 + b^2 = p$ è un primo
        di $\ZZ$ con $p \equiv 1 \pmod 4$.
    \end{itemize}
\end{proposition}

\begin{proof}
    Per quanto visto nei lemmi precedenti sappiamo che gli elementi della forma
    descritta sopra sono tutti primi di $\ZZ[i]$, vediamo che effettivamente
    non ne esistono altri.\newline
    Sia $a + bi \in \ZZ[i]$ un primo, fattorizziamo in primi di $\ZZ$ la norma 
    di $a + bi$
    \[
        a^2 + b^2 = \prod_{j = 1}^k p_j^{e_j}
    \]
    Poiché $a + bi \mid a^2 + b^2$ in $\ZZ[i]$, poiché primo si ha che esiste 
    $j_0 \in \{1, \ldots, k\}$ tale che $a + bi \mid p_{j_0}$, distinguiamo tre casi:
    \begin{itemize}
        \item se $p_{j_0}\equiv 3 \pmod 4$ allora $p_{j_0}$ è irriducibile in 
        $\ZZ[i]$, pertanto $a + bi$ è associato a $p_{j_0}$;
        \item se $p_{j_0} \equiv 1 \pmod 4$ allora si fattorizza in $\ZZ[i]$ come 
        \[
              p_{j_0} = (c + di)(c - di)
        \]
        con $c + di, c - di$ primi, quindi irriducibili, di $\ZZ[i]$ non 
        associati, pertanto $a + bi$ è associato a uno dei due;
        \item se $p_{j_0} = 2$ allora $a + bi \mid -i(1 + i)^2$. Poiché $a + bi$
        non è invertibile si ha $a + bi \mid 1 + i$, cioè $a + bi$ è 
        associato a $1 + i$.
    \end{itemize}
\end{proof}


\subsubsection{Quozienti di $\ZZ[i]$}

Abbiamo visto che $\displaystyle\frac{\ZZ[i]}{(1 + i)} \cong \FF_2$, vogliamo
determinare le classi di isomorfismo degli altri quozienti di $\ZZ[i]$ per 
ideali primi. Osserviamo che tali quozienti sono campi, infatti in un PID tutti 
gli ideali primi non nulli sono ideali massimali, pertanto il quoziente per 
un ideale primo produce un campo. In alternativa possiamo notare che tali
quozienti sono dei domini finiti, quindi dei campi.

\begin{proposition}
    Sia $p \in \ZZ$ un primo dispari:
    \begin{enumerate}[(1)]
        \item se $p \equiv 3 \pmod 4$ allora $\faktor{\ZZ[i]}{(p)}
        \cong \FF_{p^2}$;
        \item se $p \equiv 1 \pmod 4$ e $p = (a + bi)(a - bi)$ è la sua 
        fattorizzazione in primi di $\ZZ[i]$ allora $\displaystyle\frac{\ZZ[i]}{(a + bi)}
        \cong \FF_p$.
    \end{enumerate}
\end{proposition}

\begin{proof} 
    Mostriamo i due fatti separatamente:
    \begin{enumerate}[(1)]
        \item possiamo identificare in modo univoco gli elementi di 
        $\faktor{\ZZ[i]}{(p)}$ con i resti della divisione per $p$, cioè
        con l'insieme 
        \[
            \{a + bi \mid 0 \leq a \leq p - 1, 0 \leq b \leq p - 1\}
        \]
        che contiene $p^2$ elementi. Poiché il quoziente è un campo di cardinalità
        $p^2$ si ha
        \[
            \faktor{\ZZ[i]}{(p)}\cong \FF_{p^2}
        \]
        \item poiché $p$ è un elemento dell'ideale $(a + bi)$, per il 
        Secondo Teorema di Omomorfismo abbiamo 
        \[
            \frac{\ZZ[i]}{(a + bi)} \cong \frac{\faktor{\ZZ[i]}{(p)}}{\faktor{(a + bi)}{(p)}}
        \]
        Consideriamo solo la struttura di gruppo additivo, il quoziente 
        $\faktor{\ZZ[i]}{(p)}$ è isomorfo, come gruppo, all'insieme
        dei resti 
        \[
            \{a + bi \mid 0\leq a\leq p - 1, 0\leq b\leq p -1\}
        \]
        che è isomorfo a $\Zp\times\Zp$. Osserviamo che il quoziente
        $\faktor{(a + bi)}{(p)}$ ha cardinalità 1, $p$, oppure $p^2$ in quanto,
        come gruppo, è isomorfo a un sottogruppo di $\Zp\times\Zp$. Questa 
        non può essere 1 in quanto altrimenti si avrebbe l'identità
        \[
            (a + bi) = ((a + bi)(a - bi))
        \]
        che non è vera in quanto $a + bi$ e $a - bi$ non sono associati. D'altra 
        parte se fosse $p^2$ allora avremmo
        \[
            \frac{\faktor{\ZZ[i]}{(p)}}{\faktor{(a + bi)}{(p)}} = \{\overline{0}\}
        \]
        che sarebbe assurdo in quanto $a + bi$ non è invertibile. Pertanto 
        abbiamo l'isomorfismo di gruppi
        \[
            \frac{\ZZ[i]}{(a + bi)} \cong \frac{\Zp \times \Zp}{\Zp} \cong \Zp
        \]
        Pertanto il quoziente è un anello di cardinalità $p$, da cui necessariamente
        \[
            \frac{\ZZ[i]}{(a + bi)} \cong \FF_p
        \]
    \end{enumerate}
\end{proof}

\begin{remark}
    Se $p \equiv 1 \pmod 4$, gli anelli $\faktor{\ZZ[i]}{(p)}$ e $\FF_p \times \FF_p$
    sono isomorfi tramite un isomorfismo diverso da quello visto nella dimostrazione.
    Fattorizziamo in primi $p = (a + bi)(a - bi)$, poiché $\ZZ[i]$ è un PID
    gli ideali $(a + bi), (a - bi)$ sono massimali, quindi $(a + bi) + (a - bi) = \ZZ[i]$.
    Per il Teorema Cinese del Resto allora
    \[
        \faktor{\ZZ[i]}{(p)} \cong \frac{\ZZ[i]}{(a + bi)(a - bi)} \cong
        \frac{\ZZ[i]}{(a + bi)} \times \frac{\ZZ[i]}{(a - bi)} \cong \FF_p \times \FF_p
    \]
\end{remark}

\begin{remark}
    Abbiamo mostrato anche che la cardinalità del quoziente $\faktor{\ZZ[i]}{(\alpha)}$
    con $\alpha$ un primo di $\ZZ[i]$ è uguale a $N(\alpha)$.
\end{remark}

\begin{lemma}
    \label{lemma2.21}
    Siano $A$ un PID e $I\subseteq A$ un ideale. Se il quoziente $\faktor A I$
    è finito allora vale 
    \[
        \left|\faktor{A}{I^n}\right| = \left|\faktor A I\right|^n\footnote{
            Con $I^n$ intendiamo il prodotto dell'ideale $I$ con se stesso
            ripetuto $n$ volte.
        }
    \]
\end{lemma}

\begin{proof}
    Sia $I = (p)$, mostriamo la tesi per induzione su $n$. Consideriamo 
    l'omomorfismo di anelli
    \[
        \varphi: A \longrightarrow A: a \longmapsto ap
    \]
    e la proiezione al quoziente
    \[
        \pi: A \longrightarrow \faktor{A}{I^2}: a \longmapsto a + I^2
    \]
    Il nucleo della loro composizione è
    \[
        \ker\pi\circ\varphi = \{a \in A \mid pa \in I^2 = (p^2)\} = 
        \{a \in A \mid a = pb, b \in A\} = I
    \]
    e l'immagine è 
    \[
        \pi(\varphi(A)) = \pi(pA) = \pi(I) = \faktor{I}{I^2}
    \]
    Per il Primo Teorema di Omomorfismo allora
    \[
        \faktor A I \cong \faktor{I}{I^2}
    \]
    pertanto
    \[
        \faktor A I \cong \frac{\faktor{A}{I^2}}{\faktor{I}{I^2}}
    \]
    da cui ricaviamo
    \[
        \left|\faktor A I\right| = \left|\frac{\faktor{A}{I^2}}{\faktor{I}{I^2}}\right| = 
        \left|\frac{\faktor{A}{I^2}}{\faktor{A}{I}}\right| = 
        \frac{\left|\faktor{A}{I^2}\right|}{\left|\faktor A I\right|}
    \]
    Quindi abbiamo la tesi per $n = 2$:
    \[
        \left|\faktor{A}{I^2}\right| = \left|\faktor A I\right|^2
    \]
    Per $n > 2$, supponiamo che la tesi sia valida per $n - 1$. Consideriamo
    gli omomorfismi
    \[
        \varphi: A \longrightarrow A: a\longmapsto p^{n - 1}a
    \]
    \[
        \pi: A \longrightarrow \faktor{A}{I^n}: a \longmapsto a + I^n
    \]
    Il nucleo e l'immagine della loro composizione sono
    \[
        \ker \pi\circ\varphi = \{a \in A \mid p^{n - 1}a \in I^n = (p^n)\} = I
    \]
    \[
        \pi(\varphi(A)) = \pi(p^{n - 1}A) = \pi(I^{n - 1}) = \faktor{I^{n - 1}}{I^n}
    \]
    Pertanto abbiamo l'isomorfismo
    \[
        \faktor A I \cong \faktor{I^{n - 1}}{I^n}
    \]
    da cui, come sopra,
    \[
        \left|\faktor A I\right| = \frac{\left|\faktor{A}{I^n}\right|}{\left|\faktor{A}{I^{n -1}}\right|}
        = \frac{\left|\faktor{A}{I^n}\right|}{\left|\faktor A I\right|^{n-1}}
    \]
    Da cui la tesi.
\end{proof}

Consideriamo adesso il quoziente di $\ZZ[i]$ per un generico ideale $I = (z)$,
fattorizziamo $z$ in primi di $\ZZ[i]$:
\[
    z = u(1 + i)^e\prod_{j = 1}^r (a_j + b_j i)^{e_j}\prod_{h = 1}^s p_h^{e_h}
    \qquad u \in \ZZ[i]^*
\]
Gli ideali $(1 + i)^e, (a_j + b_j)^{e_j}, (p^{e_h})$ sono a due a due coprimi,
in quanto massimali, quindi per il Teorema Cinese del Resto
\[
    \faktor{\ZZ[i]}{I} \cong \frac{\ZZ[i]}{(1 + i)^e}\times
    \prod_{j = 1}^r\frac{\ZZ[i]}{(a_j + b_ji)^{e_j}}
    \times \prod_{h = 1}^s \frac{\ZZ[i]}{(p)^{e_h}}
\]
La cardinalità di questo quoziente è $N(z)$, infatti applicando il 
\hyperref[lemma2.21]{Lemma 2.21} abbiamo
\begin{multline*}
    \left|\faktor{\ZZ[i]}{I}\right| = \left|\frac{\ZZ[i]}{(1 + i)^e}\right|\cdot
    \left|\prod_{j = 1}^r\frac{\ZZ[i]}{(a_j + b_ji)^{e_j}}\right|\cdot
    \left|\prod_{h = 1}^s \frac{\ZZ[i]}{(p)^{e_h}}\right| = \\
    = \left|\frac{\ZZ[i]}{(1 + i)}\right|^e\cdot
    \left|\prod_{j = 1}^r\frac{\ZZ[i]}{(a_j + b_ji)}\right|^{e_j}\cdot
    \left|\prod_{h = 1}^s \frac{\ZZ[i]}{(p)}\right|^{e_h} = \\
    = N(1 + i)^e\prod_{j = 1}^r N(a_j + b_ji)^{e_j}\prod_{h = 1}^s N(p_h)^{e_h} = \\
    = N\left(u(1 + i)^e\prod_{j = 1}^r (a_j + b_j i)^{e_j}\prod_{h = 1}^s p_h^{e_h}\right)
    = N(z)\\
\end{multline*}

\newpage

\subsection{Esempio di dominio non euclideo}

Consideriamo l'anello $\displaystyle A = \ZZ\left[\frac{1 + \sqrt{-19}}{2}\right]$,
vogliamo mostrare che non è un dominio euclideo. 

\begin{proposition}
    \[
        \ZZ\left[\frac{1 + \sqrt{-19}}{2}\right] = 
        \left\{a + b\frac{1 + \sqrt{-19}}{2}\Bigm| a, b \in \ZZ\right\}
    \]
\end{proposition}

\begin{proof}
    Sia $\alpha = \displaystyle\frac{1 + \sqrt{-19}}{2}$, il polinomio minimo di $\alpha$
    su $\QQ$ è $\mu(x) = x^2 - x + 5$. Consideriamo l'omomorfismo di valutazione
    \[
        \varphi: \ZZ[x]\longrightarrow \CC: p(x) \longmapsto p(\alpha) 
        \qquad \mathrm{Im\varphi} = \ZZ[\alpha]
    \]
    poiché $\varphi$ è la restrizione dell'usuale omomorfismo di valutazione
    su $\QQ[x]$, che è un PID, abbiamo
    \begin{multline*}
        \ker \varphi = \ZZ[x] \cap \{f(x) \in \QQ[x]\mid f(\alpha) = 0\}
        = \ZZ[x] \cap (x^2 - x + 5)\QQ[x] = (x^2 - x + 5)\ZZ[x]
    \end{multline*}
    dove l'ultima uguaglianza è giustificata dal fatto che $\mu(x)$ è un 
    polinomio a coefficienti interi primitivo e per il Lemma di Gauss.
    Pertanto per il Primo Teorema di Omomorfismo abbiamo 
    \[
        \frac{\ZZ[x]}{(x^2 - x + 5)} \cong \mathrm{Im\varphi} = 
        \{a + b\alpha \mid a, b \in \ZZ\} = \ZZ[\alpha]
    \]
\end{proof}

\begin{remark}
    Il risultato appena visto non è un fatto ovvio. Consideriamo $\beta = 
    \displaystyle\frac{1 + \sqrt{3}}{2}$,
    il suo polinomio minimo su $\QQ$ è $\displaystyle\mu(x) = x^2 - x - \frac 1 2$.
    Ragionando in modo analogo a quanto fatto sopra, il nucleo della valutazione
    in $\beta$ è 
    \[
        \ZZ[x] \cap \left(x^2 - x - \frac 1 2\right)\QQ[x] =(2x^2 - 2x - 1)\ZZ[x]
    \]
    E $\displaystyle\frac{\ZZ[x]}{(2x^2 - x - 1)} \cong \ZZ[\beta]$.
    D'altra parte 
    \[
        \ZZ[\beta] \neq \{a + b\beta\mid a, b \in \ZZ\}
    \]
    in quanto il quoziente $\displaystyle\frac{\ZZ[x]}{(2x^2 - x - 1)}$
    contiene delle classi di resto della forma $\overline{x^k}$ per ogni $k \in \NN$,
    in quanto $x^k$ e $2x^2 - x - 1$ sono coprimi in $\ZZ[x]$. Notiamo che
    il risultato di sopra non è valido in questo caso in quanto il polinomio
    minimo di $\beta$ non ha coefficienti interi.
\end{remark}

Mostriamo che $A$ non è un domino euclideo. Sia $\omega = 
\displaystyle\frac{1 + \sqrt{-19}}{2}$, consideriamo l'applicazione
\[
    N: A \longrightarrow \NN: a + b\omega \longmapsto (a + b\omega)(a + b\overline{\omega}) =
    a^2 + 5b^2 + ab
\]
$N$ è la restrizione all'anello $A$ dell'usuale norma su $\CC$,
pertanto è moltiplicativa. Osserviamo inoltre che se $u \in A^*$ si ha 
$N(u)= 1$, infatti se $v \in A$ è tale che $uv = 1$ allora $N(uv) = N(u)N(v) = 1$
da cui $N(u) = N(v) = 1$ necessariamente. D'altra parte l'equazione 
\[
    a^2 + ab + 5b^2 = \left(a + \frac 1 2 b\right)^2 + \frac{19}{4} b^2 = 1
\]
ha soluzione se e solo se $a = \pm 1$ e $b = 0$, pertanto $A^* = \{-1, 1\}$.

Supponiamo per assurdo che $A$ sia un dominio euclideo, cioè che esista
un'applicazione 
\[
    \mathcal{N}: A\setminus \{0\} \longrightarrow \NN
\]
che rispetta gli assiomi di norma euclidea, ricordiamo che gli elementi 
invertibili di $A^*$ sono gli elementi di norma $\mathcal{N}$ minima. 
Consideriamo l'insieme $X = \{\mathcal{N}(x)\mid x \in A\setminus A^*\}$,
poiché $X$ è un sottoinsieme non vuoto di $\NN$ esiste un elemento minimo 
$m \in X$, sia $x \in A\setminus A^*$ tale che $\mathcal{N}(x) = m$.
Per definizione di dominio euclideo, per ogni $a \in A$ esistono $q, r \in A$
tali che $a = qx + r$, con $r = 0$ oppure $\mathcal{N}(r) < \mathcal{N}(x)$.
Se $r \neq 0$ allora $r \in A^*$ per minimalità di $\mathcal{N}(x)$, pertanto 
l'insieme dei possibili resti della divisione per $x$ è $\{0, 1, -1\}$.
Abbiamo quindi che l'insieme $\{0, 1, -1\}$ è un insieme di rappresentanti, 
possibilmente con ripetizioni, per gli elementi del quoziente $\faktor{A}{(x)}$,
che è quindi isomorfo a $\FF_2$ oppure $\FF_3$.\newline
Il polinomio $\mu(x) = x^2 - x + 5$ ha come soluzioni in $A$ $\omega$ e $\overline{\omega}$,
pertanto è riducibile in $A[x]$. Da questo si ricava che le classi di $\omega$
e $\overline{\omega}$ nel quoziente $\faktor{A}{(x)}$ sono le radici della 
classe del polinomio $\mu(x)$, che è assurdo in quanto $\mu(x)$ è irriducibile
in $\FF_2[x]$ e in $\FF_3[x]$. Pertanto $A$ non è un dominio euclideo.

\newpage

\section{Campi}

\subsection{Estensioni normali}

Ricordiamo che un'estensione di campi algebrica $\faktor L K$ si dice 
\vocab{normale} se per ogni immersione $\varphi: L\longhookrightarrow \ol{K}$
tale che $\varphi_{\mid K} = id_K$ vale $\varphi(L) = L$. Diciamo che 
l'estensione è \vocab{separabile} se il polinomio minimo su $K$ di ogni elemento
di $K$ ha radici distinte nel suo campo di spezzamento. Diciamo anche che un 
polinomio è separabile su $K$ se le sue radici in $\ol{K}$ sono tutte distinte. 
Chiamiamo \vocab{estensione di Galois finita} un'estensione di campi finita che sia 
normale e separabile. I campi che considereremo saranno sempre \vocab{campi perfetti},
cioè tutte le estensioni saranno separabili.

\begin{example}
    $\faktor{\QQ(\sqrt[3]{2})}{\QQ}$ non è un'estensione normale di $\QQ$.
    Infatti, poiché il polinomio minimo di $\sqrt[3]{2}$ su $\QQ$ è $\mu(x) = x^3 - 2$,
    esistono 3 immersioni $\varphi_i: \QQ(\sqrt[3]{2})\longhookrightarrow \ol\QQ$
    tali che $\varphi_{i\mid K} = id_K$, $i \in \{0, 1, 2\}$. Poiché il campo
    $\QQ$ è fissato da $\varphi_i$, è sufficiente studiare l'immagine delle
    radici di $\mu(x)$ tramite le immersioni: le possibili immagini di $\sqrt[3]{2}$
    sono $\sqrt[3]{2}, \sqrt[3]{2}\zeta_3, \sqrt[3]{2}\zeta_3^2$. Poiché i 
    tre campi $\QQ(\sqrt[3]{2}), \QQ(\sqrt[3]{2}\zeta_3), \QQ(\sqrt[3]{2}\zeta_3^2)$
    sono diversi, l'estensione non è normale.
\end{example}

\begin{example}
    $\faktor{\QQ(\sqrt{2})}{\QQ}$ è un'estensione normale di $\QQ$. Infatti,
    poiché il polinomio minimo di $\sqrt{2}$ su $\QQ$ è $x^2 - 2$, abbiamo 
    due immersioni $\varphi_1, \varphi_2:\QQ(\sqrt{2})\longhookrightarrow \ol\QQ$
    che fissano $\QQ$ tali che $\varphi_1(\sqrt{2}) = \sqrt{2}$, 
    $\varphi_2(\sqrt{2}) = -\sqrt{2}$. Pertanto le immagini di $\QQ(\sqrt 2)$
    tramite le immersioni sono
    \[
        \varphi_1(\QQ(\sqrt 2)) = \{\varphi_1(a + b\sqrt 2)\mid a, b \in \QQ\} = 
        \{a + b\sqrt 2\mid a, b \in \QQ\}
    \]
    \[
        \varphi_2(\QQ(\sqrt 2)) = \{\varphi_2(a + b\sqrt 2)\mid a, b \in \QQ\} = 
        \{a - b\sqrt 2 \mid a, b \in \QQ\}
    \]
    che sono uguali, quindi l'estensione è normale.
\end{example}

Se un'estensione $\faktor L K$ è normale possiamo definire il \vocab{Gruppo di Galois}
di $\faktor L K$ come il gruppo delle immersioni $\varphi:L\longhookrightarrow \ol{K}$
che fissano $K$. Questo coincide con il gruppo degli automorfismi di $L$
che fissano $K$, e il suo ordine è pari al grado dell'estensione.

\begin{remark}
    Un'estensione quadratica è sempre un'estensione normale. Infatti se $K$ 
    è un campo (perfetto) e $\alpha \in \ol{K}$ è tale che $\sqrt{\alpha}\notin K$,
    allora $K(\alpha)$ è il campo di spezzamento del polinomio $x^2 - \alpha$.
    Quindi $\faktor{K(\alpha)}{K}$ è normale e $\Gal(K(\alpha)/K) \cong \Z2$.
\end{remark}

Diamo qualche esempio di calcolo del gruppo di Galois di un'estensione.

\begin{example}
    Sia $L = \QQ(\sqrt 2, \sqrt 3)$ verifichiamo che $\faktor{L}{\QQ}$ è un'estensione 
    normale e calcoliamo $\Gal(L/\QQ)$. $\faktor{L}{\QQ}$ è un'estensione 
    normale in quanto $L$ è il campo di spezzamento del polinomio $(x^2 - 2)(x^2 - 3)$
    su $\QQ$, pertanto è ben definito il gruppo di Galois dell'estensione, 
    che ha ordine 4 in quanto $[L: \QQ] = 4$. Siano $\{1, \sqrt 2, \sqrt 3, \sqrt 6\}$
    una $\QQ$-base di $L$ come spazio vettoriale e $\varphi \in \Gal(L/\QQ)$,
    poiché $\varphi$ è in particolare un'applicazione lineare è sufficiente
    determinare la sua immagine sulla base. Abbiamo quindi
    \[
        \varphi(\sqrt 2) = \pm\sqrt 2
    \]
    \[
        \varphi(\sqrt 3) = \pm \sqrt 3
    \]
    \[
        \varphi(\sqrt 6) = \varphi(\sqrt 2)\varphi(\sqrt 3)
    \]
    in particolare abbiamo al più 4 omomorfismi. D'altra parte $\Gal(L/\QQ)$
    contiene esattamente 4 elementi, quindi questi sono tutti e soli gli automorfismi
    del campo $L$ che fissano $\QQ$. Si verifica che questi omomorfismi, ad 
    eccezione dell'identità, hanno ordine 2, pertanto $\Gal(L/\QQ) \cong \Z2\times\Z2$.
\end{example}

\begin{example}
    Sia $L = \QQ(\sqrt[3]2, \zeta_3)$ il campo di spezzamento del polinomio 
    $x^3 - 2$ su $\QQ$, $\faktor{L}{\QQ}$ è un'estensione normale di $\QQ$
    e $[L: \QQ] = 6$. Esplicitando le radici del polinomio $x^3 - 2$,
    abbiamo $L = \QQ(\sqrt[3]2, \sqrt[3]2\zeta_3, \sqrt[3]2\zeta_3^2)$.
    Sappiamo dalla teoria che $\Gal(L/\QQ)$ si immerge in $S_3$, poiché 
    sono entrambi gruppi finiti della stessa cardinalità si ha quindi 
    $\Gal(L/\QQ)\cong S_3$.
\end{example}

\begin{definition}
    Dato $p$ un numero primo, l'applicazione
    \[
        \Phi: \FF_{p^n} \longrightarrow \FF_{p^n}: x \longmapsto x^p
    \]
    si dice \vocab{automorfismo di Frobenius}
\end{definition}

L'automorfismo di Frobenius è effettivamente un automorfismo, poiché $\FF_{p^n}$
è un campo finito è sufficiente mostrare che è un omomorfismo iniettivo:
\begin{itemize}
    \item per ogni $x, y \in \FF_{p^n}$
    \[
        \Phi(xy) = (xy)^p = x^py^p = \Phi(x)\Phi(y)
    \]
    \[
        \Phi(x + y) = (x + y)^p =\footnote{Per il Lemma del Binomio Ingenuo.} 
        x^p + y^p = \Phi(x) + \Phi(y)
    \]
    pertanto $\Phi$ è un omomorfismo:
    \item sia $x \in \ker\Phi$, allora
    \[
        \Phi(x) = x^p = 0 \iff x = 0
    \]
    in quanto il polinomio $t^p$ ha 0 come unica radice in $\FF_{p^n}$, 
    pertanto $\Phi$ è iniettivo.
\end{itemize}

\begin{theorem}
    Per ogni primo $p$, l'estensione $\faktor{\FF_{p^n}}{\FF_p}$ è normale e 
    $\Gal(\FF_{p^n}/\FF_p) \cong \Zn$.
\end{theorem}

\begin{proof}
    L'estensione $\faktor{\FF_{p^n}}{\FF_p}$ è normale in quanto $\FF_{p^n}$
    è, per costruzione, il campo di spezzamento del polinomio $t^{p^n} - t$
    su $\FF_p$, e il grado di tale estensione è $n$. Osserviamo che l'automorfismo
    di Frobenius $\Phi$ è un elemento di $\Gal(\FF_{p^n}/\FF_p)$, infatti 
    per ogni $x \in \FF_p$ vale $\Phi(x) = x^p = x$ per il Piccolo Teorema di Fermat.
    L'ordine di $\Phi$ è $n$, infatti 
    \[
        \Phi^k = id_{\FF_{p^n}} \iff x^{p^k} = x~\forall x \in \FF_{p^n}
    \]
    e l'equazione è verificata se e solo se il polinomio $t^{p^k} - t$ ha almeno
    $p^n$ radici, cioè se $k \geq n$. D'altra parte l'ordine di $\Phi$ deve 
    dividere $n$, pertanto $\ord\Phi = n$. Quindi $\Phi$ è un generatore 
    di $\Gal(\FF_{p^n}/\FF_p)$, che è isomorfo a $\Zn$.
\end{proof}

\subsection{Estensioni ciclotomiche}

\begin{lemma}
    \label{lemma3.8}
    Dato $K$ un campo, il polinomio $x^n - 1$ è separabile su $K$ se e solo 
    se $\ch{K} \nmid n$.
\end{lemma}

\begin{proof}
    Per il Criterio della Derivata il polinomio $x^n - 1$ ha radici multiple
    in $\ol{K}$ se e solo se $(x^n - 1, nx^{n - 1}) \neq 1$. Se $\ch K = 0$
    allora $\QQ\subseteq K$ e le radici di 
    $x^n - 1$ sono le $n$ radici complesse dell'unità, che sono tutte distinte.
    Se $\ch K = p$, $p$ primo, allora $(x^n - 1, nx^{n - 1}) \neq 1$ se e solo
    se $p \mid n$, in quanto in quel caso si ha $nx^{n - 1} = 0$.
\end{proof}

\begin{theorem}
    \label{teorema3.9}
    Sia $\zeta_n\in \CC$ una radice primitiva $n$-esima dell'unità, allora
    l'estensione $\faktor{\QQ(\zeta_n)}{\QQ}$ è normale e 
    $\Gal(\QQ(\zeta_n)/\QQ)\cong \Zn^*$.
\end{theorem}

\begin{proof}
    Poiché $\zeta_n$ è una radice primitiva dell'unità, l'insieme delle sue 
    potenze coincide con l'insieme delle radici del polinomio $x^n - 1$\footnote{
        Ricordiamo che l'insieme delle radici complesse di $x^n -1$ è un gruppo 
        isomorfo a $\Zn$, i cui generatori sono le radici primitive.
    },
    pertanto $\faktor{\QQ(\zeta_n)}{\QQ}$ è normale in quanto $\QQ(\zeta_n)$
    è il campo di spezzamento di $x^n - 1$ su $\QQ$. Per comodità suddividiamo 
    la dimostrazione in passi:
    \begin{itemize}
        \item mostriamo che $[\QQ(\zeta_n):\QQ] \leq \phi(n)$. Un'immersione
        $\psi \in \Gal(\QQ(\zeta_n)/\QQ)$ è univocamente determinata dall'immagine
        di $\zeta_n$, inoltre $\psi(\zeta_n)$ è un elemento dell'insieme 
        $\{\zeta_n^k\mid k = 0, \ldots, n - 1\}$ in quanto è radice di $x^n - 1$.
        Supponiamo per assurdo che $\psi(\zeta_n) = \zeta_n^k$ con $d = (k, n)\neq 1$, 
        allora
        \[
            \psi(\zeta_n^{\frac n d}) = \psi(\zeta_n)^{\frac n d} = \zeta^{k\frac n d} = 
            \zeta^{\frac k d n} = 1
        \]
        da cui $\zeta_n^{\frac n d} = 1$ in quanto $\psi$ è iniettiva, quindi ha 
        nucleo banale. Questo è assurdo dato che $\ord \zeta_n = n$.
        Pertanto $\psi(\zeta_n) \in \{\zeta_n^k\mid k < n,~(n, k) = 1\}$, quindi
        $[\QQ(\zeta_n):\nolinebreak\QQ] \leq \phi(n)$;
        \item siano $p$ un primo che non divide $n$, $f(x)$ e $g(x)$ i polinomi 
        minimi su $\QQ$ rispettivamente di $\zeta_n$ e $\zeta_n^p$, osserviamo
        che $f(x) \mid g(x^p)$ in quanto $g(\zeta_n^p) = 0$;
        \item supponiamo per assurdo $f(x) \neq g(x)$, allora $f(x)$ e $g(x)$
        sono coprimi in $\QQ[x]$ ed entrambi dividono $x^n - 1$, pertanto $f(x)g(x) \mid x^n -1$.
        Per il Lemma di Gauss esistono $q(x), r(x) \in \ZZ[x]$ tali che 
        \[
            f(x)g(x)q(x) = x^n - 1\qquad f(x)r(x) = g(x^p)
        \]
        Riducendo modulo $p$ abbiamo 
        \[
            g(x)^p = g(x^p) = f(x)r(x)
        \]
        in $\FF_p[x]$. Pertanto se $\alpha \in \ol{\FF_p}$ è una radice di $f(x)$
        allora è anche una radice di $g(x)$. Pertanto $\alpha$ è una radice
        almeno doppia di $x^n - 1 \in \FF_p[x]$, che è assurdo in quanto $x^n -1$
        è separabile su $\FF_p$ per il \hyperref[lemma3.8]{Lemma 3.8}. Pertanto 
        $f(x) = g(x)$;
        \item abbiamo quindi che $\zeta_n$ e $\zeta_n^p$ hanno lo stesso polinomio 
        minimo su $\QQ$. Ripetendo lo stesso ragionamento con qualsiasi altro 
        primo $q$ che non divide $n$ otteniamo che $\zeta_n$ e $\zeta_n^q$ 
        hanno lo stesso polinomio minimo su $\QQ$, quindi questo è valido 
        in generale per $\zeta_n$ e $\zeta_n^k$ con $(n, k) = 1$. In particolare
        $\zeta_n^k$ è radice di $f(x)$ per ogni $k < n$ con $(n, k) = 1$, pertanto 
        $\deg f \geq \phi(n)$;
        \item poiché $\deg f = [\QQ(\zeta_n):\QQ] \leq \phi(n)$ abbiamo che 
        effettivamente $\deg f = \phi(n)$, quindi $\#\Gal(\QQ(\zeta_n)/\QQ) = \phi(n)$.
        Gli elementi di $\Gal(\QQ(\zeta_n)/\QQ)$ sono tutti e soli della forma
        \[
            \psi_k: \QQ(\zeta_n)\longrightarrow \ol{\QQ}: \zeta_n \longmapsto \zeta_n^k
        \]
        con $k < n$ e $(n, k) = 1$, inoltre $\psi_k\circ\psi_h = \psi_{kh} = \psi_{hk}$
        in quanto 
        \[
            \psi_k(\psi_h(\zeta_n)) = \psi_k(\zeta_n^h) = \psi_k(\zeta_n)^h =
            \zeta_n^{hk} = \zeta_n^{kh} = \psi_h(\zeta_n^k) = \psi_h(\psi_k(\zeta_n))
        \]
        abbiamo quindi un isomorfismo 
        \[
            \Psi: \Zn^* \longmapsto \Gal(\QQ(\zeta_n)/\QQ):k \longmapsto \psi_k
        \]
    \end{itemize}
\end{proof}

\begin{definition}
    [Polinomio ciclotomico]
    Data $\zeta_n \in \CC$ una radice primitiva $n$-esima dell'unità, chiamiamo
    \vocab{$n$-esimo polinomio ciclotomico} il polinomio minimo $\Phi_n(x)$
    di $\zeta_n$ su $\QQ$.
\end{definition}

\begin{remark}
    Poiché gli elementi di $\Gal(\QQ(\zeta_n)/\QQ)$ sono
    \[
        \psi_k: \QQ(\zeta_n) \longrightarrow \ol{\QQ}: \zeta_n \longmapsto \zeta_n^k
    \]
    per $0 \leq k \leq n$, $(k, n) = 1$, possiamo scrivere $\Phi_n(x)$ come 
    \[
        \Phi_n(x) = \prod_{\substack{0\leq k \leq n\\(k, n) = 1}}(x - \zeta_n^k)
    \]
    Notiamo che le radici di $\Phi_n(x)$ sono tutte e sole le radici primitive
    $n$-esime dell'unità e che $\deg\Phi_n = \phi(n)$.
\end{remark}

\begin{proposition}
    \[
        x^n - 1 = \prod_{d\mid n}\Phi_d(x)
    \]
\end{proposition}

\begin{proof}
    Sia $f(x) = \displaystyle\prod_{d \mid n}\Phi_d(x)$, notiamo che:
    \begin{itemize}
        \item sia $\alpha$ una radice di $x^n - 1$, allora esiste un intero $d$
        che divide $n$ tale che $\alpha^d = 1$, pertanto $\alpha$ è una radice primitiva
        $d$-esima. In particolare ogni radice di $x^n - 1$ è una radice di 
        $f(x)$, cioè $x^n - 1 \mid f(x)$;
        \item sia $\alpha$ una radice di $f(x)$, allora $\alpha$ è una radice
        primitiva $d$-esima dell'unità con $d\mid n$, in particolare$\alpha^d = 1$
        e quinci $\alpha^n = 1$. Allora $\alpha$ è una radice $n$-esima dell'unità,
        cioè $f(x) \mid x^n - 1$;
        \item dai due punti precedenti si deduce che esiste $\lambda \in \QQ^*$
        tale che $x^n - 1 = \lambda f(x)$, d'altra parte entrambi i polinomi
        sono monici, quindi $x^n - 1 = f(x)$.
    \end{itemize}
\end{proof}

\newpage

\subsection{Gruppo di Galois del traslato e del composto}

\begin{proposition}
    \label{prop3.13}
    Siano $\faktor K F$ un'estensione di Galois finita e $\faktor L F$ un'estensione
    finita, allora
    \begin{enumerate}[(1)]
        \item $\faktor{KL}{L}$ è un'estensione di Galois;
        \item $\Gal(KL/L) \cong \Gal(K/K\cap L)$.
    \end{enumerate}
\end{proposition}

\begin{proof}
    Consideriamo il seguente diagramma di campi, mostriamo i due enunciati separatamente
    \begin{center}
        \begin{tikzpicture}
            \node (Q1) at (0,0) {$F$};
            \node (Q2) at (-1.5, 1.5) {$K$};
            \node (Q3) at (1.5, 1.5) {$L$};
            \node (Q4) at (0,1) {$K\cap L$};
            \node (Q5) at (0,3) {$KL$};

            \draw (Q1)--(Q2);
            \draw (Q1)--(Q3);
            \draw (Q1)--(Q4);
            \draw (Q2)--(Q5);
            \draw (Q3)--(Q5);
            \draw (Q4)--(Q5);
            \draw (Q4)--(Q2);
            \draw (Q4)--(Q3);
        \end{tikzpicture}
    \end{center}
    \begin{enumerate}[(1)]
        \item poiché $\faktor K F$ è un'estensione di Galois finita possiamo scrivere
        $K$ come $F(\alpha_1, \ldots, \alpha_n)$, dove $\alpha_1, \ldots, \alpha_n \in \ol{K}$
        sono le radici di un certo polinomio $p(x) \in F[x]$. Allora abbiamo che
        $KL = L(\alpha_1, \ldots, \alpha_n)$ è il campo di spezzamento di 
        $p(x)$ su $L$, pertanto $\faktor{KL}{L}$ è un'estensione di Galois;
        \item l'estensione $\faktor{K}{K\cap L}$ è di Galois, in quanto lo è $\faktor K F$. Consideriamo la mappa di restrizione
        \[
            \Phi: \Gal(KL/L) \longrightarrow \Gal(K/K\cap L): \varphi\longmapsto \varphi_{\mid K}
        \]
        questa è ben definita in quanto ogni immersione $KL \longhookrightarrow \ol{F}$
        che fissa $L$ fissa anche $K\cap L$. Chiaramente $\Phi$ è un omomorfismo di gruppi, mostriamo che in realtà è un isomorfismo.
        $\Phi(\varphi) = id$ se e solo se $\varphi_{\mid K} = id$, ma questo è possibile se e solo se $\varphi = id$ in quanto se 
        $\varphi$ è la mappa identità su $K$ e su $L$ allora lo è anche sul composto $KL$, pertanto $\Phi$ è iniettivo. Mostriamo adesso
        che è anche surgettivo. Sia $H = \mathrm{Im}\Phi$, il sottocampo di $K$ fissato da $H$ è
        \[
            K^H = \{x \in K \mid \psi(x) = x~\forall \psi \in H\}
        \]
        Poiché $H$ contiene le restrizioni a $K$ degli elementi di $\Gal(KL/L)$, si ha
        \begin{multline*}
            K^H = \{x \in K \mid \psi(x) = x~\forall \psi \in \Gal(KL/L)\} = \\ = K\cap\{x \in KL \mid \psi(x) = x~\forall \psi \in \Gal(KL/L)\} =\\=
            K \cap (KL)^{\Gal(KL/L)} = K\cap L
        \end{multline*}
        pertanto $H = \Gal(K/K\cap L)$ per il Teorema di Corrispondenza di Galois. Quindi $\Phi$ è surgettivo, di conseguenza è un isomorfismo
        tra $\Gal(KL/L)$ e $\Gal(K/K\cap L)$.
    \end{enumerate}
\end{proof}

\begin{corollary}
    Siano $\faktor K F$ un'estensione di Galois finita e $\faktor L F$ un'estensione finita, se $K\cap L = F$ allora $[KL:F] = [K:F][L:F]$.
\end{corollary}

\begin{proof}
    Consideriamo il seguente diagramma di campi
    \begin{center}
        \begin{tikzpicture}
            \node (Q1) at (0,0) {$F$};
            \node (Q2) at (-1.5,1.5) {$K$};
            \node (Q3) at (1.5,1.5) {$L$};
            \node (Q4) at (0,3) {$KL$};
            \draw (Q1)--(Q2);
            \draw (Q1)--(Q3);
            \draw (Q1)--(Q4);
            \draw (Q2)--(Q4);
            \draw (Q3)--(Q4);
        \end{tikzpicture}
    \end{center}
    per il Teorema delle Torri abbiamo $[KL:F] = [KL:L][L:F]$. Poiché 
    $\Gal(KL/L) \cong \Gal(K/K\cap L) = \Gal(K/F)$ per la 
    \hyperref[prop3.13]{Proposizione 3.13}, in particolare $[KL:L] = [K:F]$,
    quindi $[KL:F] = [K:F][L:F]$.
\end{proof}

\begin{proposition}
    \label{prop3.15}
    Siano $\faktor{K_1}{F}$, $\faktor{K_2}{F}$ estensioni di Galois finite, 
    allora $\faktor{K_1K_2}{F}$ è un'estensione di Galois. Inoltre:
    \begin{enumerate}[(1)]
        \item esiste un'immersione $\Phi: \Gal(K_1K_2/F) \longhookrightarrow 
        \Gal(K_1/F)\times\Gal(K_2/F)$;
        \item $\Gal(K_1K_2/F)\cong \Gal(K_1/F)\times\Gal(K_2/F)$ se e solo 
        se $K_1\cap K_2 = F$. 
    \end{enumerate}
\end{proposition}

\begin{proof}
    Poiché $\faktor{K_1}{F}$ e $\faktor{K_2}{F}$ sono estensioni normali, 
    esistono $p_1(x), p_2(x) \in F[x]$ tali che $K_1$ e $K_2$ 
    sono rispettivamente i campi di spezzamento di $p_1(x)$ e $p_2(x)$ su 
    $F$. Allora il composto $K_1K_2$ è il campo di spezzamento  
    del polinomio $p_1(x)p_2(x)$ su $F$, quindi $\faktor{K_1K_2}{F}$ è un'estensione 
    di Galois.
    \begin{enumerate}[(1)]
        \item Consideriamo la mappa
        \[
            \Phi: \Gal(K_1K_2/F)\longrightarrow \Gal(K_1/F)\times 
            \Gal(K_2/F): \varphi\longmapsto (\varphi_{\mid K_1}, \varphi_{\mid K_2})
        \]
        chiaramente $\Phi$ è un omomorfismo di gruppi, mostriamo quindi che 
        il suo nucleo è banale. $\Phi(\varphi) = (id_{K_1}, id_{K_2})$
        se e solo se $\varphi_{\mid K_1} = id_{K_1}$ e $\varphi_{\mid K_2} =
        id_{K_2}$, ma allora $\varphi$ è l'identità anche sul composto
        $K_1K_2$, pertanto $\Phi$ è iniettivo;
        \item poiché i gruppi in questione sono finiti, è sufficiente mostrare 
        che hanno la stessa cardinalità per concludere che sono isomorfi.
        Per il Teorema delle Torri abbiamo $[K_1K_2:F] = [K_1K_2:K_1][K_1:F]$,
        d'altra parte $[K_1K_2:K_1] = [K_2:K_1\cap K_2]$ per
        la \hyperref[prop3.13]{Proposizione 3.13}. Pertanto $|\Gal(K_1K_2/F)| =
        |\Gal(K_1/F)|\cdot |\Gal(K_2/F)|$ 
        se e solo se $[K_2:K_1\cap K_2][K_1:F] = [K_1:F][K_2:F]$, cioè se e 
        solo se $[K_2:K_1\cap K_2] = [K_2:F]$, ovvero $K_1\cap K_2 = F$.
    \end{enumerate}
\end{proof}

\newpage

\subsection{Gruppo di Galois di un polinomio di grado 3}

Consideriamo un polinomio $f(x)$ di grado 3 che non sia completamente fattorizzabile 
in $\QQ[x]$, cioè che ha campo di spezzamento $K$ 
su $\QQ$ diverso da $\QQ$. Dalla teoria sappiamo che 3 divide l'ordine di 
$\Gal(K/\QQ)$ e che questo è isomorfo a un sottogruppo di 
$S_3$, pertanto $\Gal(K/\QQ) \cong S_3$ oppure $\Gal(K/\QQ) \cong \Z3$. 
Vediamo che entrambi i casi sono possibili con due esempi.
\newline
Consideriamo il polinomio $f(x) = x^3 - 2$, le sue radici in $\ol{\QQ}$ sono 
$\alpha_0 = \sqrt[3]{2}$, $\alpha_1 = \sqrt[3]2 \zeta_3$,
$\alpha_2 = \sqrt[3]2\zeta_3^2$, dove $\zeta_3$ è una radice primitiva terza 
di 1. In particolare, il campo di spezzamento di $f(x)$ su $\QQ$
è $K = \QQ(\alpha_0, \alpha_1, \alpha_2) = \QQ(\sqrt[3]2, \zeta_3)$. Infatti 
$\sqrt[3]2 = \alpha_0$ e $\zeta_3 = \frac{\alpha_1}{\alpha_0}$,
quindi $\QQ(\sqrt[3]2, \zeta_3) \subseteq \QQ(\alpha_0, \alpha_1, \alpha_2)$, 
d'altra parte $\alpha_i = \sqrt[3](2)\zeta_3^i$ per 
$i = 0, 1, 2$, pertanto si ha anche l'altra inclusione, da cui l'uguaglianza. 
Consideriamo il diagramma di campi 
\begin{center}
    \begin{tikzpicture}
        \node (Q1) at (0,0) {$\QQ$};
        \node (Q2) at (-1.5,1.5) {$\QQ(\sqrt[3]2)$};
        \node (Q3) at (1.5, 1.5) {$\QQ(\zeta_3)$};
        \node (Q4) at (0,3) {$K = \QQ(\sqrt[3]2, \zeta_3)$};

        \draw (Q1)--(Q2);
        \draw (Q1)--(Q3);
        \draw (Q1)--(Q4);
        \draw (Q2)--(Q4);
        \draw (Q3)--(Q4);
    \end{tikzpicture}
\end{center}
l'estensione $\faktor{\QQ(\sqrt[3]2)}{\QQ}$ ha grado 3 in quanto il 
polinomio minimo di $\sqrt[3]2$ su $\QQ$ è $x^3 - 2$, mentre 
l'estensione $\faktor{\QQ(\zeta_3)}{\QQ}$ ha grado 2 in quanto il polinomio
 minimo di $\zeta_3$ su $\QQ$ è $x^2 + x + 1$. Dato 
che i gradi sono coprimi, l'estensione $\faktor{K}{\QQ}$ ha grado 6, di 
conseguenza $\Gal(K/\QQ) \cong S_3$.
\newline

Adesso vogliamo determinare un polinomio il cui gruppo di Galois sia 
isomorfo a $\ZZ3$. Consideriamo l'estensione $\faktor{\QQ(\zeta_7)}{\QQ}$,
dove $\zeta_7$ è una radice primitiva settima di 1, per il 
\hyperref[teorema3.9]{Teorema 3.9} $\Gal(\QQ(\zeta_7)/\QQ) \cong \Z7^* \cong \Z6$.
Vale il seguente risultato.

\begin{proposition}
    Sia $\zeta_n \in \ol{\QQ}$ una radice primitiva $n$-esima di 1 per $n \geq 3$,
    allora $[\QQ(\zeta_n):\QQ(\zeta_n)\cap \RR] = 2$.
\end{proposition}

\begin{proof}
    Sia $\alpha = \zeta_n + \zeta_n^{-1}$, poiché $\ol{\zeta_n} = \zeta_n^{-1}$
    si ha $\ol{\alpha} = \ol{\zeta_n + \zeta_n^{-1}} = \zeta_n + \zeta_n^{-1} = \alpha$,
    cioè $\alpha \in \RR$ e quindi $\QQ(\alpha) \subseteq \QQ(\zeta_n)\cap \RR$.
    Il polinomio $x^2 - \alpha x + 1$ è a coefficienti reali e si annulla in 
    $\zeta_n$, pertanto $[\QQ(\zeta_n):\QQ(\zeta_n)\cap \RR]\leq 2$. D'altra
    parte $\QQ(\zeta_n)\neq \QQ(\zeta_n)\cap \RR$, quindi 
    $[\QQ(\zeta_n):\QQ(\zeta_n)\cap \RR] = 2$.
\end{proof}

\begin{remark}
    In realtà vale che $\QQ(\zeta_n)\cap \RR = \QQ(\zeta_n + \zeta_n^{-1})$.
    Infatti il polinomio $x^2 - \alpha x + 1$, con le notazioni di sopra, 
    è un polinomio a coefficienti in $\QQ(\alpha)$ che si annulla in $\zeta_n$,
    pertanto $[\QQ(\zeta_n):\QQ(\alpha)] \leq 2$. D'altra parte $\QQ(\zeta_n)\neq \QQ(\alpha)$,
    pertanto $[\QQ(\zeta_n):\QQ(\alpha)] = 2$ e quindi 
    $\QQ(\zeta_n + \zeta_n^{-1}) = \QQ(\zeta_n)\cap \RR$.
\end{remark}

Abbiamo quindi che la sottoestensione $\QQ(\zeta_7 + \zeta_7^{-1})$ ha grado
3 su $\QQ$, mostriamo quindi che è una sua estensione normale. Posto 
$\alpha = \zeta_7 + \zeta_7^{-1}$, le immersioni di 
$\QQ(\alpha) \longhookrightarrow \ol{\QQ}$ sono le restrizioni a $\QQ(\alpha)$
degli elementi di $\Gal(\QQ(\zeta_7)/\QQ)$, pertanto sono univocamente determinate
dalle assegnazioni 
\[
    \zeta_7 + \zeta_7^{-1} \longmapsto \zeta_7 + \zeta_7^{-1}\qquad
    \zeta_7 + \zeta_7^{-1} \longmapsto \zeta_7^2 + \zeta_7^{-2}\qquad
    \zeta_7 + \zeta_7^{-1} \longmapsto \zeta_7^3 + \zeta_7^{-3}
\]
Il polinomio minimo di $\alpha$ su $\QQ$ è quindi 
\[
    \mu_{\alpha}(x) = (x - (\zeta_7 + \zeta_7^{-1}))(x - (\zeta_7^2 + \zeta_7^{-2}))(x - (\zeta_7^3 + \zeta_7^{-3}))
    = x^3 + x^2 - 2x - 1
\]
Notiamo che $\zeta_7^2 + \zeta_7^{-2}$ e $\zeta_7^3 + \zeta_7^{-3}$ sono
elementi di $\QQ(\alpha)$, in quanto
\[
    \zeta_7^2 + \zeta_7^{-2} = (\zeta_7 + \zeta_7^{-1})^2 - 1
\]
\[
    \zeta_7^3 + \zeta_7^{-3} = (\zeta_7 + \zeta_7^{-1})^3 - 3(\zeta_7 - \zeta_7^{-1})
\]
pertanto $\faktor{\QQ(\alpha)}{\QQ}$ è un'estensione normale di grado 3 in 
quanto campo di spezzamento di $\mu_{\alpha}(x)$ su $\QQ$,
quindi il suo gruppo di Galois è isomorfo a $\Z3$.

\newpage

\subsection{Possibili gruppi di Galois}

Vogliamo vedere quali gruppi finiti si possono realizzare come gruppi di 
Galois di un'estensione di campi. 

\begin{lemma}
    \label{lemma3.18}
    Dato $p$ un primo, $S_p$ è generato da un $p$-ciclo e da una trasposizione.
\end{lemma}

\begin{proof}
    Per motivi di notazione, consideriamo $S_p$ come l'insieme delle permutazioni
    dell'insieme $\{0, 1, \ldots, p - 1\}$, che pensiamo come $\FF_p$. Siano
    $\sigma = \cycle{0, 1, \ldots, p - 1}$ un $p$-ciclo e $\tau = \cycle{a, b}$
    una trasposizione, poiché $\sigma^k\tau\sigma^{-k} = \cycle{a + k, b + k}$
    ed esiste $h \in \FF_p$ tale che $a + h = 0$, possiamo supporre
    $\tau = \cycle{0, \alpha}$ senza perdita di generalità. Notiamo che gli 
    elementi $\sigma^k\tau\sigma^{-k}$ sono tutti della forma $\cycle{k, \alpha + k}$,
    in particolare $\langle\sigma, \tau\rangle$ contiene le permutazioni 
    $\cycle{k\alpha, (k + 1)\alpha}$ al variare di $k \in \FF_p$. Possiamo
    quindi costruire in modo iterativo le permutazioni $\cycle{0, k\alpha}$,
    ad esempio
    \[
        \cycle{0, \alpha}\cycle{\alpha, 2\alpha}\cycle{0, \alpha} = \cycle{0, 2\alpha}
    \]
    \[
        \cycle{0, 2\alpha}\cycle{2\alpha, 3\alpha}\cycle{0, 2\alpha} = \cycle{0, 3\alpha}
    \]
    \[
        \cycle{0, 3\alpha}\cycle{3\alpha, 4\alpha}\cycle{0, 3\alpha} = \cycle{0, 4\alpha}
    \]
    e così via. Poiché $\FF_p$ è un campo e $\alpha \neq 0$ (altrimenti $\tau$ 
    non sarebbe una trasposizione) esiste $h \in \FF_p$ tale che $h\alpha = 1$,
    pertanto $\cycle{0, 1} \in \langle\sigma, \tau\rangle$. Poiché 
    $\langle\cycle{0, 1}, \sigma\rangle = S_p$ si ha la tesi.
\end{proof}

\begin{lemma}
    \label{lemma3.19}
    Dati $p$ un primo e $f(x) \in \QQ[x]$ un polinomio irriducibile di grado
    $p$, se $f(x)$ ha esattamente $p - 2$ radici reali e 2 radici non reali
    e $K$ è il suo campo di spezzamento su $\QQ$ allora $\Gal(K/\QQ) \cong S_p$.
\end{lemma}

\begin{proof}
    Poiché $\deg f = p$ esiste un omomorfismo iniettivo 
    \[
        \Phi: \Gal(K/\QQ) \longhookrightarrow S_p
    \]
    inoltre $p \mid [K:\QQ]$ in quanto $f(x)$ è irriducibile su $\QQ$, pertanto 
    $\Phi(\Gal(K/\QQ))$ contiene un $p$-ciclo. Notiamo che contiene anche una
    trasposizione, che corrisponde alla restrizione del coniugio complesso 
    in $\Gal(K/\QQ)$. Allora $\Phi(\Gal(K/\QQ)) = S_p$ per il 
    \hyperref[lemma3.18]{Lemma 3.18}, cioè $\Gal(K/\QQ) \cong S_p$.
\end{proof}

\begin{lemma}
    [Lemma di Artin]
    \footnote{
        La dimostrazione è da revisionare nella parte della dimostrazione 
        della finitezza dell'estensione.
    }
    \label{lemma3.20}
    Dato $K$ un campo e $G$ un sottogruppo finito di $\Aut(K)$, allora 
    $\faktor{K}{K^G}$ è un'estensione di Galois finita e $\Gal(K/K^G) = G$.
\end{lemma}

\begin{proof}
    Consideriamo un'immersione $\varphi: K \longrightarrow \ol{K}$ tale che
    $\varphi_{\mid K^G} = id_{K^G}$, per definizione di $K^G$ si ha che 
    $\varphi \in G$. In particolare $G$ è l'insieme delle immersioni di $K$
    in $\ol{K}$ che fissano $K^G$, pertanto $[K:K^G] = |G|$, cioè $\faktor{K}{K^G}$
    è un'estensione finita. 

    Per il Teorema dell'Elemento Primitivo esiste $\alpha \in K$ tale che 
    $K = K^G(\alpha)$, posto $\mu(x) \in K^G[x]$ il polinomio minimo di $\alpha$
    su $K^G$ sia $L$ il campo di spezzamento di $\mu(x)$ su $K^G$, vale
    l'inclusione $K \subseteq L$ in quanto $K = K^G(\alpha)$. Consideriamo 
    il polinomio 
    \[
        f(x) = \prod_{g \in G}(x - g(\alpha)) \in K[x]
    \]
    In realtà si ha $f(x) \in K^G[x]$, in quanto per ogni $h \in G$ vale 
    \[
        h(f(x)) = \prod_{g \in G}(x - (hg)(\alpha)) = f(x)
    \]
    in quanto la composizione per $h$ induce una bigezione tra gli elementi di $G$,
    quindi un riordinamento del prodotto. Poichè $f(\alpha) = 0$ si ha che
    $\mu(x)\mid f(x)$, pertanto le radici di $\mu(x)$ sono tutte della forma
    $g(\alpha)$ per opportuni $g \in G$. Allora le radici di $\mu(x)$ sono
    tutti elementi di $K$, pertanto $L = K$ e quindi $\faktor{K}{K^G}$ è 
    un'estensione di Galois. Sia $H = \Gal(K/K^G)$, allora
    \[
        K^H = K^{\Gal(K/K^G)} = K^G
    \]
    da cui $H = G$ per il Teorema di Corrispondenza di Galois.
\end{proof}

\begin{proposition}
    Ogni gruppo finito $G$ si realizza come gruppo di Galois di un'estensione
    di campi.
\end{proposition}

\begin{proof}
    Sia $|G| = n$ e $p \geq n$ un primo, si hanno le immersioni
    \[
        G \longhookrightarrow S_n \longhookrightarrow S_p
    \]
    Consideriamo $f(x) \in \QQ[x]$ un polinomio irriducibile di grado $p$
    avente esattamente $p - 2$ radici reali e 2 radici non reali e sia $K$ 
    il suo campo di spezzamento su $\QQ$. Per il \hyperref[lemma3.19]{Lemma 3.19}
    vale $\Gal(K/\QQ) \cong S_p$, per il \hyperref[lemma3.20]{Lemma di Artin}
    allora $\faktor{K}{K^G}$ è un'estensione di Galois finita e il suo gruppo
    di Galois è isomorfo a $G$.
\end{proof}

\newpage

\subsection{Estensioni quadratiche di $\QQ$}

\begin{theorem}
    Siano $p_1, \ldots, p_n \in \ZZ$ primi distinti, poniamo $K_n = 
    \QQ(\sqrt{p_1}, \ldots, \sqrt{p_n})$. $\faktor{K_n}{\QQ}$ è un'estensione
    di Galois e $\Gal(K_n/\QQ) \cong (\Z2)^n$.
\end{theorem}

\begin{proof}
    Mostriamo la tesi per induzione su $n \geq 1$. Per $n = 1$ si ha $K_1 = \QQ(\sqrt{p_1})$,
    che è un'estensione di Galois di $\QQ$ in quanto di grado 2, e il suo 
    gruppo di Galois è isomorfo a $\Z2$. Per $n > 1$, supponiamo che l'estensione
    $\faktor{K_n}{\QQ}$ sia di Galois e che $\Gal(K_n/\QQ) \cong (\Z2)^n$, 
    mostriamo la tesi per $n + 1$. Consideriamo il seguente diagramma di campi
    \begin{center}
        \begin{tikzpicture}
            \node (Q1) at (0,0) {$\QQ$};
            \node (Q2) at (-1.5,1.5) {$K_n$};
            \node (Q3) at (1.5, 1.5) {$\QQ(\sqrt{p_{n + 1}})$};
            \node (Q4) at (0,3) {$K_{n + 1}$};
    
            \draw (Q1)--(Q2);
            \draw (Q1)--(Q3);
            \draw (Q1)--(Q4);
            \draw (Q2)--(Q4);
            \draw (Q3)--(Q4);
        \end{tikzpicture}
    \end{center}
    l'estensione $\faktor{K_{n + 1}}{\QQ}$ è di Galois in quanto composto
    di due estensioni di Galois su $\QQ$. Notiamo che si ha la tesi nel caso 
    in cui $K_n \cap \QQ(\sqrt{p_{n + 1}}) = \QQ$, e che se questo non si 
    verifica allora $\QQ(\sqrt{p_{n + 1}}) \subseteq K_n$. Per il Teorema di 
    Corrispondenza di Galois le sottoestensioni di $K_n$ di grado due su 
    $\QQ$ sono tante quanti i sottogruppi di indice due di $(\Z2)^n$, che a 
    loro volta sono tanti quanti gli iperpiani di $(\FF_2)^n$\footnote{
        Stiamo qua considerando la struttura di spazio vettoriale di $(\FF_2)^n$.
    }, che sono $2^n - 1$.
    Consideriamo le sottoestensioni quadratiche 
    $\QQ(\sqrt{p_1^{\varepsilon_1}\ldots p_n^{\varepsilon_n}})$ con $\varepsilon_i \in \{0, 1\}$
    non tutti nulli, se queste sono due a due distinte allora sono tutte e sole
    le sottoestensioni quadratiche di $K_n$, in quanto sono $2^n - 1$. In effetti,
    le estensioni $\QQ(\sqrt{p_1^{\varepsilon_1}\ldots p_n^{\varepsilon_n}})$ e
    $\QQ(\sqrt{p_1^{\delta_1}\ldots p_n^{\delta_n}})$, con 
    $\varepsilon_i, \delta_i \in \{0, 1\}$ non tutti nulli, coincidono se e solo se 
    $(p_1^{\varepsilon_1}\ldots p_n^{\varepsilon_n})(p_1^{\delta_1}\ldots p_n^{\delta^n})$
    è un quadrato in $\QQ$, quindi in $\ZZ$. Questo è equivalente a richiedere
    $\varepsilon_i + \delta_i \equiv 0 \pmod 2$ per ogni $i$, ovvero 
    $\varepsilon_i = \delta_i$ per ogni $i$. Abbiamo quindi determinato tutte 
    e sole le sottoestensioni di $K_n$ quadratiche su $\QQ$. Notiamo quindi 
    che $\QQ(\sqrt{p_{n + 1}}) \nsubseteq K_n$ in quanto $p_{n + 1}$ non è 
    un quadrato in $\QQ$ essendo irriducibile in $\ZZ$, quindi per la 
    \hyperref[prop3.15]{Proposizione 3.15} si ha $\Gal(K_{n + 1}/\QQ) \cong
    (\Z2)^n\times \Z2 = (\Z2)^{n + 1}$.
\end{proof}

\begin{remark}
    Otteniamo come corollario che $\QQ$ ammette infinite estensioni quadratiche.
\end{remark}

\begin{remark}
    Un elemento primitivo per l'estensione $\faktor{K_n}{\QQ}$ è dato da 
    $\alpha =\displaystyle \sum_{i = 1}^n \sqrt{p_i}$. Consideriamo infatti le immersioni
    $\QQ(\alpha) \longhookrightarrow \ol{\QQ}$ che fissano $\QQ$, poiché 
    $\QQ(\alpha) \subseteq K_n$ queste si estendono a immersioni
    $K_n \longhookrightarrow \ol{\QQ}$, che sono gli elementi di $\Gal(K_n/\QQ)$.
    In particolare, se $\sigma \in \Gal(K_n/\QQ)$ si ha 
    $\sigma(\alpha) = \displaystyle\sum_{i = 1}^n a_i\sqrt{p_i}$, con $a_i \in \{1, -1\}$.
    Le immagini di $\alpha$ tramite gli elementi di $\Gal(K_n/\QQ)$ sono 
    quindi tutte distinte in quanto $\sqrt{p_1}, \ldots, \sqrt{p_n}$ sono
    elementi di una base di $K_n$ su $\QQ$, pertanto la scrittura di $\sigma(\alpha)$
    come combinazione lineare di tali elementi è unica al variare di 
    $\sigma \in \Gal(K_n/\QQ)$. In particolare $\alpha$ ha $2^n$ immagini distinte,
    pertanto $[\QQ(\alpha): \QQ] = 2^n$ e quindi $\QQ(\alpha) = K_n$.
\end{remark}

\newpage

\subsection{Gruppo di Galois di un polinomio biquadratico}

\begin{theorem}
    Siano $f(x) = x^4 + ax^2 + b \in \QQ[x]$ un polinomio irriducibile, definiamo $\Delta = a^2 - 4b$.
    Posto $K$ il campo di spezzamento di $f(x)$ su $\QQ$ si ha:
    \begin{enumerate}[(1)]
        \item se $\sqrt{b} \notin \QQ$ e $\sqrt{b\Delta} \notin \QQ$ allora $\Gal(K/\QQ) \cong D_4$;
        \item se $\sqrt{b} \in \QQ$ allora $\Gal(K/\QQ) \cong \Z2\times\Z2$;
        \item se $\sqrt{b\Delta}\in\QQ$ allora $\Gal(K/\QQ) \cong \Z4$;
    \end{enumerate}
\end{theorem}

\begin{proof}
    Sostituendo $t = x^2$ e risolvendo l'equazione $t^2 + at + b$ ricaviamo
    le radici di $f(x)$ in $\ol{\QQ}$
    \[
        x_1 = \sqrt{\frac{-a + \sqrt{\Delta}}{2}}\qquad 
        x_2 = -\sqrt{\frac{-a + \sqrt{\Delta}}{2}}\qquad
        x_3 = \sqrt{\frac{-a - \sqrt{\Delta}}{2}}\qquad
        x_4 = -\sqrt{\frac{-a - \sqrt{\Delta}}{2}}
    \]
    quindi $K = \QQ(x_1, x_2, x_3, x_4) = \QQ(x_1, x_3)$. Per ogni $i \in
    \{1, 2, 3, 4\}$ osserviamo che $\QQ(x_i^2) = \QQ(\sqrt{\Delta})$,
    consideriamo quindi il seguente diagramma di campi 
    \begin{center}
        \begin{tikzpicture}
            \node (Q1) at (0,0) {$\QQ$};
            \node (Q2) at (0,1) {$\QQ(\sqrt\Delta)$};
            \node (Q3) at (-1.5, 2.5) {$\QQ(x_1)$};
            \node (Q4) at (1.5, 2.5) {$\QQ(x_3)$};
            \node (Q5) at (0,4) {$K$};

            \draw (Q1)--(Q2);
            \draw (Q2)--(Q3);
            \draw (Q2)--(Q4);
            \draw (Q3)--(Q5);
            \draw (Q4)--(Q5);
        \end{tikzpicture}
    \end{center}
    Poiché $f(x)$ è irriducibile su $\QQ[x]$ si ha $\sqrt\Delta \notin \QQ$,
    pertanto $[\QQ(\sqrt\Delta):\QQ] = 2$. Per il Teorema delle Torri allora
    il grado di $\QQ(x_1)$ e di $\QQ(x_3)$ su $\QQ(\sqrt\Delta)$ è uguale a 2,
    quindi $[K: \QQ] \in {4, 8}$. In particolare $[K:\QQ] = 4$ se e solo se 
    $\QQ(x_1) = \QQ(x_3)$, cioè se e solo se $x_1^2x_3^2$ è un quadrato in 
    $\QQ(\sqrt\Delta)$, cioè $x_1x_3 \in \QQ(\sqrt\Delta)$.
    \[
        x_1x_3 = \sqrt{\frac{-a + \sqrt\Delta}{2}\cdot\frac{-a - \sqrt\Delta}{2}} = 
        \sqrt{\frac{a^2 - \Delta}{4}} = \sqrt b
    \]
    quindi $\QQ(x_1) = \QQ(x_3)$ se e solo se $\sqrt b \in \QQ(\Delta)$, cioè
    se e solo se $\sqrt b \in \QQ$ oppure $\sqrt{b\Delta} \in \nolinebreak\QQ$.
    Distinguiamo tre casi:
    \begin{enumerate}[(1)]
        \item se $\sqrt b \notin \QQ$ e $\sqrt{b\Delta} \notin \QQ$ allora 
        $[K:\QQ] = 8$. Allora $\Gal(K/\QQ) \cong D_4$ in quanto $\Gal(K/\QQ)$
        è isomorfo a un sottogruppo di $S_4$ e i 2-Sylow di $S_4$ sono isomorfi
        a $D_4$;
        \item se $\sqrt b \in \QQ$ (e di conseguenza $\sqrt{b\Delta}\notin \QQ$)
        allora $K = \QQ(x_1)$, quindi $[K:\QQ] = 4$ e $\Gal(K/\QQ)$ è isomorfo
        a $\Z2\times\Z2$ oppure a $\Z4$. Siano $\varphi_i \in \Gal(K/\QQ)$ per 
        $i \in \{1, 2, 3, 4\}$ gli omomorfismi determinati dalle seguenti assegnazioni
        \[
            \varphi_1: x_1\longmapsto x_1\qquad \varphi_2:x_1\longmapsto x_2\qquad
            \varphi_3: x_1\longmapsto x_3\qquad \varphi_4:x_1\longmapsto x_4
        \]
        poiché $x_2 = -x_1$ abbiamo che $\varphi_2^2 = \varphi_1 = id$,
        cioè $\varphi_2$ ha ordine 2. Sfruttando la relazione $x_1x_3 = \sqrt b$
        abbiamo 
        \[
            \varphi_3^2(x_1) = \varphi_3(x_3) = 
            \varphi_3\left(\frac{\sqrt b}{x_1}\right) = 
            \frac{\varphi_3(\sqrt b)}{\varphi_3(x_1)}\footnote{
                L'uguaglianza è data dal fatto che $\sqrt b$ è un elemento
                di $\QQ$, pertanto è fissato da tutti gli elementi di $\Gal(K/\QQ)$.
            } = \frac{\sqrt b}{x_3} = x_1
        \]
        pertanto anche $\varphi_3$ ha ordine 2. Allora $\Gal(K/\QQ)$ non è 
        ciclico, quindi è isomorfo a $\Z2\times\Z2$;
        \item se $\sqrt{b\Delta} \in \QQ$ (e di conseguenza $\sqrt b \notin \QQ)$
        scriviamo $b = \Delta q^2$, con $q \in \QQ$. Ragionando allo stesso
        modo e con le stesse notazioni si ha che $\varphi_2$ ha ordine 2 e 
        \[
            \varphi_3^2(x_1) = \varphi_3(x_3) = 
            \varphi_3\left(\frac{\sqrt b}{x_1}\right) = 
            \frac{\varphi_3(\sqrt\Delta q)}{\varphi_3(x_1)} = 
            q\frac{\sqrt\Delta}{x_3}
        \]
        Poiché $\sqrt\Delta = 2x_1^2 + a$ si ha $\varphi_3(\sqrt\Delta) =
        2x_3^2 + a = -\sqrt\Delta$, pertanto
        \[
            \varphi_3^2(x_1) = -q\frac{\sqrt\Delta}{x_3} = 
            -\frac{\sqrt b}{x_3} = -x_1
        \]
        quindi $\varphi_3$ ha ordine 4. Allora $\Gal(K/\QQ)$ è ciclico,
        in particolare è isomorfo a $\Z4$.
    \end{enumerate}
\end{proof}

\newpage

\subsection{Contare le sottoestensioni quadratiche di un campo}

Consideriamo un'estensione di Galois finta $\faktor F K$, sia $G = \Gal(F/K)$,
le sottoestensioni di $F$ di grado due su $K$ sono in corrispondenza con i
sottogruppi di $G$ di indice 2. Osserviamo che un sottogruppo $H \leqslant G$
di indice 2 contiene il sottogruppo 
\[
    \mathscr{G} = \langle g^2\mid g \in \nolinebreak 
G\rangle\footnote{
    Stiamo usando la notazione moltiplicativa. In notazione additiva
    allora $\mathscr{G} = \langle 2g \mid g \in G\rangle$.
}
\]
Infatti se consideriamo il quoziente $\faktor G H \cong \Z2$, per ogni $g \in G$
si ha
\[
    (gH)^2 = g^2H = H
\]
da cui $g^2 \in H$ e quindi anche $\mathscr{G} \leqslant H$. Per il 
Teorema di Corrispondenza tra sottogruppi abbiamo una bigezione 
\[
    \{H \leqslant G \mid [G : H] = 2\} \longleftrightarrow \left\{\mathcal{H}
    \leqslant\faktor{G}{\mathscr{G}}\Bigm| \left[\faktor{G}{\mathscr{G}}:
    \mathcal{H}\right] = 2\right\}
\]

Gli elementi di $\faktor{G}{\mathscr{G}}$ hanno ordine al più 2. Questo 
implica che sia un gruppo abeliano, infatti per ogni $a, b \in \faktor{G}{\mathscr{G}}$
si ha 
\[
    aba^{-1}b^{-1} = abab = (ab)^2 = e
\]
pertanto $ab = ba$. Essendo un gruppo finito per il Teorema di Struttura dei
Gruppi Abeliani Finiti si ha $\faktor{G}{\mathscr{G}} \cong (\Z2)^k$,
dove $k$ è un parametro che dipende da $G$. Il numero di sottogruppi di 
indice $2$ di $(\Z2)^k$ è uguale al numero di iperpiani dello spazio vettoriale
$(\FF_2)^k$, quindi $2^k - 1$, e questo è il numero di sottoestensioni di $F$
quadratiche su $K$. 

\begin{example}
    Sia $F = \QQ(i, \zeta_3, \sqrt[3]3)$, si verifica che il gruppo di Galois
    dell'estensione $\faktor{F}{\QQ}$ è isomorfo a $G = S_3\times\Z2$.
    Il sottogruppo $\langle g^2 \mid g \in G\rangle$ è isomorfo al sottogruppo
    $\mathscr{G} = \mathcal{A}_3 \times \{0\}$, pertanto $\faktor{G}{\mathscr{G}}
    \cong \Z2\times\Z2$, che contiene tre sottogruppi di indice 2. Quindi 
    $F$ contiene tre sottoestensioni quadratiche su $\QQ$, che sono 
    $\QQ(\zeta_3) = \QQ(\sqrt{-3})$, $\QQ(i)$, $\QQ(\sqrt 3)$.
\end{example}

\begin{remark}
    Possiamo ripetere la costruzione di sopra per cercare i sottogruppi 
    normali di indice $k$, in quanto questi contengono il sottogruppo
    $\langle g^k \mid g \in G\rangle$, ma la caratterizzazione del quoziente
    è più complicata in generale.
\end{remark}

\newpage

\subsection{Radici dell'unità}

Consideriamo un'estensione di Galois finita $\faktor{K}{\QQ}$ con gruppo di 
Galois $G$, sia $\zeta_n$ una radice primitiva $n$-esima dell'unità, vogliamo
capire come determinare se $\zeta_n$ è contenuta in $F$ al variare di $n \in \NN$.

\begin{center}
    \begin{tikzpicture}
        \node (Q1) at (0,0) {$\QQ$};
        \node (Q2) at (0,1.5) {$F = K \cap \QQ(\zeta_n)$};
        \node (Q3) at (-1.5,3) {$K$};
        \node (Q4) at (1.5,3) {$\QQ(\zeta_n)$};

        \draw (Q1)--(Q2);
        \draw (Q2)--(Q3);
        \draw (Q2)--(Q4);
    \end{tikzpicture}
\end{center}

Il gruppo $\Gal(F/\QQ)$ è un sottogruppo di $\Gal(\QQ(\zeta_n)/\QQ)$, che è
isomorfo a $\Zn^*$ per il \hyperref[teorema3.9]{Teorema 3.9}, in particolare
$\Gal(F/\QQ)$ è un gruppo abeliano e $\faktor{F}{\QQ}$ è un'estensione normale
in quanto tutte le sottoestensioni di $\QQ(\zeta_n)$ sono normali. D'altra 
parte $\Gal(F/\QQ)$ è isomorfo a un quoziente $\faktor G H$, con $H \trianglelefteqslant G$,
in particolare $H$ deve contenere il sottogruppo derivato $G'$ in quanto
il quoziente è abeliano (\hyperref[prop1.35]{Proposizione 1.35}).

\begin{example}
    Per $n\geq 3$, sia $f(x) \in \QQ[x]$ che il gruppo di Galois del suo campo di spezzamento
    $K$ su $\QQ$ è isomorfo a $S_n$, consideriamo i sottogruppi normali $H$ di $S_n$
    che contengono $S_n'$. Poiché $S_n' = \mathcal{A}_n$ tali sottogruppi
    sono solo $\mathcal{A}_n$ e $S_n$. Se $H = S_n$ allora $\QQ(\zeta_n) \cap K = \nolinebreak\QQ$
    in quanto $\faktor{\Gal(K/\QQ)}{H}$ è il gruppo banale, quindi le uniche
    radici dell'unità contenute in $K$ sono 1 e $-1$ (che sono rispettivamente
    $\zeta_1$ e $\zeta_2$). Se $H = \mathcal{A}_n$
    allora $[\QQ(\zeta_n)\cap K: \QQ] = 2$. Poiché una sottoestensione di 
    $\QQ(\zeta_n)$ è della forma $\QQ(\zeta_d)$ con $d \mid n$, abbiamo
    che $[\QQ(\zeta_d):\QQ] = 2$. I possibili $d$ sono quindi da determinare
    tra le soluzioni dell'equazione $\phi(m) = 2$, cioè $d \in {3, 4, 6}$.
    In particolare le uniche radici dell'unità non banali che possono essere
    contenute in $K$ sono $\zeta_3$, $\zeta_4$ e $\zeta_6$, e quali di queste
    sono effettivamente elementi del campo dipende dalle radici del polinomio $f(x)$.
\end{example}

Consideriamo adesso le estensioni ciclotomiche $\QQ(\zeta_p)$ con $p$ primo,
vogliamo determinare quando una sottoestensione quadratica di $\QQ(\zeta_p)$
è reale oppure no. Il gruppo di Galois dell'estensione è isomorfo a $\Zp^* \cong 
\Z{(p -1)}$, che è un gruppo ciclico, quindi $\QQ(\zeta_n)$ contiene un'unica 
sottoestensione quadratica su $\QQ$. Osserviamo che l'insieme dei quadrati 
di $\Zp^*$ è un sottogruppo di indice 2, in quanto la mappa 
\[
    \varphi: \Zp^* \longrightarrow \Zp^*: x \longmapsto x^2
\]
è un omomorfismo di gruppi, essendo $\Zp^*$ abeliano, e il suo nucleo è $\{1, -1\}$,
pertanto $|\mathrm{Im}\varphi| = \displaystyle\frac{p - 1}{2}$. Siano $K = \QQ(\zeta_p)\cap \RR$,
$F_2$ l'unica sottoestensione quadratica di $\QQ(\zeta_p)$, $K$ è il campo 
fissato dal coniugio complesso, pertanto corrisponde al sottogruppo
$\langle -1\rangle \leqslant \Zp^*$, mentre $F_2$ corrisponde al sottogruppo
dei quadrati di $\Zp^*$. Allora $F_2 \subseteq K$ se e solo se $-1$ è un 
quadrato in $\Zp^*$, cioè se e solo se $p \equiv 1 \pmod 4$.

Più in generale vale il seguente teorema, di cui non diamo la dimostrazione.

\begin{theorem}
    Dato $p$ un primo dispari, l'unica sottoestensione di $\QQ(\zeta_p)$ 
    quadratica su $\QQ$ è
    \begin{enumerate}[(1)]
        \item $\QQ(\sqrt p)$ se $p \equiv 1 \pmod 4$;
        \item $\QQ(\sqrt{-p})$ se $p \equiv 3 \pmod 4$.
    \end{enumerate}
\end{theorem}   

\end{document}