\documentclass[11pt]{scrartcl}
\usepackage[italian]{babel}
\usepackage[sexy]{evan}


\begin{document}
\title{Appunti Algebra 1}
\subtitle{\large\normalfont\rmfamily\scshape APPUNTI DEL CORSO DI ALGEBRA 1 TENUTO\\ DALLA PROF. DEL CORSO E DAL PROF. LOMBARDO}
\author{Diego Monaco \\ \textnormal{\href{d.monaco2@studenti.unipi.it}{d.monaco2@studenti.unipi.it}}}
\date{Anno Accademico 2022-23}
\maketitle
\newpage

\tableofcontents
\eject
\newpage

\section*{Ringraziamenti}
\textbf{Davide Ranieri}, Federico Allegri, Pietro Crovetto, \textbf{Francesco Sorce}, Leonardo Migliorini, Matteo Gori, Daniele Lapadula, Alessandro Fenu,
\textbf{Leonardo Alfani}, Clementina Salamina, Giorgia Capecchi, Gianni Bellu, Carlo Rotolo, Lorenzo Picinelli, Alessandro Moretti.

\newpage
\section{Gruppi}
\subsection{Automorfismi}
Dato un gruppo $G$ possiamo definire l'insieme degli automorfismi di $G$ come segue:
    \[ \Aut(G) = \{\varphi : G \longrightarrow G |\, \varphi \, \text{isomorfismo}\}
        \]
si verifica facilmente che $(\Aut(G), \circ)$ è un gruppo, e in particolare $\Aut(G) \leqslant S(G)$,
ovvero il gruppo delle permutazioni di $G$. Si osserva che $id \in \Aut(G)$, $\varphi \in \Aut(G) \implies 
\varphi^{-1} \in \Aut(G)$ e $\varphi,\psi \in \Aut(G) \implies \varphi \circ \psi \in \Aut(G)$.

\begin{example}
    [Esempi di automorfismi]
    Esempi di insiemi di automorfismi:
        \begin{itemize}
            \item $\Aut(\ZZ) = \{\pm id\}$.
            \item $\Aut(\Zn) \cong \Zn^*$.
            \item $\Aut(\Z2 \times \Z2) \cong S_3$.
            \item $\Aut(\underbrace{\Zp \times \ldots \times \Zp}_{n \;\text{volte}}) \cong GL_n(\Fp)$
        \end{itemize}
\end{example}

\subsection{Automorfismi interni}
\begin{definition}
    Dato un gruppo $G$ possiamo definire l'omomorfismo di \vocab{coniugio}:
        \[ \varphi_g : G \longrightarrow G : x \longmapsto gxg^{-1}
            \]
    dove l'elemento $gxg^{-1}$ si dice \vocab{coniugato} di $g$.
\end{definition}

\begin{proposition}
    \label{prop1}
    Valgono i seguenti fatti:
    \begin{enumerate}[(1)]
        \item $\varphi_g \in \Aut(G)$, $\forall g \in G$.
        \item $\{\varphi_g | g \in G\} = \Inn(G) \trianglelefteqslant \Aut(G)$.\footnote{$\Inn(G)$ si definisce \vocab{gruppo degli automorfismi interni}.}
    \end{enumerate}
\end{proposition}

\begin{proof}
    Proviamo le due affermazioni:
        \begin{enumerate}[(1)]
            \item Per verificare che $\varphi_g$ è un automorfismo bisogna verificare che $\varphi_g$ è ben definita, ma ciò segue dalla chiusura di $G$ per l'operazione.
                Verifichiamo che sia un omomorfismo:
                    \[ \varphi_g(xy) = g x y g^{-1} = gxg^{-1}gyg^{-1} = \varphi_g(x)\varphi_g(y)
                    \qquad \forall x,y \in G 
                        \]
                ci resta da verificare che sia una bigezione. Partiamo dalla surgettività, vogliamo verificare che $\forall y \in G$, $\exists x \in G :$
                    \[ \varphi_g(x) = y
                        \]
                in tal caso basta prendere $x = g^{-1}yg \in G$. Per l'iniettività si osserva:
                    \[ \ker \varphi_g = \{x \in G | \varphi_g(x) = e\} = \{x \in G | gxg^{-1} = e \iff x = e\} = \{e\}
                        \]
                pertanto $\varphi_g$ è iniettivo.
            \item Verifichiamo che $\Inn(G) \trianglelefteqslant \Aut(G)$; mostriamo prima che $\Inn(G)$ è un sottogruppo di $\Aut(G)$, infatti: 
            $id = \varphi_e \in \Inn(G)$, $\forall g_1,g_2 \in G$ vale che $\varphi_{g_{1}} \circ \varphi_{g_{2}} = \varphi_{g_1g_2} \in \Inn(G)$, infatti:
                    \[ \varphi_{g_{1}} \circ \varphi_{g_{2}}(x) = \varphi_{g_{1}}(g_2xg_2^{-1}) = g_1g_2xg_2^{-1}g_1^{-1} = \varphi_{g_1g_2}(x)
                        \]
                infine, $(\varphi_{g})^{-1} = \varphi_{g^{-1}} \in \Inn(G)$:
                    \[ (\varphi_{g})^{-1} \circ \varphi_{g}(x) = (\varphi_{g})^{-1} (gxg^{-1}) = x \iff (\varphi_{g})^{-1} = \varphi_{g^{-1}}
                        \]
                e analogamente per l'inversa a destra. Per verificare la normalità bisogna mostrare che:
                    \[ f \circ \Inn(G) \circ f^{-1} \subseteq \Inn(G)
                    \qquad \qquad \forall f \in \Aut(G)
                        \]
                ovvero:
                    \[ f \circ \varphi_g \circ f^{-1} \in \Inn(G)
                    \qquad \qquad \forall f \in \Aut(G), \forall \varphi_g \in \Inn(G)
                        \]
                si osserva che $f \circ \varphi_g \circ f^{-1} = \varphi_{f(g)} \in \Inn(G)$, infatti:
                    \[ f \circ \varphi_g \circ f^{-1} (x) = f(\varphi_g(f^{-1} (x))) = f(g(f^{-1}(x))g^{-1}) =
                        \]\[ = f(g) f(f^{-1}(x)) f(g^{-1}) = f(g) x (f(g))^{-1} = \varphi_{f(g)}(x)
                            \]
        \end{enumerate}
\end{proof}

\begin{remark}
    Se $G$ è abeliano, allora $\Inn(G) = \{id\}$, infatti:
        \[ gxg^{-1} = gg^{-1}x = x \qquad \qquad \forall x \in G, \forall g \in G
            \]
\end{remark}

\begin{proposition}
    Dato un gruppo $G$ si ha:
    \[\Inn(G) \cong \faktor{G}{Z(G)}\]
\end{proposition}

\begin{proof}
    Per dimostrare il teorema ci basta trovare un omomorfismo surgettivo da $G$ in $\Inn(G)$ e poi sfruttare il Primo Teorema di Omomorfismo. Sia:
        \[ \phi : G \longrightarrow \Inn(G) : g \longmapsto \varphi_g
            \]
    tale applicazione è chiaramente ben definita, ed è surgettiva per come abbiamo definito $\Inn(G)$. Verifichiamo che è un omomorfismo:
        \[ \phi(g_1g_2) = \varphi_{g_1g_2} = \varphi_{g_1} \circ \varphi_{g_2} = \phi(g_1) \circ \phi(g_2)
        \qquad \qquad \forall g \in G
            \]
        dove la penultima uguaglianza è vera per quanto visto nella dimostrazione del (2) della proposizione precedente. A questo punto, per il 
        primo teorema di omomorfismo si ha che:
            \begin{center}
            \begin{tikzcd}
                G \arrow[d, "\pi_{\ker{\phi}}" left, twoheadrightarrow] \arrow[r, "\phi", twoheadrightarrow] &\Inn(G)\\	
                \faktor {G}{\ker{\phi}} \arrow[ur, "\widesim{}"' above, sloped, anchor=center] & 
            \end{tikzcd}
            \end{center}
        dunque:
            \[ \frac{G}{\ker{\phi}} \cong \Inn(G)
                \]
        non ci resta che osservare:
            \begin{multline*}
                \ker{\phi} = \{g \in G | \phi(g) = \varphi_g = id\} = \{g \in G | gxg^{-1} = x, \forall x \in G\} = \\ 
                = \{g \in G | gx = xg, \forall x \in G\} = Z(G)
            \end{multline*}
\end{proof}

\begin{remark}
    L'isomorfismo trovato è del tipo $gZ(G) \longmapsto \varphi_g$, ricordiamo che è ben definito per il Primo Teorema di Omomorfismo.
\end{remark}

\begin{remark}
    Si ricorda che se $\faktor{G}{Z(G)}$ è ciclico, allora $G$ è abeliano (e quindi $\faktor{G}{Z(G)}$ è banale), infatti, sia:
        \[ \faktor{G}{Z(G)} = \left<gZ(G)\right>
            \]
    Presi $g_1,g_2 \in G$, si ha che $g_1Z(G) = g^{k_1}Z(G)$ e $g_2Z(G) = g^{k_2}Z(G)$, da cui:
        \[ g^{-k_1}g_1Z(G) = Z(G) \iff  g^{-k_1}g_1 \in Z(G)
            \]
    ovvero $\exists z_1 \in Z(G): g_1 = g^{k_1}z_1$ e analogamente $g_2 = g^{k_2}z_2$, da cui:
        \[ g_1g_2 = g^{k_1}z_1g^{k_2}z_2 = g^{k_1}g^{k_2}z_1z_2 = g^{k_1+k_2}z_1z_2
            \]
    e contemporaneamente:
        \[ g_2g_1 = g^{k_2}z_2g^{k_1}z_1 = g^{k_2}g^{k_1}z_2z_1 = g^{k_2 + k_1}z_2z_1 = g^{k_1+k_2}z_1z_2
            \]
    dove nell'ultimo passaggio si è sfruttato il fatto che $k_1,k_2 \in \ZZ$ e $z_1,z_2 \in Z(G)$. Da ciò segue che $G$ è abeliano.
\end{remark}

\begin{remark}
    Dunque $\Inn(G)$ ciclico $\implies \faktor{G}{Z(G)}$ ciclico $\implies G$ abeliano da cui:  
        \[ \Inn(G) \cong \faktor{G}{Z(G)} \cong \{e\}
            \]
\end{remark}

\begin{remark}
    $N \trianglelefteqslant G \iff \forall \varphi_g \in \Inn(G)$ si ha $\varphi_g(N) = N$ (o anche $\varphi_g(N) \subseteq N$). Equivalentemente, i sottogruppi
    normali di $G$ sono i sottogruppi \textbf{invarianti} per automorfismi interni (ovvero sono tali che $gNg^{-1} = N$, $\forall g \in G$). Se $N \trianglelefteqslant G$, si può considerare:
        \[ \Inn(G) \longrightarrow \Aut(N) : \varphi_g \longmapsto \varphi_{g \mid N}
            \]
    con $\varphi_{g \mid N} : N \longrightarrow N$ che è un automorfismo, infatti rimane iniettivo, la surgettività segue dal fatto che $\varphi_g(N) = N$, e infine, essendo $\varphi_g$ 
    un omomorfismo su tutti gli elementi di $G$, lo sarà in particolare anche su tutti gli elementi di $N$. Dunque quando si ha un sottogruppo normale, ogni automorfismo interno si restringe
    a un automorfismo di $N$.
\end{remark}

Abbiamo visto che i sottogruppi normali sono invarianti per automorfismi interni, possiamo generalizzare quest'idea e considerare i sottogruppi invarianti per automorfismi:

\begin{definition}
    Dato un sottogruppo $H \leqslant G$, esso si dice \vocab{caratteristico} se è invariante per automorfismi:
        \[ f(H) = H
        \qquad \qquad \forall f \in \Aut(G)
            \]
\end{definition}

Anche in questo caso basta verificare che $f(H) \subseteq H$, $\forall f \in \Aut(G)$, perché si ha anche che:
    \[ f^{-1}(H) \subseteq H
        \]
da cui si ottiene:
    \[ f(f^{-1}(H)) \subseteq f(H)
        \]

\begin{remark}
    Si osserva che se $H$ è caratteristico in $G$, allora è invariante per tutti gli automorfismi di $G$ (e quindi in particolare quelli interni), dunque
    se $H$ è caratteristico in $G$, allora è anche normale. Il viceversa è falso.
\end{remark}

\begin{remark}
    Se $H$ è caratteristico in $G$ (dunque normale), si può scrivere un'applicazione:
        \[ \Aut(G) \longrightarrow \Aut(H) : f \longmapsto f_{\mid H}
            \]
    dove $f_{\mid H}$ è un automorfismo di $H$.
\end{remark}

\begin{remark}
    Si osserva che se $H$ è l'unico sottogruppo di $G$ di un certo ordine, allora $H$ è caratteristico in $G$ (segue immediatamente dal fatto che gli automorfismi
    preservano gli ordini degli elementi). In modo analogo, se $H$ è caratterizzato da una proprietà invariante per automorfismo, allora è caratteristico.
\end{remark}

\begin{exercise}
    Il centro di un gruppo $Z(G)$ è un sottogruppo caratteristico.
\end{exercise}
        
\begin{soln}
    Per dimostrare che $Z(G)$ è caratteristico è sufficiente far vedere che:
        \[ f(Z(G)) \subseteq Z(G)
        \qquad \forall f \in \Aut(G)
            \]
    ovvero:
        \[ f(z) \in Z(G)
        \qquad \forall f \in \Aut(G), \forall z \in Z(G)
            \]
    dunque bisogna verificare che:
        \[ gf(z) = f(z)g \qquad \forall g \in G
            \]
    poiché $f$ è un automorfismo, allora $\exists h \in G : f(h) = g$, dunque:
        \[ gf(z) = f(h)f(z) = f(hz) = f(zh) = f(z)f(h) = f(z)g \qquad \forall g \in G
            \]
\end{soln}

\begin{example}
    Sia $G = \Z2 \times \Z2 = \{(\overline 0, \overline 0),(\overline 1, \overline 0),(\overline 0, \overline 1),(\overline 1, \overline 1)\}$, $G$ ha ordine $4$ ed ha tre sottogruppi ciclici di ordine 2:
        \[ H_1 = \left<(\overline 1, \overline 0)\right> \qquad H_2 = \left<(\overline 0, \overline 1)\right> \qquad H_3 = \left<(\overline 1, \overline 1)\right>
            \] 
    ed essendo $G$ abeliano si ha $H_1,H_2,H_3 \trianglelefteqslant G$ (e quindi i sottogruppi sono invarianti per automorfismi interni). Tuttavia nessuno dei sottogruppi è caratteristico,
    infatti possiamo prendere un automorfismo non banale (e quindi non uno interno) e vedere come i sottogruppi di questo tipo non siano invarianti:
        \[ f = \begin{cases}
            (\overline 1, \overline 0) \longmapsto (\overline 1, \overline 1)\\
            (\overline 0, \overline 1) \longmapsto (\overline 0, \overline 1)
        \end{cases}
            \]
    la definizione della mappa data tuttavia non è completa, perché abbiamo stabilito solo dove vengono mandati i generatori, dobbiamo definire cosa faccia un elemento generico:
        \[ f ((\overline a, \overline b)) = af((\overline 1, \overline 0)) + bf((\overline 0, \overline 1)) = (\overline a, \overline a) + (\overline 0, \overline b) = (\overline a, \overline{a + b})
            \]
    a questo punto abbiamo definito completamente l'applicazione (rimarrebbe da verificare che $f$ sia un omomorfismo),  e si verifica facilmente che
     $f(H_1) = H_3 $ quindi $H_1 \trianglelefteqslant G$, ma non caratteristico.
\end{example}

A questo punto è facile verificare che:
    \[ \Aut(\Z2 \times \Z2) \cong S_3
        \]
infatti, ogni automorfismo del gruppo si ottiene fissando l'elemento neutro $(\overline 0, \overline 0) \longmapsto (\overline 0, \overline 0)$, quindi il numero possibile di bigezioni è al più $3!$, occorre
verificare che tutte e $6$ le funzioni sono omomorfismi. Dimostriamo invece che:
    \[ \boxed{\Aut(S_3) \cong S_3}
        \]
Per farlo, poiché $S_3$ non è abeliano, possiamo osservare che:
    \[ \Inn(S_3) \cong \faktor{S_3}{Z(S_3)} \cong S_3
        \]
in quanto l'unico elemento che commuta con tutti gli altri in $S_3$ è l'identità, quindi $Z(S_3) = \{id\} \cong \{e\}$.
Per quanto detto si ha $\Inn(S_3) \trianglelefteqslant \Aut(S_3)$ e quindi $\Aut(S_3)$ contiene una copia isomorfa di $S_3$ come sottogruppo normale, pertanto,
se verifichiamo che $|\Aut(S_3)| \leq 6$ abbiamo concluso. Sia $f \in \Aut(S_3)$, $f$ può al più scambiare i $3$ elementi di ordine $2$, d'altra parte, fissate le 
immagini di $\tau_1,\tau_2,\tau_3$\footnote{Con $\tau_i$ si intendono le trasposizioni che lasciano fisso l'elemento $i$.}, i due 
$3$-cicli\footnote{Come si vedrà $S_3 = \left<\tau_1,\tau_2,\tau_3\right>$} sono completamente determinati,
ciò significa che si hanno al più $3!$ automorfismi, dunque:
    \[ \Aut(S_3) = \Inn(S_3) \cong S_3 \implies \Aut(S_3) \cong S_3 
        \]

\newpage
\subsection{Azione di un gruppo su un insieme}

\begin{definition}
    Sia $G$ un gruppo e $X$ un insieme, un'\vocab{azione} di $G$ su $X$ è un omomorfismo:
        \[ \varphi : G \longrightarrow S(X) : g \longmapsto \varphi_g 
            \]
    dove $\varphi_g : X \longrightarrow X : x \longmapsto \varphi_g(x)$\footnote{Alternativamente si può indicare l'immagine con $\varphi_g : x \longmapsto g \ast x$ dove il simbolo $\ast$ indica l'azione di $g$ su $x$.},
     con $\varphi_g$ bigettiva, $\forall g \in G$. Si può definire un'azione anche come:
        \[ \varphi : G \times X \longrightarrow X : (g,x) \longmapsto \varphi_g(x)
            \]
    Un'azione di $G$ su $X$ si indica con $G \circlearrowleft X$.
\end{definition}

\begin{example}
    Sia $X = G$, quindi $\varphi : G \longrightarrow S(G) : g \longmapsto \varphi_g$, con $\varphi_g$ coniugio, $\varphi$ è un'azione. Come si è visto nell'($1$)
    della \hyperref[prop1]{Proposizione 1.3} $\varphi_g$ è un automorfismo di $G$ (e quindi una bigezione), e $\varphi$ è un omomorfismo. In questo caso si ha che:
        \[ \varphi_g(x) = gxg^{-1}
            \]
\end{example}

\begin{example}
    Sia $V$ un $K$-spazio vettoriale, sia:
        \[ \varphi : K^* \longrightarrow S(V) : \lambda \longmapsto \varphi_\lambda
            \]
    con $\varphi_\lambda : V \longrightarrow V : \underline v \longmapsto \lambda \underline v$, $\varphi$ è un'azione di $K^*$ su $V$.
\end{example}

Sia $\varphi: G \longrightarrow S(X)$ un'azione, $\varphi$ definisce una relazione di equivalenza su $X$:
    \[ x \sim y \iff \exists g \in G : \varphi_g(x) = y
        \]
    ovvero due elementi sono in relazione se esiste un'applicazione $\varphi_g \in S(X)$, per cui un elemento è l'immagine dell'altro mediante 
    tale applicazione. La relazione è appunto di equivalenza, infatti: $x \sim x$, per $g = e$ si ha (essendo $\varphi$ un omomorfismo) $\varphi_e(x) = id(x) = x$, 
    $x \sim y \implies y \sim x$:
        \[ \varphi_g(x) = y \implies x = (\varphi_g(y))^{-1} = \varphi_{g^{-1}}(y)
            \]
    infine $x \sim y$, $y \sim z \implies x \sim z$, infatti si avrebbe: $\varphi_g(x) = y$, $\varphi_h(y) = z$ da cui:
        \[ z = \varphi_h(\varphi_g(x)) = \varphi_{hg}(x) \implies x \sim z
            \]

\begin{definition}
    Data la relazione di equivalenza $\sim$ si definiscono \vocab{orbite} le classi di equivalenza di $X$ rispetto alla relazione $\sim$:
        \[ \Orb(x) = \{\varphi_g(x) | g \in G\} (\subseteq X)
            \]
\end{definition}
Da cui:
    \[ X = \bigcupdot_{x \in \mathcal{R}} \Orb(x)
        \]
Con $\mathcal{R}$ insieme di rappresentanti. Un'orbita è quindi l'insieme di tutte le immagini di un elemento in un insieme, mediante tutte le possibili 
applicazioni (permutazioni) dell'insieme $\varphi(G)$.

\begin{definition}
    Per ogni $x \in X$ si dice \vocab{stabilizzatore} di $x$:
        \[ \St(x) = \{g \in G | \varphi_g(x) = x\}
            \]
\end{definition}
Cioè lo stabilizzatore è l'insieme degli elementi di $G$, che danno origine mediante $\varphi$ alle applicazioni $\varphi_g \in S(X)$, che lasciano fisso un determinato elemento.

\begin{example}
    Se $X = \RR^2$ e $G$ è il gruppo di traslazioni di vettore $\ul v = (0, l)$, allora:
        \[ \varphi : G \longrightarrow S(X) : \tau_{(0,l)} \longmapsto \tau_{(0,l)} \footnote{Si osserva che il primo $\tau_{(0,l)}$ è un elemento del gruppo $G$, mentre il secondo è un'applicazione bigettiva di $X$.}
            \]
    con:
        \[ \Orb(x,y) = \{(x,y + l) | l \in \RR\}
        \quad \text e \quad
        \St(x,y) = \{\tau_{(0,l)} | (x, y + l) = (x,y)\}= \{id\}
            \]
\end{example}

\begin{example}
    Se $X = \RR^2$ e $G$ è il gruppo delle rotazioni di centro $O$, allora:
        \[ \varphi: G \longrightarrow S(\RR^2) : r_\theta \longmapsto r_\theta
            \]
    con:
        \[ \St(x,y) =
        \begin{cases}
            \{id\} & \text{se $(x,y) \ne (0,0)$}\\
            G & \text{se $(x,y) = (0,0)$}
        \end{cases}
            \]
    e, detta $\omega$ la circonferenza di centro $O$ e raggio $\sqrt{x^2+y^2}$:
        \[ \Orb(x,y) = \{(x^{\prime},y^{\prime}) \in \RR^2 | (x^{\prime}, y^{\prime}) \in \omega\}
            \]
\end{example}

\begin{proposition}
    [$\St(x) \leqslant G$]
    Dato un gruppo $G$ e un'azione $\varphi : G \longrightarrow S(X)$, si ha che $\St(x) \leqslant G$.\footnote{In generale lo
     stabilizzatore non è un sottogruppo normale.}
\end{proposition}

\begin{proof}
    Si osserva che $e \in \St(x)$, in quanto $\varphi_e(x) = id(x) = x$, inoltre, presi $g,h \in \St(x)$, ovvero $\varphi_g (x) = \varphi_h(x) = x$, allora:
        \[ \varphi(gh)(x) = \varphi_{gh}(x) = \varphi_g \circ \varphi_h (x) = \varphi_g(\varphi_h(x)) = \varphi_g(x) = x \implies gh \in \St(x)
            \]
    dove si ha che $ \varphi_{gh}(x) = \varphi_g \circ \varphi_h (x)$ in quanto $\varphi$ è un omomorfismo.
    Infine, preso $g \in \St(x)$, si ha $g^{-1} \in \St(x)$, infatti $\varphi_g$ è bigettiva e quindi ammette inversa:
        \[ (\varphi_g)^{-1} \circ \varphi_g (x) = x \implies (\varphi_g)^{-1}(\varphi_g(x)) = x \implies (\varphi_g)^{-1}(x) = x
            \]
    con $(\varphi_g)^{-1}(x) = (\varphi(g))^{-1}(x) = (\varphi(g^{-1}))(x) = \varphi_{g^{-1}}(x)$ e per quanto detto:
        \[ \varphi_{g^{-1}} (x) = x \implies g^{-1} \in \St(x)
            \]
\end{proof}

\pagebreak
\begin{remark}[$\Orb(x) \longleftrightarrow \lbrack G : \St(x) \rbrack$]
    Sia $x \in X$ e $g,h \in G$, allora:
        \[ \varphi_g(x) = \varphi_h(x) \iff \varphi_{h^{-1}}(\varphi_g(x)) = x
            \]
    e per le proprietà di omomorfismo dell'azione $\varphi$, si ha:
        \[ \varphi_{h^{-1}}(\varphi_g(x)) = x \iff \varphi_{h^{-1}g}(x) = x \iff h^{-1}g \in \St(x)
            \]
    ovvero $g \St(x) = h \St(x)$, in quanto $\St(x) \leqslant G$ e la condizione ottenuta è esattamente quella dell'equivalenza modulo $\St(x)$,
    quindi:
    \[ \Orb(x) \longleftrightarrow \text{classi laterali di $\St(x)$ in $G$} = [G : \St(x)]
        \]
    cioè due elementi danno la stessa immagine (di un fissato elemento $x \in X$) se e solo se stanno nella stessa classe laterale modulo $\St(x)$, e la corrispondenza biunivoca tra orbita e classi
    laterali è data da:
    \[ g\St(x) \longmapsto \varphi_g(x) \qquad \text e \qquad h\St(x) \longmapsto \varphi_h(x)
                \]
    che è ben definita e per quanto detto all'inizio è iniettiva:
        \[ \varphi_g(x) = \varphi_h(x) \iff g\St(x) = h\St(x)
            \]
    (quindi due elementi di un'orbita sono uguali se e solo se lo sono le classi laterali dei rispettivi elementi che generano le applicazioni sono uguali modulo $\St(x)$, dunque per ogni elemento
    dell'orbita c'è una e una sola classe laterale modulo $\St(x)$) e surgettiva:
        \[ \forall y \in \Orb(x), y = \varphi_g(x) \implies g\St(x) \longmapsto y
            \]
    e quindi concludiamo che il numero di classi laterali di $\St(x)$ in $G$ è lo stesso della cardinalità di $\Orb(x)$.
\end{remark}

Per quanto detto si ha:
    \[ |G| = |\St(x)| [G : \St(x)]
        \]
ma $[G : \St(x)]$ è il numero di classi laterali di $\St(x)$ in $G$, che è proprio uguale a $|\Orb(x)|$ pertanto vale la seguente:

\begin{proposition}
    [Lemma orbita-stabilizzatore]
    \label{p:1.25}
    Sia $G$ un gruppo finito e $X$ un insieme, allora:
        \[ |G| = |\Orb(x)||\St(x)|
            \qquad \forall x \in X
            \]
\end{proposition}

\begin{remark}
    Si osserva che essendo $\St(x) \leqslant G$, allora è ovvio (per Lagrange) che $|\St(x)| \mid |G|$, tuttavia, per la proposizione precedente, si ha che:
    $|\Orb(x)| \mid |G| $ con $\Orb(x) \subseteq X$.
\end{remark}

\pagebreak
Ricordando che:
    \[ X = \bigcupdot_{x \in \mathcal R} \Orb(x)
        \]
se $|X|<+\infty$ si ha:
    \[ \colorboxed{red}{
        |X| = \sum_{x \in \mathcal{R}} |\Orb(x)| = \sum_{x \in \mathcal{R}} \frac{|G|}{|\St(x)|}
    }\]


\newpage
\subsection{Azione di coniugio}
\begin{definition}
    Si parla di \vocab{azione di coniugio}, quando si ha un'azione di $G$ su $G$ stesso:
        \[ \varphi : G \longrightarrow \Inn(G) (\trianglelefteqslant \Aut(G)) : g \longrightarrow \varphi_g
            \]
\end{definition}

Abbiamo già osservato che è un'azione (ovvero che $\varphi$ è un omomorfismo). In questo caso:
    \[ \Orb(x) = \{\varphi_g(x) | g \in G\} = \{ gxg^{-1} | g \in G\} = \Cl_G(x)
        \]
dove $\Cl_G(x)$ prende il nome di \vocab{classe di coniugio} di $x$\footnote{Si può indicare anche con $C_x$.}. Mentre:
    \[ \St(x) = \{g \in G | \varphi_g(x) = gxg^{-1} = x\} = \{g \in G | gx = xg\} = Z_G(x)
        \]
dove $Z_G(x)$ si dice \vocab{centralizzatore} di $x$. Per quanto detto in precedenza si ha:
    \[ |G| = |\Cl_G(x)||Z_G(x)|
        \]
In particolare $|\Cl_G(x)| \mid |G|$ e :
    \[ |G| = \sum_{x \in \mathcal{R}} |\Cl_G(x)| = \sum_{x \in \mathcal{R}} \frac{|G|}{|Z_G(x)|}
        \]

\begin{remark}
    $\Cl_G(x)$ è un sottoinsieme, non un sottogruppo di $G$, poiché non c'è mai l'identità.
\end{remark}

\begin{remark}
    Osserviamo che $Z_G(x) = G \iff x \in Z(G)$, infatti per un elemento del centro si ha che 
    $\forall g \in G$ l'elemento commuta, e dunque il suo centralizzatore è tutto il gruppo.
\end{remark}

\begin{remark}
    Per un'azione di coniugio si ha che $x \in Z(G)$ se e solo se $\Orb(x) = \{x\}$ e $\St(x) = G$ (ovvero $\varphi_g(x) = x$, $\forall g \in G$).
\end{remark}

    \[ |G| = \sum_{x \in Z(G)} \frac{|G|}{|Z_G(x)|} + \sum_{x \in \mathcal{R}\setminus Z(G)} \frac{|G|}{|Z_G(x)|}
        \]
ma, per quanto detto, se $x \in Z(G)$, allora $\displaystyle\frac{|G|}{|Z_G(x)|} = |\Cl_G(x)| = \{x\}$, segue dunque la relazione:
    \[ \colorboxed{red}{|G| = |Z(G)| + \sum_{x \in \mathcal{R}\setminus Z(G)} \frac{|G|}{|Z_G(x)|}}
        \]
che prende il nome di \vocab{formula delle classi} (di coniugio).

\newpage
\subsection{Applicazioni ai $p$-gruppi}
\label{pgruppi}
\begin{definition}
    Si definisce \vocab{$p$-gruppo} un gruppo di ordine $p^n$, con $p$ primo e $n \geq 1$.
\end{definition}

Se $G$ è un $p$-gruppo la formula delle classi diventa:
    \[ p^n = |G| = |Z(G)| + \sum_{x \in \mathcal{R}\setminus Z(G)} \frac{|G|}{|Z_G(x)|}
        \]
con $|Z(G)| = p^z$, $0 \leq z \leq n$, facciamo due osservazioni fondamentali:
    \begin{enumerate}[(1)]
        \ii Il centro di un $p$-gruppo non è mai banale, infatti, se osserviamo la formula delle classi, si ha:
            \[ p^n = |Z(G)| +  \sum_{x \in \mathcal{R}\setminus Z(G)} \frac{|G|}{|Z_G(x)|} \implies
             |Z(G)| +  \sum_{x \in \mathcal{R}\setminus Z(G)}\frac{|G|}{|Z_G(x)|} \equiv 0 \pmod p
                \]
            con $\displaystyle \frac{|G|}{|Z_G(x)|} > 1$, poiché $Z_G(x) = G$ se e solo se $x \in Z(G)$,
            viceversa deve essere che $\displaystyle \frac{|G|}{|Z_G(x)|} = p^{k_x}$, $k>0$, poiché $G$ è un $p$-gruppo (e quindi anche $Z_G(x)$), dunque:
                \[ |Z(G)| \equiv 0 \pmod p \implies |Z(G)| \geq 2
                    \]
            e quindi il centro di un $p$-gruppo non è mai banale.
        \ii Un gruppo di ordine $p^2$ è abeliano, infatti, si ha:
            \[ |G| = p^2 \implies |Z(G)| = \begin{cases}
                                                1 &\text{non può accadere per (1)} \\
                                                p &\text{no perché allora $G/Z(G)$ ciclico, ma $G$ non è abeliano}\\
                                                p^2
                                            \end{cases}
                \]
            dunque l'unica possibilità è che $Z(G) = G \iff G$ abeliano.
    \end{enumerate}

\newpage
\subsection{Teorema di Cauchy}

\begin{theorem}
    [Teorema di Cauchy]
    \label{p:Cauchy}
    Dato un gruppo $G$ e un primo $p$, se $p \mid |G|$, allora $\exists x \in G : \ord_G(x) = p$. \footnote{Si considera già noto il teorema per gruppi abeliani.}
\end{theorem}

\begin{proof}
    Sia $|G| = pn$, procediamo per induzione su $n$, nel caso $n = 1$ il teorema è ovvio. Supponiamo vera la tesi per i gruppi di ordine $pm$, con 
    $1 \leq m < n$ e proviamola per $n$. Distinguiamo due casi:
        \begin{itemize}
            \item Se esiste $H \lneq G$ con $p \mid |H|$, ovvero $|H| = pm \implies$ vale il teorema di Cauchy per ipotesi induttiva (essendo $m<n$), quindi 
            $\exists x \in H : \ord_H(x) = p$, ma essendo $H \subset G \implies x \in G$ e quindi la tesi è vera.
            \item Se $\forall H \lneq G$ si ha $p \nmid |H|$, allora si può applicare a $G$ la formula delle classi rispetto all'azione di coniugio:
                \[ pn = |G| = |Z(G)| + \sum_{x \in \mathcal{R}\setminus{Z(G)}} \frac{|G|}{|Z_G(x)|}
                    \]
                ricordando che il centralizzatore di $x$ è uno stabilizzatore (e quindi un sottogruppo di $G$), si ha $p \nmid |Z_G(x)|$, e quindi:
                    \[ p \Bigm| \sum_{x \in \mathcal{R}\setminus{Z(G)}} \frac{|G|}{|Z_G(x)|}
                        \]
                da cui segue che $p \mid |Z(G)| = \underbrace{|G|}_{= pn} - \sum pl_x$\footnote{Con $pl_x$ indichiamo le varie cardinalità del rapporto tra $|G|$ e $|Z_G(x)|$ al variare di $x$.}, per quanto premesso ($\forall H \lneq G$ si ha $p \nmid |H|$), ed essendo $Z(G) \leqslant G$, l'unica 
                possibilità è che $Z(G) = G$ e vale il teorema poiché è già stato dimostrato per il caso in cui $G$ è abeliano.
        \end{itemize}
\end{proof}

\newpage
\subsection{Azione di coniugio su un sottogruppo}
Sia $X = \{H \leqslant G\}$ e $\varphi : G \longrightarrow S(X) : g \longmapsto \varphi_g(X)$, con $\varphi_g : X \longrightarrow X : H \longmapsto gHg^{-1} = \{ghg^{-1} | h \in H\}$. 
    Si verifica facilmente che $\varphi$ è un omomorfismo; mostriamo invece che $\varphi_g$ è una permutazione (cioè bigettiva), per l'iniettività si osserva che:
        \[ \varphi_g(H) = \varphi_g(K) \iff gHg^{-1} = gKg^{-1} \iff H = K
            \]
    mentre per la surgettività si ha che $\forall H \in X, \exists L \in X$:
        \[ \varphi_g(L) = H \iff gLg^{-1} = H \iff L = g^{-1}Hg
            \]
    inoltre si ha anche:
        \[ \Orb(H) = \{\varphi_g(H) | g \in G\} = \{gHg^{-1}| g \in G\} \quad \St(H) = \{g \in G | \varphi_g(H) = H\} = N_G(H)
            \]
    dove $\Orb(H)$ è l'insieme dei coniugati di $H$, mentre $\St(H) = N_G(H)$ prende il nome di \vocab{normalizzatore} di $H$.

\begin{remark}
    Si osserva che $H \trianglelefteqslant G$ se e solo se $\Orb(H) = \{H\} \iff N_G(H) = G$, ovvero se $H$ è sempre chiuso per coniugio in $G$.
\end{remark}

Per quanto affermato nella \hyperref[p:1.25]{Proposizione 1.25} si ha:
    \[ |G| = |\Orb(H)||N_G(H)| \implies |\Orb(H)| = \frac{|G|}{|N_G(H)|}
    \]
\begin{remark}
    Quindi in generale, dato $H \leqslant G$ si ha che $\#\{gH\} = [G:H]$ e $\#\{gHg^{-1}\} = [G : N_G(H)]$.
\end{remark}

\begin{remark}
    [Sulla definizione di sottogruppo normale]
    I sottogruppi normali possono essere ridefiniti nella maniera seguente, $H \trianglelefteqslant G$ se e solo se:
        \[ H = \bigcup_{h \in H} \Cl_h
            \]
    cioè un sottogruppo è normale se e solo se è l'unione delle classi di coniugio dei suoi elementi. Infatti:
        \[ H \trianglelefteqslant G \iff ghg^{-1} \in H \qquad \forall h \in H, \forall g \in G
            \]
    che equivale a:
        \[ \Cl_h = \{ghg^{-1} | h \in H\} \subseteq H  \quad \forall h \in H \implies \bigcup_{h \in H} \Cl_h \subseteq H
            \]
    d'altra parte se $H$ è normale è chiuso per coniugio, ovvero il coniugio di ogni suo elemento è ancora in $H$
    ($ghg^{-1} = h^{\prime}$, $\forall h \in H$) e in particolare ciò significa che:
        \[ H \subseteq \bigcup_{h \in H} \Cl_h
            \] 
    
\end{remark}

\newpage
\subsection{Teorema di Cayley}

\begin{theorem}
    [Teorema di Cayley]
    \label{p:Cayley}
    Ogni gruppo è isomorfo ad un sottogruppo di un gruppo di permutazioni. In particolare, se $|G| = n$, allora 
    $G$ è isomorfo a un sottogruppo di $S_n$.
\end{theorem}

\begin{proof}
    Definiamo la mappa:
        \[ \lambda : G \varlonghookrightarrow S(G) : g \longmapsto \varphi_g
            \]
    con $\varphi_g : G \longrightarrow G : x \longmapsto gx$, l'applicazione $\lambda$ prende il nome di \vocab{rappresentazione regolare a sinistra} di $G$, si 
    vuole dimostrare che $\lambda$ è un omomorfismo iniettivo.
    Osserviamo innanzitutto che $\lambda$ è ben definita, cioè $\varphi_g \in S(G)$, infatti $\varphi_g$ è iniettiva (segue dalle leggi di cancellazione) e 
    surgettiva, perché $\forall y \in G$, $\exists g^{-1}y \in G : \varphi_g(g^{-1}y) = y$. Verifichiamo che $\lambda$ è un omomorfismo:
        \[ \lambda(g_1g_2) = \varphi_{g_1g_2}
            \]
    con $\varphi_{g_1g_2} (x) = \varphi_{g_1} \circ \varphi_{g_2} (x)$, $\forall x \in G$, e quindi:
        \[ \lambda(g_1g_2) = \lambda(g_1) \lambda(g_2)
        \qquad \forall g_1,g_2 \in G
            \]
    infine, per l'iniettività si ha che:
        \[ \ker \lambda = \{g \in G | \lambda(g) = \varphi_g = id = \varphi_e\} = \{e\}
            \]
    da ciò segue che $G \cong \Imm(G) \leqslant S(G)$, e se $|G| = n$ si ha che $\Imm(G) \leqslant S_n$.
\end{proof}

\begin{remark}
    In generale, dato $G = \left\{g_1 = e,g_2, \ldots, g_n\right\}$ e $\lambda : G \varlonghookrightarrow S(G) \cong S_n$, si ha che:
        \[ g_1 = e \longmapsto \lambda_{g_1} \qquad \text{con} \qquad \lambda_{g_1}: G \longrightarrow G : g_i \longmapsto g_i
            \]
        \[ g_2 \longmapsto \lambda_{g_2} \qquad \text{con} \qquad \lambda_{g_2} : G \longrightarrow G : x \longmapsto g_2x \longmapsto g_2^2x \longmapsto \ldots \longmapsto g_2^{k-1}x
            \]
        con $k = \ord_G(g_2)$. $\lambda_{g_2}$ può essere rappresentata mediante la notazione dei cicli:
            \[ (x,g_2x,\ldots,g_2^{k-1}x)
                \]
        preso poi $y \not\in \lambda_{g_2}(G)$, si ha analogamente:
        \[ (y,g_2y,\ldots,g_2^{k-1}y)
            \]
\end{remark}

\begin{example}
    Nel caso in cui $G = \Z8$ consideriamo l'azione:
        \[ \lambda : G \longrightarrow S(\Z8) \cong S_8\footnote{Perché appunto $S(\Z8)$ è l'insieme di permutazioni di un insieme di $8$ elementi.} : \overline a \longmapsto \lambda_a
            \]
    che, per quanto visto  genera ad esempio le applicazioni:\footnote{Per $+$ si intende la somma modulo $8$.}
            \begin{align*}
            \begin{array}{ccc}
            1 &\longmapsto \lambda_1 : X \longrightarrow X : a \longmapsto 1 + a \implies &(0,1,\ldots,7) \\
            2 &\longmapsto \lambda_2 : X \longrightarrow X : a \longmapsto 2 + a \implies &(0,2,4,6)(1,3,5,7)\\
            4 &\longmapsto \lambda_4 : X \longrightarrow X : a \longmapsto 4 + a \implies &(0,4)(1,5)(2,6)(3,7)\\		
            \end{array}
            \end{align*}
    che permutano gli elementi di $X$ secondo i cicli trovati.
\end{example}

\nopagebreak

\begin{definition}
    Un'azione $\lambda$ si dice \vocab{fedele} se è iniettiva.
\end{definition}
Ad esempio l'azione di rappresentazione regolare a sinistra è fedele:
    \[ \ker \lambda = \{g \in G | \lambda(g) = id\} = \{g \in G | \lambda_g(e) = e\} = \{g \in G | ge = e\} = \{e\}
        \]
infatti $\lambda(e) = \lambda_e = id$ e inoltre
    \[ \lambda(g) = \lambda_g = id \implies \lambda_g(e) = e \implies ge = e \implies g = e
        \]
da cui $\lambda$ fedele.

\begin{remark}
    Esiste anche un'applicazione $\rho : G \longrightarrow S(G) (\cong S_n)$, ($n = |G|$), detta azione di \vocab{rappresentazione regolare a destra}, con:
        \[ g \longmapsto \rho_g : x \longmapsto xg^{-1}
            \]
\end{remark}

\begin{lemma}[Lemma di Ranieri]
    \label{davide}
    Sia $G$ un gruppo abeliano di ordine $n$, allora $\forall d\mid n, \exists H \leqslant G : |H| = d$.\footnote{Il nome ovviamente non è ufficiale, ma deriva da un curioso aneddoto in cui è coinvolto il buon Davide
    Ranieri, pertanto vi sconsiglio di citarlo con questo nome in contesti ufficiali, ma se voleste farlo comunque è a vostro rischio e pericolo :)}\footnote{La dimostrazione non è stata fatta durante il corso, ma è stata comunque aggiunta per completezza.}
\end{lemma}

\begin{proof}
    Si consideri innanzitutto il caso $d=p^k$, $p$ primo, e mostriamolo per induzione:
    per $k=1$ la tesi è equivalente al \hyperref[p:Cauchy]{Teorema di Cauchy} (anche solo per i gruppi abeliani).
    Supponiamo la tesi per $k-1$. Poiché in particolare $p\mid |G|$ scegliamo un sottogruppo $H$ di $G$ di ordine $p$;
    tale sottogruppo è normale poiché $G$ è abeliano. $p^{k-1}\mid |G/H|\implies$ per ipotesi induttiva $\exists K \leqslant G/K : |K|=p^{k-1}$. \\
    Prendendo la controimmagine di $K$ tramite la proiezione al quoziente troviamo il sottogruppo di G cercato. \\
    A questo punto possiamo scrivere in generale $d=p_1^{k_1}\ldots p_s^{k_s}$; per ogni $i$ troviamo sottogruppi $H_i$ di ordini $p_i^{k_i}$ (tutti normali), poiché considerando il generato dagli elementi di ordine potenza di $p_i$:
    questo è un $p_i$-gruppo ed è normale per abelianità, se non avesse ordine $p_i^{k_i}$ allora quozientando $G$ per questo avremmo per \hyperref[Cauchy]{Cauchy} un elemento di ordine $p_i$ e considerando la controimmagine della proiezione avremmo
    perso un elemento di ordine potenza di $p_i$ (dato che avevamo quozientato per un $p_i$-gruppo). Si ha quindi che $H_1H_2\leqslant G$ per normalità,
    inoltre $|H_1\cap H_2|=1$ poiché l'ordine di un elemento in tale intersezione deve dividere $(p_1^{k_1}, p_2^{k_2})=1$. Pertanto $|H_1H_2|=p_1^{k_1}p_2^{k_2}$.
    Ragionando per induzione otteniamo che il sottogruppo $H_1\ldots H_k$ ha ordine $d$ come voluto.
\end{proof}

\begin{exercise}
    Sia $G$ un gruppo, se $|G| = p^n$, allora esiste:
        \[ \{e\} = H_n < H_{n-1} < \ldots < H_1 < G
            \]
    con $H_i \trianglelefteqslant G$ e $|H_i| = p^{n-i}$, $\forall i \in \{1,\ldots,n\}$.
\end{exercise}

\begin{soln}
    Procediamo per induzione su $n$, per $n = 1$ è ovvio, infatti si ha $H_1 = \{e\} \trianglelefteqslant G$.
    Supponiamo la tesi vera $\forall 1 \leq k \leq n-1$, osserviamo che $G$ è un $p$-gruppo, pertanto il suo centro non è banale:
        \[ |Z(G)| = p^z \qquad z\geq 1
            \]
    sia $\mathcal{G} = \faktor{G}{Z(G)}$, essendo $|G/Z(G)| < p^n$ (perché deve essere $|Z(G)| \geq p$), allora vale l'ipotesi induttiva, dunque
    $|\mathcal{G}| = p^m$, con $m = n-z\ (<n)$, allora esiste:
        \[ \mathcal{H}_m = \{e_{\mathcal{G}}\} < \mathcal{H}_{m-1} < \ldots < \mathcal{H}_1 < \mathcal{G}
            \]
    con $|\mathcal{H}_i| = p^{m-i}$ e $\mathcal{H}_i \trianglelefteqslant \mathcal{G}$. Data la proiezione al quoziente:
        \[ \pi_{Z(G)} : G \longrightarrow \mathcal{G}
            \]
    per il Teorema di Corrispondenza dei sottogruppi, esiste una bigezione tra i sottogruppi di $\faktor{G}{Z(G)}$ e i sottogruppi di
    $G$ che contengono $Z(G)$, la quale preserva normalità e indice del sottogruppo, pertanto preso $\mathcal{H}_i \leqslant \faktor{G}{Z(G)}$ è sufficiente
    applicare $\pi_{Z(G)}^{-1}$ alla catena scritta sopra e troviamo:
        \[ Z(G) = \pi_{Z(G)}^{-1}(\mathcal{H}_{m}) < \ldots < \pi_{Z(G)}^{-1}(\mathcal{H}_{1}) < \pi_{Z(G)}^{-1}(\mathcal{G}) (= G)
            \]
    Segue per il teorema di corrispondenza che $\pi_{Z(G)}^{-1}(\mathcal{H}_i) = H_i \trianglelefteqslant G$, ovvero si preserva la normalità dei sottogruppi, inoltre,
    segue sempre dal teorema che:
        \[ p^i = [\mathcal{G} : \mathcal{H}_i] = [G : H]
            \]
    dunque la catena esiste  e $|H_i| = p^{n-i}$ per $1\leq i \leq m$. Essendo $Z(G)$ abeliano, i sottogruppi di ogni suo ordine (che esistono sempre
    per il \hyperref[davide]{Lemma Di Ranieri}) sono normali in $Z(G)$, inoltre $|Z(G)| = p^z$ (dunque si hanno sottogruppi normali di ordine $p^l$ per $l \mid z$), pertanto esiste la catena:
        \[ \{e\} = H_n < \ldots < H_m = Z(G)
        \qquad \text{con $|H_j| = p^{n-j}$, $\forall m \leq j \leq n$} 
            \]
    Bisogna infine verificare che $H_j \trianglelefteqslant G$, dunque:
        \[ gH_jg^{-1} = H_j \qquad \forall g \in G
            \]
    ma $H_j \subset Z(G)$ (quindi è invariante per coniugio rispetto a ogni $g \in G$) dunque è sempre verificata l'ultima uguaglianza.
\end{soln}

\newpage
\subsection{Permutazioni}
Ricordiamo brevemente che:

\begin{definition}
    Dato un insieme $X$ si definisce \vocab{permutazione} un'applicazione bigettiva di $X$ in se stesso.
\end{definition}

Indichiamo con $S(X)$ il gruppo delle permutazioni di $X$ e con $S_n$ il gruppo delle permutazioni di un insieme di cardinalità $n$, che 
per semplicità indichiamo con $\{1,\ldots,n\}$.
Le permutazioni si possono indicare in vari modi, ad esempio, preso $\sigma \in S_{12}$ si può rappresentare mediante la matrice di permutazione:
    \[ \sigma = \left(\begin{array}{cccccccccccc}
        1 & 2 & 3 & 4 & 5 & 6 & 7 & 8 & 9 & 10 & 11 & 12 \\
        3 & 2 & 4 & 5 & 1 & 9 & 8 & 7 & 6 & 12 & 11 & 10 
        \end{array}\right)
        \]
o anche con la notazione dei cicli:
    \[ \sigma = \cycle{1,3,4,5} \cycle{6,9} \cycle{7,8} \cycle{10,12}
            \]
ogni ciclo prende il nome di \vocab{$k$-ciclo} (dove $k$ indica la sua lunghezza), come si osserva i cicli di lunghezza $1$ sono stati omessi,
in quanto lasciano fissi gli elementi, inoltre, i $2$-cicli prendono il nome di \vocab{trasposizioni}.\\
Formalmente, sia $\sigma \in S_n$ una permutazione di un insieme di $n$ elementi, per descrivere tale permutazione
 possiamo considerare l'insieme $X$, con $|X| = n$, il gruppo $G = \left<\sigma\right>$ e definire l'azione:
    \[ \varphi : G = \left<\sigma\right> \varlonghookrightarrow S(X) \cong S_n : \sigma \longmapsto \sigma
        \]
con $\sigma \in S_n$ e $\sigma : i \longmapsto \sigma(i)$, dunque abbiamo definito l'azione $\left<\sigma\right> \circlearrowleft X$ data dall'inclusione, la quale ci permetterà di descrivere come
$\sigma$ agisce sull'insieme $\{1,\ldots,n\}$. Osserviamo che:
    \[ \Orb(x) = \{\sigma(x) | \sigma \in \left<\sigma\right>\} = \{\sigma^l(x) | l \in \NN\} = \{x, \sigma(x), \sigma^2(x),\ldots,\sigma^{m_x-1}(x)\}
        \]
con $|\Orb(x)| = m_x$, con $m_x = \min\{k > 0 | \sigma^k(x) = x\}$, perché se $\sigma^k(x) = x$, allora $\sigma^{k+1}(x) = \sigma(x)$, pertanto, sia $k \in \NN$
tale che $\sigma^k(x) \in \{x,\ldots,\sigma^{k-1}(x)\}$, allora $\exists h:$
    \[ \sigma^k(x) = \sigma^h(x)
    \qquad \text{con $0 \leq h \leq k-1$}
        \]
Dunque vale che $\sigma^{k-h}(x) = x \in \{x,\ldots,\sigma^{k-1}(x)\}$ e per la minimalità di $k$ si ha che $h = 0$.
L'azione di $\left<\sigma\right>$ su $X$ divide $X$ in orbite e su ogni orbita $\sigma$ agisce ciclicamente (ovvero $\sigma(\Orb(x)) = \Orb(x)$).

\begin{definition}
    Si dice \vocab{ciclo} di $\sigma \in S_n$ l'orbita di un elemento $x \in \{1,\ldots,n\}$ vista come insieme ordinato:
        \[ (x,\sigma(x),\ldots,\sigma^{m_x-1}(x))
            \]
\end{definition}

\begin{remark}
    Un ciclo di lunghezza $k$ (un $k$-ciclo) ha $k$ scritture distinte, in quanto possiamo scegliere arbitrariamente il primo elemento.
\end{remark}

\begin{remark}
    Data $\sigma \in S_n$, essa è determinata dalle immagini di $\{1,\ldots,n\}$, dunque è determinata dai suoi cicli.
\end{remark}

\begin{example}
    Presa ad esempio $\sigma \in S_{10}$:
        \[ \sigma = \cycle{1,2,3}\cycle{4,5}\cycle{6,7,8,9}
            \]
    chiamiamo i suoi cicli:
        \[ \sigma_1 = \cycle{1,2,3} \qquad \sigma_2 = \cycle{4,5} \qquad \sigma_3 = \cycle{6,7,8,9}
            \]
    dove appunto $\sigma_1,\sigma_2,\sigma_3 \in S_{10}$ e:
        \[ \sigma = \sigma_1 \circ \sigma_2 \circ \sigma_3
            \]
\end{example}

\begin{definition}
    Una permutazione si dice \vocab{ciclica} se ha un unico ciclo (orbita) non banale.\footnote{D'ora in avanti si utilizzeranno i termini "permutazione ciclica" e "ciclo" come sinonimi, in quanto una permutazione ciclica è appunto un singolo ciclo non banale.}
\end{definition}

\begin{remark}
    Si osserva che:
        \begin{itemize}
            \item Cicli disgiunti commutano.
            \item L'ordine di una permutazione ciclica è la lunghezza del suo ciclo:
                    \[ \sigma = (x_1,\ldots,x_k) \implies \ord \sigma = k
                        \]
                quindi $\sigma^k = id$ e se $d < k$, allora $\sigma^d(x_1) = x_{d+1} \ne x_1$.
        \end{itemize}
\end{remark}

\begin{proposition}
    [Struttura Delle Permutazioni]
    \label{perm}
    Ogni permutazione si scrive in modo unico (a meno dell'ordine e della scrittura di cicli) come prodotto di cicli disgiunti,
     ovvero come composizione di permutazioni cicliche che agiscono su insiemi disgiunti.
\end{proposition}

\begin{proof}
    I cicli della permutazione sono univocamente determinati in quanto orbite della permutazione, sappiamo che ogni permutazione si scrive come prodotto dei suoi cicli,
    e per concludere basta osservare che i cicli disgiunti commutano.
\end{proof}

\begin{remark}
    Si osserva che l'unicità della scrittura di una permutazione vista nella \hyperref[perm]{Proposizione 1.50} è effettivamente valida solo nel caso di cicli disgiunti, infatti, prendendo ad esempio:
        \[ \sigma = \cycle{1,2}\cycle{2,4} \in S_4 \qquad \text{con}\qquad \sigma_1 = \cycle{2,4} \quad \text e \quad \sigma_2 = \cycle{1,2}
            \]
    non essendo $\sigma_1,\sigma_2$ cicli disgiunti, si osserva che $\sigma_2 \circ \sigma_1 = \cycle{2,4,1}$ e quindi $\sigma$ era in realtà un $3$-ciclo, e la sua fattorizzazione è unica come tale 
    (mentre non era unica come prodotto di cicli non disgiunti). 
\end{remark}

\pagebreak

\begin{corollary}
    $S_n$ è generato dalle permutazioni cicliche.
\end{corollary}

\begin{proof}
    Segue immediatamente dal fatto che ogni permutazione si ottiene mediante composizione di permutazioni cicliche. 
\end{proof}

\begin{example}
    Per esempio, preso $S_4$, le permutazioni possibili sono cicli del tipo:
        \[ id \qquad \cycle{a,b} \qquad \cycle{a,b,c} \qquad \cycle{a,b,c,d} \qquad \cycle{a,b}\cycle{c,d}
            \]
    per contare il numero di $2$-cicli, ci basta scegliere $2$ elementi dell'insieme in $\binom{4}{2}$ modi e poi considerare tutti i possibili 
    riordinamenti ciclici (dove la scelta del primo elemento è arbitraria), e ciò può essere fatto in $\frac{2!}{2}$ modi, per un totale di:
        \[ \binom{4}{2}\frac{2!}{2} = 6
            \]
    e ragionando analogamente per i $3$-cicli e i $4$-cicli si ottiene:
        \[ \binom{4}{3}\frac{3!}{3} = 8 \qquad \text e \qquad \binom{4}{4}\frac{4!}{4} = 6
            \]
    infine, per quanto riguarda le permutazioni ottenute dalla composizione di due $2$-cicli, possiamo scegliere e permutare due coppie di elementi, come 
    nei casi precedenti, tuttavia, essendo i cicli disgiunti, questi commutano (banalmente perché lasciano fissi gli altri elementi del dominio), quindi bisogna anche 
    dividere per il numero di permutazioni dei cicli della stessa lunghezza, ovvero $2!$ dunque:
        \[ \binom{4}{2}\frac{2!}{2}\binom{2}{2}\frac{2!}{2} \cdot \frac{1}{2!} = 3
            \]
    e dal conteggio delle permutazioni di $S_4$ divise per cicli di diversa lunghezza si ottiene: $1+6+8+6+3 = 24 = |S_4|$.
\end{example}

\begin{remark}
    Quanto visto nell'esempio precedente può essere generalizzato ottenendo:
        \[ \#\{\sigma \in S_n | \sigma\,\text{è un $k$-ciclo}\} = \binom{n}{k}\frac{k!}{k} = \binom{n}{k}(k-1)!
            \]
\end{remark}

\begin{example}
    Per quanto detto risulta semplice ad esempio calcolare:
        \[ \#\{\sigma \in S_{20} | \text{$\sigma$ si fattorizza in cicli del tipo $2+2+2+4+5+5$}\}
            \]
    applicando quanto detto nell'osservazione precedente si trovano:
        \[ \frac{\binom{20}{2}\binom{18}{2}\binom{16}{2}1!1!1!}{3!} \cdot \binom{14}{4}3! \cdot \frac{\binom{10}{5}\binom{5}{5}4!4!}{2!}
            \]
\end{example}

\begin{proposition}
    [Ordine di una permutazione]
    Data $\sigma \in S_n$ con $\sigma = \sigma_1\ldots\sigma_k$, con $\sigma_i$ cicli disgiunti, allora:
        \[ \ord \sigma = [\ord \sigma_1,\ldots, \ord\sigma_k]
            \]
\end{proposition}

\begin{proof}
    Sia $\sigma_i$ un $l_i$-ciclo, ovvero $\ord\sigma_i = l_i$, vogliamo dimostrare che:
        \[ \ord\sigma = [l_1,\ldots,l_k] = d
            \]
    osserviamo che $\sigma^d = (\sigma_1\ldots\sigma_k)^d = \sigma_1^d\ldots\sigma_k^d$, in quanto i cicli $\sigma_i$ sono disgiunti (pertanto commutano),
    ed essendo $d = [l_1,\ldots,l_k]$ si ha che $l_i \mid d$, $\forall \in \{1,\ldots,k\}$, pertanto:
        \[ \sigma^d = \sigma_1^d\ldots\sigma_k^d = id \implies \ord \sigma = m \mid d
            \]
    d'altra parte, si ha che:
        \[ \sigma^m = \sigma_1^m\ldots\sigma_k^m = id \iff \sigma_i^m = id, \forall i \in\{1,\ldots,k\}
            \]
    dunque $\ord \sigma_i = l_i \mid m$, $\forall i \in\{1,\ldots,k\}$, ovvero $[l_1,\ldots,l_k] \mid m$ da cui si conclude che $m = [l_1,\ldots,l_k]$.
\end{proof}

\begin{proposition}
    [$S_n$ è generato dalle trasposizioni]
    \label{trasp}
    Le trasposizioni generano $S_n$, $\forall n \geq 2$.
\end{proposition}

\begin{proof}
    Per dimostrare l'affermazione bisogna mostrare che ogni permutazione è prodotto di trasposizioni (in generale non disgiunte).
    Poiché ogni permutazione, per quanto affermato nella \hyperref[perm]{Proposizione 1.50}, è il prodotto di cicli (permutazioni cicliche) disgiunti,
    è sufficiente mostrare che i cicli sono tutti prodotto di trasposizioni, infatti si può osservare che:
        \[ \cycle{1,\ldots,k} = \cycle{1,k}\cycle{1,k-1}\ldots\cycle{1,2}
            \]
    dove l'uguaglianza è tra funzioni, quindi ci basta mostrare che danno la stessa immagine. Se $i>k$, allora entrambe le funzioni mandano $i \longmapsto i$, se 
    $i \leq k$, allora la funzione a sinistra manda $i \longmapsto i+1$ e $k \longmapsto 1$, quella a destra lascia fisso $i$ fino al ciclo $\cycle{1,i}$ che manda $i \longmapsto 1$,
    e il ciclo successivo (alla sinistra del precedente) $\cycle{1,i+1}$ manda $1 \longmapsto i+1$, infine i cicli successivi lasciano fisso $i+1$ (quindi complessivamente abbiamo $i \longmapsto i+1$), 
    mentre $k$ viene lasciato fisso da tutti i cicli tranne $\cycle{1,k}$, quindi $k \longmapsto 1$.
\end{proof}

\begin{remark}
    La scrittura di una permutazione come prodotto di trasposizioni non è unica. Ad esempio in $S_4$:
        \[ \sigma = \cycle{1,2}\cycle{2,4} = \cycle{1,2}\cycle{3,4}\cycle{3,4}\cycle{2,4}
            \]
\end{remark}

La seguente proposizione ci mostra invece che è fissata la parità della decomposizione in trasposizioni, cioè se $\sigma$ si scompone come prodotto di $m$ trasposizioni,
ogni altra decomposizione come prodotto di trasposizioni ha un numero di trasposizioni con la stessa parità.

\begin{proposition}[Segno di una permutazione]
    L'applicazione:
        \[ sgn : S_n \longrightarrow \{\pm1\} : \sigma \longmapsto sgn(\sigma) = \prod_{1 \leq i < j \leq n} \frac{\sigma(i) - \sigma(j)}{i - j}
            \]
    è un omomorfismo di gruppi. Inoltre, se $\sigma$ è una trasposizione, allora $sgn(\sigma) = -1$.
\end{proposition}

\begin{proof}
    Osserviamo inizialmente che $sgn$ è ben definita cioè:
        \[ sgn(\sigma) = \prod_{1 \leq i < j \leq n} \frac{\sigma(i) - \sigma(j)}{i - j} \in \{\pm 1\}
            \]
    al denominatore del prodotto vi sono tutte le possibili coppie $i - j$ (in $\{1,\ldots,n\}$, prese tutte ordinatamente con $i<j$) e anche al numeratore poiché $\sigma$ è bigettiva, l'unica cosa che 
    può cambiare è l'ordine (ovvero potrebbe comparire $i - j$ al numeratore e $j - i$ al denominatore), quindi $sgn(\sigma) \in \{\pm 1\}$. Mostriamo che $sgn$ 
    è un omomorfismo:
        \[ sgn(\sigma \circ \tau) = \prod_{i < j}\frac{\sigma(\tau(i)) - \sigma(\tau(j))}{i - j} =  \prod_{i < j}\frac{\sigma(\tau(i)) - \sigma(\tau(j))}{i - j}\frac{\tau(i) - \tau(j)}{\tau(i) - \tau(j)}
            \]
    da cui:
        \[ \prod_{i < j}\frac{\sigma(\tau(i)) - \sigma(\tau(j))}{\tau(i) - \tau(j)}\frac{\tau(i) - \tau(j)}{ i - j} =
        \underbrace{\prod_{i < j} \frac{\sigma(\tau(i)) - \sigma(\tau(j))}{\tau(i) - \tau(j)}}_{sgn(\sigma)} \underbrace{\prod_{i < j}\frac{\tau(i) - \tau(j)}{ i - j}}_{sgn(\tau)}
        \qquad \forall \sigma,\tau \in S_n
            \]
    Ci resta da verificare che il segno di una trasposizione è $-1$. Sia $\sigma = \cycle{a,b}$, analizzando il segno delle varie coppie, distinguiamo le seguenti possibilità per i vari fattori del prodotto:
    \begin{itemize}
        \item $\{i,j\} \cap \{a,b\} = \emptyset $, in tal caso $\sigma$ lascia fissi gli elementi, $\sigma(i) = i, \sigma(j) = j \implies \frac{\sigma(i) - \sigma(j)}{i - j} = 1$.
        \item $\{i,a\}$ con $i \ne b$ (il caso $\{i,b\}$ con $i \ne a$ è analogo), in tal caso $\frac{\sigma(i) - \sigma(a)}{i - a} = \frac{i - b}{i - a}$, però vi è anche il fattore $\frac{\sigma(i) - \sigma(b)}{i - b} = \frac{i - a}{i - b}$ e 
            il loro prodotto dà $1$.
        \item Infine, nel caso in cui $\{i,j\} = \{a,b\}$ si ha:
            \[ \frac{\sigma(a) - \sigma(b)}{a - b} = \frac{b - a}{a - b} = -1
                \]
    \end{itemize}
    Dunque si conclude che $sgn(\cycle{a,b}) = -1$.
\end{proof}

\begin{remark}
    La proposizione appena vista dimostra quanto detto sopra, ovvero:
        \[ \sigma = \tau_1 \ldots \tau_m \qquad \text{con $\tau_i$ trasposizione}
            \]
    allora $sgn(\sigma) = \prod_{1\leq i \leq m}sgn(\tau_i) = (-1)^m$.
\end{remark}

\begin{definition}
    Una permutazione $\sigma \in S_n$ si dice \vocab{pari} se $sgn(\sigma) = 1$, \vocab{dispari} se $sgn(\sigma) = -1$.
\end{definition}

\begin{definition}
    Dato l'omomorfismo $sgn : S_n \longrightarrow \{\pm 1\}$, si definisce \vocab{gruppo alterno}:
        \[ \mathcal{A}_n = \ker sgn = \{\sigma \in S_n |\, \text{$\sigma$ è pari}\}
            \]
\end{definition}

\begin{remark}
    Si osserva che $\mathcal{A}_n \trianglelefteqslant S_n$ e $\displaystyle|\mathcal{A}_n| = \frac{n!}{2}$ poiché $\faktor{S_n}{\mathcal{A}_n} \cong \{\pm 1\}$.
\end{remark}

\begin{remark}
    Per quanto detto nella \hyperref[trasp]{Proposizione 1.57}, un $k$-ciclo si può scrivere nella forma:
        \[ \cycle{1,\ldots,k} = \underbrace{\cycle{1,k}\cycle{1,k-1}\ldots\cycle{1,2}}_{k-1 \, \text{trasposizioni}}
            \]
    dunque un $k$-ciclo è pari se $k \equiv 1 \pmod 2$, dispari se $k \equiv 0 \pmod 2$.
\end{remark}

\pagebreak

\subsection{Classi di coniugio in $S_n$}
\begin{theorem}
    Due permutazioni in $S_n$ sono coniugate se e solo se hanno la stessa decomposizione in cicli disgiunti.
\end{theorem}

\begin{proof}
    Mostriamo le due implicazioni:
        \begin{itemize}
            \item Presa $\sigma = \cycle{a_1,\ldots,a_k}$ e $\tau \in S_n$, vogliamo dimostrare che $\tau\circ\sigma\circ\tau^{-1}$ è ancora un $k$-ciclo.
                Sia $\tau(a_i) = b_i$, allora si ha che $\tau\sigma\tau^{-1} = \cycle{b_1,\ldots,b_k}$, con $b_i \ne b_j$, $\forall i \ne j$, poiché $\tau$ è bigettiva; verifichiamo
                l'uguaglianza mostrando che le due funzioni coincidono per tutti gli elementi. Si osserva che nel ciclo a destra accade semplicemente che $b_i \longmapsto b_{i+1}$, a sinistra invece:
                    \[ b_i \xmapsto{\tau^{-1}} a_i \xmapsto{\sigma} a_{i+1} \xmapsto{\tau} b_{i+1} \qquad \forall i \in \{1,\ldots,k\}
                        \]
                Se, invece, $x \ne b_i$, a sinistra si ha $\tau^{-1}(x) \ne a_1,\ldots,a_k$ (perché non si parte da alcun $b_i$), quindi $\sigma(\tau^{-1}(x)) = \tau^{-1}(x)$, e quindi
                $\tau\sigma\tau^{-1}(x) = \tau \tau^{-1} (x) = x$; a destra invece, essendo $x \ne b_i \forall i$ viene lasciato fisso, ciò conclude che le due funzioni sono uguali e quindi $\tau\sigma\tau^{-1}$ è un $k$-ciclo.
            \item Mostriamo ora che due permutazioni con la stessa fattorizzazione in cicli disgiunti sono coniugate. Siano:
                \[ \sigma = \cycle{a_1, \ldots, a_l}\cycle{b_1, \ldots, b_s} \ldots \cycle{z_1, \ldots, z_t}
                    \]\[ \rho = \cycle{a_1^{\prime}, \ldots, a_l^{\prime}}\cycle{b_1^{\prime}, \ldots, b_s^{\prime}} \ldots \cycle{z_1^{\prime}, \ldots, z_t^{\prime}}
                        \]
                per dimostrare la tesi è sufficiente trovare $\tau \in S_n$ tale che $\tau\circ\sigma\circ\tau^{-1} = \rho$. Scegliamo $\tau$ definita da:
                    \[ \tau(a_i) = a_i^{\prime}, \tau(b_i) = b_i^{\prime}, \ldots, \tau(z_i) = z_i^{\prime}
                        \]
                ed eventualmente si aggiungono altri elementi. Verifichiamo allora che $\tau\circ\sigma\circ\tau^{-1} = \rho$, consideriamo (WLOG) il primo ciclo:
                    \[ a_i^{\prime} \xmapsto{\tau^{-1}} a_i \xmapsto{\sigma} a_{i+1} \xmapsto{\tau} a_{i+1}^{\prime}
                        \]
                e quindi $a_i^{\prime} \longmapsto a_{i+1}^{\prime}$, pertanto $\tau\circ\sigma\circ\tau^{-1}$ e $\rho$ coincidono sempre.
        \end{itemize}
\end{proof}

\begin{example}
    In $S_5$ la classe di coniugio di $\sigma = \cycle{1,2}\cycle{3,4}$ è $C_\sigma = \{\cycle{a,b}\cycle{c,d} \in S_5\}$, con:
        \[ \#C_\sigma = \frac{\binom{5}{2}\binom{3}{2}1!1!}{2!} = 15
            \]
    e da ciò si ricava anche che:
        \[ \#Z_{S_5}(\sigma) = \frac{|S_5|}{|C_\sigma|} = \frac{5!}{15} = 8
            \]
\end{example}

\begin{example}
    Sia $\sigma = \cycle{3,5}\cycle{14} \in S_5$ e sia $\rho = \cycle{1,2}\cycle{3,4}$, cerchiamo $\tau \in S_5$ tale che:
        \[ \tau\circ\sigma\circ\tau^{-1} = \rho
            \]
    si può scegliere $\tau = \cycle{1,3}\cycle{2,5}$, da cui:
        \[ \cycle{1,3}\cycle{2,5} \circ \cycle{3,5}\cycle{14} \circ \cycle{1,3}\cycle{2,5} = \cycle{1,2}\cycle{3,4} = \rho
            \]
\end{example}

\begin{corollary}
    \label{c:1.68}
    Valgono i seguenti fatti:
        \begin{enumerate}[(1)]
            \item Il numero di classi di coniugio in $S_n$ è uguale al numero di partizioni di $n$.
            \item Se $H \leqslant S_n$, allora $H \trianglelefteqslant S_n$ se e solo se contiene tutte le permutazioni di un certo tipo o nessuna.
        \end{enumerate}
\end{corollary}

\pagebreak

\subsection{Prodotto diretto}
Ricordiamo brevemente che se $G_1$ e $G_2$ sono gruppi, allora l'insieme $G_1 \times G_2$ con l'operazione fatta componente per componente prende il nome di \vocab{prodotto diretto}.

\begin{example}
    Presi ad esempio $\Z7$ e $S_4$, si ha $\Z7 \times S_4$, con $\sigma = (\overline 1, \cycle{1,2,3})$ e $\rho = (\overline 4, \cycle{1,4,2,4})$ in $\Z7 \times S_4$ e l'operazione:
        \[ \sigma \cdot \rho = (\overline 1 + \overline 4, \cycle{1,2,3} \circ \cycle{1,4,2,3}) = (\overline 5, \cycle{1,4,3,2})
            \]
\end{example}

\begin{remark}
    Si ricordano i seguenti fatti:
    \begin{itemize}
        \item Se $H,K \leqslant G$ in generale $HK$ non è un sottogruppo, ma $HK \leqslant G \iff HK = KH$. Ovviamente se uno tra $H$ e $K$ è normale in $G$, allora questo
        è sempre vero.
        \item $H \times K \leqslant G \times G$.
    \end{itemize}
\end{remark}

\begin{lemma}
    \label{l:1.71}
    Siano $H,K \trianglelefteqslant G$ e $H \cap K = \{e\}$, allora $hk = kh$, $\forall h \in H$, $\forall k \in K$.
\end{lemma}

\begin{proof}
    Preso $hkh^{-1}k^{-1}$, si ha:
        \[ hkh^{-1}k^{-1} = \underbrace{\underbrace{(hkh^{-1})}_{= k^{\prime}}k^{-1}}_{\in K} = \underbrace{h\underbrace{(kh^{-1}k^{-1})}_{= h^{\prime}}}_{\in H}
            \]
    dunque $hkh^{-1}k^{-1} \in H \cap K \implies hkh^{-1}k^{-1} = e$, da cui segue la tesi.
\end{proof}

\begin{theorem}
    [Decomposizione in prodotto diretto]
    \label{t:1.72}
    Sia $G$ un gruppo e siano $H,K \trianglelefteqslant G$ tali che:
        \begin{enumerate}[(1)]
            \item HK = G.
            \item $H \cap K = \{e\}$.
        \end{enumerate}
    Allora $G \cong H \times K$.
\end{theorem}

\begin{proof}
    Definiamo l'applicazione:
        \[ \varphi : H \times K \longrightarrow G : (h,k) \mapsto hk
            \]
    Si verifica che è un omomorfismo:
        \[\varphi((h_1,k_1)(h_2,k_2)) = \varphi((h_1h_2,k_1k_2)) = h_1h_2k_1k_2
            \]
    per il \hyperref[l:1.71]{Lemma 1.71} si ha che $h_1h_2k_1k_2 = h_1k_1h_2k_2 = \varphi((h_1,k_1))\varphi((h_2,k_2))$, $\forall h_1,h_2 \in H$, $\forall k_1,k_2 \in K$.
    Si osserva ora che $\varphi$ è surgettiva, per l'ipotesi ($1$); infine, è iniettiva in quanto:
        \[ \ker \varphi = \{(h,k) \in H \times K | hk = e\} = \{(h,k) \in H \times K | h = k^{-1}\} = \{e\}
            \]
    dove nell'ultima uguaglianza si è usato il fatto che $H \cap K = \{e\}$.
\end{proof}

\begin{remark}
    Se abbiamo due sottogruppi $G_1$ e $G_2$ e costruiamo $G = G_1 \times G_2$, allora presi:
        \[ H = G_1 \times \{e_2\} \trianglelefteqslant G \qquad \text e \qquad K = \{e_1\} \times G_2 \trianglelefteqslant G
            \]
    $H,K$ sono normali, hanno intersezione banale e sono tali che $HK = G$, quindi verifichiamo le ipotesi del teorema, pertanto $G \cong H \times K$.
\end{remark}

\begin{example}
    Sia $G$ un gruppo con $|G| = p^2$, dalla formula delle classi avevamo ottenuto che $G$ è necessariamente abeliano, quindi $G$ è 
    isomorfo a $\ZZ/p^2\ZZ$ o $\Zp \times \Zp$. Se $G$ è ciclico, allora $G \cong \ZZ/p^2\ZZ$. Mostriamo che se non lo è, allora $G \cong \Zp \times \Zp$ e in 
    questo caso tutti gli elementi di $G$ hanno ordine $p$. \\
    Consideriamo $(e \ne) x \in G$ e $H = \left<x\right> \trianglelefteqslant G$ (in quanto $G$ abeliano); prendiamo $y \in G \setminus\left<x\right>$
    e analogamente $K = \left<y\right> \trianglelefteqslant G$, da ciò segue che $H \cap K = \{e\}$, infatti $H$ e $K$ sono sottogruppi ciclici di $G$
    aventi ordini due primi distinti e quindi hanno in comune solo l'elemento neutro.
    Osservando infine che per cardinalità $HK = G$, per cardinalità:
        \[ |HK| = \frac{|H||K|}{|H \cap K|} = \frac{p\cdot p}{1} = p^2
            \]
    le ipotesi del \hyperref[t:1.72]{Teorema 1.72} sono verificate, dunque:
        \[ G \cong H \times K \cong \Zp \times \Zp
            \]
\end{example}

\newpage
\subsection{Prodotto semidiretto}

\begin{definition}
    Dati due gruppi $H,K$ e l'azione:
        \[ \varphi : K \longrightarrow \Aut(H) (\leqslant S(H)) : k \longmapsto \varphi_k
            \]
    si dice \vocab{prodotto semidiretto} di $H$ e $K$ via $\varphi$:
        \[ H \rtimes_{\varphi} K
            \]
    (o anche $K {_\varphi}\ltimes H$) l'insieme ottenuto come prodotto cartesiano $H \times K$ con l'operazione definita da:
        \[ (h,k) (h^{\prime},k^{\prime}) = (h \cdot_H \varphi_k(h^{\prime}), k \cdot_K k^{\prime})
            \]
\end{definition}

\begin{proposition}
    [Il Prodotto Semidiretto è un gruppo]
    Dati due gruppi $H,K$, allora $H \rtimes_{\varphi} K$ è un gruppo.
\end{proposition}

\begin{proof}
    Come si verifica facilmente l'operazione indotta dal prodotto semidiretto è associativa, verifichiamo che $(e_H,e_K)$ è 
    l'elemento neutro:
        \[ (h,k)(e_H,e_K) = (h \cdot \varphi_k(e_H), k e_K) = (h e_H, k) = (h,k)
            \]
    dove $\varphi_k(e_H) = e_H$ poiché $\varphi_k$ è un automorfismo (e quindi in particolare un omomorfismo), a sinistra, invece, si ha:
        \[(e_H,e_K)(h,k) = (e_H \cdot \varphi_{e_K}(h), e_Kk) = (e_H \cdot id(h), k) = (e_H h, k) = (h,k)
            \]
    Per l'inverso si osserva:
        \[ (h,k)^{-1} = ((\varphi_k)^{-1}(h^{-1}), k^{-1}) = (\varphi_{k^{-1}}(h^{-1}), k^{-1})\,\footnote{L'uguaglianza $(\varphi_k)^{-1} = \varphi_{k^{-1}}$
         segue dal fatto che $\varphi$ è un omomorfismo e quindi manda inversi in inversi.}
            \]
    dunque si verifica a destra:
        \begin{multline*}
            (h,k)(\varphi_{k^{-1}}(h^{-1}), k^{-1}) = (h \cdot \varphi_k(\varphi_{k^{-1}}(h^{-1})), kk^{-1}) = \\
            =(h \cdot id(h^{-1}), e_K) = (hh^{-1}, e_K) = (e_H, e_K)
        \end{multline*}
    e analogamente a sinistra:
        \begin{multline*}
            (\varphi_{k^{-1}}(h^{-1}), k^{-1})(h,k) = (\varphi_{k^{-1}}(h^{-1}) \cdot \varphi_{k^{-1}}(h), k^{-1}k) = \\
            = (\varphi_{k^{-1}}(h^{-1}h), e_K) = (\varphi_{k^{-1}}(e_H), e_K) = (e_H, e_K)
        \end{multline*}
\end{proof}

\pagebreak

\begin{remark}
    Si osserva che $H \rtimes_{\varphi} K$ è il prodotto diretto se e solo se $\varphi_k = id_H$, $\forall k \in K$.
    Infatti:
        \[ (h,k)(h^{\prime},k^{\prime}) = (h\cdot \varphi_k(h^{\prime}), kk^{\prime}) = (hh^{\prime}, kk^{\prime}) \iff \varphi_k(h^{\prime}) = h^{\prime}
        \qquad \forall k \in K
            \]
        e dunque $\varphi_k = id_H$.
\end{remark}

\begin{theorem}
    [Decomposizione in prodotto semidiretto]
    \label{t:1.78}
    Sia $G$ un gruppo e siano $H,K \leqslant G$, con $H \trianglelefteqslant G$, tali che:
        \begin{enumerate}[(1)]
            \item $HK = G$.
            \item $H \cap K = \{e\}$. 
        \end{enumerate}
    Allora $G \cong H \rtimes_{\varphi} K$, dove $\varphi : K \longrightarrow \Aut(H) : k \longmapsto \varphi_k$, con $\varphi_k : H \underset{\sim}{\longrightarrow} H : h \longmapsto khk^{-1}$.
\end{theorem}

\begin{proof}
    Costruiamo esplicitamente un isomorfismo tra i due gruppi:
        \[ \mathcal{F} : H \rtimes_{\varphi} K \longrightarrow G : (h,k) \longmapsto hk
            \]
    Verifichiamo che è un omomorfismo:
        \[
            \mathcal{F}((h,k)(h^{\prime},k^{\prime})) = \mathcal{F}(h\cdot \varphi_k(h^{\prime}),kk^{\prime}) = \mathcal{F}(h\underbrace{kh^{\prime}k^{-1}}_{= \varphi_k(h^{\prime})},kk^{\prime})
            = hkh^{\prime}k^{-1}kk^{\prime} = \underbrace{hk}_{= \mathcal{F}(h,k)}\underbrace{h^{\prime}k^{\prime}}_{= \mathcal{F}(h^{\prime},k^{\prime})}
        \]
    Si vede inoltre che $\mathcal{F}$ è surgettiva per l'ipotesi ($1$) e iniettiva per la ($2$), infatti:
        \[ \ker\mathcal{F} = \{(h,k) \in  H \rtimes_{\varphi} K | \mathcal{F}(h,k) = hk = e\} = \{e\}
            \]
\end{proof}

\begin{remark}
    Si osserva che $\varphi_k$ è la restrizione al sottogruppo $H$ dell'automorfismo interno $g \longmapsto kgk^{-1}$, poiché
    $H \trianglelefteqslant G$, allora la restrizione a $H$ di ogni elemento di $\Inn(G)$ è un automorfismo di $H$.
\end{remark}

\begin{remark}
    Sapendo che $G \cong H \rtimes_{\varphi} K$ e seguendo i passaggi della verifica di omomorfismo al contrario, si ricava che necessariamente $\varphi$ è esattamente
    l'azione di coniugio su $H$.
\end{remark}

\pagebreak

\begin{remark}
    Siano $\overline H = H \times \{e_K\}$ e $\overline K = \{e_H\} \times K$, si osserva che $\overline H, \overline K \leqslant G = H \rtimes_{\varphi} K$, infatti sono chiusi per prodotto (ristretto):
        \[ (h,e_K)(h^{\prime},e_K) = (h \cdot \varphi_{e_K}(h^{\prime}), e_K) = (h \cdot id(h^{\prime}), e_K) = (hh^{\prime},e_K)
        \]\[ (e_H,k)(e_H,k^{\prime}) = (e_H \cdot \varphi_k(e_H), kk^{\prime}) = (e_H,kk^{\prime})
            \]
    e si verifica facilmente anche per inverso. Si osserva che $\overline H \trianglelefteqslant G$\footnote{$\overline K$ in generale non è normale, lo è solo se il prodotto è diretto, infatti in quel caso vale il \hyperref[t:1.72]{Teorema 1.72}.},
    in quanto $\ol H = \ker \pi$, con:
        \[ \pi : H \rtimes_{\varphi} K \longrightarrow K: (h,k) \longmapsto k
            \]
    con $\pi$ omomorfismo come si vede:
        \[ \pi((h,k)(h^{\prime},k^{\prime})) = \pi(h \cdot \varphi_k(h^{\prime}),kk^{\prime}) = kk^{\prime} = \pi((h,k))\pi((h^{\prime},k^{\prime}))
            \]
    Per come li abbiamo presi si nota subito che $\overline H \, \overline K = G$ e $\overline H \cap \overline K = \{e\}$, quindi valgono le ipotesi del \hyperref[t:1.78]{Teorema 1.78}, pertanto:
        \[ G \cong H \rtimes_{\varphi} K \cong \ol H \rtimes_{\varphi} \ol K
            \]
\end{remark}

\begin{example}
    [$S_n \cong \mathcal{A}_n \rtimes_{\varphi} \left<\cycle{1,2}\right>$]
    Verifichiamo che $S_n$ è prodotto semidiretto di $H = \mathcal{A}_n$ e $ K = \left<\cycle{1,2}\right>$ \footnote{In generale va bene qualsiasi trasposizione (che esiste sempre in $S_n$ per $n \geq 2$).}
    usando il \hyperref[t:1.78]{Teorema 1.78}, per quanto detto nel ($1$) del \hyperref[c:1.68]{Corollario 1.68} sappiamo che $\mathcal{A}_n \triangleleft S_n$, inoltre, sempre per il punto ($1$), essendo $|\mathcal{A}_n| = \frac{n!}{2}$, segue
    per cardinalità che $HK = S_n$. Essendo $\mathcal{A}_n = \ker sgn$ e $\left<\cycle{1,2}\right>$ una trasposizione $H \cap K = \{e\}$ (in quanto il nucleo dell'omomorfismo segno contiene solo permutazioni pari), pertanto segue la tesi:
        \[ S_n \cong \mathcal{A}_n \rtimes_{\varphi} \left<\cycle{1,2}\right>
            \]
    Osserviamo inoltre che:
        \[ \varphi : \left<\cycle{1,2}\right> \longrightarrow \Aut(\mathcal{A}_n) : \cycle{1,2} \longmapsto \varphi_{\cycle{1,2}}, id \longmapsto id
            \]
    con $\varphi_{\cycle{1,2}} : \mathcal{A}_n \longrightarrow \mathcal{A}_n : \rho \longmapsto \cycle{1,2}\rho\cycle{1,2}^{-1}$.
\end{example}

\begin{example}
    [$D_n \cong \Zn \rtimes_{\varphi}\Z2$]
    Ricordando che $D_n = \left<r,s | r^n = s^2 = id, srs^{-1} = r^{-1}\right>$, possiamo osservare ancora una volta che le ipotesi del \hyperref[t:1.78]{Teorema 1.78} sono soddisfatte. Poiché $\ord r = n$, allora $|\left<r\right>| = n$, 
    e in particolare $[D_n : \left<r\right>] = 2 \implies \left<r\right> \triangleleft D_n$; inoltre, $\left<r\right> \cap \left<s\right> = \{id\}$ perché $\det(r_i) = 1$, mentre $\det(sr_i) = -1$, $\forall i \in \{1,\ldots,n\}$. Infine, essendo $\ord s = 2$, allora il prodotto di sottogruppi avrà cardinalità:
        \[ |\left<r\right>\left<s\right>| = \frac{|\left<r\right>||\left<s\right>|}{|\left<r\right> \cap \left<s\right>|} = \frac{2n}{1} = 2n
            \]
    dunque $\left<r\right>\left<s\right> = D_n$. Pertanto $D_n \cong \left<r\right> \rtimes_{\varphi} \left<s\right>$, dove $\left<r\right> \cong \Zn$ e $\left<s\right> \cong \Z2$, quindi:
        \[ D_n \cong \Zn \rtimes_{\varphi} \Z2
            \]
    con:
        \[ \varphi : \left<s\right> \longrightarrow \Aut(\left<r\right>) : s \longmapsto \varphi_s
            \]
    dove $\varphi_s : \left<r\right> \longrightarrow \left<r\right> : r \longmapsto srs^{-1} (= r^{-1})$. Si osserva che deve essere $\ord \varphi_s | \ord s = 2$, quindi ci sono soltanto due possibilità:
        \[ \varphi_s = \begin{cases}
            id \\
            r \longmapsto r^{-1}
            \end{cases}
        \]
    nel caso in cui $\varphi_s = id$ si ottiene il prodotto diretto, nell'altro caso si ottiene il prodotto semidiretto che definisce $D_n$. Se in $\Aut(\Zn)$ ci sono altri elementi di ordine due
     (ad esempio se $\Aut(\Z8) \cong \Z8^* \cong \Z2 \times \Z2$) si possono definire anche altri prodotti semidiretti:
    \[ \Zn \rtimes_{\varphi} \Z2
        \]
    Rimane il problema di verificare se danno o meno due gruppi isomorfi.
\end{example}

\begin{example}
    [Gruppi di ordine $pq$]
    Sia $|G| = pq$, per il \hyperref[p:Cauchy]{Teorema Di Cauchy} esistono $x,y \in G$ tali che $\ord x = q$, $\ord y = p$, assumiamo (WLOG) $q>p$, allora si ha che:
        \[ H = \left<x\right> \triangleleft G
            \]
    poiché $[G : H] = p$, con $p$ più piccolo primo che divide $|G|$. Alternativamente si può vedere che $H$ è caratteristico in $G$ poiché è l'unico sottogruppo di quell'ordine;
    se $H^{\prime} < G$ e $|H^{\prime}| = q$, se fosse $H \ne H^{\prime}$, allora $H \cap H^{\prime} = \{e\}$ e quindi:
    \[ |HH^{\prime}| = \frac{|H||H^{\prime}|}{|H \cap H^{\prime}|} = \frac{q \cdot q}{1} = q^2 > pq
        \]
    quindi $H^{\prime}$ non può essere un sottogruppo di $G$. Si verifica che, detto $K = \left<y\right>$, le ipotesi del \hyperref[t:1.78]{Teorema 1.78} sono soddisfatte:
        \[ HK = G \qquad H \cap K = \{e\} \qquad H \triangleleft G
            \]
da ciò segue che ogni gruppo di ordine $pq$ è prodotto semidiretto: $G \cong H \rtimes_{\varphi} K$.
\end{example}

Per classificare tutti i gruppi di ordine $pq$ bisogna classificare tutti i possibili prodotti semidiretti $\ZZ/q\ZZ \rtimes_{\varphi} \Zp$ a meno di isomorfismo.
Osserviamo che un prodotto semidiretto deve avere un'operazione definita da:
\[ \varphi : \Zp \longrightarrow \Aut(\ZZ/q\ZZ) \cong \ZZ/q\ZZ^* \cong \ZZ/(q-1)\ZZ
            \]
Essendo $\Zp = \left<y\right>$ e $\ZZ/q\ZZ = \left<x\right>$ possiamo scrivere:
    \[ \varphi : \left<y\right> \longrightarrow \Aut(\left<x\right>) (\cong \ZZ/q\ZZ^* \cong \ZZ/(q-1)\ZZ) : y \longmapsto \varphi_y
            \]
dove $\varphi_y : \left<x\right> \longrightarrow \left<x\right> : x \longmapsto x^l$ (poiché gli automorfismi di un gruppo ciclico mandano un elemento in una sua potenza, o prodotto se la notazione è additiva).
Per definire $\varphi$ su $\left<y\right>$ (un dominio ciclico)
basta assegnare $\varphi_y$ con la condizione $\ord \varphi_y \mid \ord y = p$, inoltre, $\varphi_y \in \Aut(\Z{q}) \cong \Z{(q-1)} \implies \ord \varphi_y \mid q-1$, 
quindi $\ord \varphi_y \mid (p,q-1)$. Distinguiamo due casi:
    \begin{itemize}
        \item Se $p \nmid q-1$, si ha che $\ord \varphi_y \mid 1 \implies \varphi_y = id$, dunque l'unico automorfismo possibile di $\Z{q}$ è l'identità, pertanto si ha un prodotto diretto tra $\Z{p}$ e
            $\Z{q}$ e quindi esiste ed è unico il gruppo di ordine $pq$: $\Z{pq}$.  
        \item Se $p \mid q-1$, allora o $\ord \varphi_y = 1$ e quindi ancora $\varphi_y = id$; oppure $\ord \varphi_y = p$, e poiché ci sono $p-1$ elementi di ordine $p$ in $\Z{(q-1)}$, abbiamo $p-1$
            scelte per $\varphi_y$ che danno un prodotto semidiretto.
    \end{itemize}

Si osserva che $\ord_{\Aut(\left<x\right>)} \varphi_y = \ord_{\Z{q}^*}(\overline l)$, in quanto:
\[ \varphi_y(x) = x^l \implies (\varphi_y(x))^k\footnote{In questo caso si intende $\underbrace{\varphi_y \circ \ldots \circ \varphi_y}_{\text{$k$-volte}}(x)$, da cui l'esponente $l^k$ di $x$.} = x^{l^k}
    \]
quindi $\ord \varphi_y = p \iff l^p \equiv 1 \pmod q \iff \ord(\ol l) = p$.

Possiamo verificare che le $p-1$ scelte per $\varphi_y$ danno tutte gruppi isomorfi, quindi se $p \mid q-1$ ci sono esattamente due gruppi di ordine $pq$ a meno di isomorfismo. Detti:
    \[ G_1 \cong \left<x\right> \rtimes_{\varphi} \left<y\right>
    \qquad \text e \qquad 
    G_2 \cong \left<x\right> \rtimes_{\psi} \left<y\right>
        \]
con $\varphi_y(x) = x^l$, $\psi_y(x) = x^\lambda$ e $\ord_{\Z{q}^*}(\ol\lambda) = \ord_{\Z{q}^*}(\ol l) = p$ (necessariamente, per quanto detto sopra è l'unico altro ordine possibile, oltre ad 1), abbiamo $\left<l\right> = \left<\lambda\right>$ se e solo se 
$l = \lambda^r$, con $0 < r < p$. Consideriamo l'applicazione:
    \[ \mathcal{F} : G_1 \longrightarrow G_2 : x \longmapsto x, y \longmapsto y^r
        \]
essa definisce un isomorfismo tra i due gruppi $G_1$ e $G_2$. Per verificare che la mappa sia effettivamente un isomorfismo, consideriamo le presentazioni dei due gruppi:
    \[ G_1 = \left<x,y \m x^q = y^p = e_1, yxy^{-1} = x^l\right> \quad \text e \quad G_2 = \left<x,y \m x^q = y^p = e_2, yxy^{-1} = x^\lambda\right>
        \]
affinché $\mathcal{F}$ sia un isomorfismo, deve rispettare gli ordini dei generatori e verificare che sia preservata la relazione di commutazione su di essi definita; osserviamo che:
    \[ \mathcal{F}(x^q) = (\mathcal{F}(x))^q = x^q = e_2 \qquad \text{in quanto $x^q = e_2$ in $G_2$}
        \]
e anche:
    \[ \mathcal{F}(y^p) = (\mathcal{F}(y))^p = (y^r)^p = (y^p)^r = e_2 \qquad \text{in quanto $y^p = e_2$ in $G_2$}
        \]
ed infine:
    \[ \mathcal{F}(yxy^{-1}) = \mathcal{F}(x^l)
        \]
in quanto:
    \[ \mathcal{F}(yxy^{-1}) = \mathcal{F}(y) \mathcal{F}(x) \mathcal{F}(y^{-1}) = \underbrace{y^rxy^{-r}}_{\in G_2} = (\varphi_y(x))^r \footnote{Anche in questo caso si intende la composizione $r$ volte.} = x^{\lambda^{r}} =\footnote{Qui stiamo usando l'ipotesi per cui $l = \lambda^r$.} x^l = \mathcal{F}(x^l)
        \]
ciò garantisce che $\mathcal{F}$, ottenuto estendendo l'assegnamento $x \longmapsto x, y \longmapsto y^r$ a tutto il gruppo, è un omomorfismo; si verifica inoltre che è anche una bigezione e quindi è un isomorfismo.

\newpage
\subsection{Teorema di struttura per i gruppi abeliani finiti}

\begin{theorem}
    [Teorema Di Struttura Dei Gruppi Abeliani Finiti]
    \label{t:struttura}
    Sia $G$ un gruppo abeliano finito, allora $G$ è prodotto diretto di gruppi ciclici, cioè:
        \[ G \cong \Z{n_1} \times \ldots \times \Z{n_s}
            \]
    Inoltre tale scrittura è unica se $n_{i+1} \mid n_i$, $\forall i \in \{1,\ldots,s-1\}$.
\end{theorem}

\begin{remark}
    [Schema della dimostrazione]
    Sia:
        \[ G(p) = \{g \in G | \ord(g) = p^k, k \in \NN\}
            \]
    $G(p)$ prende il nome di \vocab{$p$-componente} o componente di \vocab{$p$-torsione}.
    Si osserva che:
        \begin{itemize}
            \item $G(p)$ è un sottogruppo di $G$ perché $G$ è abeliano, dunque:
                \[ \ord(xy) \mid [\ord(x),\ord(y)] \qquad \forall x,y \in G
                    \]
                quindi se $x$ ed $y$ hanno per ordine una potenza di $p$, anche il prodotto ha per ordine una potenza di $p$, quindi $xy \in G(p)$, ed essendo 
                $G$ finito allora $G(p)$ è un sottogruppo. \footnote{Si osserva che le $p$-componenti sono $p$-gruppi.}
            \item $G(p)$ è un sottogruppo caratteristico di $G$ (ciò segue dal fatto che gli automorfismi conservano l'ordine degli elementi, e quindi $G(p)$ viene mandato in $G(p)$).
        \end{itemize}
\end{remark}

\begin{theorem}
    [I gruppi abeliani sono prodotto delle loro $p$-componenti]
    \label{t:t1}
    Sia $G$ un gruppo abeliano, con $|G| = n = p_1^{e_1}\ldots p_s^{e_s}$, con i primi $p_i \ne p_j$, $\forall i \ne j$, allora:
        \[ G \cong G(p_1) \times \ldots \times G(p_s)
            \]
    Inoltre la decomposizione di $G$ come prodotto di $p$-gruppi di ordine tra loro coprimi è unica.
\end{theorem}

\begin{theorem}
    [I $p$-gruppi si spezzano come prodotto di $p$-gruppi ciclici]
    \label{t:t2}
    Sia $G$ un $p$-gruppo abeliano. Esistono e sono univocamente determinati $r_1,\ldots,r_s$ tali che $r_1 \geq r_2 \geq \ldots \geq r_t$\footnote{L'ordine
     degli esponenti assicura l'unicità della fattorizzazione.}, per i quali:
        \[ G \cong \Z{p^{r_1}} \times \ldots \times \Z{p^{r_t}}
            \]
\end{theorem}

\pagebreak

Segue la dimostrazione del \hyperref[t:struttura]{Teorema Di Struttura Dei Gruppi Abeliani Finiti}:

\begin{proof}
    \underline{\textbf{Esistenza}}: Dato il gruppo $G$, abeliano e finito, per il \hyperref[t:t1]{Teorema 1.87} si ha:
        \[ G \cong G(p_1) \times \ldots \times G(p_s)
            \]
    possiamo applicare il \hyperref[t:t2]{Teorema 1.88} ad ognuno dei fattori $G(p_i)$ ed ottenere:
        \begin{multline*}
            G \cong G(p_1) \times \ldots \times G(p_s) \cong \\
            \cong (\Z{p_1^{r_{1_1}}} \times \ldots \Z{p_1^{r_{1_{t_1}}}}) \times \ldots \times (\Z{p_s^{r_{s_1}}} \times \ldots \Z{p_s^{r_{s_{t_s}}}})
        \end{multline*}
    con $r_{i_1} \geq \ldots \geq r_{i_{t_i}}$. Per il Teorema Cinese del Resto possiamo rimettere assieme i termini formati da primi distinti in modo da mantenere la relazione di divisibilità (e quindi unicità)
    richiesta dal teorema:
    \[ \Z{\underbrace{(p_1^{r_{1_1}}\ldots p_s^{r_{s_1}})}_{n_1}} \times \ldots \times \Z{\underbrace{(p_1^{r_{1_t}}\ldots p_s^{r_{s_t}})}_{n_t}}
        \]
    dove $t = \max\{t_1,\ldots,t_s\}$ e poniamo $r_{i_h} = 0$ se $h>t_i$. Si osserva che, per come abbiamo riscritto la fattorizzazione si ha: $n_t \mid n_{t-1} \mid \ldots \mid n_1$.\\
    \underline{\textbf{Unicità:}} Segue dall'unicità del \hyperref[t:t1]{Teorema 1.87} e del \hyperref[t:t2]{Teorema 1.88}, infatti se ci fossero due decomposizioni di 
    $G$ diverse con ordini che si dividono in catena, ripercorrendo gli isomorfismi, avremmo all'inizio due diverse decomposizioni per $G(p)$ (o per $G$ come prodotto di $p$-componenti).
\end{proof}

\begin{example}
    Sia $G \cong \Z{100} \times \Z8 \times \Z6 \times \Z{15} \cong \Z{2^2} \times \Z{5^2} \times \Z{2^3} \times \Z2 \times \Z3 \times \Z3 \times \Z5$, e raggruppando in base all'ordine degli elementi
    otteniamo i $p$-sottogruppi:
        \[ G \cong \underbrace{(\Z{2^3} \times \Z{2^2} \times \Z{2})}_{G(2)} \times \underbrace{(\Z3 \times \Z3)}_{G(3)} \times \underbrace{(\Z{5^2} \times \Z5)}_{G(5)}
            \]
    e per il \hyperref[t:struttura]{Teorema Di Struttura} possiamo riscrivere il prodotto in ordine decrescente (rimettendo assieme $p$-gruppi ciclici di ordine massimo):
        \[ G \cong \Z{(2^3\cdot 3 \cdot 5^2)} \times \Z{(2^2 \cdot 3 \cdot 5)} \times \Z2
            \]
\end{example}

\begin{example}
    Classificare i gruppi abeliani di ordine $1000$. Per fare ciò osserviamo che $1000 = 2^3 \cdot 5^3$, allora:
    \[  G = G(2) \times G(5)
        \]
    con $|G(2)| = 2^3$, e $|G(5)| = 5^3$ pertanto le $p$-componenti possono essere scritti come prodotto di gruppi ciclici nei seguenti modi:
    \[ G(2) \cong \begin{cases}
                    \Z{2^3} \\
                    \Z{2^2} \times \Z2 \\
                    \Z2 \times \Z2 \times \Z2
                \end{cases}
    \qquad \text e \qquad
    G(5) \cong \begin{cases}
        \Z{5^3} \\
        \Z{5^2} \times \Z5 \\
        \Z5 \times \Z5 \times \Z5
    \end{cases}
        \]
    Dunque i gruppi abeliani di ordine $1000$ (a meno di isomorfismo) sono $3\cdot 3 = 9$, in quanto per il \hyperref[t:struttura]{Teorema di Struttura} abbiamo
    una fattorizzazione unica come prodotto di gruppi cicli finiti, e per tale fattorizzazione abbiamo $3$ scelte per la $2$-componente e $3$ scelte per la $5$-componente.
\end{example}

Dimostriamo ora il \hyperref[t:t1]{Teorema 1.87}

\begin{proof}
    \underline{\textbf{Esistenza}}: Sia $|G| = n$, con $n = p_1^{e_1} \ldots p_s^{e_s}$, procediamo per induzione su $s$. Nel caso in cui $s = 1$, si ha $|G| = p_1^{e_1} \implies G = G(p_1)$. Supponiamo 
    la tesi vera $\forall m : 2 \leq m < n$, possiamo scrivere $n = m m^{\prime}$ con $(m,m^{\prime}) = 1$ e $m,m^{\prime} < n$, allora (in notazione additiva) vogliamo verificare che:
        \[ G \cong mG \times m^{\prime}G
            \]
        È facile verificare che $mG,m^{\prime}G < G$ (basta vedere la chiusura per l'operazione), ed essendo $G$ abeliano si ha anche $mG,nG \triangleleft G$; si osserva inoltre che, essendo $(m,m^{\prime}) = 1$, allora $\exists h,k \in \ZZ$:
            \[ mh + m^{\prime}k = 1 \implies m(gh) + m^{\prime}(gk) = g \qquad \forall g \in G \implies G \subseteq mG + m^{\prime}G
                \]
        il contrario è ovvio, dunque:
            \[ mG + m^{\prime}G = G
                \]
        Inoltre, sia $x \in mG \cap m^{\prime}G$, ovvero $x = mg = m^{\prime}g^{\prime}$, allora si osserva che $m^{\prime}x = m^{\prime}mg = ng = 0$ e $mx = mm^{\prime}g^{\prime} = ng^{\prime} = 0$, dunque:
            \[ \ord(x) \mid m \qquad \text e \qquad \ord(x) \mid m^{\prime} \implies \ord(x) \mid (m,m^{\prime}) = 1 \implies x = 0
                \]
        Quindi $mG \cap m^{\prime}G = \{e\}$, pertanto sono verificate ipotesi del \hyperref[t:1.72]{Teorema 1.72}, dunque è vero che $G \cong mG \times m^{\prime}G$. Osserviamo che:
            \[ mG = G_{m^{\prime}} = \{g \in G | m^{\prime}g = 0\} \qquad \text e \qquad m^{\prime}G = G_m = \{g \in G | mg = 0\}
                \]
        Verifichiamo (WLOG) $m^{\prime}G = G_m$ mostrando la doppia inclusione tra insiemi; $m^{\prime}G \subseteq G_m$, ovvero $m^{\prime}x \in G_m$, perché $mm^{\prime}x = nx = 0$, viceversa, preso 
        $x \in G_m$, ovvero $mx = 0$, per quanto visto sopra abbiamo che:
            \[ \underbrace{mx}_{= 0}h + m^{\prime}kx = x \implies x = m^{\prime}(kx) \implies x \in m^{\prime}G
                \]
        quindi $G_m \subseteq m^{\prime}G \implies m^{\prime}G = G_m$. Pertanto possiamo scrivere:
            \[ G \cong G_m \times G_{m^\prime}
                \]
        Poiché $|G_m|,|G_{m^\prime}| < |G|$, perché $G_m$ contiene tutti e soli gli elementi di $G$ di ordine che divide $m$, inoltre $G_m \ne \{0\}$ (per \hyperref[p:Cauchy]{Cauchy}, dato che $1 < m < n$),
        quindi $G_{m^{\prime}} \lneq G$ e viceversa. Possiamo quindi applicare l'ipotesi induttiva e scrivere:
            \[ G_m = \prod_{i \in I}G(p_i) \qquad \text e \qquad G_{m^{\prime}} = \prod_{j \in J}G(p_j)
                \]
        con $I \cup J = \{1,\ldots,s\}$ e $I \cap J = \emptyset$ (poiché $(m,m^{\prime}) = 1$). \\
        \underline{\textbf{Unicità}}: La scrittura come prodotto di $p$-componenti è unica, perché se $G$ fosse anche isomorfo ad altri $p$-gruppi:
            \[ G \cong H_1 \times \ldots \times H_n \qquad \text{con $H_i$ $p_i$-gruppo e $H_i<G$}
                \]
        allora $H_i \subseteq G(p_i)$ (in quanto $G(p_i)$ contiene tutti gli elementi di ordine potenze di $p_i$), ma:
            \[ |G| = |H_1| \ldots |H_s| = |G(p_1)| \ldots |G(p_s)| \implies\footnote{Ciò deriva dal fatto che per coprimalità tra gli altri fattori $|G(p_i)|$ divide $|H_i|$, dunque vale anche la divisibilità inversa tra gli ordini.} |H_i| = |G(p_i)| \qquad \forall i \in \{1,\ldots,s\}
                \]
        quindi segue che $H_i = G(p_i)$, $\forall i \in \{1,\ldots,s\}$. 
\end{proof}

\begin{lemma}
    \label{l:1.91}
    Sia $G$ un $p$-gruppo abeliano e sia $x_1$ un elemento di ordine massimo in $G$, preso $\overline x \in \faktor{G}{\left<x_1\right>}$
     esiste $y \in \pi^{-1}_{\left<x_1\right>}(\overline x) : \ord_G(y) = \ord_{G/\left<x_1\right>}(\overline x)$, ovvero preso un elemento nel quoziente, esiste sempre un elemento nella sua fibra con lo stesso ordine.
\end{lemma}

\begin{proof}
    Nelle ipotesi in cui siamo, sia $\pi^{-1}_{\left<x_1\right>}(\overline x) = \pi^{-1}_{\left<x_1\right>}(x + \left<x_1\right>)$, dunque l'elemento $y \in \pi^{-1}(\overline x)$ che cerchiamo è della forma:
        \[ y = x + ax_1
            \]
    Sappiamo che $\pi_{\left<x_1\right>}(y) = \pi_{\left<x_1\right>}(x) = \overline x$ (stiamo considerando due elementi nella stessa classe laterale); dato che il quoziente è ancora un $p$-gruppo, sia $p^r = \ord_{\left<x_1\right>}(\pi_{\left<x_1\right>}(y)) = \ord_{\left<x_1\right>}(\overline x) \mid \ord_G(y)$ (per le proprietà di omomorfismo),
    possiamo scegliere $y$ (scegliendo $a$\footnote{Infatti $x_1$ è fissato per ipotesi, mentre $x$ è fissato perché determinato da $\ol x$, sempre per ipotesi.}) in modo che il suo ordine sia esattamente $p^r$ (dalla divisibilità precedente sappiamo che $p^r$ divide il suo ordine):
        \[ (p^ry =) p^rx + p^rax_1 = 0 \iff p^rx = -p^rax_1 
            \]
    dove essendo $\ord_{\left<x_1\right>}(\overline x) = p^r $ allora $p^rx \in \left<x_1\right>$ (ovvero la sua proiezione modulo $\left<x_1\right>$, sta nella classe laterale banale) dunque $p^rx = bx_1$. Per ipotesi avevamo assunto che $x_1$ ha ordine massimo,
    chiamiamolo $p^{r_1}$, deve essere che $r \leq r_1$, ma:
        \[ 0 = p^{r_1}x \iff\footnote{Abbiamo moltiplicato e diviso per $p^r$.} p^{r_1 - r}p^rx = 0 \iff  p^{r_1 - r}bx_1 = 0
            \]  
    quindi l'ultima uguaglianza è vera se e solo se $p^{r_1}\mid p^{r_1 - r}b$ (essendo $\ord_G(x_1) = p^{r_1}$), dunque se e solo se $p^r \mid b \implies b = p^rb_1$. Infine, scegliendo $a = -b_1$ e sostituendo nell'espressione iniziale, si ha:
        \[ p^ry = p^rx - p^rb_1x_1 = bx_1 - \underbrace{p^rb_1}_{= b}x_1 = 0
            \]
    pertanto $y = x -b_1x_1 \in G$ realizza la proprietà richiesta.
\end{proof}

\pagebreak

Dimostriamo ora il \hyperref[t:t2]{Teorema 1.88}:

\begin{proof}
    \underline{\textbf{Esistenza}}: Sia $G$ un $p$-gruppo, $|G| = p^n$, proviamo la tesi per induzione su $n$. Per $n = 1$ si ha che $|G| = p \implies G \cong \Zp$, e quindi la tesi è verificata.
    Supponiamo la tesi vera per $1 \leq m <n$ e proviamola per $n$; sia $x_1 \in G$ un elemento di ordine massimo, $\ord(x_1) = p^{r_1}$:
    \begin{itemize}
        \item Se $r_1 = n$, allora $G$ è ciclico $\implies G \cong \Z{p^n}$.
        \item Se $r_1 < n$, poiché $G$ è abeliano si ha $\left<x_1\right> \triangleleft G$, quindi possiamo considerare $\faktor{G}{\left<x_1\right>}$ che ha ordine $p^{n-r_1} < p^n$, dunque vale l'ipotesi induttiva ed il gruppo quoziente può essere fattorizzato come prodotto di gruppi ciclici:
            \[ \faktor{G}{\left<x_1\right>} \cong \left<\overline{x_2}\right> \times \ldots \times \left<\overline{x_t}\right> \footnote{Dunque si ha $|\left<\overline{x_2}\right> \times \ldots \times \left<\overline{x_t}\right>| = p^{n - r_1}$.}
                \]
            sia $\ord(\overline{x_i}) = p^{r_i}$, e supponiamo di aver scritto il prodotto diretto in modo ordinato, con $r_2 \geq \ldots \geq r_t$. Consideriamo la proiezione al quoziente:
            \[ \pi : G \longrightarrow \faktor{G}{\left<x_1\right>} \cong \left<\overline{x_2}\right> \times \ldots \times \left<\overline{x_t}\right>\footnote{L'isomorfismo tra i due gruppi è quello che manda $ \left(G/\left<x_1\right>\ni\right) \overline g = a_2\overline{x_2} + \ldots + a_t\overline{x_t}$ 
            (poiché $G/\left<x_1\right>$ è finito è anche finitamente generato) in $( a_2\overline{x_2} , \ldots , a_t\overline{x_t}) (\in \left<\overline{x_2}\right> \times \ldots \times \left<\overline{x_t}\right>)$.}
                \]
            per il \hyperref[l:1.91]{Lemma 1.91} esistono $x_2,\ldots,x_t \in G$ tali che $\ord_G(x_i) = \ord_{G/\left<x_1\right>}(\overline{x_i}) = p^{r_i}$. Vogliamo mostrare allora che:
                \[ H = \left<x_2,\ldots,x_t\right> \cong \left<{x_2}\right> \times \ldots \times \left<{x_t}\right>
                    \]
            ovvero che il sottogruppo di $G$ finitamente generato da $x_2,\ldots,x_t$ è isomorfo al prodotto diretto dei singoli sottogruppi ciclici generati dai medesimi elementi. Consideriamo di nuovo la proiezione al quoziente modulo $\left<x_1\right>$, ma ristretta ad $H$:
                \[ \pi_{\mid H} : H \longrightarrow \faktor{G}{\left<x_1\right>} \cong \left<\overline{x_2}\right> \times \ldots \times \left<\overline{x_t}\right> : a_2x_2 + \ldots + a_tx_t \longmapsto (a_2\overline{x_2}, \ldots, a_t\overline{x_t})
                    \]
            è un isomorfismo, infatti $\pi$ è un omomorfismo, è surgettivo (in quanto si possono mandare tutti i generatori $x_i$ di $H$ nelle $t$-uple di generatori di $G/\left<x_1\right>$); per l'iniettività si osserva che gli elementi del nucleo sono del tipo:
                \[ \pi(a_2x_2 + \ldots + a_tx_t) = (a_2\overline{x_2}, \ldots, a_t\overline{x_t}) = (0,\ldots,0) \iff a_i\overline{x_i} = 0 \quad \forall i \in\{2,\ldots,t\}
                    \]
            cioè se e solo se $\ord_{G/\left<x_1\right>}(\overline{x_i}) =\footnote{Qui stiamo usando ancora il \hyperref[l:1.91]{Lemma 1.91}} p^{r_i} \mid a_i$, $\forall i \in \{2,\ldots,t\}$. Segue che $\pi_{\mid H}$ è un isomorfismo e si ha:
                \[ H \cong \left<\overline{x_2}\right> \times \ldots \times \left<\overline{x_t}\right> \cong \left<{x_2}\right> \times \ldots \times \left<{x_t}\right>
                    \]
            Dove l'ultimo isomorfismo deriva dal fatto che abbiamo scelto elementi di ordini uguali, che quindi generano gli stessi gruppi ciclici a meno di isomorfismo.
            Mostriamo che $G \cong \left<x_1\right> \times H (\cong \left<x_2\right> \times \ldots \times \left<x_t\right>)$ e per farlo verifichiamo che le ipotesi del \hyperref[t:1.72]{Teorema 1.72} siano soddisfatte. \\
            Per mostrare che l'intersezione è banale, consideriamo un elemento in quest'ultima, ovvero un elemento che può essere scritto come:
                \[ a_1x_1 = a_2x_2 + \ldots + a_tx_t
                    \]
            applicando $\pi$ alle due scritture si ha:
                \[ \overline 0 =  a_2\overline{x_2} + \ldots + a_t\overline{x_t} \iff (a_2\overline{x_2}, \ldots, a_t\overline{x_t}) = (\overline 0,\ldots, \overline 0)
                    \]
            in quanto $G/\left<x_1\right> \cong \prod_{i= 2}^{t}\left<\overline{x_i}\right>$, dunque l'unica possibilità di annullare la somma scritta è che $a_i \equiv 0 \pmod{p^{r_1}}$ (ovvero $a_i$ è multiplo dell'ordine di $\overline{x_i}$), $\forall i \in \{2,\ldots,t\}$, da ciò segue che
            anche nel gruppo di partenza $a_i = 0$ e quindi $a_1x_1 = 0$, pertanto $\left<x_1\right> \cap H = \{0\}$.
            Per mostrare che $\left<x_1\right> + H = G$, osserviamo che $\left<x_1\right> + H \subseteq G$ e che la sua cardinalità è:
                \[ |\left<x_1\right> + H| = \frac{|\left<x_1\right>||H|}{|\left<x_1\right> \cap H|} = \frac{p^{r_1}\cdot p^{n - r_1}}{1} = p^n
                    \]
            Le ipotesi sono soddisfatte e quindi $G \cong \left<x_1\right> \times H \cong \left<x_1\right> \times \ldots \times \left<x_t\right>$.
    \end{itemize}
\underline{\textbf{Unicità}}: Sia $|G| = p^n$ e procediamo ancora per induzione su $n$. Per $n = 1$ segue sempre $G \cong \Zp$ e quindi la tesi è verificata. Supponiamo la tesi vera per $m<n$ e proviamola per $n$; sia:
        \[ G \cong \Z{p^{r_1}} \times \ldots \times \Z{p^{r_t}} \cong \Z{p^{k_1}} \times \ldots \times \Z{p^{k_s}}
            \]
        dove supponiamo $r_1 \geq \ldots \geq r_t$ e $k_1 \geq \ldots \geq k_s$. Deve essere necessariamente che $t = s$, perché, considerando:
            \[ G_p = \{g \in G | pg = 0\}
                \]
        con $G_p$ gruppo caratteristico (poiché gli isomorfismi conservano gli ordini degli elementi) e quindi:
            \[ G_p \cong \left(\Zp\right)^t \cong \left(\Zp\right)^s \implies t = s
                \]
        Quindi le lunghezze delle fattorizzazioni sono uguali, per concludere ci basta utilizzare l'ipotesi induttiva al gruppo $pG$ (con $|pG| = p^{n-t}$):
            \[ pG \cong \frac{p\ZZ}{p^{r_1}\ZZ} \times \ldots \times \frac{p\ZZ}{p^{r_t}\ZZ} \cong \Z{p^{r_1-1}} \times \ldots \times \Z{p^{r_t-1}} \cong \Z{p^{k_1-1}} \times \ldots \times \Z{p^{k_t-1}}
                \]
        quindi $pG$ ha decomposizione unica (per ipotesi induttiva), da cui:
            \[ r_1 - 1 = k_1 -1, \ldots, r_t - 1 = k_t - 1 \iff r_1 = k_1, \ldots , r_t = k_t
                \]
\end{proof}

\begin{remark}
    Il \hyperref[l:1.91]{Lemma 1.91} non vale in generale per quozienti qualsiasi, ad esempio:
        \[ \faktor{\Z{p^2}}{\left<p\right>} \cong \frac{\Z{p^2}}{\Zp}\cong \Zp
            \]
    e con la proiezione:
        \[ \pi: \Z{p^2} \longrightarrow \frac{\Z{p^2}}{\Zp} \cong \Zp : 1 \longmapsto \overline 1
            \]
    con $\overline 1$ che ha ordine $p$ nel gruppo di arrivo, mentre:
        \[ \pi^{-1}(\overline 1) = \{1 + kp\}_{k = 1,\ldots,p-1}
            \]
    con $1+kp$ che ha ordine $p^2$, $\forall k : 1 \leq k \leq p$, dunque stiamo quozientando per un elemento che non ha ordine massimo;
     nelle condizioni del lemma, invece, stiamo quozientando per un elemento di ordine massimo.
\end{remark}


\newpage
\subsection{Teorema di Sylow}

\begin{remark}
    Dato un gruppo $G$ finito cosa possiamo dire dell'esistenza di elementi e sottogruppi di un certo ordine? Riepiloghiamo di seguito i principali risultati visti:
        \begin{itemize}
            \item $H \leqslant G \implies |H| \mid |G|$ (Teorema Di Lagrange).
            \item $\forall p$ primo tale che $p \mid |G|$, $\exists x \in G : \ord_G(x) = p$ (\hyperref[p:Cauchy]{Teorema Di Cauchy}).
            \item Se $G$ è ciclico, $\forall d \mid |G|$, $\exists x \in G : \ord_G(x) = d$ (dalla definizione di gruppo ciclico).
            \item $G$ è ciclico se e solo se $d = |G|$ (esiste $x \in G$ tale che $d = \ord_G(x)$).
            \item Se $G$ è abeliano $\forall d \mid |G|$, $\exists H \leqslant G$ tale che $|H| = d$ (\hyperref[davide]{Lemma Di Ranieri}).
        \end{itemize}
\end{remark}
L'ultimo fatto può essere ricavato (alternativamente) dal \hyperref[t:struttura]{Teorema di Struttura}, infatti:
    \[ G = G_{p_1} \times \ldots \times G_{p_r}
        \]
con $|G| = p_1^{e_1}\ldots p_r^{e_r}$, se $d = p_1^{a_1}\ldots p_r^{a_r}$, bisogna verificare che per ogni $i$ esiste $H_{p_i} \leqslant G_{p_i}$ tale che $|H_{p_i}| = p^{a_i}$.
Poiché:
    \[ G = \Z{p^{n_1}} \times \ldots \Z{p^{n_s}} \qquad \text{con $\sum n_i = e$}
        \]
possiamo costruire sottogruppi di ogni ordine \footnote{Ad esempio $|H_p| = p^{72}$, preso $G_p = \Z{p^{30}} \times \Z{p^{30}} \times \Z{p^{30}}$, può essere ottenuto come $H_p = \Z{p^{30}} \times \Z{p^{30}} \times p^{18} \Z{p^{30}}$.};
inoltre, dato che $G$ è abeliano il prodotto di sottogruppi è un sottogruppo:
    \[ H_{p_1}\ldots H_{p_r} < H
        \]
e inoltre:
    \[ H_{p_1}\ldots H_{p_r} \cong H_{p_1} \times \ldots \times H_{p_r}
        \]
poiché $H_{p_i} \cap H_{p_j} = \{e\}$, dunque:
    \[ |H_{p_1}\ldots H_{p_r}| = \prod |H_{p_i}| = \prod p_{i}^{a_i} = d
        \]
e quindi otteniamo il sottogruppo di ordine $d$ voluto.
    
\begin{remark}
    Se $G$ non è abeliano e $d \mid |G|$ non è detto che $G$ abbia sottogruppi di ordine $d$.
\end{remark}

\begin{example}
    [$\mathcal{A}_4$ non contiene sottogruppi di ordine $6$]
    Sappiamo che $|\mathcal{A}_4| = 4!/2 = 12$, se $\exists H < \mathcal{A}_4$ di ordine $6$, allora $H \triangleleft \mathcal{A}_4$; per \hyperref[p:Cauchy]{Cauchy}
    $\exists x \in H : \ord(x) = 2$, con $x = \cycle{a,b}\cycle{c,d}$, deve essere quindi che:
        \[ \Cl_{\mathcal{A}_4}(x) \subset H
            \]
    poiché $H \triangleleft \mathcal{A}_4$ e per definizione è unione di classi di coniugio in $\mathcal{A}_4$. Sappiamo che:
        \[ \Cl_{\mathcal{A}_4}(x) = \{ \cycle{1,2}\cycle{3,4}, \cycle{1,3}\cycle{2,4}, \cycle{1,4}\cycle{2,3}\}
            \]
    Visto che $|\Cl_{\mathcal{A}_4}(\cycle{a,b} \cycle{c,d})| = 3$, allora $\Cl_{\mathcal{A}_4}(\cycle{a,b} \cycle{c,d}) = \Cl_{S_4}(\cycle{a,b} \cycle{c,d})$,
    dunque se $H \triangleleft \mathcal{A}_4 \implies H \supset \{e, \cycle{1,2}\cycle{3,4}, \cycle{1,3}\cycle{2,4}, \cycle{1,4}\cycle{2,3}\} = V$ \footnote{$V$ prende il nome di \vocab{gruppo di Klein} o \vocab{Klein 4-group}.},
    allora $V < H$, ma $4 \nmid 6 \implies$ assurdo.
\end{example}

\begin{lemma}
    \label{l:1.95}
    Sia $G$ un $p$-gruppo e $H \lneq G$, allora $H \lneq N_G(H)$.
\end{lemma}

\begin{proof}
    Essendo $G$ un $p$-gruppo abbiamo che $|G|=p^n$. Procediamo per induzione su $n$. Se $n=0$ non c'è niente da dimostrare. Se $n>0$ consideriamo due casi: 
    \begin{itemize}
        \item Se $Z(G)\not\subseteq H$, dato che $H\cup Z(G)\subseteq N_G(H)$, abbiamo che $\emptyset\neq Z(G)\setminus H\subseteq N_G(H)\setminus H$ da cui la tesi.
        \item Se $Z(G)\subseteq H$ osserviamo che $Z(G)$ è normale in $G$ e che $Z(G)$ non è banale perché $G$ è un $p-$gruppo, dunque possiamo considerare $\faktor{G}{Z(G)}$
        di ordine strettamente minore all'ordine di $G$. Sia $\pi:G \longrightarrow \faktor{G}{Z(G)}$ la mappa di proiezione al quoziente. Per ipotesi induttiva $N_{G/Z(G)}\left(\faktor{H}{Z(G)}\right)$ contiene strettamente $\faktor{H}{Z(G)}$, quindi per il teorema di corrispondenza si ha che anche
        le loro controimmagini tramite $\pi$ rispettano un contenimento stretto (perché si preservano gli indici). Sempre per corrispondenza $\pi^{-1}\left(N_{G/Z(G)}\left(\faktor{H}{Z(G)}\right)\right)=H$, quindi basta mostrare che $\pi^{-1}\left(N_{\faktor{G}{Z(G)}}\left(\faktor{H}{Z(G)}\right)\right)\subseteq N_G(H)$, e questo deriva 
        dal fatto che se $g\in \pi^{-1}\left(N_{G/Z(G)}\left(\faktor{H}{Z(G)}\right)\right)$ allora $gHg^{-1}\subseteq HZ(G)=H$.\footnote{Dimostrazione proposta da Francesco Sorce.}
    \end{itemize}
\end{proof}

\begin{definition}
    Dato $G$ un gruppo finito e $p$ un primo, tali che $|G| = p^nm$, con $p^n \pdiv |G|$\footnote{Il simbolo $\pdiv$ indica la divisibilità esatta, ovvero $p^n$ è la massima potenza di $p$ che divide $|G|$.} 
    e $n \geq 1$ e $(m,p) = 1$, allora un sottogruppo di $G$ di ordine $p^n$ prende il nome di \vocab{$p$-sottogruppo di Sylow} (\vocab{$p$-Sylow}).\footnote{I $p$-sottogruppi di Sylow possono anche essere pensati come $p$-sottogruppi di ordine massimale.}
\end{definition}

\newpage

\begin{theorem}
    [Teorema Di Sylow]
    \label{Sylow}
    Sia $G$ un gruppo finito, con $|G| = p^nm$, con $p$ primo, $n \geq 1$ e $(m,p) = 1$ \footnote{Ovvero $p^n \pdiv |G|$, o anche $\nu_p(|G|) = n$ (dove con $\nu_p$ intendiamo la
     \href{https://it.wikipedia.org/wiki/Valutazione_p-adica}{\textcolor{purple}{valutazione $p$-adica}}).}, allora:
        \begin{enumerate}[(1)]
            \item $\forall \alpha : 0 \leq \alpha \leq n$, $\exists H \leqslant G$ : $|H| = p^\alpha$. {\color{orange}(Esistenza)}
            \item $\forall \alpha : 0 \leq \alpha \leq n-1$, ogni sottogruppo di ordine $p^\alpha$ è contenuto in un sottogruppo di ordine $p^{\alpha+1}$. In particolare,
                ogni $p$-sottogruppo è contenuto in un $p$-sottogruppo di Sylow. {\color{orange}(Inclusione)}
            \item Due qualunque $p$-sottogruppi di Sylow di $G$ sono coniugati (quindi tutti i $p$-sottogruppi di ordine massimale sono isomorfi). {\color{orange}(Coniugio)}
            \item Sia $n_p$ il numero di $p$-sottogruppi di Sylow di $G$, allora: {\color{orange}(Numero)}
                \[ n_p \mid |G| \qquad \text{e} \qquad n_p \equiv 1 \pmod p \qquad \text e \qquad n_p = [G : N_G(S)] \footnote{Con $S$ ci si riferisce a un qualsiasi $p$-Sylow, per un $p$ fissato.}
                    \]
        \end{enumerate}
\end{theorem}

\begin{proof}
    Dimostriamo tutte le affermazioni del teorema:
        \begin{enumerate}[(1)]
            \item Dimostriamo che $\forall \alpha : 0 \leq \alpha \leq n$ esiste almeno un sottogruppo di ordine $p^{\alpha}$; sia $\mathcal{M} = \{X \subset G | \#X = p^{\alpha}\}$, allora:
                \[ |\mathcal{M}| = \binom{|G|}{|X|} = \binom{p^nm}{p^\alpha} = \frac{p^nm(p^nm -1)\ldots(p^nm - p^\alpha + 1)}{p^\alpha(p^\alpha - 1)\ldots (p^{\alpha} - p^{\alpha} + 1)} \,\footnote{Si osserva che abbiamo semplificato al numeratore e al denominatore il termine $(p^{n}m - p^{\alpha})!$.}
                    \]
                Possiamo riscrivere il prodotto dei termini nel modo seguente:
                \[ \prod_{i=0}^{p^\alpha-1}\frac{p^nm - i}{p^\alpha - i} = p^{n-\alpha}m\prod_{i=1}^{p^\alpha-1}\frac{p^nm - i}{p^\alpha - i} 
                        \]
                dove nell'ultimo passaggio abbiamo raccolto il primo termine, $p^{n-\alpha}m$, e lo abbiamo portato fuori dalla produttoria.\\
                Osserviamo a questo punto che $p^{n-\alpha}$ è la più grande potenza di $p$ che divide $|\mathcal{M}|$\footnote{O anche $p^{n - \alpha} \pdiv |\mathcal{M}|$, o ancora $\nu_p(|\mathcal{M}|) = n - \alpha$.}, infatti,
                si osserva che $p \nmid \prod_{i=1}^{p^\alpha-1}\frac{p^nm - i}{p^\alpha - i}$, cioè $\forall \in \{1,\ldots,p^\alpha - 1\}$ si ha che $p \nmid \frac{p^nm - i}{p^\alpha - i}$, come si osserva infatti:
                \[ \nu_p(p^nm-i) = \nu_p(p^\alpha - i) = \nu_p(i)
                    \]
                dunque, se $p \nmid i \implies p^nm - i$ e $p^{\alpha} - i$ non sono divisibili per $p$; se fosse $i = p^kj$, con $(j,p) = 1$, allora
                 $p^{\alpha} - i = p^{\alpha} - p^kj = p^k\underbrace{(p^{\alpha - k} - j)}_{\text{non divisibile per $p$}}$, con $k < \alpha$, (analogamente per $p^nm - i$), per quanto abbiamo detto deve essere necessariamente che:
                \[ p^{n - \alpha} \pdiv |\mathcal{M}|
                    \] 
                ovvero $p^{n - \alpha}$ è l'esatta potenza di $p$ che divide $|\mathcal{M}|$.
                Consideriamo $M \in \mathcal{M}$, allora $gM \in \mathcal{M}$, $\forall g \in G$, dunque possiamo considerare l'azione:
                \[ \phi : G \longrightarrow S(\mathcal{M}) : g \longmapsto \varphi_g
                    \]
                dove $\varphi_g : \mathcal{M} \longrightarrow \mathcal{M} : M \longmapsto gM$ è una bigezione. Data l'azione $\phi$ sappiamo che:
                \[ \mathcal{M} = \bigcupdot_{i = 1}^{s}\Orb(M_i) \implies |\mathcal{M}| = \sum_{i = 1}^{s} |\Orb(M_i)| = \sum_{i = 1}^s \frac{|G|}{|\St(M_i)|}
                    \]
                unendo ciò a quanto detto si ha che $p^{n - \alpha} \pdiv \sum_{i = 1}^s \frac{|G|}{|\St(M_i)|}$, quindi non tutte le orbite possono essere divisibili per una potenza maggiore di $p^{n - \alpha}$, ovvero esiste almeno un $i$
                tale per cui $p^{n - \alpha + 1} \nmid |\Orb(M_i)|$(ovvero non può essere diviso per una potenza più grande di quanto detto), da ciò segue: $ p^{n - \alpha + 1} \nmid |\Orb(M_i)| = \frac{|G|}{|\St(M_i)|} = \frac{p^nm}{|\St(M_i)|}$,
                pertanto deve essere necessariamente che:
                \[ p^{\alpha} \mid |\St(M_i)| = t
                    \]
                cioè, affinché il rapporto non sia divisibile per $p^{\alpha}$, al denominatore deve esserci una potenza di $p$ maggiore o uguale ad $\alpha$. D'altra parte, sia $x \in M_i$, la funzione:
                \[ \varphi_x : \St(M_i) \longrightarrow M_i : y \longmapsto yx 
                    \]
                è iniettiva\footnote{Si vede che $\varphi_x(y) = \varphi_x(z) \iff yx = zx \iff y = z$.}, dunque $t = |\St(M_i)| \leq |M_i| = p^{\alpha}$, segue quindi $t = p^{\alpha}$, pertanto $\St(M_i)$ è il sottogruppo di ordine $p^{\alpha}$ cercato.
            \item Sia $S$ un $p$-sottogruppo di Sylow di $G$, con $|S| = p^n$, e sia $H \leqslant G$, con $|H| = p^{\alpha}$; consideriamo l'insieme $\faktor{G}{S} = X$ dato dalle classi laterali di $S$ in $G$, allora:
                \[ |X| = [G : S] = \frac{p^nm}{p^n} = m
                    \]
                Consideriamo l'azione di $H$ su $X$ data da:
                \[ \varphi: H \longrightarrow S(X) : h \longmapsto \varphi_h
                    \]
                con $\varphi_h : X \longrightarrow X : gS \longmapsto hgS$ bigezione; per la formula delle classi si ha:
                \[ m = |X| = \sum_{i = 1}^{r}|\Orb(g_iS)| = \sum_{i = 1}^{r} \frac{|H|}{|\St(g_iS)|} = \sum_{i = 1}^{r} p^{a_i}
                    \]
                (essendo $p$-gruppi). Poiché  per ipotesi $p \nmid m$, allora esiste $i$ tale che $a_i = 0$ (dunque c'è un $1$ nella fattorizzazione
                che impedisce la divisibilità di $m$ per $p$) $ \implies \Orb(g_iS) = \{g_iS\} \implies \St(g_iS) = H$ (ovvero per tale $i$ si ha una classe
                laterale $g_iS$ la cui orbita è solo se stessa, e quindi il suo stabilizzatore è tutto $H$). Da ciò segue che $\forall h \in H$:
                \[ hg_iS = g_iS \iff hg_i \in g_iS \iff h \in g_iSg_{i}^{-1}\iff H \subset g_iSg_i^{-1}
                    \]
                dove $|g_iSg_i^{-1}| = |S|$ dunque $g_iSg_i^{-1}$ è un $p$-Sylow ed $H$ di ordine $p^{\alpha}$ è contenuto in un $p$-Sylow. Questo dimostra il punto $(3)$,
                ovvero due $p$-Sylow di $G$ sono coniugati, infatti la relazione trovata vale per ogni $\alpha$ ed in particolare prendendo $|H| = p^n \implies H \leqslant g_iSg_{i}^{-1}$ ma
                i due sottogruppi hanno lo stesso ordine, quindi $H = g_iSg_{i}^{-1}$; pertanto, tutti i $p$-Sylow per ogni $p$ sono coniugati tra loro in $G$. \\
                Per completare la dimostrazione del punto $(2)$ utilizziamo il risultato del \hyperref[l:1.95]{Lemma 1.95}, considerando $|H| = p^{\alpha}$, con $\alpha \leq n - 1$ e $H \lneq S$ (stiamo supponendo che $H$ stia in $S$), dunque $H \lneq N_S(H)$ \footnote{Si noti che abbiamo preso il normalizzatore di $H$ in $S$.},
                sia ora $\frac{N_{S}(H)}{H}$, esso è un $p$-gruppo non banale e per il \hyperref[p:Cauchy]{Teorema di Cauchy} esiste una classe laterale $\overline x (= xH)$ di ordine $p$, infine, per il Teorema di Corrispondenza \footnote{Tra i sottogruppi di $\frac{N_{S}(H)}{H}$ ed i
                sottogruppi di $N_{S}(H)$ che contengono $H$.}, $\pi^{-1}_H(\left<\overline x\right>)$ è un sottogruppo di $N_S(H)$ che contiene $H$ (sempre per il Teorema Di Corrispondenza) ed ha ordine $p^{\alpha+1}$ (poiché stiamo considerando la controimmagine di un sottogruppo con $p$ elementi,
                ciascuno dei quali fatto da classi laterali di $p^{\alpha}$ elementi, dunque la cardinalità della controimmagine si ottiene moltiplicando la fibra di ciascun elemento, che appunto ha ordine $p^{\alpha}$, per il numero di elementi $p$).
                \skipitems{1}
            \item Sia $n_p$ il numero dei $p$-sottogruppi di Sylow, per quanto detto al punto $(3)$ i $p$-sottogruppi di Sylow sono tutti coniugati,
            dunque per ciò che abbiamo visto sul numero di coniugi rispetto all'azione di coniugio si ha $n_p = |\mathcal{C}\ell(S)|=[G:N_G(S)]$, da cui:
                        \[ n_p = \frac{|G|}{|N_G(S)|} \implies |G| = n_p|N_G(S)| \implies n_p \mid |G|
                            \]
                Sia $X$ l'insieme dei $p$-Sylow di $G$, consideriamo l'azione di coniugio:
                    \[ \phi : S \longrightarrow S(X) : s \longmapsto \varphi_s
                        \]
                con $\varphi_s : X \longrightarrow X : H \longmapsto sHs^{-1}$ bigezione; $\phi$ ha un'unica orbita banale, ovvero quella del gruppo $S$, $\Orb(S) = \{S\}$, infatti, per ogni altra orbita si ha:
                    \[ \Orb(H) = \{sHs^{-1} | s \in S\} = \{H\} \iff sHs^{-1} = H \qquad \forall s \in S
                        \]
                ovvero:
                    \[ S \subset N_G(H)
                        \]
                ma sappiamo anche che $H \lneq N_G(H)$, pertanto si deve avere che:
                    \[ HS < N_G(H)
                        \]
                (poiché $S$ normalizza $H$ il prodotto di sottogruppi da un sottogruppo), ma questo è assurdo se $S \ne H$, perché avremmo:
                    \[ |SH| = \frac{|S||H|}{|S \cap H|} = \frac{p^n \cdot p^n}{p^k} \footnote{$k<n$.} = p^{2n - k} \nmid |G|
                        \]
                Quindi esiste un'unica orbita banale e applicando la formula delle classi otteniamo:
                    \[ n_p = |X| = \sum_{i = 1}^{r} \underbrace{|\Orb(H_i)| }_{p^{a_i} \ne 1} + \underbrace{|\Orb(S)|}_{= 1} = pf +1 \qquad f \in \ZZ
                        \]
                o equivalentemente $n_p \equiv 1 \pmod p$.
                \end{enumerate}
\end{proof}

\pagebreak

\begin{corollary}
    Sia $G$ un gruppo abeliano finito, $\forall p$ primo tale che $p \mid |G|$, $G(p)$ è l'unico $p$-Sylow di $G$. Inoltre $G$ è il prodotto diretto dei suoi $p$-Sylow:
        \[ G \cong G(p_1) \times \ldots \times G(p_r)
            \]
    con $|G| = \prod p_i^{e_i}$.
\end{corollary}

\begin{proof}
    
\end{proof}
\nopagebreak 

\begin{example}
    [Classificazione dei gruppi di ordine $12$]
    Poiché $12 = 2^2 \cdot 3$, per Sylow, sappiamo che $\exists P_2,P_3$, con $P_2$ $2$-Sylow, $P_3$ $3$-Sylow e $|P_2| = 4$, $|P_3| = 3$;
    abbiamo che $P_2 \cap P_3 = \{e\}$ poiché $p$-gruppi distinti, dunque $G = P_2P_3$, in quanto:
        \[ |P_2P_3| = \frac{|P_2||P_3|}{|P_2 \cap P_3|} = \frac{4 \cdot 3}{1} = 12
            \]
    inoltre, almeno uno tra $P_2$ e $P_3$ è normale. Se $P_3 \triangleleft G$ allora abbiamo un sottogruppo normale; se $P_3 \ntriangleleft G$, allora osserviamo che, per quanto detto al punto $(4)$
    del \hyperref[Sylow]{Teorema Di Sylow}, possiamo avere solo che $n_3 = 1,4$, ma non essendo $P_3$ normale $n_3$ non può essere $1$, dunque $n_3 = 4$;
    da ciò segue che in $G$ ci sono $8$ elementi di ordine $3$\footnote{$4 \cdot 3 - 4 = 8$.} e $4$ elementi di ordine diverso da $3$, che quindi formano 
    l'unico $2$-Sylow, equivalentemente $n_2 = 1$, e quindi $P_2$ è normale. Osserviamo che supponendo invece $P_2 \ntriangleleft G$, si arriva simmetricamente a concludere
    che $P_3 \triangleleft G$, pertanto uno dei due sottogruppi di Sylow è necessariamente normale e in entrambi i casi sono soddisfatte le ipotesi del \hyperref[t:1.78]{Teorema 1.78},
    segue che $G$ è un prodotto semidiretto tra $P_2$ e $P_3$. Studiamo separatamente i due casi.
\end{example}
\nopagebreak 
\begin{example}[$G \cong P_2 \rtimes_{\varphi} P_3$]
    Se $P_2 \triangleleft G$, allora $G \cong P_2 \rtimes_{\varphi} P_3$. $P_2$ ha ordine $4$, dunque è $\Z{4}$ o $\Z{2} \times \Z{2}$, mentre $P_3$ è necessariamente $\Z{3}$; nel primo caso abbiamo:
                \[ \Z{4} \rtimes_{\varphi} \Z{3} \qquad \text{con} \qquad \varphi : \Z{3} \longrightarrow \Aut(\Z{4}) \cong \Z{2}
                    \]
    in questo caso l'unica possibilità è $[1]_3 \longmapsto id$, dunque il prodotto semidiretto è in realtà sempre un prodotto diretto, dunque il primo gruppo trovato è:
                \[ \color{LimeGreen!90!black}{\Z{4} \times \Z{3} \cong \Z{12}}
                    \]
    nel secondo caso abbiamo:
                \[ (\Z{2} \times \Z{2}) \rtimes_{\varphi} \Z{3} \qquad \text{con} \qquad \varphi : \Z{3} \longrightarrow \Aut(\Z{2} \times \Z2) \cong S_3
                    \]
    a questo punto, possiamo o mandare $[1]_3 \longmapsto id$ ottenendo il prodotto diretto:
            \[ \color{LimeGreen!90!black}{\Z{2} \times \Z{2} \times \Z{3} \cong \Z{2} \times \Z{6}}
                \]
    oppure mandare $[1]_3$ in un altro elemento il cui ordine divida $3$ (in questo caso uno dei due $3$-cicli),
    dunque abbiamo due scelte per $\varphi([1]_3)$; entrambe le scelte danno origine a due prodotti semidiretti isomorfi\footnote{Come nel caso dei gruppi di ordine $pq$.}.
    Osserviamo che abbiamo:
                \[ (\Z{2} \times \Z{2}) \rtimes_{\varphi} \Z{3} = G \varlonghookrightarrow S_4
                    \]
    infatti, $G$ agisce per coniugio sull'insieme $\{P_3,P_3^{\prime},P_3^{\prime\prime},P_3^{\prime\prime\prime}\}$ dei quattro $3$-Sylow di $G$, pertanto abbiamo l'azione transitiva
    $\phi : G \longrightarrow S(X) \cong S_4$, con $\ker \phi = \{id\}$ (dunque è un'azione fedele).
    Si verifica facilmente che l'unica possibilità è che $G$ sia isomorfo al gruppo alternante di $4$ elementi, dunque abbiamo ottenuto il gruppo:
        \begin{center}
            $\color{LimeGreen!90!black}{\mathcal{A}_4}$
        \end{center}
\end{example}
\nopagebreak 

\begin{example}[$G \cong P_3 \rtimes_{\varphi} P_2$]
    Se $P_3 \triangleleft G$, allora $G \cong P_3 \rtimes_{\varphi} P_2$. Analogamente a quanto visto prima $P_2$ ha ordine $4$, dunque è $\Z{4}$ o $\Z{2} \times \Z{2}$, e $P_3$ è $\Z{3}$.
    Il primo prodotto che abbiamo è:
        \[ \Z{3} \rtimes_{\varphi} \Z{4} \qquad \text{con} \qquad \varphi : \Z{4} \longrightarrow \Aut(\Z{3}) \cong \Z{2}
            \]
    dunque $[1]_4 \longmapsto id,-id$, nel primo caso riotteniamo il prodotto diretto e $\Z{12}$, nel secondo caso invece otteniamo un prodotto semidiretto che ci dà il gruppo:
        \[ \color{LimeGreen!90!black}{\Z{3} \rtimes_{\varphi} \Z{4}}
            \]
    L'ultimo prodotto possibile è:
        \[ \Z{3} \rtimes_{\varphi} (\Z{2} \times \Z{2}) \qquad \text{con} \qquad \varphi : \Z{2} \times \Z{2} \longrightarrow \Aut(\Z{3}) \cong \Z{2}
            \]
    se mandassimo tutti gli elementi nell'identità otterremmo un prodotto semidiretto, alternativamente, riscrivendo $\Z{2} \times \Z{2}$ come $\left<x\right> \times \left<y\right>$ (i cui elementi saranno $\{e,x,y,xy\}$), 
    abbiamo due elementi di ordine $2$ che vanno in $-id$ e l'elemento neutro e un altro elemento di ordine $2$ che vanno in $id$. Possiamo dunque costruire tre prodotti semidiretti che danno 
    origine a gruppi isomorfi, supponiamo (WLOG) che:
        \[ \varphi_x = id \qquad \varphi_y = -id \qquad \varphi_{xy} = -id
            \]
    dunque abbiamo:
        \[ \left<x\right> \times \left<y\right> \cong \Z{2} \times \Z{2} \qquad \text e \qquad \left<z\right> \cong \Z{3}
            \]
    possiamo osservare che:
        \[ \varphi_{x}(Z) = xzx^{-1} = id(z) = z \implies \text{$x$ commuta con $z$}
            \]
    similmente:
        \[ \varphi_{y}(z) = yzy^{-1} = -id(z) = -z
            \]
    dunque il sottogruppo generato da $y$ e $z$ è:
        \[ \left<y,z | y^2 = 1, z^3 = 1, yzy^{-1} = z^{-1}\right> \cong D_3
            \]
    quindi il gruppo che si ottiene con i tre prodotti semidiretti è $\Z{2} \times D_3$ (il prodotto diretto deriva
    dal fatto che $x$ commuta sia con $y$ che con $z$), ovvero:
        \begin{center}
            $\color{LimeGreen!90!black}{D_6}$
        \end{center}
\end{example}

Abbiamo quindi classificato tutti i gruppi di ordine 12:
    \[ \color{LimeGreen!90!black}{\Z{12} \qquad \Z{2} \times \Z{6} \qquad \mathcal{A}_4 \qquad \Z{3} \rtimes_{\varphi} \Z{4} \qquad D_6}
        \]

\newpage
\subsection{Gruppo dei Quaternioni}

\begin{definition}
    Si definisce gruppo dei \vocab{quaternioni} il gruppo con la seguente presentazione:
        \[ Q_8 = \left<i,j | i^4 = 1, i^2 = j^2, ij = j^3i\right>
            \]
\end{definition}

\begin{remark}[Ordini di $i$ e $j$]
    Osserviamo che $\ord(i) = 4$, per la definizione che ne abbiamo dato, da ciò si ricava che, essendo $j^2 = i^2$,
    allora $j^4 = 1 \implies \ord(j) \mid 4$, ciò unito al fatto che:
        \[ \ord(j^2) = \frac{\ord(j)}{(2,\ord(j))} = \ord(i^2) = 2
            \]
    implica che $\ord(j) = 4$. Dunque abbiamo due gruppi ciclici di ordine $4$, $\left<i\right>$ e $\left<j\right>$, con
    $\left<i\right> \cap \left<j\right> = \{1,i^2 = j^2\}$.
\end{remark}

Dalla presentazione del gruppo, sappiamo che $Q_8 = \left<i\right>\left<j\right>$ dunque possiamo stabilire l'ordine:
    \[ |Q_8| = |\left<i\right>\left<j\right>| = \frac{|\left<i\right>||\left<j\right>|}{|\left<i\right> \cap \left<j\right>|} = \frac{4 \cdot 4 }{2} = 8
        \]
quindi il gruppo dei quaternioni ha $8$ elementi, dati da:
    \[ Q_8 = \{1, i, j, i^2 = j^2, i^3, j^3, ij, i^3j\}
        \]

\begin{remark}
    $Q_8$ non è abeliano perché:
        \[ ij = j^3i = j^{-1}i \ne ji
            \]
\end{remark}

\begin{remark}
    Osserviamo che $\left<i\right>,\left<j\right> \triangleleft Q_8$ perché hanno indice $2$, inoltre $\left<i^2\right>, \left<j^2\right> \triangleleft Q_8$ (per verifica diretta). 
\end{remark}

Ricordando che un sottogruppo di ordine $2$ è normale se e solo se è un sottogruppo di $Z(G)$ \footnote{Infatti, preso $H = \{e,h\} \trianglelefteqslant G$, allora $gHg^{-1} = H$, $\forall g \in G$, ovvero $ghg^{-1} \in H \iff ghg^{-1} = h \iff
gh = hg$, $\forall g \in G$, dunque $h \in Z(G)$ (nel caso in cui $ghg^{-1} = e$, allora $h = e$, e ovviamente appartiene al centro), pertanto $H \leqslant Z(G)$.}, possiamo osservare che:

\begin{remark}
    $\left<i^2\right> = Z(Q_8)$, infatti, per quanto detto si deve avere che $\left<i^2\right> \leqslant  Z(Q_8)$, inoltre $Q_8$ è un $p$-gruppo non abeliano, ed
    essendo $|Q_8| = p^3$ segue che:
        \[ |Z(Q_8)| = \begin{cases}
                        1 & \text{assurdo per quanto detto sui \hyperref[pgruppi]{p-gruppi}} \\
                        p \\
                        p^2 & \text{ma allora $Q_8/Z(Q_8)$ ciclico $\implies Q_8$ abeliano} \\
                        p^3 & \text{$\implies Z(Q_8) = Q_8$, assurdo}
                    \end{cases}
        \]
      ovvero $|Z(Q_8)|=2$ e quindi è proprio $\left<i^2\right>$.
\end{remark}

Posto convenzionalmente $ij = k$, gli elementi si possono riscrivere anche come:
    \[ Q_8 = \left\{\pm 1, \pm i, \pm j, \pm k\right\}
        \]
con $i^2 = -1$, $i^3 = -i$, $j^3 = -j$, $i^3j = -k$.

\begin{remark}
    [Prodotto in $Q_8$]
    I prodotti tra gli elementi di $Q_8$ seguono il $3$-ciclo:
    \[  \begin{tikzpicture}[->,scale=.7] 
        \foreach \a/\t in {90/i,-30/j,210/k}{
          \node (\t) at (\a:1cm) {$\t$};
          \draw (\a-20:1cm)  arc (\a-20:\a-100:1cm);
        } 
        \end{tikzpicture}
    \]
    che percorso in senso orario ci dà i prodotti:
    \[ ij = k \qquad jk = i \qquad ki = j
        \]
    ed in senso antiorario:
    \[ ji = -k \qquad ik = -j \qquad kj = -i
        \]
    Le operazioni fatte in questo modo sono equivalenti a quelle che si ottengono con le regole di commutazione della presentazione, ad esempio:
        \[ k^2 = (ij)^2 = i j i j = ijj^3i = i^2 
            \]
\end{remark}

\begin{remark}
    [Ordine degli elementi]
    Dunque in $Q_8$ 1 ha ordine 1, $-1$ ha ordine 2, mentre $i$, $-i$, $j$, $-j$, $k$, $-k$ hanno ordine 4.   
\end{remark}

Abbiamo visto che $Q_8$ è un gruppo di ordine $8$ non è abeliano, e per quanto detto $Q_8 \ncong D_4$,
poiché ha $Q_8$ ha sei elementi di ordine 4, mentre $D_4$ ne ha soltanto uno. \\

\begin{remark}
    [Sottogruppi di $Q_8$]
    Per quanto riguarda i sottogruppi di $Q_8$ osserviamo in primis che $\left<-1\right> = Z(Q_8)$ ed è caratteristico (perché è il centro oppure perché è l'unico sottogruppo di ordine $2$);
    $\left<i\right>$, $\left<j\right>$, $\left<k\right>$ sono sottogruppi di ordine 4, dunque sono normali. Abbiamo quindi dimostrato che tutti i sottogruppi (includendo ovviamente quelli banali) di $Q_8$ sono normali.
\end{remark}

Concludiamo la discussione su $Q_8$ osservando che non può essere prodotto semidiretto di due suoi sottogruppi, infatti $\forall H_1,H_2 \leqslant Q_8$ si ha $H_1 \cap H_2 \ne \{1\}$, infatti, l'intersezione 
contiene sempre il sottogruppo $\{1,-1\}$.

\begin{exercise}
    Dimostrare che $Q_8 \varlonghookrightarrow GL_2(\CC)$.
\end{exercise}

\begin{soln}
\end{soln}

\newpage
A questo punto siamo pronti per classificare tutti i gruppi di ordine 8:

\begin{example}
    [Classificazione dei gruppi di ordine $8$]
    Distinguiamo innanzitutto i gruppi in base all'abelianità:
    \begin{itemize}
        \item Se $G$ è abeliano, allora per il nel \hyperref[t:struttura]{Teorema di Struttura} abbiamo che $G \cong G(2)$ e per la $2$-componente abbiamo le seguenti possibilità:
            \begin{center}
                $\color{LimeGreen!90!black}{\Z{8} \qquad \Z{4} \times \Z{2} \qquad \Z{2} \times \Z{2} \times \Z{2}}$
            \end{center}
        \item Se $G$ non è abeliano, allora ha almeno un elemento di ordine 4 (se avesse tutti elementi di ordine 2 sarebbe isomorfo a $(\Z{2})^3$), sia $a \in G$ tale che 
            $\ord(a) = 4$, allora $\left<a\right> \triangleleft G$ e:
                \[ \faktor{G}{\left<a\right>} = \{\left<a\right>, b\left<a\right>\} \qquad b \in G\setminus{\left<a\right>}
                    \]
            dove deve essere $b^2\left<a\right> = \left<a\right>$, infatti se fosse $b^2\left<a\right> = b\left<a\right> \implies b\left<a\right> = \left<a\right> \implies b \in \left<a\right>$, che è assurdo, dunque:
                \[ b^2\left<a\right> = \left<a\right> \implies b^2 \in \{e,a,a^2,a^3\}
                    \]
            ma non può essere che $b^2 = a,a^3$, altrimenti $b$ avrebbe ordine 8, dunque rimangono soltanto i casi $b^2 = 1$ e $b^2 = a^2$.
            \begin{enumerate}[(1)]
                \item Se $a^4 = 1$ e $b^2 = 1$, allora $G = \{1,a,a^2,a^3,b,ba,ba^2,ba^3\}$ da cui (si verificano facilmente le ipotesi del \hyperref[t:1.78]{Teorema 1.78}) segue:
                      \[ G \cong \left<a\right> \rtimes_{\varphi} \left<b\right> \cong \color{LimeGreen!90!black}{D_4}
                    \]
                    dove $\varphi : \left<b\right> \longmapsto \Aut(\left<a\right>) \cong \Z{2} : b \longmapsto \varphi_b$ e $\varphi_b : \left<a\right> \longmapsto \left<a\right> : a \longmapsto a^{-1}$ 
                    (ovvero $\varphi_b = -id$, se avessimo scelto l'identità avremmo ottenuto uno dei prodotti diretti già visti sopra).
                \item Se $a^4 = 1$ e $b^2 = a^2$, osserviamo che $bab^{-1} \in \left<a\right>$ (essendo il generato da $a$ normale in $G$), inoltre non può essere che $bab^{-1} = 1$ (altrimenti $a = 1$) o $bab^{-1} = a^2$ (poiché il coniugio conserva l'ordine degli elementi) e non 
                    piò nemmeno essere che $bab^{-1} = a$ (poiché abbiamo supposto che $G$ non sia commutativo). Pertanto abbiamo necessariamente $bab^{-1} = a^3 \iff ba = a^3b$, da cui segue:
                        \[ G \cong \color{LimeGreen!90!black}{Q_8}
                            \]
                    dove l'isomorfismo manda $a \longmapsto i$ e $b \longmapsto j$.
            \end{enumerate}
    \end{itemize}
\end{example}

Dunque i gruppi di ordine 8 sono:
    \[ \color{LimeGreen!90!black}{\Z{8} \qquad \Z{4} \times \Z{2} \qquad \Z{2} \times \Z{2} \times \Z{2} \qquad D_4 \qquad Q_8}
        \]

\newpage
\begin{exercise}
    Determinare il minimo $n$ tale che $Q_8 \varlonghookrightarrow S_n$.
\end{exercise}

\begin{soln}
    Osserviamo inizialmente che per il \hyperref[p:Cayley]{Teorema di Cayley} $n \leq 8$ e che per quello di Lagrange l'ordine dell'immagine di $Q_8$ deve dividere quello di $S_n$, pertanto $n \geq 4$, dunque abbiamo un numero finito di possibilità:
        \[ S_4,S_5,S_6,S_7,S_8
            \]
    Se $Q_8$ si immergesse in $S_4$, con $|S_4| = 2^3 \cdot 3$, sarebbe un suo $2$-Sylow; poiché $D_n$ si immerge sempre in $S_n$ \footnote{In tal caso infatti basta mandare $x \in D_4$ nella corrispondente permutazione dei vertici.}, sappiamo che $D_4 \varlonghookrightarrow S_4$, 
    ed in particolare $D_4$ è un $2$-Sylow di $S_4$, ma ciò significa che $Q_8$ non è in $S_4$, poiché non è un coniugato di $D_4$. \\
    Si ragiona in maniera analoga per $S_5$, infatti $|S_5| = 2^3 \cdot 3 \cdot 5$ e $D_4 \subset S_4 \subset S_5$, dunque i due $2$-Sylow di $S_4$ sono isomorfi a quelli di $S_5$, ed ancora una volta ciò significa che $Q_8$ non si immerge nel gruppo.\\
    Sia $|S_6| = 2^4 \cdot 3^2 \cdot 5$, detto $P_2$ un $2$-Sylow di $S_6$, osserviamo che se fosse $Q_8 \varlonghookrightarrow S_6$, dovremmo avere:
        \[ i \longmapsto \sigma \qquad j \longmapsto \rho \qquad k \longmapsto \sigma\rho = \eta
            \]
    con $\ord(\sigma) = \ord(\rho) = 4$ e $\sigma^2 = \rho^2 = \eta^2$, dove $\ord(\sigma^2) = \ord(\rho^2) = \ord(\eta^2) = 2$. Osserviamo che le permutazioni di ordine $4$ in $S_6$ possono essere soltanto $4$-cicli o 4-cicli uniti a 2-cicli, mentre
    le permutazioni di ordine 2 sono prodotto di trasposizioni (al più tre, essendo in $S_6$).

    \begin{remark}
        Osserviamo che una permutazione è un quadrato se e solo se i cicli di lunghezza pari compaiono a coppie. Infatti:
            \begin{itemize}
                \item Se $\eta$ è un $k$-ciclo, con $k$ dispari, $\eta$ è un quadrato di un $k$-ciclo, ovvero:
                        \[ \eta = \eta^{k+1} = \left(\eta^{\frac{k+1}{2}}\right)^2
                            \]
                Se $\eta$ è un $k$-ciclo, con $k$ pari, allora si verifica che:
                        \[ \cycle{a_1,\ldots,a_k}\cycle{b_1,\ldots,b_k} = \cycle{a_1,b_1,\ldots,a_k,b_k}^2
                            \]
                \item Se $x^2 = \cycle{\eta_1, \ldots, \eta_2}^2 = \eta_1^2 \ldots \eta_s^2$, allora otteniamo cicli di lunghezza dispari e coppie di cicli.
            \end{itemize}
        Ad esempio, in $S_6$, una coppia di 3-cicli può essere sia un quadrato di un ciclo di lunghezza pari, sia il quadrato di altri due 3-cicli:
            \[ \cycle{1,2,3}\cycle{4,5,6} = \cycle{1,4,2,5,3,6}^2 = \left(\cycle{1,2,3}\cycle{4,5,6}\right)^2
                \]
        mentre in $S_4$ una coppia di cicli di lunghezza pari può essere soltanto il quadrato di un 4-ciclo:
            \[ \cycle{1,2}\cycle{3,4} = (\cycle{1,4,2,3})^2 = (\cycle{1,3,2,4})^2
                \]
    \end{remark}
    Dunque il fatto che $\sigma^2 = \rho^2 = \eta^2$ hanno ordine $2$ (quindi sono fatte da sole trasposizioni) e che sono quadrati (quindi i cicli di lunghezza pari compaiono a coppie),
    ci dice che le trasposizioni sono prodotti di un numero pari di trasposizioni, pertanto l'unica possibilità è che:
        \[ \sigma^2 = \rho^2 = \eta^2 = \cycle{a,b}\cycle{c,d}
            \]
    Risolvendo $x^2 = \cycle{1,2}\cycle{3,4}$, otteniamo:
        \[ x_1 = \cycle{1,3,2,4} \qquad x_2 = \cycle{1,4,2,3} \qquad x_3 = \cycle{1,3,2,4}\cycle{5,6} \qquad x_4 = \cycle{1,4,2,3}\cycle{5,6}
            \]
    abbiamo quindi 4 soluzioni in $S_6$, mentre in $Q_8$ ne avevamo 6, pertanto nemmeno $S_6$ contiene una copia isomorfa di $Q_8$.\\
    $Q_8$ non si immerge nemmeno in $S_7$ perché i $2$-Sylow di $S_7$ sono isomorfi a quelli di $S_6$, e quindi siamo nello stesso caso di prima. \\
    Dunque per esclusione deve essere necessariamente che:
        \[ Q_8 \varlonghookrightarrow S_8 \implies n = 8
            \]
    Per \hyperref[p:Cayley]{Cayley} l'immersione è di $Q_8$ in $S(Q_8)$, dunque la mappa che realizza ciò è data da:
        \[ i \longmapsto \varphi_i \qquad \text{con} \qquad \varphi_i : Q_8 \longrightarrow Q_8 : x \longmapsto ix
            \]
    in particolare con la notazione dei cicli abbiamo che l'immagine di $\varphi_i$ di $Q_8$ è data da:
        \[ \cycle{1,i,-1,i}\cycle{j,k,-j,-k}
            \]
    analogamente per $\varphi_j(Q_8)$:
        \[ \cycle{1,j,-1,-j}\cycle{i,-k,-i,k}
            \]
    e numerando in qualsiasi ordine gli elementi di $Q_8$ possiamo scrivere le permutazioni corrispondenti in $S_8$:
        \[ i \longmapsto \cycle{1,2,3,4}\cycle{5,6,7,8} \qquad j \longmapsto \cycle{1,5,2,6}\cycle{3,8,4,7}
            \]
\end{soln}

\begin{example}
    [Classificazione dei gruppi di ordine 30]
    Osserviamo che $|G| = 2 \cdot 3 \cdot 5$ e distinguiamo due casi:
    \begin{itemize}
        \item Se $G$ è abeliano, allora per il \hyperref[t:struttura]{Teorema di Struttura} $G \cong G(2) \times G(3) \times G(5)$, dunque l'unica possibilità è che il gruppo sia ciclico:
            \begin{center}
                $G \cong \color{LimeGreen!90!black}{\Z{2} \times \Z{3} \times \Z{5} \cong \Z{30}}$
            \end{center}
        \item Se $G$ non è abeliano, osserviamo che (in generale) $30 = 2d$, con $d$ dispari, dunque $G$ ha un sottogruppo di ordine 15, che è normale in quanto ha indice 2 ed è ciclico, in quanto è un gruppo di ordine $pq$
            con $p \nmid q-1$, pertanto $G$ contiene una copia isomorfa di $\Z{15}$. Per \hyperref[p:Cauchy]{Cauchy} esiste un elemento di ordine $2$ e quindi anche una copia isomorfa a $\Z{2}$ in $G$ (in particolare potevamo prendere
            direttamente il $2$-Sylow), dunque i due gruppi verificano le ipotesi del \hyperref[t:1.78]{Teorema 1.78}, da cui: \footnote{La direzione del prodotto semidiretto è data dal fatto che $\Z{15}$ è l'unico normale tra i due sottogruppi.}
                \[ G \cong \Z{15} \rtimes_{\varphi} \Z{2}
                    \]
            con:
                \[ \varphi : \Z{2} \longrightarrow \Aut(\Z{15}) \cong \Z{15}^* \cong \Z{3}^* \times \Z{5}^* \cong \Z{2} \times \Z{4}
                    \]
            dove abbiamo che $[1]_2 \longmapsto \varphi_y$, e adottando la notazione moltiplicativa, $\varphi_y : \Z{15} \longmapsto \Z{15} : \overline{x} \longmapsto \overline{x}^l$, abbiamo 
            $\ord(\varphi_y) \mid 2$, dunque ci sono due possibilità, o $\varphi_y = id$ (quindi $l = 1$), o $\varphi_y^2 = id \implies \varphi_y^2(x) = (x^l)^l = x^{l^2} = x$, da cui segue (essendo $x$ un generatore di $\Z{15}$):
                \[ l^2 \equiv 1 \pmod{15} \implies x\equiv \pm 1, \pm 4 \pmod{15}
                    \]
            Dunque, per $l = 1$ otteniamo il prodotto diretto già trovato sopra, per gli altri tre possibili $l$ invece otteniamo 3 gruppi non isomorfi di ordine $30$, infatti, per $l = -1$, abbiamo:
                \[ \varphi_y(x) = x^{-1} \iff yxy^{-1} = x^{-1} \implies \Z{15} \rtimes_{\varphi} \Z{2} \cong  \color{LimeGreen!90!black}{D_{15}}
                    \]
            Per $l = 4$ invece si ottiene $\color{LimeGreen!90!black}{D_5 \times \Z{3}}$ e per $l = -4$ si ottiene $\color{LimeGreen!90!black}{D_3 \times \Z{5}}$ \footnote{Andrebbe aggiunto il perché ma non è chiarissimo dalle note della Del Corso.},
            i quali sono gruppi non isomorfi, ad esempio perché hanno centri diversi:
                \[ Z(D_{15}) = \left<id\right> \qquad Z(D_5 \times \Z{3}) = Z(D_5) \times Z(\Z{3}) \cong \Z{3} \]
                \[ Z(D_3 \times \Z{5}) \cong \Z{5}
                    \]
    \end{itemize}
\end{example}
I gruppi di ordine 30 sono quindi:
    \begin{center}
        $\color{LimeGreen!90!black}{\Z{30} \qquad D_{15} \qquad D_5 \times \Z{3} \qquad D_3 \times \Z{5}}$
    \end{center}
    
\newpage


\section{Anelli}
\subsection{Riepilogo sugli anelli}
\begin{definition}
    Un \vocab{anello} è un insieme non vuoto munito di due operazioni $(A,+,\cdot)$ tali che:
    \begin{itemize}
        \item $(A,+)$ è un gruppo abeliano.
        \item $\cdot$ è associativa.
        \item Valgono le leggi distributive a destra e sinistra:
            \[ a(b+c) = ab + ac \qquad \text e \qquad (a+b)c = ac + bc \qquad \qquad \forall a,b,c \in A
                \]
    \end{itemize}
\end{definition}

\begin{example}[Anelli]
    Esempi di anelli:
    \begin{itemize}
        \item $\ZZ$, $\QQ$, $\RR$, $\CC$, $\Zn$.
        \item Dato un anello $A$, $A[x]$, l'insieme dei polinomi a coefficienti in $A$ è un anello.
        \item $M_{m \times m}(K)$.
        \item $\End(G) = \Hom(G,G)$, con $G$ gruppo abeliano e le operazioni di somma e composizione. 
    \end{itemize}
\end{example}

Riepiloghiamo brevemente \footnote{In caso di dubbi sulle definizioni si può fare riferimento alle \href{https://github.com/diego-unipi/Appunti-Aritmetica}{\textcolor{purple}{dispense di Aritmetica}}
dove sono state trattate più ampiamente.} le definizioni che riguardano gli anelli: \footnote{Per convenzione adotteremo lo $0$ per indicare l'elemento neutro rispetto all'operazione $+$ e l'$1$ per indicare l'elemento neutro rispetto all'operazione $\cdot$.}

\begin{definition}
    Un anello $A$ si dice \vocab{commutativo} se l'operazione $\cdot$ è commutativa.
\end{definition}

\begin{definition}
    Un anello $A$ si dice \vocab{con identità} se esiste $1 \in A$ elemento neutro per il prodotto.
\end{definition}

\begin{definition}
    Un anello $(A,+,\cdot)$ si dice \vocab{campo} se $(A\setminus\{0\},\cdot)$ è un gruppo abeliano.
\end{definition}

\begin{definition}
    Un anello $(A,+,\cdot)$ si dice \vocab{corpo} se $(A\setminus\{0\},\cdot)$ è un gruppo.
\end{definition}

\begin{definition}
    Dato un anello $A$, $x \in A$ si dice \vocab{divisore di zero} se $\exists y \in A$, $y \ne 0$ tale che $xy=yx=0$.
\end{definition}

\begin{definition}
    Dato un anello $A$, $x \in A$ si dice \vocab{nilpotente} se $\exists n \in \NN$ tale che $x^n = 0$.
\end{definition}

\begin{definition}
    Dato un anello $A$, $x \in A$ si dice \vocab{invertibile} se $\exists y \in A$ tale che $xy=yx=1$.
\end{definition}

\begin{definition}
    Un anello $A$ si dice \vocab{dominio d'integrità} se:
        \[ D(A) = \{x \in A |\, \text{$x$ è un divisore di $0$}\} = \{0\}
            \]
\end{definition}

Definiamo inoltre l'insieme degli elementi invertibili di $A$:
    \[ A^* = \{x \in A |\, \text{$x$ è invertibile}\}
        \]
e dei nilpotenti:
    \[ \mathcal{N} = \{x \in A |\, \text{$x$ è nilpotente}\} 
        \]

\begin{exercise}
    Calcolare i divisori di zero, gli invertibili ed i nilpotenti di:
        \[ \ZZ, \Zn, \Zn \times \Zn
            \]
\end{exercise}

\begin{soln}
    
\end{soln}

\begin{proposition}
    Dato A un anello commutativo con identità:
    \begin{enumerate}[(1)]
        \item $(A^*,\cdot)$ è un gruppo abeliano.
        \item $A^* \cap D(A) = \emptyset$.
        \item Se $A$ è un anello finito, allora $A = D(A) \cup A^*$. In particolare, un dominio
            d'integrità finito è un campo.
    \end{enumerate}
\end{proposition}

\begin{proof}
    Proviamo le affermazioni:
    \begin{enumerate}[(1)]
        \item Per provare che $(A^*, \cdot)$ è un gruppo abeliano, è sufficiente verificare le proprietà richieste dalla definizione:
        \begin{enumerate}[(a)]
            \item Chiusura: osserviamo che $\forall x,y \in A^*$, allora $\exists x^{-1}y^{-1} \in A$, pertanto $xy \in A^*$, poiché $y^{-1}x^{-1} \in A^*$. 
            \item Associatività: poiché $A \subseteq A^*$, allora, essendo $A$ associativo rispetto al $\cdot$, allora anche gli elementi di un suo qualsiasi sottoinsieme saranno associativi tra loro. 
            \item Elemento Neutro: $1 \in A$, infatti, l'inverso di $1$ è se stesso, quindi $1$ è invertibile.
            \item Inverso: Segue per la stessa definizione di $A^*$ che ogni suo elemento debba avere inverso moltiplicativo nel gruppo, $\forall x \in A, \exists x^{-1} \in A$:
                \[ x \cdot x^{-1} = x^{-1} \cdot x = 1
                \qquad
                \forall x \in A
                \]
            \item L'abelianità segue immediatamente dall'abelianità di $A$ (infatti $A^* \subset A$, dunque l'abelianità vale in particolare per gli elementi di $A^*$).
        \end{enumerate}
        \item Supponiamo per assurdo che $D(A) \cap A^* \neq \emptyset$, e consideriamo $x \in D(A) \cap A^*$, poiché $x \in D(A)$, allora $\exists z \in A, z \ne 0$, tale per cui:
		    \[ xz = zx = 0
		        \]
            d'altra parte, poiché $x \in A^*$, allora $\exists x^{-1}$ tal per cui:
            \[  xy = yx = 1
                    \]
            da cui segue:
            \[ (zx)y = z(xy) \implies 0 \cdot y = z \implies z = 0
                    \]
            ma ciò è assurdo, pertanto l'ipotesi $D(A) \cap A^*$ è vuoto.
        \item Il contenimento $D(A) \cup A^* \subseteq A$ è ovvio in quanto i primi due sono sottoinsiemi del primo, ci resta da verificare quello opposto. Sia $x \in A$, se $x \in D(A)$ abbiamo concluso,
            se $x \in A \setminus D(A)$, allora possiamo definire l'omomorfismo di gruppi:
                \[ \varphi_x : A \longrightarrow A : a \longmapsto xa
                    \]
            con:
                \[ \ker \varphi_{x} = \{y \in A | \varphi_x(y) = xy = 0\} = \{0\}
                    \]
            infatti, non essendo $x$ un divisore di zero, l'unica possibilità, in base all'annullamento del prodotto è che $y = 0 \implies xy = 0$. Poiché $|A| < +\infty$ l'omomorfismo è anche surgettivo, dunque è una bigezione,
            pertanto $1 \in \text{Im} \varphi_x \implies \exists a \in A$ tale che $\varphi_x(a) = xa = 1 \implies x \in A^*$.
    \end{enumerate}
\end{proof}

\begin{definition}
    Dato $B \subset A$ non vuoto, si dice che $B$ è un \vocab{sottoanello} di $A$ se è chiuso rispetto alle operazioni $+$ e $\cdot$ ristrette a $B$.
\end{definition}

\begin{definition}
    Dato $I \subset A$, con $A$ anello commutativo, si dice che $A$ è un \vocab{ideale} di $A$ se:
    \begin{itemize}
        \item $(I,+) < (A,+)$.
        \item Vale la \vocab{proprietà di assorbimento} a destra e sinistra:\footnote{Se $A$ non è un anello commutativo e $(I,+) < (A,+)$, possono
        valere separatamente le proprietà di assorbimento, nel caso in cui valga $aI \subset I$, $\forall a \in A$, si parla di \vocab{ideale sinistro}, mentre nel caso 
        $Ia \subset I$, $\forall a \in A$, si parla invece di \vocab{ideale destro}.}
            \[ aI \subset I \qquad \text e \qquad Ia \subset I \qquad \forall a \in A
                \]
    \end{itemize}
\end{definition}

\begin{remark}
    Per verificare che un sottoinsieme di un anello commutativo con identità è un ideale ci basta verificare soltanto che $(I,+)$ è chiuso per l'operazione $+$ e che valga la proprietà di assorbimento, infatti, da ciò segue 
    che $(-1)a \in I$, dove $(-1)$ esiste in $A$ è un gruppo rispetto al $+$.
\end{remark}

Da questo momento in poi, anche se non specificato, assumiamo di star operando sempre in anelli commutativi con identità.

\begin{example}[Ideali]
    Gli ideali vanno ricercati tra i sottogruppi di un anello, ad esempio:
    \begin{itemize}
        \item Considerando l'anello $\ZZ$, abbiamo gli ideali dati da $\{n\ZZ\}_{n \in \NN}$, infatti:
            \[ xn\ZZ \subset n\ZZ \qquad \forall x \in \ZZ
                \]
        ed essendo $\ZZ$ abeliano, abbiamo un ideale.
        \item I sottogruppi $\{0\}$ (ideale \vocab{banale}) e $A$ (ideale \vocab{improprio}) sono ideali dell'anello $A$.
    \end{itemize}
\end{example}

\begin{exercise}
    Dato l'anello delle matrici $A = M_{n \times n}(K)$ dimostrare che non ha ideali bilateri diversi da $\{0\}$ e $A$.
\end{exercise}

\begin{soln}
Sia $J\subseteq A$ un ideale non banale e $M\in J$ non nulla. Osserviamo che l'Algoritmo di Gauss applicato ad $M$ può essere espresso come moltiplicazione di $M$
a sinistra e a destra per opportuni elementi di $A$. Dunque, se $k=\mathrm{rnk}\, M>0$, la matrice 
\[ N=\left(\begin{array}{c|c}
    I_k & 0\\
    \hline
    0 & 0
\end{array}\right)
    \]
appartiene a $J$. Allora anche
\[
    N'=N\left(\begin{array}{c|c}
        1 & 0 \\
        \hline
        0 & 0
    \end{array}\right)
=\left(\begin{array}{c|c}
    1 & 0 \\
    \hline
    0 & 0
\end{array}\right)\]
è elemento di $J$. Coniugando $N'$ per matrici di permutazione di base troviamo nell'ideale matrici diagonali con diagonale nulla tranne per un 1
nella $i$-esima riga per ogni $i$, e sommando tutte queste matrici otteniamo $I_n\in J$, ovvero $J=A$.\footnote{Dimostrazione proposta da Davide Ranieri.}
\end{soln}

\begin{definition}
    Dato un sottoinsieme non vuoto di un anello $S \subset A$, si definisce \vocab{ideale generato} da $S$ in $A$:
        \[ (S) \vcentcolon =  \left\{\sum_{i = 1}^{n}a_is_i \m a_i \in A, s_i \in S, n \in \NN\right\}
            \]
\end{definition}

Osserviamo che se $S = \{x\}$ possiamo definire l'ideale generato da un elemento:
        \[ (x) = \{ax | a \in A\} = Ax
            \]
in tal caso l'ideale prende anche il nome di \vocab{ideale principale}.
        
\begin{proposition}
    L'ideale generato da un sottoinsieme $S$ di un anello $A$ è un ideale.
\end{proposition}

\begin{proof}
    Per verificare che un ideale generato sia effettivamente un ideale, bisogna verificare che sia un sottogruppo del gruppo abeliano $(A,+)$ e che valga la proprietà di assorbimento (bilaterale in questo caso,
    poiché stiamo operando nel caso di anelli commutativi). Presi $x,y \in S$, ovvero della forma:
        \[ x = \sum_{i = 1}^n a_is_i \qquad \text e \qquad y = \sum_{j=1}^m\alpha_j \sigma_j \qquad a_i,\alpha_j \in A \qquad s_i,\sigma_j \in S 
            \]
    si osserva che:
        \[ x + y = \sum_{i = 1}^n a_is_i + \sum_{j=1}^m\alpha_j \sigma_j \in (S)
            \]
    dunque $I$ è chiuso per la somma. Infine, per ogni $a \in A$ si ha:
        \[ ax = a\sum_{i = 1}^n a_is_i = \sum_{i = 1}^n \underbrace{aa_i}_{\in A}s_i \in (S)
            \]
    dunque $(S)$ è un ideale.
\end{proof}

\begin{example}[Ideali generati]
    Alcuni esempi di ideali generati possono essere:
    \begin{itemize}
        \item $n\ZZ = (n)$, con $n \in \ZZ$.
        \item Dato $K \subset F$ e $\alpha \in F$ algebrico su $K$, sia $\mu_{\alpha}(x) \in K[x]$ il polinomio minimo di $\alpha$, sappiamo che:
            \[ (\mu_{\alpha}(x)) = \{p(x) \in K[x] | p(\alpha) = 0\}
                \]
    \end{itemize}
\end{example}


\newpage
\subsection{Operazioni tra ideali}
\begin{proposition}[Operazioni tra ideali]
    Dato $A$ un anello commutativo e $I,J \subset A$ ideali, abbiamo che:
    \begin{itemize}
        \item $I \cap J$ è un ideale.
        \item $I + J = (I,J) = \{i + j | i \in I, j \in J\}$ è un ideale.
        \item $IJ = (\{xy|x \in I, y \in J\})$ è un ideale.
        \item $\sqrt{I} = \{x \in A | \exists n \in \NN : x^n \in I\}$ è un ideale. In particolare $\sqrt{0} = \mathcal{N}$ è un ideale.
        \item $(I : J) = \{x \in A | xJ \subseteq I\}$ è un ideale.
    \end{itemize}
\end{proposition}

\begin{proof}
    Verifichiamo tutte le affermazioni:
    \begin{itemize}
        \item $I \cap J$ è un sottogruppo di $A$ e, $\forall x \in I \cap J$ si ha:
            \[ ax \in I \qquad \text e \qquad ax \in J \qquad \forall a \in A
                \]
            dunque $I \cap J$ assorbe e quindi è un ideale (bilatero in quanto abbiamo supposto l'anello commutativo).
        \item Dato $I + J = \{i + j | i \in I, j \in J\}$, presi $x,y \in I + J$, ovvero:
            \[ x = i_1 + j_1 \qquad \text e \qquad y = i_2 + j_2 \implies x + y = (\underbrace{i_1 + i_2}_{\in I}) + (\underbrace{j_1 + j_2}_{\in J}) \in I + J
                \]
            inoltre, $\forall a \in A$ si ha che:
            \[ ax = \underbrace{ai_1}_{\in I} + \underbrace{aj_1}_{\in J} \in I + J \qquad \forall x \in I + J
                \]
            dunque $I+J$ è un ideale. Verifichiamo che $I+J = (I,J)$; osserviamo che ovviamente:
            \[ \forall i+j \in I+J, i+j \in (I,J) \implies I + J \subseteq (I,J)
                \]
            per verificare l'altro contenimento bisogna verificare che $I,J \subset I + J$ e da queste inclusioni e dal fatto che $I + J$ è un ideale segue che
            $I + J$ contiene il più piccolo ideale di $A$ che contiene sia $I$ che $J$ (e quindi contiene il loro generato). Osserviamo innanzitutto che in generale:
                \[ (S) = \bigcap_{\substack{S \subseteq X \subseteq A\\ X\,\text{ideale}}} X
                    \]
            dove l'intersezione è appunto il più piccolo ideale di $A$ che contiene $S$.
            Dobbiamo dimostrare ora quanto detto; osserviamo che $(S)$ è contenuto nell'intersezione in quanto
            è uno dei termini di quest'ultima; il contenimento opposto segue dal fatto che $\forall x \in S$ si ha $x = \sum a_is_i \in X$ (poiché $X$ è un ideale che contiene $S$, per come l'abbiamo definito),
            d'altra parte, per vedere che un ideale generato $(S)$ è contenuto a sua volta in un ideale $\mathcal{I}$ di $A$, basta vedere che $S \subset \mathcal{I}$ (ed è ciò che abbiamo appena fatto con $S$).
            A questo punto, tornando all'inclusione iniziale, ci basta verificare che, come abbiamo anticipato, $I+J$ contenga sia $I$ che $J$; essendo $0 \in J$ abbiamo:
                \[ I \subset I + J
                    \]
            infatti basta considerare sempre $j = 0$ per ottenere tutti gli elementi di $I$; in maniera simmetrica si dimostra la stessa cosa per $J$, dunque $I,J \subset I + J \implies (I,J) \subset I + J \implies I + J = (I,J)$.
        \item $IJ = (\{xy | x \in I, y \in J\})$ è un ideale per definizione.
        \item Verifichiamo che $\sqrt{I} = \{x \in A | x^n \in I, n \in \NN\}$ è un ideale, presi $x,y \in \sqrt{I}$, ovvero $x^n,y^m \in I$, $n,m \in \NN$, vogliamo 
            provare che $x + y \in \sqrt{I}$ (ovvero che esiste $d \in \NN$ tale che $(x+y)^d \in I$), osserviamo che:
                \[ (x + y)^{n + m} = \sum_{i = 0}^{n + m} \binom{n+m}{i} x^i y^{m+n-i}
                    \]
            dove $\forall i \in \{0,\ldots,m+n\}$ si ha che $i \geq n \implies x^i \in I$, oppure che $n+m-i \geq m \implies y^{n+m-i} \in I$, dunque tutti i termini di $(x+y)^{n+m}$ stanno in $I$
            e quindi $x + y \in \sqrt{I}$. Osserviamo che $\forall a \in A$ si che:
                \[ (ax)^n = a^n\underbrace{x^n}_{\in I} \in I \implies ax \in \sqrt{I}
                    \]
        \item Dato $(I : J) = \{x \in A | xJ \subseteq I\}$ e presi $x,y \in (I : J)$ si ha che:
            \[ (x+y)J = \underbrace{xJ}_{\subseteq I} + \underbrace{yJ}_{\subseteq I} \implies x+y \in (I:J)
                \]
        inoltre, $\forall a \in A$ abbiamo:
            \[ axJ = a(xJ) \subseteq aI \subseteq I \implies ax \in (I:J) \qquad \forall x \in (I:J)
                \]
    \end{itemize}
\end{proof}

\begin{remark}[$I \cup J$]
    $I \cup J$ in generale non è un ideale.
\end{remark}

\begin{remark}[$IJ \subset I \cap J$]
    Osserviamo che $IJ \subset I \cap J$, infatti presi $x \in I$ e $y \in J$ si ha dalla proprietà di assorbimento che:
        \[ \underbrace{x}_{\in I}\underbrace{y}_{\in A} \in I \qquad \text e \qquad \underbrace{x}_{\in A}\underbrace{y}_{\in J} \in  J \implies xy \in I \cap J
            \]
\end{remark}

\begin{remark}[$IJ = I \cap J$]
    \label{2.24}
    Se $I + J = A$, allora $IJ = I \cap J$. Dall'ipotesi possiamo dedurre che:
        \[ i+j = 1
            \]
    vogliamo verificare che $\forall x \in I \cap J$ si ha $x \in IJ$ (dall'osservazione precedente 
    sappiamo già che $IJ \subset I \cap J$, quindi stiamo verificando il contenimento opposto), possiamo scrivere:
        \[ x \cdot 1 = x(i+j) = \underbrace{xi}_{\in IJ} + \underbrace{xj}_{\in IJ} \in IJ
            \]
    dove l'appartenenza segue dal fatto che stiamo considerando la somma di due elementi in $IJ$ (che è un gruppo additivo).
\end{remark}

\begin{example}
    [Operazioni tra ideali in $\ZZ$]
    Osserviamo che presi ad esempio gli elementi nell'intersezione degli ideali $m\ZZ$ e $n\ZZ$, questi sono i multipli comuni sia ad $m$ che ad $n$ in $\ZZ$, ovvero:
        \[ m\ZZ \cap n\ZZ = [m,n]\ZZ
            \]
    Mentre, il prodotto tra i due ideali contiene gli interi multipli si di $m$ che di $n$:
        \[ m\ZZ \cdot n\ZZ = mn\ZZ
            \]
    Osserviamo poi che la somma è data da tutti gli interi multipli del loro M.C.D.:
        \[ m\ZZ + n\ZZ = (m,n)\ZZ
            \]
    infatti, per l'identità di Bézout, si ha:
        \[ m\ZZ + n\ZZ = \{am + bn | a,b \in \ZZ\} = \{dx | x \in \ZZ\}
            \]
    Consideriamo ora $n = p_1^{e_1} \ldots p_r^{e_r}$, possiamo considerare:
        \[ \sqrt{n\ZZ} = p_1 \ldots p_r\ZZ
            \]
    poiché:
        \[ \sqrt{n\ZZ} = \{x \in \ZZ | x^k \in n\ZZ, k \in \ZZ\} = \{x \in \ZZ | n \mid x^k, k \in \ZZ\}
            \]
    ma $n \mid x^k \implies p_i \mid x^k$, $\forall i \in \{1,\ldots,r\}$, ovvero $p_1 \ldots p_r \mid x \implies x \in p_1\ldots p_r\ZZ$.
    Viceversa $x = p_1 \ldots p_r m \in \sqrt{n\ZZ}$ perché, detto $e = \max e_i$:
        \[ x^e = p_1^e \ldots p_r^e m^e = ny \in n\ZZ
            \]
    quindi ad esempio:
        \[ \sqrt{100\ZZ} = 10\ZZ
            \]
    Infine, osserviamo che:
        \[ (m\ZZ : n\ZZ) = \frac{m}{(m,n)}\ZZ
            \]
    quindi ad esempio:
        \[ (75\ZZ : 18\ZZ) = \frac{75}{(75,18)} = 25\ZZ
            \]
    questo poiché:
        \[ (75\ZZ : 18\ZZ) = \{x \in \ZZ | 18x \ZZ \subset 75\ZZ\} = 25\ZZ
            \]
    infatti $18x \ZZ \subset 75\ZZ \iff 75 \mid 18x \iff 25 \mid 6x \iff 25 \mid x$.
\end{example}

\newpage
\begin{proposition}
[Proprietà ideali propri]
\label{2.26}
Valgono i seguenti fatti:
\begin{enumerate}[(1)]
    \item Dato $I \subset A$ ideale, $I$ è un \vocab{ideale proprio} $(I \subsetneq A)$ se e solo se $I \cap A^* = \emptyset$.
    \item $A$ è un campo se e solo se gli unici ideali di $A$ sono $\{0\}$ e $A$.
\end{enumerate}
\end{proposition}

\begin{proof}
    Dimostriamo singolarmente i fatti:
    \begin{enumerate}[(1)]
        \item Se $I \cap A^* = \emptyset$, poiché vale sempre che $1 \in A^*$, allora c'è almeno un elemento di $A$ che non sta nell'ideale, quindi $I \subsetneq A$.
            Viceversa, sia $I$ ideale proprio e supponiamo $x \in I \cap A^*$, allora $x$ è invertibile, dunque $\exists y \in A$ tale che $xy = 1$, ma:
                \[ 1 = \underbrace{x}_{\in I}\underbrace{y}_{\in A} \in I \implies a \cdot 1 \in I\footnote{In pratica se c'è l'identità in $I$ c'è esattamente ogni elemento dell'anello che contiene l'ideale.} \qquad \forall a \in A \implies A \subset I
                    \]
            che è assurdo in quanto avevamo supposto $I \subset A$, dunque $I \cap A^* = \emptyset$.
        \item $A$ è un campo se e solo se $A^* = A \setminus \{0\}$, ma per il punto (1) l'unico elemento fuori da $A^*$ è 0, dunque $I = \{0\}$ e $I = A$ sono gli unici ideali che possiamo avere.
    \end{enumerate}
\end{proof}

\newpage
\subsection{Anelli quoziente e omomorfismi di anelli}

\begin{definition}
    Dati $A$ e $B$ anelli, $f : A \longrightarrow B$ è un \vocab{omomorfismo di anelli} se:
    \begin{itemize}
        \item $f(a_1 + a_2) = f(a_1) + f(a_2)$, $\forall a_1,a_2 \in A$.
        \item $f(a_1a_2) = f(a_1)f(a_2)$, $\forall a_1,a_2 \in A$.
    \end{itemize}
\end{definition}

\begin{remark}
    Se $A$ e $B$ sono anelli commutativi con identità in genere si richiede anche:
        \[ f(1_A) = 1_B
            \]
    poiché tale condizione non è già implicata da altro; ad esempio:
        \[ f(a) = f(1_Aa) = f(1_A)f(a) \implies f(a) - f(1_A)f(a) = (1_B - f(1_A))f(a) = 0
            \]
    ma non abbiamo la legge di cancellazione in quanto non è detto che $A$ sia un dominio d'integrità.
    Se $B$ è un dominio e $f(a) \ne 0$, allora, da quanto detto sopra segue $f(1_A) = 1_B$, ma se $f(A) \subset D(B)$ non è detto che 
    $f(1_A) = 1_B$.
\end{remark}

\begin{definition}
    Sia $A$ un anello e $I \subseteq A$ un suo ideale, il gruppo quoziente $\left(\faktor{A}{I}, +\right)$ ha anche una struttura di anello con l'operazione:
        \[ (a + I) \cdot (b + I) \stackrel{\text{def}}{=} ab + I
            \]
\end{definition}

\begin{remark}
    Si verifica facilmente che l'operazione è ben definita, infatti, presi:
        \[ a + I = a^{\prime} + I \qquad \text e \qquad b + I = b^{\prime} + I
            \]
    segue:
        \[ (a^{\prime} + I) \cdot (b^{\prime} + I) = a^{\prime}b^{\prime} + I = (a + I)(b + I) + I = ab + I
            \]
\end{remark}

\begin{remark}
    Si verifica facilmente che $\left(\faktor{A}{I},+,\cdot\right)$ è un anello.
\end{remark}

\begin{remark}
    Possiamo definire una proiezione all'anello quoziente:
        \[ \pi_I : A \longrightarrow \faktor{A}{I} : a \longmapsto a + I
            \]
    con $\pi_I$ omomorfismo di anelli surgettivo e $\ker \pi_I = I$.
\end{remark}

\begin{proposition}
    Gli ideali sono tutti e soli i nuclei degli omomorfismi di anello definiti su $A$.
\end{proposition}

\begin{proof}
    Sia $\varphi : A \longrightarrow B$ un omomorfismo di anelli, allora $\ker \varphi$ è un ideale di $A$, infatti $\ker \varphi < A$ perché $\varphi$ 
    è in particolare un omomorfismo di gruppi, inoltre, $\forall a \in A$ si ha che:
        \[ ax \in \ker \varphi \qquad \forall x \in \ker \varphi
            \]
    in quanto $\varphi(ax) = \varphi(a)\varphi(x) = \varphi(a) \cdot 0 = 0$; dunque i nuclei degli omomorfismi di anelli sono ideali. Viceversa, tutti gli ideali 
    sono nuclei degli omomorfismi di proiezione al quoziente $\pi_I$.
\end{proof}

\begin{theorem}
    [Teorema di Omomorfismo di Anelli]
    \label{omo}
    Dati $A,B$ anelli e $f : A \longrightarrow B$ omomorfismo (di anelli), esiste un unico omomorfismo (di anelli) $\varphi$ che fa commutare il diagramma:
    \begin{center}
	\setbox0=\hbox{
	\begin{tikzcd}
		A \arrow[d, "\pi" left, twoheadrightarrow] \arrow[r, "f"] & B \\	
		\faktor{A}{\ker f} \arrow[ur,dashed, "\varphi"] & 
	\end{tikzcd}}
	\savebox{\tmp}{\box0}
	{%
	\stackinset{c}{-0.17cm}{c}{0.26cm}{\Large$\circlearrowleft$}{%
	\usebox{\tmp}}}
	\end{center}
    cioè tale che $f = \varphi \circ \pi$, con $\varphi$ iniettivo e $\text{Im}\varphi = \text{Im}f$.
\end{theorem}

\begin{proof}
    Per il Teorema di Omomorfismo i gruppi, essendo $f$ in particolare un omomorfismo di gruppi, posto $I = \ker f$, sappiamo che
    esiste ed è unico l'omomorfismo:
        \[ \varphi : \faktor{A}{I} \longrightarrow B
            \]
    tale che $f = \varphi \circ \pi_I$, con $\varphi$ è iniettivo e $\text{Im} \varphi = \text{Im} f$. Non ci resta altro da fare che verificare che $\varphi$ è 
    anche un omomorfismo di anelli:
        \[ \varphi((a+I)(b+I)) = \varphi(ab+I) = f(ab) \qquad \forall a,b \in A
            \]
    viceversa:
        \[ f(ab) = f(a)f(b) = \varphi(a+I)\varphi(b+I) \qquad \forall a,b \in I
            \]
    dove la seconda uguaglianza è vera per ipotesi.
\end{proof}

\begin{lemma}
    [Gli ideali si comportano come i sottogruppi normali con gli omomorfismi]
    \label{2.35}
    Dato $f : A \longrightarrow B$ omomorfismo di anelli vale che:
    \begin{enumerate}[(1)]
        \item $\forall J \subset B$ ideale si ha $f^{-1}(J)$ è un ideale di $A$.
        \item Se $f$ è surgettiva $\forall I \subset A$ ideale si ha $f(I)$ ideale di $B$
    \end{enumerate}
\end{lemma}

\begin{proof}
    Dimostriamo le proposizioni:
    \begin{enumerate}[(1)]
        \item Sappiamo già che $f^{-1}(J)$ è un sottogruppo di $A$, verifichiamo che valga la proprietà di assorbimento, ovvero:
            \[ af^{-1}(J) \subset f^{-1}(J) \qquad \forall a \in A
                \]
            sia $x \in f^{-1}(J) \implies f(x) \in J$, allora:
            \[ \underbrace{f(a)}_{\in B}\underbrace{f(x)}_{\in J} = f(ax) \in J \qquad \forall x \in f^{-1}(J)
                \]
            da cui $ax \in f^{-1}(J)$.
        \item Sappiamo che $f(I)$ è un sottogruppo di $B$, verifichiamo l'assorbimento, sia $b \in B$, poiché $f$ è surgettiva esiste $a \in A$ tale che $b = f(a)$, dunque:
            \[ bf(x) = f(a)f(x) = f(\underbrace{ax}_{\in I}) \in f(I)
                \]
    \end{enumerate}
\end{proof}

\begin{theorem}
    [Teorema di Corrispondenza tra Ideali]
    \label{corrispondenza}
    Sia $I \subset A$ un ideale e $\pi_I$ la proiezione all'anello quoziente modulo $I$, $\pi_I$ induce una corrispondenza biunivoca tra gli ideali di $\faktor{A}{I}$ e 
    gli ideali di $A$ che contengono $I$, e tale corrispondenza preserva l'ordinamento.
\end{theorem}

\begin{proof}
Per il Teorema di Corrispondenza tra sottogruppi abbiamo già la bigezione tra questi, dobbiamo tuttavia verificare che restringendo la corrispondenza agli ideali  questa associ ancora un 
ideale di $A$ ad un ideale di $\faktor{A}{I}$ e viceversa, cioè l'immagine e la controimmagine di un ideale mediante $\pi_I$ è ancora un ideale (in particolare, per la Corrispondenza tra Sottogruppi sappiamo già
che le controimmagini contengono l'ideale per il quale si quozienta). Siano:
    \[ X = \{J \subseteq A \, \text{ideale}| I \subset J\} \qquad \text e \qquad Y=\left\{\mathcal{J} \subset \faktor{A}{I} \m \,\text{$\mathcal{J}$ ideale}\right\}
        \]
per il \hyperref[2.35]{Lemma 2.35}, essendo $\pi_I$ surgettivo, si ha che le immagini e la controimmagini via $\pi_I$:
    \[ J \longmapsto \pi_I(J) \qquad \text e \qquad \mathcal{J} \longmapsto \pi_{I}^{-1}(\mathcal{J})
        \]
sono ideali, e ciò conclude la dimostrazione.
\end{proof}

\begin{example}
    Se nel \hyperref[2.35]{Lemma 2.35} $f$ non fosse surgettiva, allora l'immagine di un ideale non sarebbe un ideale, ad esempio presa:
        \[ f : \ZZ \varlonghookrightarrow \QQ : (2) \longmapsto 2\ZZ
            \]
    con $2\ZZ$ che non è un ideale di $\QQ$ perché $\QQ$ è un campo e quindi i suoi ideali sono soltanto $\{0\}$ e $\QQ$.
\end{example}

\begin{definition}
    Dato un omomorfismo di anelli $f: A \longrightarrow B$ e un ideale $I \subset A$ definiamo \vocab{estensione} di $I$ a $B$ via $f$ l'ideale generato in $B$ da $f(I)$:
        \[ (f(I)) = f(I)B = IB
            \]
\end{definition}

\begin{definition}
    Dato un omomorfismo di anelli $f: A \longrightarrow B$ e un ideale $J \subset B$ definiamo \vocab{contrazione} di $J$ ad $A$ via $f$ l'ideale $f^{-1}(J)$.
\end{definition}

\begin{remark}
    Gli omomorfismi sono sostanzialmente inclusioni a meno di isomorfismo, conoscendo la corrispondenza tra ideali indotta da $\pi_I$ osserviamo che:
        \[ A \varlonghookrightarrow B : I \longmapsto IB
            \]
    che manda ogni ideale nella propria estensione ad un \vocab{sovra anello} e:
        \[ J \longmapsto J\cap A
            \]
    che manda ogni ideale di $B$ nella propria contrazione ad un \vocab{sottoanello} fanno si che l'applicazione:
        \[ \varphi : A \varlonghookrightarrow B \longrightarrow \faktor{B}{J}
            \]
    sia tale che:
        \[ \ker \varphi = \{a \in A | \varphi(a) = a + J = J\} = \{a \in A | a \in J\} = J \cap A = \pi_{I}^{-1}(J)
            \]
    da cui si ha anche che:
        \[ \frac{A}{J \cap A} \varlonghookrightarrow \faktor{B}{J}
            \]
    per il Primo Teorema di Omomorfismo.
\end{remark}

\begin{remark}
Dal \hyperref[omo]{Teorema di Omomorfismo di Anelli} si deducono anche il secondo ed il terzo teorema di omomorfismo, ovvero:
    \[ \frac{A/I}{J/I} \cong \faktor{A}{J} \qquad \text e \qquad \frac{I+J}{J} \cong \frac{I}{I \cap J}
        \]
dove in entrambi i casi gli isomorfismi sono di anelli.    
\end{remark}

\newpage
\subsection{Prodotto diretto di anelli}
\begin{definition}
    Dati gli anelli $A,B$ il prodotto cartesiano $A \times B$ può essere dotato di una struttura di anello con le operazioni:
        \[ (a_1,b_1) + (a_2,b_2) = (a_1+a_2,b_1+b_2) \quad \text e \quad (a_1,b_1) \cdot (a_2,b_2) = (a_1 \cdot a_2,b_1 \cdot b_2) 
            \]
    $\forall a_1,a_2 \in A, \forall b_1,b_2 \in B$, l'insieme $A \times B$ con queste operazioni si dice \vocab{prodotto diretto} di anelli.
\end{definition}

\begin{theorem}
    [Teorema Cinese Del Resto Per Anelli]
    \label{t:cinese}
    Dato $A$ anello commutativo con unità, $I,J$ suoi ideali, allora la mappa di doppia proiezione:
    \[ f: A \longrightarrow \faktor{A}{I} \times \faktor{A}{J} : a \longmapsto (a+I,a+J)
        \]
    è un omomorfismo di anelli, con $\ker f = I \cap J$. Inoltre, $I+J = A$ se e solo se $f$ è surgettiva, ed in tal caso si ottiene:
    \[ \faktor{A}{IJ} \cong \faktor{A}{I} \times \faktor{A}{J}
        \]
\end{theorem}

\begin{proof}
    Verifichiamo in primis che $f$ sia un omomorfismo di anelli:
        \[ f(a+b) = ((a+b)+I,(a+b)+J) = (a+I,a+J) + (b+I,b+J) = f(a)+f(b) \quad \forall a,b \in A
            \]
    dove la terza uguaglianza è assicura dalla struttura di anello del quoziente; analogamente:
        \[ f(ab) = (ab+I,ab+J) = (a+I,a+J)(b+I,b+J) = f(a)f(b) \quad \forall a,b \in A
            \]
    Osserviamo ora che:
        \[ \ker f =\{a \in A | f(a) = (a+I,a+J) = (I,J)\} = \{a\in A | a \in I, a \in J\} = I \cap J
            \]
    Verifichiamo separatamente le due implicazioni della seconda parte del teorema:
    \begin{itemize}
        \item Supponiamo che $I+J = A$, ovvero che esistono $i$ e $j$ tali che $i+j = 1$\footnote{Poiché
        l'identità è in $I+J$, allora per la proprietà di assorbimento ogni altro elemento di $A$ è in $I+J$.},
        e verifichiamo che $f$ è surgettiva. Per verificare che $f$ è surgettiva dobbiamo far vedere che:
            \[ \forall a,b \in A, \exists x \in A : f(x) = (a+I,b+J)
                \]
        per ipotesi sappiamo che $x \in A \implies x \in I+J$ quindi possiamo prendere $x = bi + aj \in A$, per $i \in I$ e $j \in J$, dunque:
            \[ f(x) = (\underbrace{bi}_{\in I} + aj + I, bi + \underbrace{aj}_{\in J} + J) = (aj + I, bi +J)
                \]
        da cui, osservando che $j = 1-i$ e $i = 1 - j$ per ipotesi abbiamo:
            \[ (aj+I,bi+J) = (a(1-i)+I,b(1-j)+J) = (a+I,b+J)
                \]
        e pertanto abbiamo ottenuto $f(x) = (a+I,b+J)$.
        \item Supponiamo ora che $f$ sia surgettiva e proviamo che $I+J = A$. Se $f$ è surgettiva abbiamo che:
            \[ \exists i \in A : f(i) = (I,1+J)
                \]
        dunque per tale $i$ si ha che:
            \[ i \in I \qquad \text e \qquad i \equiv 1 \pmod J
                \]
        da cui si ricava che: $\underbrace{i}_{\in I} = 1 + \underbrace{j}_{\in J} \implies 1 \in I+J \implies I+J = A$.
    \end{itemize}
    Per il Primo Teorema di Omomorfismo abbiamo a questo punto che se $f$ è surgettiva (ed equivalentemente $I+J = A$), allora:
            \[ \frac{A}{\ker f} \cong \faktor{A}{I} \times \faktor{A}{J} \implies \frac{A}{I \cap J} \cong \faktor{A}{I} \times \faktor{A}{J}
                \]
    D'altra parte, per l'\hyperref[2.24]{Osservazione 2.24}, essendo $I+J = A$, allora $I \cap J = IJ$, da cui la tesi:
        \[\faktor{A}{IJ} \cong \faktor{A}{I} \times \faktor{A}{J}
            \]
\end{proof}

\begin{remark}
    Per il Teorema Cinese Del Resto tra gruppi sapevamo che:
        \[ \Z{mn} \cong \Z{m} \times \Z{n} \iff (m,n) = 1
            \]
    per il \hyperref[t:cinese]{Teorema Cinese Del Resto} tra anelli ora sappiamo che:
        \[ f: \ZZ \longrightarrow \Z{m} \times \Z{n}
            \]
    con $\ker f = m\ZZ \cap n\ZZ = [m,n]\ZZ$, da cui:
        \[ \Z{[m,n]} \varlonghookrightarrow \Z{m} \times \Z{n}
            \]
    avevamo visto che $f$ è surgettiva se e solo se $(m,n) = 1$, ed in questo modo $[m,n] = mn$ (o equivalentemente $n\ZZ + m\ZZ = (n,m)\ZZ = \ZZ$), dunque:
        \[ \Z{[m,n]} = \Z{mn} \cong \Z{m} \times \Z{n}
            \]
    pertanto la nuova versione del Teorema Cinese del Resto è una generalizzazione della precedente.
\end{remark}

\newpage
\subsection{Ideali primi e massimali}

\begin{definition}
    Sia $(\mathcal{F}, \leqslant)$ un insieme parzialmente ordinato e sia $X \subset \mathcal{F}$ un suo sottoinsieme, diciamo che 
    $M \in \mathcal{F}$ è un \vocab{maggiorante} per $X$ se:
        \[ A \leqslant M \qquad \forall A \in X
            \]
\end{definition}

\begin{definition}
    Sia $(\mathcal{F}, \leqslant)$ un insieme parzialmente ordinato, diciamo che 
    $A \in \mathcal{F}$ è un elemento \vocab{massimale} per $\mathcal{F}$ se:
        \[ \forall B \in \mathcal{F} : A \leqslant B \implies A = B
            \]
\end{definition}

\begin{definition}
    Sia $(\mathcal{F}, \leqslant)$ un insieme parzialmente ordinato, diciamo che 
    $A \in \mathcal{F}$ si dice \vocab{massimo} per $\mathcal{F}$ se:
    \[ \forall B \in \mathcal{F} : B \leqslant A
        \]
\end{definition}

\begin{definition}
    Sia $(\mathcal{F}, \leqslant)$ un insieme parzialmente ordinato, una \vocab{catena} di $\mathcal{F}$ è un sottoinsieme di $\mathcal{F}$ totalmente 
    ordinato.
\end{definition}

\begin{definition}
    Sia $(\mathcal{F}, \leqslant)$ un insieme parzialmente ordinato, $(\mathcal{F}, \leqslant)$ si dice \vocab{induttivo} se ogni catena di $\mathcal{F}$
    ammette un maggiorante in $\mathcal{F}$.
\end{definition}

\begin{lemma}
    [Lemma di Zorn]
    \label{zorn}
    Sia $(\mathcal{F}, \leqslant)$ un insieme non vuoto, parzialmente ordinato e induttivo, allora $\mathcal{F}$ contiene elementi massimali.
\end{lemma}

\begin{remark}
    Spesso il \hyperref[zorn]{Lemma di Zorn} viene usato su famiglie $\mathcal{F}$ di ideali ordinati secondo la relazione di inclusione $\subseteq$.
\end{remark}

\begin{definition}
    Dato un ideale proprio $I \subsetneq A$, $I$ si dice \vocab{primo} se:
        \[ xy \in I \implies x \in I \vee y \in I \qquad \forall x,y \in A
            \]
    ovvero se ogni volta che contiene un prodotto, contiene uno dei due fattori.
\end{definition}

\begin{definition}
    Un ideale $I$ si dice \vocab{massimale} se è un elemento massimale della famiglia $\mathcal{F}$ di tutti gli ideali propri di $A$, ovvero:
        \[ I \, \text{è massimale} \iff \forall J \subsetneq A : I \subseteq J \implies I = J
            \]
\end{definition}

\begin{example}
    [Ideali primi di $\ZZ$]
    Gli ideali primi di $\ZZ$ sono $(p)$ con $p$ primo, infatti:
    \[ xy \in (p) \iff p \mid xy \iff p \mid x \vee p \mid y
        \]
    ovvero se $x \in (p)$ o $y \in p$. Se consideriamo invece $(m)$, con $m$ non primo, dunque riducibile $m = ab$, con $1<a<m$ e $1<b<m$, allora:
    \[ ab \in (m) \qquad \text{ma} \quad a \not\in (m) \quad \text e \quad b \not\in (m)
        \]
    dunque $(m)$ non è primo.
\end{example}

\begin{proposition}[Proprietà degli Ideali Massimali]
    Dato un anello $A$ allora:
    \begin{enumerate}[(1)]
        \item Ogni ideale proprio di $A$ è contenuto in un ideale massimale.
        \item Ogni elemento non invertibile di $A$ è contenuto in un ideale massimale.
    \end{enumerate}
\end{proposition}

\begin{proof}
    Verifichiamo le affermazioni:
    \begin{enumerate}
        \item Sia $I \subsetneq A$ un ideale proprio e sia $\mathcal{F}$ la famiglia di tutti gli ideali propri che lo contengono:
            \[ \mathcal{F} = \{J \subsetneq A | I \subseteq J\}
                \]
            osserviamo che $I \in \mathcal{F} \implies \mathcal{F} \ne \emptyset$, inoltre $(\mathcal{F},\subseteq)$ è induttivo, infatti, detta $\mathscr{C}$ una catena, essa
            sarà un sottoinsieme di $\mathcal{F}$ totalmente ordinato della forma:
            \[ \mathscr{C} = \{J_n\}\footnote{I vari $J_i$ sono contenuti tutti uno dentro l'altro "in catena".} \subseteq \mathcal{F}
                \]
            allora posta $\bigcup J_n\footnote{Andrebbe dimostato che l'unione di ideali in catena, analogamente a quanto accade per i sottogruppi, è un ideale.} = J \in \mathcal{F}$, verifichiamo che
            $J$ è maggiorante di $\mathscr{C}$. Si ha che:
            \begin{itemize}
                \item $\forall J_n \in \mathscr{C} : J_n \subseteq J$, segue ovviamente da come abbiamo definito $J$, avendolo costruito come l'unione di tutti i $J_n$.
                \item $J \in \mathcal{F}$, poiché $I \subset J_n \subset J$, $\forall J_n \in \mathcal{F}$, e infine $J$ è un ideale proprio, infatti, se per assurdo fosse
                $1 \in J = \bigcup J_n \implies \exists n$ tale che $1 \in J_n \subsetneq A$ che è assurdo (se un ideale contenesse l'identità del prodotto, allora conterrebbe
                tutti gli elementi dell'anello).
            \end{itemize}
            Dunque ogni catena $\mathscr{C}$ di $\mathcal{F}$ ammette maggiorante\footnote{Abbiamo verificato addirittura che tale maggiorante sia un massimo della catena.}, pertanto $\mathcal{F}$ è induttivo
            e vale il \hyperref[zorn]{Lemma di Zorn}, per il quale la famiglia $\mathcal{F}$ ammette almeno un elemento massimale $M$. \\
            Resta da verificare che tale elemento massimale $M$ sia un ideale massimale dell'anello (poiché abbiamo dimostrato che è massimale per la famiglia $\mathcal{F}$ degli ideali che ne contengono uno proprio,
            la quale ovviamente non è la famiglia di tutti gli ideali propri di $A$), ciò segue subito osservando che,
            supponendo $L \subsetneq A$ ideale proprio con $M \subseteq L$, allora:
                \[ I \subseteq M \subseteq L \implies L \subset \mathcal{F}
                    \]
            dunque $L$ è un elemento della famiglia $\mathcal{F}$, e per la massimalità di $M$ in $\mathcal{F}$, segue che $M = L$.
        \item Segue immediatamente dal punto (1), infatti, sia $x \in A \setminus A^*$, allora per la \hyperref[2.26]{Proposizione 2.26} l'ideale generato da $x$ è proprio,
        $(x) \subsetneq A$, e quindi vale il punto (1) appena dimostrato:
            \[ (x) \subseteq M \implies x \in M
                \]
        con $M$ ideale massimale di $A$.
    \end{enumerate}
\end{proof}

\begin{proposition}[Caratterizzazione degli ideali primi e massimali]
    \label{2.56}
    Dato un ideale proprio di $I \subsetneq A$, allora:
    \begin{enumerate}[(1)]
        \item $I$ è primo se e solo se $\faktor{A}{I}$ è un dominio.
        \item $I$ è massimale se e solo se $\faktor{A}{I}$ è un campo.
    \end{enumerate}
\end{proposition}

\begin{proof}
    Verifichiamo le affermazioni:
    \begin{enumerate}[(1)]
        \item Presi $x,y \in A$, per definizione abbiamo che $I$ è primo se e solo se $xy \in I \implies x \in I$ o $y \in I$, d'altra parte, $\faktor{A}{I}$ è un dominio se e solo se:
            \[ (x+I)(y+I) = xy + I = I \iff xy \in I \implies x \in I \quad\text o\quad y \in I
                \]
            ovvero se e solo se, quando un prodotto di elementi si annulla (quindi fa la classe laterale neutra in questo caso) uno dei due elementi è già nella classe laterale neutra 
            dell'anello quoziente (quindi è già l'unico elemento neutro del prodotto, come richiesto dal fatto che l'anello sia un dominio), ma come si vede ciò è equivalente a dire che $I$ è primo.
        \item Per il (2) della \hyperref[2.26]{Proposizione 2.26} $\faktor{A}{I}$ è un campo se e solo se gli unici ideali che contiene sono quelli impropri, $\overline{(0)}$ e $\faktor{A}{I}$, dunque per il 
        \hyperref[corrispondenza]{Teorema di Corrispondenza} ciò è equivalente a dire che gli ideali di $A$ che contengono $I$ sono soltanto $A$ ed $I$ stesso \footnote{Infatti si ha che
        $\pi_I^{-1}\left(\faktor{A}{I}\right) = A$ e $\pi_I^{-1}(\overline{(0)}) = I$.}, ovvero $I$ è un ideale massimale di $A$.
    \end{enumerate}
\end{proof}

\begin{corollary}[Caratterizzazione degli ideali primi e massimali 2]
    \label{2.57}
    Dato $A$ un anello si ha:
    \begin{enumerate}[(1)]
        \item $A$ è un dominio se e solo se $(0)$ è un ideale primo.
        \item $A$ è un campo se e solo se $(0)$ è un ideale massimale.
        \item $I$ massimale $\implies I$ primo.
    \end{enumerate}
\end{corollary}

\begin{proof}
    Proviamo le affermazioni:
    \begin{enumerate}[(1)]
        \item Per l'(1) della \hyperref[2.56]{Proposizione 2.56} sappiamo che $(0)$ è primo se e solo se $\faktor{A}{(0)}$ è un dominio, ma:
            \[ \faktor{A}{(0)} \cong A
                \]
            da cui segue che $A$ è un dominio.
        \item Per il punto (2) della \hyperref[2.56]{Proposizione 2.56} sappiamo che $(0)$ è massimale se e solo se $\faktor{A}{(0)}$ è un campo, ma:
            \[ \faktor{A}{(0)} \cong A
                \]
            da cui segue che $A$ è un campo.
        \item Per quanto detto nel (2) della \hyperref[2.56]{Proposizione 2.56}, $I$ è massimale se e solo se $\faktor{A}{I}$ è un campo, in particolare ciò 
            significa che $\faktor{A}{I}$ è un dominio d'integrità, ma per l'(1) della \hyperref[2.56]{Proposizione 2.56} ciò è equivalente a dire che $I$ sia primo.
    \end{enumerate}
\end{proof}

\begin{example}
    [$\ZZ$ è un dominio ma non un campo]
    Si osserva che l'ideale $(0)$ è un ideale primo (poiché $xy \in (0) \iff xy = 0 \implies x \in (0)$ o $y \in (0)$, poiché $\ZZ$ è un dominio), ma non massimale, infatti:
        \[ (0) \subset (m) \qquad \forall m \in \ZZ
            \]
\end{example}

\begin{corollary}
    La corrispondenza biunivoca tra ideali per mezzo della proiezione:
    \[ \pi_I : A \longrightarrow \faktor{A}{I}
        \]
    conserva ideali primi e massimali.\footnote{Ovviamente gli ideali considerati devono contenere $I$, altrimenti non c'è nulla da preservare in arrivo.}
\end{corollary}

\begin{proof}
    Osserviamo preliminarmente che si ha $I \subseteq J \subseteq A$, dunque nella proiezione $\pi_I$ si ha che:
        \[ J \longmapsto \pi_I(J) = \faktor{J}{I}
            \]
    Dobbiamo dimostrare che $J$ è primo (massimale) in $A$ se e solo se $\faktor{J}{I}$ è primo (massimale) in $\faktor{A}{I}$. Per quanto detto nella \hyperref[2.56]{Proposizione 2.56} $J$ primo (massimale) 
    è equivalente al fatto che $\faktor{A}{J}$ sia un dominio (campo), e, ugualmente deve essere che $\displaystyle\frac{A/I}{J/I}$ è un dominio (campo) ma dal Secondo Teorema di Omomorfismo di Anelli si ha:
    \[ \frac{A/I}{J/I} \cong \faktor{A}{J}
        \]
    che in entrambi i casi verifica la tesi.
\end{proof}

\newpage
\subsection{Anello delle frazioni di un dominio}

\begin{definition}
    Consideriamo un anello commutativo con identità $A$, che sia un dominio di integrità. Sia $S \subset A$ con le seguenti proprietà:
    \begin{itemize}
        \item $0 \not\in S$.
        \item $1 \in S$.
        \item $S$ è moltiplicativamente chiuso: $xy \in S$, $\forall x,y \in S$.
    \end{itemize}
    Il sottoinsieme $S$ con queste proprietà si dice \vocab{parte moltiplicativa} di $A$.
\end{definition}

\begin{definition}
    Dato un anello commutativo con identità $A$, che sia un dominio di integrità, e $S$ la sua parte moltiplicativa, allora possiamo definire l'insieme delle \vocab{frazioni di un dominio}:
    \[ S^{-1}A = \left.\left\{\frac{a}{s} \m a\in A, s \in S\right\} \middle/ \sim \right. = \frac{A \times S}{\sim}
        \]
    con la relazione $\sim$ data da $\displaystyle\frac{a}{s} \sim \frac{b}{t} \iff at = bs$.\footnote{Alternativamente possiamo scrivere la relazione come: $(a,s) \sim (b,t) \iff at = bs$, ed indicare con $\displaystyle\frac{a}{s}$ la classe di equivalenza dei due elementi.}
\end{definition}

\begin{remark}
    La relazione $\sim$ usata nella definizione precedente è una relazione di equivalenza, infatti:
    \begin{itemize}
        \item $\sim$ è riflessiva in quanto $\displaystyle\frac{a}{s} \sim \frac{a}{s} \iff as = sa$, che è vero in quanto abbiamo supposto $A$ commutativo.
        \item $\sim$ è simmetrica in quanto $\displaystyle\frac{a}{s} \sim \frac{b}{t} \iff at = bs \iff \frac{b}{t} \sim \frac{a}{s}$.
        \item $\sim$ è transitiva in quanto, dati $\displaystyle\frac{a}{s} \sim \frac{b}{t}$ e $\displaystyle\frac{b}{t} \sim \frac{c}{u}$ abbiamo che:
            \[ at = bs \qquad \text e \qquad bu = tc
                \]
        da cui, moltiplicando la prima per $u$ si ha:
            \[ aut = bus = tcu \implies aut = cts \iff t(au-cs) = 0
                \]
        essendo per ipotesi $A$ un dominio\footnote{È importante notare che qui stiamo usando il fatto che $A$ è un dominio.} e $t \in S$ (dunque $t \ne 0$) segue:
            \[ au = cs \iff \frac{a}{s} \sim \frac{c}{u}
                \]
    \end{itemize}
\end{remark}

\begin{proposition}[Anello delle frazioni di un dominio]
    L'insieme delle frazioni di un dominio munito con le operazioni di:
    \[ \frac{a}{s} + \frac{b}{t} = \frac{at+bs}{st} \qquad \text e \qquad \frac{a}{s} \cdot \frac{b}{t} = \frac{ab}{st}
        \]
    è un anello commutativo con identità.\footnote{Con l'identità data dall'elemento $1/1$.}
\end{proposition}

\begin{proof}
    Bisogna verificare in primis che le operazioni sono ben definite (in quanto le abbiamo definite tra classi di equivalenza), consideriamo $\displaystyle\frac{a}{s}\sim\frac{a^{\prime}}{s^{\prime}}$ e
    $\displaystyle\frac{b}{t}\sim\frac{b^{\prime}}{t^{\prime}}$, vogliamo verificare che le due somme:
    \[ \frac{a}{s} + \frac{b}{t} = \frac{at+bs}{st} \qquad \text e \qquad \frac{a^{\prime}}{s^{\prime}} + \frac{b^{\prime}}{t^{\prime}} = \frac{a^{\prime}t^{\prime}+b^{\prime}s^{\prime}}{s^{\prime}t^{\prime}}
        \]
    diano lo stesso risultato; per ipotesi sappiamo che $as^{\prime}=a^{\prime}s$ e $bt^{\prime}=b^{\prime}t$, osserviamo che l'uguaglianza tra le due somme è vera se e solo se:
    \[ (at+bs)s^{\prime}t^{\prime} = (a^{\prime}t^{\prime}+b^{\prime}s^{\prime})st
        \]
    sviluppando l'LHS otteniamo:
    \[ att^{\prime}s^{\prime} + bss^{\prime}t^{\prime} = a^{\prime}stt^{\prime}+b^{\prime}tss^{\prime} = (a^{\prime}t^{\prime}+b^{\prime}s^{\prime})st
        \]
    che dimostra che l'operazione $+$ è ben definita.\footnote{Le restanti (lunghissime e noiosissime) 9 verifiche verranno aggiunte in seguito :).}
\end{proof}

\begin{example}
    [Anello delle frazioni di $\ZZ$]
    Preso $A = \ZZ$ e $S = \{10^k\}_{k\geq 0}$ (si verifica facilmente che $S$ rispetta le tre proprietà richieste dalla definizione) abbiamo che l'anello delle frazioni di $\ZZ$ è dato da:
    \[ S^{-1}A = \left\{\frac{z}{10^k} \m z \in \ZZ, k \geq 0\right\}
        \]
    con ad esempio $\displaystyle \frac{5}{10} = \frac{1}{2} \in S^{-1}A$.
\end{example}

\begin{remark}
    Nel caso dell'esempio precedente si osserva che $\displaystyle\frac{2}{1} \in S^{-1}$ ed è invertibile:
        \[\frac{2}{1} \cdot \frac{5}{10} = \frac{1}{1}
            \]
\end{remark}

\begin{proposition}[$S^{-1}A$ come estensione di $A$]
    \label{2.66}
    Dato un dominio $A$ e il suo anello delle frazioni, l'applicazione:
    \[ f : A \longrightarrow S^{-1}A : a \longmapsto \frac{a}{1}
        \]
    è un omomorfismo iniettivo di anelli.\footnote{Dunque $A\subset S^{-1}A$, cioè $S^{-1}A$ è un'estensione di $A$.}
\end{proposition}

\begin{proof}
    Verifichiamo che $f$ sia un omomorfismo di anelli:
    \[ f(a+b) = \frac{a+b}{1} = \frac{a}{1} + \frac{b}{1} = f(a) + f(b) \qquad \forall a,b \in A
        \]
    e analogamente:
    \[ f(ab) = \frac{ab}{1} = \frac{a}{1} \cdot \frac{b}{1} = f(a)f(b) \qquad \forall a,b \in A
        \]
    Per l'iniettività studiamo il nucleo:
    \[ \ker f = \left\{a \in A \m f(a) = \frac{a}{1} = \frac{0}{1}\right\}\footnote{Ricordiamo che $0/1$ è l'elemento neutro di $S^{-1}A$.} = \{a \in A | a \cdot 1 = 0 \cdot 1 = 0\} = \{0\}
        \]
    dunque l'omomorfismo è iniettivo.
\end{proof}

\begin{remark}
    [$S = A \setminus\{0\}$]
    Se $A$ è un dominio, allora $S = A \setminus\{0\}$\footnote{Sarebbe $A^*$ se $A$ fosse finito.} è una parte moltiplicativa, infatti, $\forall x,y \in S$, ovvero $x \ne 0$ e $y \ne 0$, dunque 
    $xy \in S$, $xy \ne 0$.
\end{remark}

\begin{definition}
    Dato un dominio $A$, definiamo \vocab{campo dei quozienti} di $A$:
    \[ S^{-1}A = Q(A)
        \]
    l'anello delle frazioni con parte moltiplicativa $S = A\setminus\{0\}$.
\end{definition}

\begin{proposition}
    [$A \subset Q(A)$]
    \label{2.69}
    Dato $A$ dominio e la sua parte moltiplicativa $S = A\setminus\{0\}$, l'anello delle frazioni $S^{-1}A = Q(A)$ è il più 
    piccolo campo che contiene $A$.
\end{proposition}

\begin{proof}
    Verifichiamo prima che $Q(A)$ è un campo e poi la tesi:
    \begin{itemize}
        \item Per verificare che $Q(A)$ sia un campo, ci basta verificare che esistono gli inversi moltiplicativi, e ciò segue immediatamente
         dal fatto che $\forall a \in A$, $a \ne 0$, allora $\displaystyle\frac{1}{a} \in Q(A)$ (in questo modo tutti gli elementi di $A$, eccetto lo 0,
         possono essere scritti come $\displaystyle\frac{1}{a}$) ed è tale per cui:
            \[ \frac{a}{1} \cdot \frac{1}{a} = \frac{1}{1}
                \]
        \item Per la \hyperref[2.66]{Proposizione 2.66} sappiamo già che $A \subset S^{-1}A$, ed ora abbiamo dimostrato che $S^{-1}A$ è un campo, ci resta da verificare 
        che $S^{-1}A (= Q(A))$ sia effettivamente il più piccolo campo che contiene $A$. Sia $K$ è un campo tale che $A \subset K$, allora $\displaystyle \frac{1}{a} \in K$, $\forall a \in A \setminus\{0\}$, ovvero $K$ 
        contiene tutti gli inversi degli elementi di $A$, allora, $\forall b \in A$, $\forall a \in A \setminus \{0\}$, cioè $K$ contiene tutti gli elementi di $S^{-1}A$:
            \[ \frac{b}{a} \in K \implies Q(A) = S^{-1}A \subset K
                \]
        pertanto, essendo contenuto in ogni campo che contiene $A$, e contenendolo a sua volta, $Q(A)$ è il più piccolo campo che contiene $A$.
    \end{itemize}
\end{proof}

\begin{example}
    Vediamo alcuni esempi di anelli delle frazioni di domini e campi quoziente:
    \begin{itemize}
        \item Consideriamo $A = \ZZ$, $S_1 = \{10^k\}_{k\geq 0}$ e $S_0 = A \setminus\{0\}$, allora si ha che:
            \[ \ZZ \subset S_1^{-1}\ZZ \subset S_0^{-1}\ZZ = Q(\ZZ) = \QQ
                \]
        \item Considerando $A = K[x]$ si ha che:
            \[ Q(A) = K(x) = \left\{\frac{f(x)}{g(x)} \m f(x),g(x) \in K[x], g(x) \ne 0\right\}
                \]
        \item Sia $A$ un dominio e $P \subset A$ un suo ideale primo, possiamo considerare $S = A \setminus P$ che è una parte moltiplicativa, in quanto 
            $0 \not\in S$, $1 \in S$ e $\forall x,y \in S$ si ha:
            \[ x,y \not\in P \implies xy \not\in P
                \]
            poiché $P$ è primo, dunque $xy \in A \setminus P = S$. In questo caso indichiamo $S^{-1}A = A_p$ e prende il nome di \vocab{localizzato} dell'anello $A$ all'ideale $P$.
    \end{itemize}
\end{example}

\begin{remark}
    Dato il localizzato di $A$ a $P$, $A_p$, si osserva che esso è un \vocab{anello locale}, ovvero un anello che ha un unico ideale massimale.
\end{remark}

\begin{example}[Localizzato di un ideale primo]
    Se consideriamo $A = \ZZ$ e $P = 2\ZZ$, allora la parte moltiplicativa è data da $S = \ZZ\setminus 2\ZZ$ (i numeri dispari), da cui abbiamo che:
    \[ S^{-1}\ZZ = \ZZ_{(2)} = \left\{\frac{a}{b} \m a \in \ZZ, b \equiv 1 \, \text{(mod 2)}\right\}
        \]
\end{example}

\begin{exercise}
    Dati $A = \ZZ$, $P = 2\ZZ$ e $S = \ZZ\setminus 2\ZZ$, verificare che l'ideale $(2)\ZZ_{(2)}$ è l'unico ideale massimale di $\ZZ_{(2)}$.
\end{exercise}

\begin{soln}
    La tesi è equivalente a mostrare che $\ZZ_{(2)}^*=\ZZ_{(2)}\setminus (2)\ZZ_{(2)}$. Infatti, sappiamo già che 
    $(2)\ZZ_{(2)}$ è un ideale, mentre qualunque ideale non contenuto in esso contiene necessariamente un elemento invertibile ed è perciò non proprio.
    Se $\displaystyle\frac{a}{b}\notin (2)\ZZ_{(2)}$ allora sia $a$ che $b$ sono dispari, dunque $\displaystyle\frac{b}{a}\in\ZZ_{(2)}$ ed è chiaramente l'inverso di $\displaystyle\frac{a}{b}$.
    Viceversa se $\displaystyle\frac{a}{b}$ è invertibile esiste $\displaystyle\frac{c}{d}\in\ZZ_{(2)}$ tale che $\displaystyle\frac{ac}{bd}=1$, cioè $ac=bd$. Se uno tra $a$ e $c$ fosse pari lo sarebbe
    anche $bd$, e poiché 2 è primo uno tra $b$ e $d$ sarebbe pari, contraddicendo la definizione di $\ZZ_{(2)}$. Dunque $\displaystyle\frac{a}{b}\in\ZZ_{(2)}\setminus (2)\ZZ_{(2)}$.\footnote{Dimostrazione proposta da Davide Ranieri.}
\end{soln}

\begin{remark}
    [Elementi invertibili di $S^{-1}A$]
    Osserviamo che gli invertibili dell'anello $S^{-1}A$ sono:
    \[ (S^{-1}A)^* = \left\{\frac{a}{s} \,\m\, \frac{s}{a} \in S^{-1}A\right\}
        \]
    ovvero esistono $b \in A$ e $t \in S$ tali che $\displaystyle \frac{s}{a} = \frac{b}{t} \iff st = ab \in S$ (cioè esiste una scrittura di questo tipo in $S^{-1}A$, ma poiché 
    non è detto che $a$ appartenga ad $S$, sappiamo che, per quanto scritto, almeno un suo multiplo c'è), dunque:
    \[ (S^{-1}A)^* = \left\{\frac{a}{s} \,\m\, \exists b \in A \;\, \text{t.c.} \;\, ab \in S\right\}
        \]
    Ad esempio, nel caso di $A = \ZZ$ e $S = \{10^k\}_{k\geq 0}$, abbiamo che:
    \[ \frac{5}{10}=\frac{1}{2} \in (S^{-1}A)^* \qquad \text{ma} \qquad 2 \not \in S
        \]
    dunque $2 \in (S^{-1}A)^*$, poiché il suo inverso, $\displaystyle \frac{1}{2} = \frac{5}{10}$, ha una scrittura che rispetta la proprietà richiesta dall'insieme (e tale scrittura è appunto un multiplo di quella iniziale).
\end{remark}

\begin{remark}
    [Ideali di $S^{-1}A$]
    Sia $I \subset A$ un ideale di $A$, possiamo costruire l'insieme:
    \[ S^{-1}I = \left\{\frac{x}{s} \in S^{-1}A \m x \in I, s \in S \right\} \cong \frac{I \times S}{\sim}
        \]
    per tale insieme valgono le proprietà espresse dalla proposizione seguente.
\end{remark}

\begin{proposition}
    [Ideali di $S^{-1}A$]
    Sia $I \subset A$ e sia $S^{-1}A$ l'insieme costruito come sopra, allora:
    \begin{enumerate}[(1)]
        \item $S^{-1}I$ è un ideale di $S^{-1}A$.
        \item $\forall J \subset S^{-1}A$, $\exists I \subset A$ tale che $J = S^{-1}I$ (cioè ogni ideale di $S^{-1}A$ si ottiene da un ideale di $A$, considerandone il relativo anello delle frazioni).
        \item $S^{-1}I$ è un ideale proprio di $S^{-1}A$ se e solo se $I \cap S = \emptyset$.
        \item Sia $P$ un ideale primo di $A$, con $P \cap S = \emptyset$, allora $S^{-1}P$ è un ideale primo di $S^{-1}A$.
    \end{enumerate}
\end{proposition}

\begin{proof}
    Dimostriamo le singole affermazioni:
    \begin{enumerate}[(1)]
        \item Per verificare che $S^{-1}I$ sia un ideale verifichiamo prima la chiusura per somma:
            \[ \frac{x}{s} + \frac{y}{t} = \frac{\overbrace{xt + ys}^{\in I}}{\underbrace{st}_{\in S}} \in S^{-1}I \qquad \forall x,y \in I
                \]
            dove l'appartenenza del numeratore deriva dal fatto che $x$, $y$ siano elementi di un ideale. Per verificare la proprietà di assorbimento osserviamo che:
            \[ \frac{a}{s} \cdot \frac{x}{t} = \frac{\overbrace{ax}^{\in I}}{\underbrace{st}_{\in S}} \in S^{-1}I \qquad \forall \frac{a}{s} \in S^{-1}A 
                 \]
        \item Sia $J \subset S^{-1}A$ un ideale, per quanto detto nella \hyperref[2.66]{Proposizione 2.66}, sappiamo che $S^{-1}A$ è un'estensione di $A$, inoltre se consideriamo $f^{-1}(J)$, che per il \hyperref[2.35]{Lemma 2.35}, 
            sappiamo essere un ideale, ed in particolare una contrazione di $J$ ad $A$, abbiamo che:
            \[ f^{-1}(J) = J \cap A = I \subset A
                \]
            vogliamo mostrare che vale $J = S^{-1}I$. Osserviamo che $\forall x \in I$ si ha $\displaystyle f(x) = \frac{x}{1} \in J$, dunque:
            \[ \underbrace{\frac{1}{s}}_{\in S^{-1}A} \cdot \frac{x}{1} = \frac{x}{s} \in J \implies S^{-1}I \subseteq J
                \]
            cioè per assorbimento di $J$ ci sono tutti gli elementi di $S^{-1}I$. Viceversa si ha che $\displaystyle\forall \frac{x}{s} \in J$ possiamo scrivere:
            \[ \frac{x}{1} = \frac{x}{s}\cdot \frac{s}{1} \in J \implies x = f^{-1}\left(\frac{x}{1}\right) \in I
                \]
            ovvero il numeratore di ogni elemento in $S^{-1}J$ è un elemento di $I$, dunque considerando l'anello delle frazioni $S^{-1}I$ esso contiene tutte quelle di $S^{-1}J$, da cui si conclude $\displaystyle \frac{x}{s} \in S^{-1}I \implies J \subseteq S^{-1}I$.
        \item Dimostriamo la negazione\footnote{Poiché trattandosi di un'equivalenza logica va bene lo stesso.}, ovvero $S^{-1}I$ non proprio equivale a $S^{-1}I = S^{-1}A$, ma essendo il primo un ideale questo è vero se e solo se:
            \[ \frac{1}{1} \in S^{-1}I \iff \exists x \in I, \exists s \in S : \frac{1}{1} = \frac{x}{s}
                \]
            che, per la relazione definita sugli anelli di frazioni è equivalente a chiedere che $I \ni x = s \in S \iff I \cap S \ne \emptyset$.
        \item Sia $P$ un ideale primo, se fosse $P \cap S \ne \emptyset$, allora per quanto visto al punto (3) non sarebbe proprio (e dunque nemmeno primo), viceversa, se $P \cap S = \emptyset$ vogliamo dimostrare che $S^{-1}P$ primo in $S^{-1}A$; consideriamo:
                \[ \frac{a}{s} \cdot \frac{b}{t} \in S^{-1}P
                    \]
            ciò è equivalentemente al fatto che $\exists \sigma \in S$ e $\exists p \in P$ tali per cui:
                \[ \frac{ab}{st} = \frac{p}{\sigma} \iff ab\sigma = \underbrace{p}_{\in P}st \in P \implies ab\sigma \in P
                    \]
            ma, essendo per ipotesi che $\sigma \in P$ e $P \cap S = \emptyset$, allora $ab \in P$, e poiché $P$ è primo si deve avere $a \in P$ p $b \in P$, e quindi la frazione di uno dei due deve essere quella in $S^{-1}P$: $\displaystyle\frac{a}{s} \in S^{-1}P \qquad \text o \qquad \frac{b}{t} \in S^{-1}P$
            e quindi $S^{-1}P$ primo.
    \end{enumerate}
\end{proof}

\newpage
\subsection{Divisibilità nei domini}
\begin{definition}
    Sia $A$  un dominio e siano $a,b \in A$, con $a \ne 0$, si dice che $a \mid b$ ($a$ \vocab{divide} $b$) se:
    \[ \exists c \in A : b = ac
        \]
\end{definition}

\begin{remark}
    Osserviamo che $a \mid b \iff (b) \subseteq (a)$, infatti:
    \[ a \mid b \iff \exists c \in A : b = ac \iff b \in (a) \iff (b) \subseteq (a)
        \]
\end{remark}

\begin{definition}
    Dato $A$ dominio e $a$, $a^{\prime}$, diciamo che $a$ ed $a^{\prime}$ sono \vocab{associati}, $a \sim a^{\prime}$,
    se vale una delle seguenti tre condizioni equivalenti:
    \begin{enumerate}[(i)]
        \item $a \mid a^{\prime}$ e $a^{\prime} \mid a$.
        \item $\exists u \in A^*$ tale che $a = ua^{\prime}$.
        \item $(a) = (a^{\prime})$.
    \end{enumerate}
\end{definition}

\begin{remark}
    [Equivalenza delle condizioni]
    Osserviamo che le tre condizioni date sono equivalenti, infatti, per quanto riguarda (i) e (iii) si ha:
    \[ a \mid a^{\prime} \iff (a^{\prime}) \subseteq (a) \qquad \text e \qquad a^{\prime}\mid a \iff (a) \subseteq (a^{\prime})
        \]
    dunque se sono vere entrambe le condizioni (i) e (iii) sono equivalenti. Dobbiamo da verificare che (i)$\implies$(ii), dalle due divisibilità segue che:
    \[ a^{\prime} = xa \qquad \text e \qquad a = y a^{\prime} \implies a = yxa \implies a(1-xy) = 0
        \]
    poiché $a \ne 0$, ed $A$ dominio per ipotesi si ha che $xy = 1 \implies y \in A^*$, ovvero la (ii). Viceversa, assumiamo (ii) e deduciamo 
    (iii):\footnote{A questo punto sappiamo che già che (i) e (iii) sono equivalenti, quindi non è necessario fare verifiche distinte.}
    \[ a = ua^{\prime} \implies a \in (a^{\prime}) \implies (a^{\prime}) \subseteq (a)
        \]
    con $u \in A^*$, pertanto $\exists v \in A^*$ tale che $uv = vu = 1$, moltiplicando la prima relazione per $v$ si ottiene:
    \[ a^{\prime} = va \implies a^{\prime} \in (a) \implies (a) \subseteq (a^{\prime})
        \]
    e si conclude $(a) \subseteq (a^{\prime})$.
\end{remark}

\begin{definition}
    Dati $a,b \in A$ dominio, non entrambi nulli, diciamo che $d \in A$ è un \vocab{massimo comun divisore} per $a$ e $b$ se:
    \begin{enumerate}[(1)]
        \item $d \mid a$ e $d \mid b$.
        \item $\forall x \in A$ tale che $x \mid a$ o $x\mid b$, allora $x \mid d$.
    \end{enumerate}
\end{definition}

\begin{proposition}[Gli M.C.D. di due elementi in un dominio sono associati]
    Dati $d,d^{\prime} \in A$, essi sono due massimi comun divisori di una stessa coppia di elementi $a$ e $b$ di $A$, $d=(a,b)$ e $d^{\prime} = (a,b)$, se e solo se sono associati, $d \sim d^{\prime}$.
\end{proposition}

\begin{proof}
Se $d$ e $d^{\prime}$ sono due massimi comun divisori di $a$ e $b$, allora vale che:
        \[ d \mid a \land d \mid b
        \qquad
        \text e
        \qquad
        x \mid a \land x \mid b \implies x \mid d
            \]
    e contemporaneamente:
        \[ d^{\prime} \mid a \land d^{\prime} \mid b
        \qquad
        \text e
        \qquad
        x \mid a \land x \mid b \implies x \mid d^{\prime}
            \]
    dunque, considerando $d$, esso deve essere diviso da $d^{\prime}$ in quanto massimo comune divisore:
        \[ d = ud^{\prime}
            \]
    e simmetricamente:
        \[ d^{\prime} = vd
            \]
    da cui, sfruttando il fatto che $A$ è un dominio segue:
        \[ d = ud^{\prime} = uvd \implies d(1-uv) = 0 \implies uv = 1 \implies u,v \in A^*
            \]
    e quindi $d \sim d^{\prime}$ per definizione.
\end{proof}

\begin{definition}
    Dato un dominio $A$ e $x\in A$, con $x \not\in A^* \cup \{0\}$, $x$ si dice \vocab{primo} se $\forall a,b \in A$:
    \[ x \mid ab \implies x \mid a \vee x \mid b
        \]
\end{definition}

\begin{definition}
    Dato un dominio $A$ e $x\in A$, con $x \not\in A^* \cup \{0\}$, $x$ si dice \vocab{irriducibile} se $\forall a,b \in A$:
    \[ x = ab \implies a \in A^* \vee b \in A^*
        \]
\end{definition}

\begin{proposition}[primo$\implies$irriducibile]
    \label{2.58}
    Dato $A$ dominio, se $x$ è primo, allora è irriducibile.
\end{proposition}

\begin{proof}
    Supponiamo che:
    \[ x = ab
        \]
    essendo $x$ primo, allora $x \mid a$ o $x \mid b$, assumiamo (WLOG) che $x \mid a$, allora:
    \[ a = xc \implies x = bcx \implies x(1-bc) = 0
        \]
    poiché $A$ è un dominio, e poiché $x \ne 0$ per ipotesi segue che:
    \[ bc = 1 \implies b,c \in A^*
        \]
    in particolare ciò significa che $x$ è irriducibile, in quanto scrivendolo come $x = ab$, abbiamo verificato che $b \in A^*$.
\end{proof}

\begin{proposition}[Caratterizzazione di primi ed irriducibili in domini]
    \label{2.86}
    Dato un dominio $A$ si ha che:
    \begin{enumerate}[(1)]
        \item $x$ è primo se e solo se $(x)$ è un ideale primo non nullo.
        \item $x$ è irriducibile se e solo se $(x)$ è un ideale massimale nell'insieme degli ideali principali.
    \end{enumerate}
\end{proposition}

\begin{proof}
    Verifichiamo entrambe le proprietà:
    \begin{enumerate}[(1)]
        \item Sia $(x)$ un ideale primo, ovvero:
            \[ ab \in (x) \iff a \in (x) \vee b \in (x)
                \]
        ciò è equivalente al richiedere che $x \mid a$ o $x \mid b$, ovvero che $x$ sia primo in $A$.
        \item Dimostriamo separatamente le due implicazioni. Sia $x$ irriducibile e supponiamo che sia $(x) \subseteq (y) \subsetneq A$,
            dunque $\exists z \in A$ tale che $x = yz$, sappiamo che $y \not \in A^*$ (altrimenti sarebbe l'ideale conterebbe l'identità
            e avremmo $(y) = A$), poiché $x$ deve essere irriducibile segue necessariamente che $z \in A^*$ e quindi $x \sim y$, cioè:
            \[ (x) = (y)
                \]
            pertanto $(x)$ è massimale tra gli ideali principali. Per il viceversa dimostriamo la contronominale; sia $x$ riducibile, allora:
            \[ x = yz \qquad \text{con $y,z \not \in A^*$}
                \]
            e per quanto detto segue che:
            \[ (x) \subsetneq (y) \subsetneq A
                \]
            dove la seconda inclusione non può essere un'uguaglianza in quanto $y \not \in A^* \implies 1 \not \in (y)$, mentre la prima segue dal 
            fatto che $z \not \in A^*$ (e quindi $x$ e $y$ non sono associati), pertanto $(x)$ non è massimale tra gli ideali principali.
    \end{enumerate}
\end{proof}

\begin{example}
    Osserviamo che se $x$ è primo nel dominio d'integrità $A = K[x,y]$, allora:
    \[ \faktor{A}{(x)} \cong K[y]
        \]
    che sappiamo essere un dominio, dunque per la \hyperref[2.56]{Proposizione 2.56} $(x)$ è primo. Poiché $x$ è primo è anche irriducibile, quindi $(x)$
    è massimale tra gli ideali principali di $A$, ma non è un ideale massimale di $A$ in quanto $K[y]$ non è un campo, infatti:
    \[ (x) \subsetneq (x,y) \subsetneq A
        \]
    ovvero $(x)$ non è massimale in quanto è contenuto nell'ideale proprio $(x,y)$.
\end{example}

\newpage
\subsection{Domini euclidei (ED)}
\begin{definition}
    Un dominio di integrità $A$ si dice \vocab{dominio euclideo} (ED) se esiste una funzione:
    \[ d : A \setminus\{0\} \longrightarrow \NN
        \]
    detta \vocab{grado}, con le seguenti proprietà:
    \begin{enumerate}[(i)]
        \item $d(x) \leq d(xy)$, $\forall x,y \in A \setminus\{0\}$.
        \item $\forall x \in A$, $\forall y \in A \setminus\{0\}$, $\exists q,r \in A$ tali che:
        \[ x = yq+r
            \]
        con $d(r) < d(y)$\footnote{Ciò non assicura l'unicità di $q$ ed $r$, ma non è richiesto dalla definizione.} oppure $r = 0$.
    \end{enumerate}
\end{definition}

\begin{remark}
    In un dominio euclideo ogni elemento si può "ben approssimare" con un multiplo di ogni altro elemento. La funzione
    grado serve per dire cosa significa esattamente "approssimare bene".
\end{remark}

\begin{example}
    [Domini Euclidei]
    Vediamo alcuni esempi di domini euclidei:
    \begin{enumerate}[(1)]
        \item $(\ZZ, |\cdot|)$ è un dominio euclideo, infatti sappiamo che per le proprietà del valore assoluto vale che:
        \[ |xy| = |x||y| \geq |x| \qquad \forall y \ne 0
            \]
        ed esiste la divisione euclidea con il resto:
        \[ \forall x,y \in \ZZ, \exists q,r \in \ZZ: x = qy + r
            \]
        con $|r| < |y|$ oppure $r = 0$, la prima condizione posta non ci assicura l'unicità (infatti possiamo approssimare sia dal basso che dall'alto 
        prendendo anche resti negativi), ma poiché sappiamo che in realtà vale:
        \[ 0 \leq r < |y|
            \]
        allora tale scrittura è unica.
        \item $(K[x],\deg)$ è un dominio euclideo, infatti:
        \[ \deg (f(x)g(x)) = \deg f(x) + \deg g(x) \geq \deg f(x)
            \]
        e vale la divisione euclidea tra polinomi con resto:
        \[ \forall f(x),g(x) \in K[x], \exists q(x),r(x)\in K[x] : f(x) = g(x)q(x) + r(x)
            \]
        con $\deg r(x) < \deg g(x)$ oppure $\deg r(x) = 0$.
        \item $(\ZZ[i],N)$, l'anello degli \vocab{interi di Gauss}, $\ZZ[i] = \{a + ib | a,b \in \ZZ\}$, con la norma data da:
            \[ N : \ZZ[i] \longrightarrow \NN : (a +ib =)\alpha \longmapsto N(\alpha) = \alpha \overline{\alpha} = a^2+b^2
                \]
    \end{enumerate}
\end{example}

Osserviamo che $\ZZ[i] \subset \QQ(i) \subset \CC$, dunque $\ZZ[i]$ è un dominio. Osserviamo inoltre che valgono le proprietà richieste dalla definizione di dominio euclideo per la norma 
definita in precedenza:
\[ N(\alpha\beta) = |\alpha\beta|^2 = |\alpha|^2|\beta|^2 \geq |\alpha|^2 = N(\alpha)
    \]
che è vera in quanto $|\beta|^2 = x^2+y^2 \geq 1$. Per la proprietà (ii), vorremmo che $\forall \alpha,\beta \in \ZZ[i]$, con $\beta \ne 0$, si può "approssimare" $\alpha$ con un multiplo di $\beta$;
abbiamo che $\displaystyle \frac{\alpha}{\beta} \in \CC$, ma in generale non in $\ZZ[i]$, se $\displaystyle\frac{\alpha}{\beta} \in \ZZ[i]$, vogliamo trovare semplicemente il valore del rapporto in
$\ZZ[i]$ (ed abbiamo che $r = 0$), se $\displaystyle\frac{\alpha}{\beta} \not\in \ZZ[i]$, allora vogliamo scrivere:
\[ \alpha = \beta q + r \qquad \text{con}\qquad N(r) < N(\beta)
    \]
Osserviamo il reticolo formato dai multipli di $\beta$ nel piano complesso:\footnote{Immagine provvisoria, non appena ho tempo la disegno in TikZ.}

\begin{figure}[h]
    \centering
    \includegraphics[width=8cm]{Gauss.png}
    \caption{Reticolo dei multipli interi di $\beta$ nel piano complesso.}
\end{figure}

Come si osserva $\alpha$ cade in un quadrato (compreso il bordo) e del quale $q\beta$ è il multiplo di $\beta$ più vicino ad $\alpha$, pertanto, scelto $q\beta$ abbiamo:
\[ r = \alpha - q\beta
    \]
con $\displaystyle N(r) < \frac{1}{2} \; \text{diagonale quadrato} < \text{lato} = N(\beta)$.

\begin{proposition}[Algoritmo di Euclide]
    Dato un dominio euclideo $A$, $\forall a,b \in A$ non entrambi nulli esiste l'MCD $(a,b)$, determinato mediante l'algoritmo di Euclide.
\end{proposition}

\begin{proof}
    Poiché in un dominio euclideo esiste una funzione grado (che è appunto il caso generale del valore assoluto) ed è possibile la divisione euclidea con resto per definizione,
    allora vale l'Algoritmo di Euclide con la stessa dimostrazione già vista per $\ZZ$, e con in questo caso l'applicazione grado generica al posto del valore assoluto:
    \begin{itemize}
        \item \underline{\textbf{L'algoritmo termina}}: L'algoritmo termina perché la successione dei resti $r_n$ è una successione a termini
         in $\NN$ strettamente decrescente, pertanto è definitivamente costante (in questo caso 0).\\
         \item \underline{\textbf{Correttezza dell'algoritmo}}: Si dimostra l'algoritmo per induzione sul numero degli $N$ passi richiesti.
         Sia $\mathcal{P}(n)$ :"Se l'algoritmo termina in N passi, allora funziona.", per $N = 1$, si ha che:
             \[ a = qb + 0
                 \]
         con $b = r_0$ (ovvero $b \mid a)$, e quindi $(a,b) = r_0$. Supponiamo per ipotesi induttiva che l'Algoritmo di 
         Euclide funzioni per ogni numero di passi $m \leq N - 1$ e proviamo che è vero per $N$. Sia:
             \[ \begin{cases}
                 a = qb + r_1 & 0 \leq r_1 < |b| \\
                 r_0 = q_1r_1 + r_2 & 0 \leq r_2 < r_1 \\
                 \vdots \\
                 r_{n-2} = q_{n-1}r_{n-1} + r_n & 0 \leq r_n < r_{n-1} \\
                 r_{n-1} = q_nr_n + 0
             \end{cases}
                 \]
         possiamo applicare l'algoritmo di Euclide per $(r_0,r_1)$ (in tal modo si ha una sequenza di $N - 1$ passi) ed ottenere
         $r_n = (b, r_1)$:
             \[\begin{cases}
                 r_0 = q_1r_1 + r_2 & 0 \leq r_2 < r_1 \\
                 \vdots \\
                 r_{n-2} = q_{n-1}r_{n-1} + r_n & 0 \leq r_n < r_{n-1} \\
                 r_{n-1} = q_nr_n + 0
             \end{cases}
                 \]
         da cui $r_n = (b,r_1) = (b, a - q_0b) = (a,b)$, dove l'ultima uguaglianza è giustificata dalla proprietà dell'M.C.D.
    \end{itemize}
\end{proof}

\begin{proposition}[Elementi invertibili]
    Dato un dominio euclideo $A$, gli elementi di grado minimo sono gli elementi di $A^*$.
\end{proposition}

\begin{proof}
    Consideriamo $d(A\setminus\{0\}) \subset \NN$, ovvero l'immagine del dominio euclideo mediante l'applicazione grado, tale immagine è non vuota\footnote{Praticamente è ovvio, altrimenti non avrebbe nemmeno senso
    parlare di $A$ come dominio.} ed è un sottoinsieme di $\NN$, pertanto, per il Principio del Minimo ammette elemento minimo $d_0$, sia $x \in A\setminus\{0\}$ un elemento nella controimmagine di $d_0$, quindi
    tale che $d(x) = d_0$, tale elemento è quindi di grado minimo, vediamo che è invertibile. Sappiamo che $\forall y \in A\setminus\{0\}$ possiamo fare la divisione euclidea per $x$:
    \[ y = qx + r
        \]
    ed in questo caso non può essere che $d(r) < d(x) = d_0$, in quanto abbiamo supposto $d_0$ minimo, dunque l'unica possibilità è che $r = 0$ e $y = qx$, in particolare, per $y = 1$, $\exists q \in A$ tale che:
    \[ 1 = qx \implies x \in A^*
        \]
    Viceversa, sia $x \in A^*$, allora $(x) = A$ (poiché in $(x)$ vi è 1), pertanto, $\forall a \in A$:
    \[ a = qx \qquad q \in A
        \]
    inoltre, sappiamo per le proprietà del grado che $d(x) \leq d(qx)$, $\forall q \in A\setminus\{0\}$, ovvero:
    \[ d(x) \leq d(a) \qquad \forall a \in A
        \]
    quindi $x$ è un elemento di grado minimo.
\end{proof}

\begin{proposition}[Ideali di un dominio euclideo]
    \label{2.93}
    Dato un dominio euclideo $A$, tutti gli ideali di $A$ sono principali\footnote{Quindi ogni dominio euclideo è un PID.} e generati da un elemento di grado minimo.
\end{proposition}

\begin{proof}
    Sia $I \subset A$ un ideale, se $I = \{0\}$, allora è principale, altrimenti, preso $I \ne \{0\}$, vogliamo mostrare che $I$ è generato da un singolo elemento di grado minimo.
    Sia $x \in I$ un elemento di grado minimo (esiste perché $d(I) \subset \NN$ e non vuoto), allora è ovvio che $(x) \subseteq I$, viceversa, si ha che $\forall a \in A$ si può fare la divisione euclidea per $x$:
    \[ a = qx + r \qquad d(r) < d(x) \vee r = 0
        \]
    da cui segue che $r = a - qx \in I$ da cui $d(x) \leq d(r) \implies r = 0$, ovvero:
    \[ a = qx \in (a) \implies I \subseteq (a)
        \]
\end{proof}


\newpage
\subsection{Domini a ideali principali (PID)}
\begin{definition}
    Un dominio $A$ si dice a \vocab{ideali principali} (PID) se ogni ideale di $A$ è principale:
    \[ \forall I \subseteq A, I \, \text{ideale}, \exists x \in I : I = (x)
        \]
\end{definition}

\begin{proposition}[Ideali primi di un PID]
    \label{2.95}
    Dato un dominio ad ideali principali $A$, gli unici ideali primi di $A$ sono $(0)$ e gli ideali massimali.
\end{proposition}

\begin{proof}
    Dall'(1) del \hyperref[2.57]{Corollario 2.57} sappiamo che il fatto che $A$ sia un dominio è equivalente al fatto che $(0)$ sia un ideale
    primo, inoltre, dal (3) del \hyperref[2.57]{Corollario 2.57} sappiamo anche che tutti gli ideali massimali di un anello sono primi.\\
    Viceversa, vogliamo dimostrare che gli ideali primi diversi da $(0)$ di un dominio a ideali principali sono solo quelli massimali; sia $P$ un ideale primo diverso da $(0)$, essendo in un PID si ha che
    $P = (x)$, verifichiamo che $P$ è massimale osservando che, essendo $A$ un dominio, per l'(1) della \hyperref[2.86]{Proposizione 2.86}, $P$ è primo se e solo se $x$ è primo, e dunque $x$ irriducibile (è vero in ogni dominio per la \hyperref[2.58]{Proposizione 2.58}).
    Essendo $x$ irriducibile segue che $(x)$ è massimale tra gli ideali principali di $A$ (per il (2) della \hyperref[2.86]{Proposizione 2.86}), e poiché $A$ è un PID (dunque tutti gli ideali sono principali), allora $(x)$ è un ideale massimale per $A$. 
\end{proof}

\begin{remark}[M.C.D. nei PID]
    Se $A$ è un PID e $x,y \in A$, non entrambi nulli, osserviamo che l'ideale generato da $x$ e $y$ deve essere tale che:
    \[ (x,y) = (d)
        \]
    con $d$ un M.C.D.\footnote{M.C.D. a meno di prodotto per elementi di $A^*$.} di $x$ e $y$. Infatti:
    \[ x \in (d) \qquad \text e \qquad y \in (d)
        \]
    dunque $d \mid x$ e $d \mid y$, inoltre, se $c \mid x$ e $c \mid y$, allora $x,y \in (x)$, da cui:
    \[ (d) = (x,y) \subseteq (c) \implies d \in (c) \implies c \mid d
        \]
    dunque $d$ è un M.C.D. tra $x$ ed $y$.
\end{remark}


\newpage
\subsection{Domini a fattorizzazione unica (UFD)}
\begin{definition}
    Dato $A$ un dominio, esso si dice \vocab{a fattorizzazione unica} (UFD) se $\forall x \in A$, $x \not\in A^*\setminus\{0\}$ si scrive in modo
    unico, a meno dell'ordine di fattori e di moltiplicazione per elementi invertibili, come prodotto di elementi irriducibili.
\end{definition}

\begin{example}
    [Esempi di UFD]
    Esempi noti di UFD sono il dominio degli interi $\ZZ$, e quello dei polinomi in un'ordinata a coefficienti in un campo $K[x]$.
\end{example}

\begin{proposition}[UFD$\implies\exists$M.C.D.]
    Sia $A$ un dominio a fattorizzazione unica, allora presi $a,b \in A$, non entrambi nulli, esiste $\text{M.C.D.}(a,b)$.
\end{proposition}

\begin{proof}
    Sia $d$ il prodotto dei fattori irriducibili comuni fra $a$ e $b$, presi con il minimo esponente con cui compaiono,
    allora per verifica diretta si mostra che $d$ è l'M.C.D.\footnote{Si tratta praticamente solo di verifiche formali, pertanto conto di aggiungerla in seguito.}
\end{proof}

\begin{remark}
    Osserviamo che nei tre tipi di domini analizzati esiste sempre l'M.C.D., ma con delle differenze:
    \begin{itemize}
        \item Se $A$ è un dominio euclideo (ED): con l'algoritmo di Euclide si può determinare l'M.C.D. di due elementi:
        \[ d = (a,b) \qquad a,b \in A
            \]
        e i coefficienti $x_0,y_0 \in A$ per i quali vale l'identità di Bézout:
        \[ d = ax_0 + by_0
            \]
        \item Se $A$ è un dominio a ideali principali (PID): sappiamo che presi due elementi si ha:
        \[ (a,b) = (d) \qquad a,b,d \in A
            \]
        con $d$ che è l'M.C.D. tra $a$ e $b$, ma non disponiamo di un algoritmo "facile" per poterlo determinare, tuttavia esistono $x_0, y_0 \in A$ tali che:
        \[ d = ax_0 + by_0
            \]
        e ciò deriva semplicemente dal fatto che $d \in (a,b)$, dunque si può scrivere come loro combinazione lineare (ugualmente però non disponiamo di algoritmi "facili" per poter determinare $x_0$ e $y_0$).
        \item Se $A$ è un dominio a fattorizzazione unica (UFD): allora $\forall a,b \in A$ non entrambi nulli, esiste $d = \text{M.C.D.}(a,b)$, ma \textbf{non è detto che} $(d) = (a,b)$ e quindi neppure che $d = ax_0 + by_0$.
        Infatti, sicuramente è vero che $(a,b) \subset (d)$, ma non è detto che valga il contenimento opposto, il quale vale solo se $d = ax_0 + by_0$. Ad esempio, preso l'UFD $\ZZ[x]$, l'ideale:
        \[ I = (2,x)
            \]
        è tale che $\text{M.C.D.}(2,x) = 1$, ma $1 \not \in (2,x)$, perché se non fosse coì avremmo:
        \[ 1 = 2a(x) + xb(x)
            \]
        che per $x = 0$ porta a:
        \[ 1 = 2a(0) + 0 \implies 2 \mid 1
            \]
        che è assurdo.
    \end{itemize}
\end{remark}

\begin{theorem}
    [Caratterizzazione degli UFD]
    \label{2.101}
    Dato un dominio $A$, sono fatti equivalenti:
    \begin{enumerate}[(1)]
        \item $A$ è un UFD.
        \item Valgono le due condizioni seguenti:
        \begin{enumerate}[(i)]
            \item Ogni elemento irriducibile è primo.
            \item Ogni catena discendete di divisibilità è stazionaria, ovvero se $\{a_i\} \subset A$, con:
            \[ a_{i+1} \mid a_i \qquad \forall i \geq 0
                \]
            allora $\exists n_0$ tale che $a_i \sim a_{n_0}$, $\forall i \geq n_0$.
        \end{enumerate}
    \end{enumerate}
\end{theorem}

\begin{remark}
    La condizione (i) permette di dimostrare che la fattorizzazione in primi è unica, dunque è equivalente a questo fatto; mentre, la condizione (ii) 
    permette di dimostrare l'esistenza della fattorizzazione, e quindi vi è equivalente, pertanto il teorema dice che essere un UFD è equivalente al 
    fatto che in un dominio valga il teorema di fattorizzazione unica.
\end{remark}

\begin{remark}
    Osserviamo che la condizione (ii) può essere riformulata equivalentemente come: ogni catena ascendente di ideali principali è stazionaria, ovvero,
    considerata $\{(a_i)\}_{i \geq 0}$, catena  ascendente di ideali di $A$:
    \[ (a_1) \subseteq (a_2) \subseteq \ldots
        \]
    allora $\exists n_0$ tale che $(a_i) = (a_{n_0})$, $\forall i \geq n_0$.\footnote{Tra l'altro ciò, estendendo la condizione di stazionarietà ad ogni catena di ideali (quindi non solo principali), definisce un \vocab{anello noetheriano}.}
\end{remark}

\begin{corollary}
    [A PID$\implies$A UFD]
    Se $A$ è un dominio a ideali principali, allora è anche un dominio a fattorizzazione unica.
\end{corollary}

\begin{proof}
    Per dimostrare il corollario ci basta verificare che per ogni PID valgono le condizioni (i) e (ii) del \hyperref[2.101]{Teorema 2.101}, e ciò è equivalente al dire che ogni PID è un UFD.
    Sia $A$ un PID, e sia $x \in A$ un elemento irriducibile, per quanto visto nella \hyperref[2.95]{Proposizione 2.95} l'ideale $(x)$ è massimale in $A$, ma dalla (3) del \hyperref[2.57]{Corollario 2.57}, sappiamo quindi che 
    $(x)$ è primo, e poiché siamo in un dominio vale la (1) della \hyperref[2.86]{Proposizione 2.86} che ci assicura che $x$ è primo, dunque vale la (i).\\
    Consideriamo ora una catena ascendente di ideali (principali):
    \[ (a_1) \subseteq (a_2) \subseteq \ldots
        \]
    e sia:
    \[ I = \bigcup_{i \geq 0} (a_i)
        \]
    tale unione di ideali in catena è un ideale\footnote{Come già osservato in precedenza ciò andrebbe dimostrato.} di $A$, ed è pertanto principale, dunque $I = (a)$.
    Osserviamo che $a$ appartiene all'unione, per cui $a \in (a_{n_0})$, e quindi:
    \[ I = (a) \subseteq (a_{n_0})
        \]
    ma d'altra parte avevamo detto che $I$ è l'unione di tutti gli ideali principali di $A$, quindi $(a_{n_0}) \subseteq (a) = I $, dunque $I = (a_{n_0})$; a questo punto sappiamo che:
    \[ (a_i) \subset (a_{n_0}) \qquad \forall i \geq 0
        \]
    dove si ha $(a_i) = (a_{n_0})$ quando vi è anche il contenimento opposto, e ciò avviene, poiché stiamo considerando ideali in catena, $\forall i \geq n_0 \implies \{(a_i)\}$ è stazionaria e quindi è verificata la (ii).
\end{proof}

\begin{remark}
    L'ultimo corollario ci permette di poter mettere in relazione i vari tipi di domini d'integrità:
    \[ \underbrace{\text{ED}}_{\text{Domini euclidei}} \subset \underbrace{\text{PID}}_{\text{Domini a ideali principali}} \subset \underbrace{\text{UFD}}_{\text{Domini a fattorizzazione unica}}
        \]
\end{remark}

\begin{example}[Anello senza la (ii)]
    Consideriamo l'estensione $K[\{x^{\frac{1}{n}}\}_{n\geq 1}]$, essa non è un UFD, in quanto non vale la (ii) dell'\hyperref[2.101]{Teorema 2.101}:
        \[ x^{\frac{1}{2^{n+1}}} \mid x^{\frac{1}{2^n}} \mid \ldots \mid x^{\frac{1}{4}} \mid x^{\frac{1}{2}} \mid x
            \]
        dove tale catena non è definitivamente stazionaria (il successivo può sempre dividere il precedente) dunque non esiste la fattorizzazione di $x$.
\end{example}

\begin{example}[Anello senza la (i)]
    Consideriamo l'anello $\ZZ[\sqrt{-5}] = \{a + b\sqrt{-5} | a,b \in \ZZ\} \subset \QQ(\sqrt{-5}) \subset \CC$, tale anello non è un UFD poiché non vale la (i), infatti, ad esempio,
    $2$ è irriducibile ma non primo in $\ZZ[\sqrt{-5}]$:
    \[ 2 = (a + b\sqrt{-5})(c + d\sqrt{-5}) \implies N(2) = (a^2 + 5b^2)(c^2 + 5d^2)
        \]
    da cui possiamo dedurre che le uniche possibilità sono $a = \pm 2, c = \pm 1, b = 0, d = 0$ (oppure invertiti), pertanto 2 è irriducibile, ma non è primo perché:
    \[ 2 \mid 6 = (1 + \sqrt{-5})(1 - \sqrt{-5}) \qquad \text{ma} \qquad 2 \nmid (1 + \sqrt{-5}),(1 - \sqrt{-5})
        \]
    poiché, $\displaystyle\alpha = \frac{1 \pm \sqrt{-5}}{2} = \frac{1}{2} \pm \frac{\sqrt{-5}}{2} \not\in \ZZ[\sqrt{-5}]$. In $\ZZ[\sqrt{-5}]$ abbiamo quindi due fattorizzazioni distinte per 6:
    \[ 6 = 2 \cdot 3 = (1 + \sqrt{-5})(1 - \sqrt{-5})\footnote{Per essere precisi dovremmo verificare che 2,3,$1 \pm \sqrt{-5}$ sono irriducibili in $\ZZ[\sqrt{-5}]$.}
        \]
\end{example}

\begin{remark}
    $\ZZ[\sqrt{-5}]$ non essendo un UFD, non è neppure un PID, infatti, ad esempio, l'ideale $(2,1 + \sqrt{-5})$ non è principale.
\end{remark}

\begin{theorem}[$A$ UFD$\implies A\lbrack x\rbrack$ UFD]
    \label{2.109}
    Se $A$ è un dominio a fattorizzazione unica, allora anche $A[x]$ è un anello a fattorizzazione unica.
\end{theorem}


\begin{corollary}[$A$ UFD$\implies A\lbrack x_1,\ldots,x_n\rbrack$ UFD]
    Se $A$ un dominio a fattorizzazione unica, allora anche $A[x_1,\ldots,x_n]$ è un anello a fattorizzazione unica.
\end{corollary}

\begin{proof}
    Dal \hyperref[2.109]{Teorema 2.109} segue il caso base $A[x]$, per induzione si suppone vera la tesi per $A[x_1,\ldots,x_{n-1}] = B$, e poi si considera $B[x_n]$, che è un UFD per il caso iniziale,
    dunque la tesi è vera per induzione.
\end{proof}

\begin{remark}
    [Schema della dimostrazione del Teorema 2.109]
    La dimostrazione del \hyperref[2.109]{Teorema 2.109} si articola in tre tappe:
    \begin{enumerate}[(1)]
        \item $A[x]$ dominio.
        \item Ogni irriducibile di $A[x]$ è primo (condizione (i)).
        \item Ogni catena discendete di divisibilità è stazionaria (condizione (ii)).
    \end{enumerate}
    Da cui si conclude per il \hyperref[2.101]{Teorema di Caratterizzazione degli UFD}.
\end{remark}

\begin{proof}
    Iniziamo quindi verificando il primo punto come segue.
    \begin{enumerate}[(1)]
        \item Siano $f(x),g(x) \in A[x] \setminus\{0\}$, e $\deg f(x) = n \geq 0$, $\deg g(x) = m \geq 0$, essendo $a_n, b_m \ne 0$, segue che
        $a_nb_m \ne 0$ poiché per ipotesi $A$ un dominio di integrità, pertanto $f(x) \cdot g(x) \ne 0$ se $f(x),g(x) \ne 0 \implies A[x]$ è un dominio d'integrità.
        \begin{remark}
            Osserviamo anche che da ciò discende che $(A[x])^* = A^*$, infatti, considerando $f(x) \in (A[x])^*$, $\exists g(x) \in A[x]$ tale che:
            \[ f(x)g(x) = 1 \implies \deg f(x) + \deg g(x) = 0
                \]
            ovvero $\deg f(x) = \deg g(x) = 0$, per cui $f(x),g(x) \in A$, ed in particolare $f(x) = a$, $g(x) = b$, pertanto:
            \[ ab = 1 \implies f(x) \in A^*
                \]
        \end{remark}
    \end{enumerate}
\end{proof}

Per verificare la condizione (i) dobbiamo prima caratterizzare gli irriducibili di $A[x]$, e fare ciò abbiamo bisogno del Lemma di Gauss.

\begin{definition}
    Dato $A$ un UFD e $f(x) \in A[x]$, con $f(x) = \sum_{i=0}^n a_ix^i$, si dice \vocab{contenuto} di $f(x)$ l'M.C.D. dei suoi coefficienti:
    \[ c(f(x)) = (a_0,\ldots,a_n)
        \]
\end{definition}

\begin{remark}
    Il contenuto di un polinomio a coefficienti in un UFD è definito a meno di associati.
\end{remark}

\begin{definition}
    Dato $A$ un UFD e $f(x) \in A[x]$, $f(x)$ si dice \vocab{primitivo} se $c(f(x)) \sim 1$.
\end{definition}

\begin{remark}
    Ovviamente dato $f(x) \in A[x]$ si ha che:
    \[ f(x) = c(f(x))f^{\prime}(x) \qquad c(f^{\prime}(x)) = 1
        \]
    dove $f^{\prime}(x) \in A[x]$ e:
    \[ f^{\prime}(x) = \sum_{i=0}^n \frac{a_i}{d}x^i \qquad \frac{a_i}{d} \in A, \left(\frac{a_0}{d},\ldots,\frac{a_n}{d}\right) = 1
        \]
\end{remark}

\begin{lemma}
    [Lemma di Gauss]
    \label{gauss}
    Dati $f(x),g(x) \in A[x]$, allora:
    \[ c(f(x)g(x)) = c(f(x))c(g(x))
        \]
\end{lemma}

\begin{proof}
    Distinguiamo due casi:
    \begin{itemize}
        \item Se $c(f(x)) = c(g(x)) = 1$, dunque $f(x)$ e $g(x)$ sono primitivi, vogliamo verificare che $c(f(x)g(x)) = 1$; se $f(x)g(x)$ non fosse primitivo (ovvero associato ad 1)
            allora $c(f(x)g(x))$ non sarebbe invertibile (gli elementi invertibili sono associati ad 1), ovvero esisterebbe $p$ primo tale che $p \mid c(f(x)g(x))$, consideriamo la proiezione modulo $(p)$:
            \[ \pi_{(p)} : A[x] \longrightarrow \frac{A}{(p)}[x] : f(x) \longmapsto \overline{f(x)}
                \]
            con $\overline{f(x)} \ne 0$ in quanto $p \nmid c(f(x))$, analogamente $\pi_{(p)}(g(x)) = \overline{g(x)} \ne 0$, poiché $p \nmid c(g(x))$, ma $\pi_{(p)}(f(x)g(x)) = \overline{f(x)}\,\overline{g(x)} = 0$, 
            perché avevamo supposto che $p \mid c(f(x)g(x))$, ma questo è assurdo in quando $\displaystyle \frac{A}{(p)}$ è un dominio e quindi, per quanto detto, $A$ dominio$\implies A[x]$ dominio, ovvero $\displaystyle \frac{A}{(p)} [x]$
            è un dominio, da cui $c(f(x)g(x)) = 1$.
        \item Consideriamo ora il caso generale, sia $f(x) = c(f(x))f^{\prime}(x)$, con $f^{\prime}(x)$ primitivo, e analogamente $g(x) = c(g(x))g^{\prime}(x)$, con $g^{\prime}(x)$ primitivo, abbiamo che:
            \[ h(x) = f(x)g(x) = c(f(x))c(g(x))g^{\prime}(x)f^{\prime}(x)
                \]
            dove $h^{\prime}(x) = g^{\prime}(x)f^{\prime}(x)$ è primitivo perché prodotto di polinomi primitivi, dunque:
            \[ h(x) = c(h(x))h^{\prime}(x)
                \]
            e uguagliando i contenuti si ha:
            \begin{multline*}
                c(h(x))\underbrace{c(h^{\prime}(x))}_{= 1} = c(f(x))c(g(x))\underbrace{c(g^{\prime}(x)f^{\prime}(x))}_{= 1} \implies \\
                c(h(x)) = c(f(x)g(x)) = c(f(x))c(g(x))
            \end{multline*}
    \end{itemize}
\end{proof}

\begin{corollary}
    \label{2.118}
    Siano $f(x),g(x) \in A[x]$, con $c(f(x)) = 1$ e $f(x) \mid g(x)$ in $K[x]$, con $K$ campo dei quozienti di $A$, allora $f(x) \mid g(x)$ in $A[x]$.
\end{corollary}

\begin{proof}
    Per ipotesi sappiamo che $f(x) \mid g(x)$ in $K[x]$, ovvero $\exists h(x) \in K[x]$ tale che $g(x) = f(x)h(x)$, allora $\exists d \in A$ tale che:
    \[ h_1(x) = dh(x) \in A[x]
        \]
    (stiamo "cancellando" il denominatore), dunque:
    \[ dg(x) = f(x)h_1(x) \in A[x]
        \]
    da cui per il \hyperref[gauss]{Lemma di Gauss}:
    \[ dc(g(x)) = c(f(x)h_1(x)) = c(f(x))c(h_1(x)) = c(h_1(x)) \implies d \mid c(h_1(x))
        \]
    dove abbiamo usato nell'ultimo passaggio il fatto che $c(f(x)) = 1$, dunque abbiamo $\displaystyle\frac{h_1(x)}{d} = h(x) \in A[x]$
    e quindi la divisibilità iniziale era anche in $A[x]$. 
\end{proof}

\begin{corollary}
    \label{2.119}
    Sia $f(x) \in A[x]$ e $f(x) = g(x)h(x)$ in $K[x]$ (con $K$ campo dei quozienti di $A$), con $\deg f(x),\deg g(x) \geq 1$,
    allora esiste $\delta \in K^*$ tale che $g_1(x) = \delta g(x) \in A[x]$, $h_1(x) = \delta^{-1}h(x) \in A[x]$ e $f(x) = g_1(x)h_1(x)$.
\end{corollary}

\begin{proof}
    Analogamente a quanto fatto per il corollario precedente, sappiamo che $\exists d \in A$ tale che $g_1(x) = dg(x) \in A[x]$ (stiamo di nuovo "eliminando" i denominatori, ad esempio moltiplicando per l'm.c.m.),
    dunque:
    \[ f(x) = dg(x)d^{-1}h(x) = g_1(x)(d^{-1}h(x)) = c(g_1(x))g_1^{\prime}(x)(d^{-1}h(x))
        \]
    con $g_1^{\prime}(x) \in A[x]$ primitivo (dividendo per il contenuto, che è un invertibile di $A$, siamo rimasti in $A[x]$), pertanto abbiamo:
    \[ f(x) = g_1^{\prime}(x)(\underbrace{c(g_1(x))d^{-1}h(x)}_{\in K[x]})
        \]
    ovvero $g_1^{\prime}(x) \mid f(x)$ in $K[x]$, ma allora per il \hyperref[2.118]{Corollario 2.118} segue che $g_1^{\prime}(x) \mid f(x)$ in $A[x]$, dunque:
    \[ h_1(x) = \underbrace{\frac{c(g_1(x))}{d}}_{= \delta^{-1}}h(x)
        \]
    abbiamo quindi determinato $\delta$ e $\delta^{-1}$ richiesti dalla tesi.
\end{proof}

\begin{remark}
    Il teorema non ci dice altro che se $f(x)$ è riducibile in $K[x]$, allora è riducibile anche in $A[x]$, con polinomi dello stesso grado (per la precisione associati a quelli iniziali).\footnote{In
    \href{https://github.com/diego-unipi/Appunti-Aritmetica}{\textcolor{purple}{Aritmetica}}, avevamo trattato il caso particolare del Lemma di Gauss applicato a $\QQ[x]$ e $\ZZ[x]$.}
\end{remark}

\begin{example}
    Un esempio del discorso appena fatto è rappresentato dalla fattorizzazione di un polinomio in $\QQ[x]$ e $\ZZ[x]$:
    \[ (x^2 - 1) = \left(\frac{100}{7}x+\frac{100}{7}\right)\left(\frac{7}{100}x - \frac{7}{100}\right) = (x + 1)(x - 1)
        \]
\end{example}

\begin{theorem}
    [Caratterizzazione degli irriducibili di $A\lbrack x\rbrack$]
    Dato $A$ UFD, gli elementi irriducibili di $A[x]$ sono tutti e soli quelli che soddisfano una tra le seguenti:
    \begin{enumerate}[(1)]
        \item $f(x) \in A$ irriducibile in $A$.
        \item $f(x) \in A[x]$, con $\deg f(x) \geq 1$, $c(f(x)) = 1$ e $f(x)$ irriducibile in $K[x]$ (anello dei polinomi a coefficienti nel campo dei quozienti di $A$).
    \end{enumerate}
\end{theorem}

\begin{proof}
    Distinguiamo i due casi:
    \begin{enumerate}
        \item Se $f(x) \in A$, dunque è costante, allora, come già osservato in precedenza, si ha:
        \[ f(x) = g(x)h(x) \implies \deg g(x) + \deg h(x) = \deg f(x) = 0
            \]
        dove l'implicazione è data dal fatto che siamo in un dominio, dunque segue che $\deg g(x) = \deg h(x) =  0$, pertanto $g(x),h(x) \in A$,
        per cui $f(x)$ è irriducibile in $A[x]$ se e sole $f(x)$ è irriducibile in $A$ (che è la stessa cosa che avevamo già osservato dicendo che $(A[x])^* = A^*$).
        \item Sia $f(x)$ con $\deg f(x) \geq 1$. Supponiamo che $f(x)$ sia irriducibile in $A[x]$, abbiamo che:
        \[ f(x) = c(f(x))f^{\prime}(x)
            \]
        con $c(f(x))$ invertibile in $A[x]$, $c(f(x)) \in (A[x])^* = A^*$ (per quanto detto al punto (1)), d'altra parte, sia $f(x) = g(x)h(x)$ in $K[x]$, allora per il
        \hyperref[2.119]{Corollario 2.119}, possiamo scriverlo come prodotto di polinomi dello stesso grado in $A[x]$:
        \[ f(x) = g_1(x)h_1(x) \qquad \text{con} \qquad \deg g_1(x) = \deg g(x), \deg h_1(x) = \deg h(x)
            \]
        poiché $f(x)$ è irriducibile in $A[x]$ deve essere che $g_1(x)$ o $h_1(x)$ sono invertibili. Abbiamo quindi che $\deg g_1(x) = 0$ o $\deg h_1(x) = 0$, da cui
        $\deg g(x) = 0$ o $\deg h(x) = 0$, ovvero $g(x) \in (K[x])^*$ o $h(x) \in (K[x])^*$, dunque $f(x)$ è irriducibile in $K[x]$. \\
        Verifichiamo il viceversa, sia $f(x)$ primitivo ed irriducibile in $K[x]$, e sia $f(x) = g(x)h(x)$ in $A[x]$ (e anche in $K[x]$), poiché $f(x)$ è irriducibile in $K[x]$ 
        $g(x)$ o $h(x)$ sono invertibili in $K[x]$ e quindi costanti, supponiamo quindi ad esempio che sia $g(x) \in A$, da ciò segue che:
        \[ 1 = c(f(x)) = c(g(x)h(x)) = c(g(x))c(h(x)) = g c(h(x))
            \]
        dove nell'ultima uguaglianza abbiamo usato il fatto che, essendo $g(x)$ costante, allora è uguale al suo contenuto, dunque $g \in A^* (= (A[x])^*)$, pertanto $f(x)$ è irriducibile in $A[x]$.
    \end{enumerate}
\end{proof}

\begin{proposition}[Condizione (i) per $A\lbrack x\rbrack$]
    Dato $A$ UFD, in $A[x]$ ogni irriducibile è primo.
\end{proposition}

\begin{proof}
    Sia $f(x) \in A[x]$ irriducibile, per dimostrare che è primo bisogna verificare che $\forall g(x),h(x) \in A[x]$ si ha che:
    \[ f(x) \mid g(x)h(x) \implies f(x) \mid g(x) \quad \text o \quad f(x) \mid h(x) \quad \text{in $A[x]$}
        \]
    Distinguiamo due casi:
    \begin{itemize}
        \item Se $\deg f(x) = 0$, ovvero $f(x) = f \in A$, dunque se $f$ è irriducibile in $A$, essendo $A$ UFD, allora $f$ è primo in $A$;
        infatti abbiamo che:
        \[ f \mid gh \implies f = c(f) \mid c(gh) = c(g)c(h) \implies f \mid c(g) = g \quad \text o \quad f \mid c(h) = h
            \]
        dunque $f$ primo.
        \item Sia $f(x)$ primitivo e irriducibile in $K[x]$, con $\deg f(x) \geq 1$, si osserva che $K[x]$ è euclideo, dunque $f(x)$ è primo
        in $K[x]$ (poiché avevamo detto che ED $\subset$ UFD), dunque se:
        \[ f(x) \mid g(x)h(x) \qquad \text{in $A[x]$}
            \]
        allora $f(x) \mid g(x)$ o $f(x) \mid h(x)$ in $K[x]$. Avendo supposto che $f(x)$ è primitivo, allora per il \hyperref[2.118]{Corollario 2.118}:
        \[ f(x) \mid g(x)h(x) \qquad \text{in $A[x]$}
            \]
    \end{itemize}
\end{proof}

\begin{proposition}[Condizione (ii) per $A\lbrack x\rbrack$]
    Sia $\{f_n(x)\}$ una successione di elementi di $A[x]$ tale che:
    \[ f_{i+1}(x) \mid f_i(x)
        \]
    allora è stazionaria, ovvero $\exists n_0$ tale che $f_i(x) \sim f_{n_0}(x)$, $\forall i \geq n_0$.
\end{proposition}

\begin{proof}
    Si osserva che per il \hyperref[gauss]{Lemma di Gauss} si ha che:
    \[ f(x) \mid g(x) \implies c(f(x)) \mid c(g(x)) \qquad \text e \qquad f^{\prime}(x) \mid g^{\prime}(x)
        \]
    infatti, se $g(x) = f(x) h(x) \implies c(g(x))g^{\prime}(x) = c(f(x))f^{\prime}(x)c(h(x))c^{\prime}(x)$,
    ma per quanto detto $c(g(x)) = c(f(x)h(x)) = c(f(x))c(hc(x))$, da cui $f^{\prime}(x) \mid g^{\prime}(x)$.
    Alla successione $\{f_i(x)\}$ possiamo quindi associare le successioni $\{c(f_i(x))\}$ e $\{f_i^{\prime}(x)\}$, per quanto abbiamo detto si ha:
    \[ c(f_{i+1}(x)) \mid c(f_{i}(x)) \qquad \text e \qquad f_{i+1}^{\prime}(x) \mid f_i^{\prime}(x) \qquad \forall i \geq 0
        \]
    dove la successione $\{c(f_i(x))\}$ è stazionaria, in quanto è una catena discendete di divisibilità dell'UFD $A$, dunque:
    \[ \exists m_0 : c(f_i(x)) \sim c(f_{m_0}(x)) \qquad \forall i \geq m_0
        \]
    Consideriamo ora $\{f_i^{\prime}(x)\}$ e associamo a questa la successione $\{\deg f_i^{\prime}(x)\}$, ma dalla condizione:
    \[ f_{i+1}^{\prime}(x) \mid f_i^{\prime}(x) \implies \deg f_{i+1}^{\prime}(x) \leq \deg f_i^{\prime}(x)
        \]
    dunque $\{\deg f_i^{\prime}(x)\}$ è una successione di numeri naturali decrescente, pertanto si stabilizza:
    \[ \exists d_0 : \deg f_{i}^{\prime}(x) \leq \deg f_{d_0}^{\prime}(x) \qquad \forall i \geq d_0
        \]
    pertanto $\forall i \geq d_0$ abbiamo che $f_i^{\prime}(x)$ e $\deg f_{i+1}^{\prime}(x)$ hanno lo stesso grado e $f_{i}^{\prime}(x) \mid f_{d_0}^{\prime}(x)$,
    cioè differiscono per una costante, ma essendo entrambi primitivi la costante deve essere un'unità, per cui:
    \[ f_{i}^{\prime}(x) \sim f_{d_0}^{\prime}(x) \qquad \forall i \geq d_0
        \]
    dunque, detto $n_0 = \max\{m_0,d_0\}$, $\forall i \geq n_0$ vale contemporaneamente che:
    \[ c(f_i(x)) \sim c(f_{m_0}(x)) \qquad \text e \qquad f_{i}^{\prime}(x) \sim f_{d_0}^{\prime}(x)
        \]
    da cui la tesi:
    \[ f_i(x) = c(f_i(x))f_{i}^{\prime}(x) \sim c(f_{n_0}(x))f_{n_0}^{\prime}(x) \qquad \forall i \geq n_0
        \]
\end{proof}

Con la dimostrazione di quest'ultima proposizione abbiamo concluso la dimostrazione del \hyperref[2.109]{Teorema di Caratterizzazione degli UFD}.

\begin{remark}
    Osserviamo che in generale se $A$ è un PID, allora non è sempre vero che $A[x]$ sia un PID, ad esempio nel caso di $\ZZ$ sappiamo che $\ZZ[x]$ non è un PID, in quanto,
    ad esempio, l'ideale $I = (2,x)$ non è principale. Analogamente, in generale se $A$ è un ED, allora non è sempre vero che $A[x]$ sia un ED, anche qui come controesempio possiamo 
    considerare il caso di $\ZZ$ e $\ZZ[x]$.
\end{remark}

\begin{remark}
    Se $K$ è un campo, allora abbiamo che $K[x]$ è euclideo, mentre $K[x,y]$ è un UFD (poiché vale sempre il \hyperref[2.109]{Teorema di Caratterizzazione degli UFD}, e $K[x]$ è un UFD), ma
    $K[x,y]$ non è un PID, in quanto ad esempio $I = (x,y)$ non è principale.
\end{remark}

\begin{exercise}
    Dimostrare che dato $K[x,y]$ campo, l'ideale $I = (x,y)$ non è principale.
\end{exercise}

\begin{soln}
    
\end{soln}

\begin{proposition}
[Criterio di irriducibilità Eisenstein]
\label{eisenstein}
Dato $A$ UFD e $f(x) \in A[x]$ primitivo, con $f(x) = \sum_{i=0}^n a_ix^i$, e $p \in A$ un primo tale che:
\begin{enumerate}[(1)]
    \item $p \nmid a_n$.
    \item $p \mid a_i$, $\forall i \in \{0,\ldots,n-1\}$.
    \item $p^2 \nmid a_0$.
\end{enumerate}
Allora $f(x)$ è irriducibile in $A[x]$ (e in $K[x]$, con $K$ campo dei quozienti di $A$).
\end{proposition}

\begin{proof}
    La dimostrazione è identica a quella già trattata in \href{https://github.com/diego-unipi/Appunti-Aritmetica}{\textcolor{purple}{Aritmetica}}, con 
    la sola differenza che in questo caso utilizziamo un UFD generico al posto di $\ZZ$.\\
    Supponiamo per assurdo che $f(x)$, con $\deg(f(x)) = n$ sia riducibile, allora:
	\[ f(x) = g(x)h(x)
	\]
    con $\deg f(x) = m \geq 1$, $\deg(h(x)) = n-m \geq 1$. Possiamo applicare la proiezione al quoziente modulo $(p)$, e per le ipotesi si ha:
	\[ \overline{f(x)} = \overline{a_n}x^n (\ne \overline 0)
	\] 
    e inoltre:
	\[ \pi_{(p)}(f(x)) =  \pi_{(p)}(g(x)) \pi_{(p)}(h(x))
	\]
    con:
	\[ \pi_{(p)}(g(x)) = \overline{b_m}x^m + \ldots + \overline{b_0}
	\qquad
	\text e
	\qquad
	\pi_{(p)}(h(x)) = \overline{c_{n-m}}x^{n-m} + \ldots + \overline{c_0}
	\]
    Si ha che $\displaystyle\pi_{(p)}(f(x)) \in \frac{A}{(p)}[x]$ (che è un campo), quindi UFD, da ciò segue che $\pi_{(p)}(g(x))$ e 
    $\pi_{(p)}(h(x))$ sono fattori del polinomio $\overline{a_n}x^n$, da cui segue, per la definizione di prodotto tra polinomi che:
	\[  \pi_{(p)}(g(x)) = \overline{b_m}x^m
	\qquad
	\text e
	\qquad
	\pi_{(p)}(h(x)) = \overline{c_{n-m}}x^{n-m}
	\]
    ma da ciò segue che $b_0 \equiv c_0 \equiv 0 \pmod p$, da cui $a_0 \equiv b_0c_0 \equiv 0 \pmod{p^2}$, ovvero $p^2 \mid a_0$, ma ciò è assurdo.
\end{proof}

\newpage
\subsection{Terne pitagoriche}
\begin{definition}
    Si definiscono \vocab{terne pitagoriche} le terne di soluzioni intere dell'equazione:
    \[ x^2 + y^2 = z^2 \qquad x,y,z \in \ZZ
        \]
    con  $(x,y,z) = 1$.
\end{definition}

Osserviamo che possiamo riformulare il problema negli interi di Gauss nel modo che segue:
\[ x^2+y^2 = (x+iy)(x-iy) = z^2
    \]
dunque determinare le terne pitagoriche significa risolvere questo problema moltiplicativo in $\ZZ[i]$.

\begin{remark}
    Osserviamo che la coprimalità dei tre fattori la implica anche a coppie: $(x,y,z) = 1 \implies (x,y) = (x,z) = (y,z) = 1$.
\end{remark}

\begin{remark}
    Abbiamo che $x \not\equiv y \pmod 2$, infatti, studiando l'equazione modulo 4:
    \[ x^2 + y^2 \equiv z^2 \pmod 4
        \]
    dove essendo un quadrato modulo 4 abbiamo che $z^2 \in \{0,1\}$, dunque $x^2,y^2 \in \{0,1\}$, ma non possono essere contemporaneamente pari,
    altrimenti avremmo che $(x,y) \ne 1$, e neppure contemporaneamente dispari, altrimenti la somma dei loro quadrati modulo 4 sarebbe 2,
    pertanto $x$ ed $y$ sono uno pari e l'altro dispari.
\end{remark}

\begin{remark}
    Consideriamo l'ideale $I = (x+iy,x-iy)$, verifichiamo che $I = (1)$, o equivalentemente che $x+iy$ e $x-iy$ sono coprimi 
    (abbiamo visto che in un ED l'ideale generato da due elementi è quello generato dal loro M.C.D.). Osserviamo che:
    \[ 2x = (x+iy) + (x-iy) \in I
        \]
    Analogamente, considerando la differenza si ha $2y \in y$ e, considerando il prodotto, $x^2+y^2 \in I$; poiché $x$ ed $y$ hanno
    diversa parità, $x^2+y^2$ è dispari, mentre $2x$ è pari, dunque si ha che:
    \[ 1 \in (2x,2y,x^2+y^2) \subset \ZZ \subset \ZZ[i]
        \]
    infatti, se non ci fosse 1, $\exists p$ tale che $p \mid 2x$, $p \mid 2y$, da cui $p = 2$ (in quanto $(x,y) = 1$), ma $2 = p \mid x^2+y^2$,
    che però è dispari, dunque:
    \[ I = (1)
        \]
    o equivalentemente $x+iy$ e $x-iy$ sono coprimi e quindi entrambi dei quadrati a meno di unità.
\end{remark}

\pagebreak
\begin{remark}
    [Invertibili di $\ZZ\lbrack i \rbrack$]
    Osserviamo che gli elementi di $(\ZZ[i])^*$ sono gli $u \in \ZZ[i]$ tali per cui $\exists v \in \ZZ[i]$:
    \[ uv = 1\footnote{Stiamo usando il fatto che nei complessi il prodotto fa 1 se è tra elementi coniugati.} \implies v = \overline u \implies u\overline u = 1
        \]
    ovvero $u = \varepsilon + i\delta \in \ZZ[i]$ dove:
    \[ u \overline u = \varepsilon^2 + \delta^2 = 1
        \]
    che ha per soluzioni:
    \[ \begin{cases}
        \varepsilon = \pm 1\\
        \delta = 0
    \end{cases}
    \qquad \text e \qquad
    \begin{cases}
        \varepsilon = 0\\
        \delta = \pm 1
    \end{cases}
        \]
    dunque $(\ZZ[i])^* = \{\pm 1, \pm i\}$.
\end{remark}

Allora, per quanto detto sappiamo che:
\[ x + iy = u\alpha^2 \qquad \alpha \in \ZZ[i], u \in \{\pm 1, \pm i\}
\]
e analogamente:
\[ x - iy = \overline u \,\overline\alpha^2 \qquad \overline\alpha \in \ZZ[i], \overline u \in \{\pm 1, \pm i\}
    \]
Considerando $\alpha = a+ib$ si ottiene:
\[ x + iy = u(a^2-b^2-2iab)
    \]
distinguiamo ora due casi:
\begin{itemize}
    \item Se $u = \pm 1$, allora si ottiene:
    \[ \begin{cases}
        x = \pm(a^2-b^2) \\
        y = \mp 2ab \\
        z = \pm (a^2+b^2)
    \end{cases}
        \]
    \item Se $u = \pm i$, allora si ottiene:
    \[ \begin{cases}
        x = \mp 2ab \\
        z = \pm(a^2-b^2-2) \\
        y = \pm (a^2+b^2)
    \end{cases}
    \]
\end{itemize}
che forniscono la parametrizzazione delle terne pitagoriche e rispondono al problema iniziale di determinarle.

\newpage
\begin{remark}[Ultimo teorema di Fermat]
    Il procedimento risolutivo appena utilizzato nel caso delle terne pitagoriche non può essere generalizzato al caso dell'equazione:
    \[ x^n + y^n = z^n \qquad n\geq 3
        \]
    inizialmente, possiamo considerare solo il caso in cui $n = p$, infatti, se non ci fossero soluzioni nemmeno nel caso di un primo, non potremmo
    trovarle nel caso generale. Distinguiamo due casi:
    \begin{itemize}
        \item Se $p \nmid xyz$, abbiamo $x^p+y^p = z^p$ che può essere fattorizzata:
            \[ (x+y)(x+\zeta_py) \ldots (x + \zeta_p^{p-1}y) = z^p \qquad \text{in $\ZZ[\zeta_p]$}
                \]
            si può dimostrare in questo caso che i fattori sono a due a due coprimi,
            ma da ciò non possiamo concludere che sono tutte potenze $p$-esime (altrimenti a questo punto potremmo dimostrare che non esistono soluzioni),
            perché in generale $\ZZ[\zeta_p]$ \textbf{non} è un UFD, ad esempio per $p = 23$ non è vero\footnote{Ed anzi è proprio il più piccolo primo per il quale non funziona.}.
        \item Se $p$ divide esattamente uno tra $x$,$y$ e $z$, ma non discutiamo ora questo caso.
    \end{itemize}
\end{remark}

\newpage
\section{Campi}
\subsection{Riepilogo sulle estensioni di campi}
\begin{definition}
    Dato un campo $K$ ed una sua estensione $L$, $\alpha \in L$ si dice \vocab{algebrico} su $K$ se:
        \[ \exists f(x) \in K[x]\setminus\{0\} : f(\alpha) = 0
        \] 
\end{definition}

\begin{definition}
    Dato un campo $K$ ed una sua estensione $L$, $\alpha \in L$ si dice \vocab{trascendente} su $K$ se non è algebrico, ovvero:
        \[ \not\exists f(x) \in K[x]\setminus\{0\} : f(\alpha) = 0
        \] 
\end{definition}

\begin{definition}
    Dato un campo $K$, una sua estensione $L$, e $\alpha \in L$, possiamo definire l'\vocab{omomorfismo di valutazione} di $\alpha$ su $K[x]$ come:
    \[ \varphi_{\alpha} : K[x] \longrightarrow K[\alpha] (\subset L) : f(x) \longmapsto f(\alpha)
        \]
\end{definition}

Per tale omomorfismo vale il diagramma:
\begin{center}
	\begin{tikzcd}
		 K[x] \ar[d, "\pi_{\ker{\varphi_{\alpha}}}" left, twoheadrightarrow] \ar[r, "\varphi_\alpha", twoheadrightarrow] & K[\alpha](\subset L) \\
		\displaystyle\frac{K[x]}{\ker{\varphi_\alpha}} \ar[ur,  "\overline{\varphi_{\alpha}}"' rotate = -29, "\widesim{}"' above, sloped, anchor=center] & 
	\end{tikzcd}
\end{center}

Da cui abbiamo che:
\[ K[\alpha] \cong \frac{K[x]}{\ker \varphi_{\alpha}}
    \]
dove $K[\alpha]$ è un dominio perché è un sottoanello di un campo, dunque $\ker \varphi_{\alpha} \subset K[x]$ è un ideale primo (per l'(1) della \hyperref[2.56]{Proposizione 2.56}).

\begin{remark}
    Ricordiamo che, data un'estensione $K \subset L$, si ha per $\alpha \in L$ che:
	\begin{itemize}
	\ii $\alpha$ è trascendente su $K$ $\iff \ker{\varphi_\alpha} = \{0\} \iff \varphi_\alpha$ è iniettivo $\iff K[x] \cong K[\alpha]$.
	\ii $\alpha$ è algebrico su $K$ $\iff \ker{\varphi_\alpha} \ne \{0\} \iff \varphi_\alpha$ non è iniettivo.
	\end{itemize}
\end{remark}

Se consideriamo $\alpha \in L$ algebrico su $K[x]$, dunque $\ker \varphi_{\alpha} \ne \{0\}$, poiché $K[x]$ è un PID, allora $\ker \varphi_{\alpha}$ è un ideale massimale di $K[x]$ (poiché 
vale la \hyperref[2.95]{Proposizione 2.95}), dunque $\displaystyle \frac{K[x]}{\ker \varphi_{\alpha}}$ è un campo, e lo è quindi $K[\alpha]$, di conseguenza, coincide col
suo campo dei quozienti\footnote{Volendo per la \hyperref[2.69]{Proposizione 2.69}.} $K[\alpha] = K(\alpha)$.

\begin{remark}
    Poiché $K[x]$ è un PID, si ha che $\ker \varphi_{\alpha} = (\mu_{\alpha}(x))$, con $\mu_\alpha(x) \in \ker \varphi_\alpha$, ed essendo un ideale massimale $\mu_\alpha(x)$ è irriducibile in $K[x]$ (\hyperref[2.86]{Proposizione 2.86}).
    Inoltre, scegliamo $\mu_\alpha(x)$ come l'unico generatore monico, infatti essendo $K[x]$ anche un ED, per la \hyperref[2.93]{Proposizione 2.93} sappiamo che i suoi ideali sono generati da elementi di grado minimo, e tali elementi differiscono per un elemento di $K \setminus\{0\}$.
\end{remark}

\begin{definition}
    Data l'estensione $\faktor{L}{K}$ ($K \subseteq L$), si dice \vocab{grado} di $\faktor{L}{K}$:
    \[ [L:K] = \dim_K L
        \]
    se $[L:K] < +\infty$, diciamo che $\faktor{L}{K}$ è un'estensione \vocab{finita} (o di \vocab{grado finito}).
\end{definition}

\begin{proposition}[Grado di un'estensione semplice]
    \label{3.7}
    Data l'estensione $\faktor{L}{K}$ e $\alpha \in L$, allora il grado dell'estensione semplice $K \subseteq K(\alpha)$ è dato da:
    \[ [K(\alpha) : K] = \begin{cases}
        +\infty & \text{se $\alpha$ è trascendente su $K$}\\
        \deg \mu_\alpha(x) & \text{se $\alpha$ è algebrico su $K$}
    \end{cases}
        \]
\end{proposition}

\begin{proof}
    Osserviamo che per quanto già detto, $\alpha$ è trascendente su $K$ se e solo $\ker \varphi_\alpha = \{0\}$ (a questo punto abbiamo già concluso la dimostrazione della prima parte), quindi se e solo se l'omomorfismo di valutazione è iniettivo, ed essendo 
    surgettivo per definizione, si ha che:
    \[ K[x] \cong K[\alpha] \iff K(x) \cong K(\alpha)
        \]
    Analogamente, per quanto detto, $\alpha$ algebrico su $K$ è equivalente a dire:
    \[ K(\alpha) \cong K[\alpha] \cong \frac{K[x]}{(\mu_\alpha(x))}
        \]
    con l'isomorfismo che manda la base $\{\overline 1, \overline x, \ldots, {\overline x}^{n-1}\}$ di $\displaystyle \frac{K[x]}{(\mu_\alpha(x))}$, in $\{1,\alpha,\ldots,\alpha^{n-1}\}$, ovvero la $K$-base di $K(\alpha)$, da cui segue 
    che:
    \[ \dim_K K(\alpha) = n = \deg \mu_\alpha(x)
        \]
\end{proof}

\begin{proposition}
    [Proprietà delle torri di estensioni]
    \label{torri}
    Data una torre di estensioni $K \subset F \subset L$, $\faktor{F}{K}$ è finita se e solo se $\faktor{F}{L}$ e $\faktor{L}{K}$ sono finite e inoltre:
    \[ [F:K] = [F:L][L:K]
        \]
\end{proposition}

\begin{proof}
    La dimostrazione è identica a quella già vista in \href{https://github.com/diego-unipi/Appunti-Aritmetica}{\textcolor{purple}{Aritmetica}}. Siano $[F:K] = n$ e $[L:F] = m$, verifichiamo che $[L:K]=nm$;
    per definizione sappiamo che $[F:K] = n$ ovvero $F$ è un $K$-spazio vettoriale con $\dim_KF = n$, e ugualmente, $[L:F] = m$ ovvero $L$ è un $F$-spazio vettoriale con $\dim_FL = m$, possiamo considerare allora una $K$-base di $F$, $\{v_1,\ldots,v_n\}$, ed una $F$-base di $L$, $\{w_1,\ldots,w_m\}$, per dimostrare la tesi, dobbiamo dimostrare che $\dim_KL = nm$, ovvero $L$ ammette una $K$-base di cardinalità $nm$. Consideriamo l'insieme $\{v_iw_j\}_{i=1,\ldots,n}^{j=1,\ldots,m}$, come si osserva facilmente, esso ha cardinalità $nm$, dimostriamo quindi che tale insieme è una $K$-base di $L$, per fare ciò verifichiamo separatamente che gli elementi di $\{v_iw_j\}_{i=1,\ldots,n}^{j=1,\ldots,m}$ generano tutti gli elementi di $L$ e che sono tra loro linearmente indipendenti:
	\begin{itemize}
	\ii Sia $\alpha \in L$, poiché $L$ è per ipotesi un $F$-spazio vettoriale, quindi $L = \left<w_1,\ldots,w_m\right>_F$, si ha che:
		\[ \alpha = \sum_{j=1}^{m} \lambda_jw_j
		\qquad\qquad \lambda_j \in F
		\]
		d'altra parte, poiché $F$ è un $K$-spazio vettoriale, quindi $F = \left<w_1,\ldots,w_m\right>_K$, si ha che:
		\[ \lambda_j = \sum_{i=1}^{n} a_{j_i}v_i
		\qquad\qquad a_{j_i} \in K
		\]
		e sostituendo si ottiene:
		\[ \alpha = \sum_{j=1}^{m} \left( \sum_{i=1}^{n} a_{j_i}v_i\right)w_j = \sum_{j=1}^{m}\sum_{i=1}^{n} a_{j_i}v_iw_j
		\qquad\qquad a_{j_i} \in K
		\]
    e quindi l'insieme $\{v_iw_j\}_{i=1,\ldots,n}^{j=1,\ldots,m}$ genera $L$ su $K$\footnote{Ovviamente tutti i prodotti $v_iw_j$ sono contenuti in $L$ per le proprietà di campo e perché appartengono a sottocampi del campo considerato.}.
	\ii Per dimostrare che l'insieme $\{v_iw_j\}_{i=1,\ldots,n}^{j=1,\ldots,m}$ è costituito da elementi linearmente indipendenti, è sufficiente mostrare che la somma:
		\[ \sum_{j=1}^{m}\sum_{i=1}^{n} a_{j_i}v_iw_j = 0
		\]
    se e solo se sono nulli tutti i coefficienti $a_{j_i}$. Scriviamo esplicitamente la somma esterna:
	\[ \left(\sum_{i=1}^{n} a_{1_i}v_i\right)w_1 +  \left(\sum_{i=1}^{n} a_{2_i}v_i\right)w_2 + \ldots +  \left(\sum_{i=1}^{n} a_{m_i}v_i\right)w_m = 0
	\]
    i prodotti $a_{j_i}v_i$, essendo $v_i \in F$ e $a_i \in K$, sono contenuti in $F$, pertanto, la somma appena scritta è una combinazione lineare dei $w_j$ (della $F$-base di $L$), che sappiamo essere linearmente indipendenti su $F$ (perché appunto sono una base di $L$), quindi la somma fa $0$ se e solo se i coefficienti sono tutti nulli, ovvero:
	\[ \sum_{i=1}^{n} a_{1_i}v_i =\ldots=\sum_{i=1}^{n} a_{m_i}v_i = 0
	\]
	\end{itemize}
    essendo $\{v_1,\ldots,v_n\}$ una $K$-base di $F$, le singole somme $\sum_{i=1}^{n} a_{j_i}v_i$ sono nulle se e solo se $a_{j_i} = 0$, $\forall i \in\{1,\ldots,n\}$, quindi la somma iniziale è nulla se e solo se $a_{j_i} = 0$, $\forall i \in\{1,\ldots,n\}$, $\forall j \in \{1,\ldots,m\}$, quindi gli elementi di $\{v_iw_j\}_{i=1,\ldots,n}^{j=1,\ldots,m}$ sono linearmente indipendenti.
\end{proof}

\begin{proposition}
    [Proprietà del composto di estensioni]
    \label{3.9}
    Date due torri di estensioni di $K$, $K \subset L \subset FL$ e $K \subset F \subset FL$, con $[L:K] = m$ e $[F:K] = n$, allora $[FL:K] = d < +\infty$ e $[m,n] \mid d$.
\end{proposition}

\begin{proof}
Per dimostrare la proposizione, osserviamo che, avendo supposto $n,m < +\infty$, allora possiamo considerare $FL$ come $L$-spazio vettoriale con un insieme di generatori di $n$ elementi,
da cui si ricava che $FL$ è anche un $K$-spazio vettoriale di dimensione finita; per la precisione applicando due volte il \hyperref[torri]{Teorema delle torri di estensioni} all'estensione:
\begin{center}
    \begin{tikzpicture}
    \node (Q1) at (0,0) {$K$};
    \node (Q2) at (1.5,1.5) {$F$};
    \node (Q3) at (0,3) {$FL$};
    \node (Q4) at (-1.5,1.5) {$L$};
    \draw (Q1)--(Q2) node [pos=0.7, below,inner sep=0.25cm] {$n$};
    \draw (Q1)--(Q4) node [pos=0.7, below,inner sep=0.25cm] {$m$};
    \draw (Q3)--(Q4) node [pos=0.7, above]{};
    \draw (Q2)--(Q3) node [pos=0.3, above]{};
    \draw (Q1)--(Q3) node [pos=0.7, below, right,inner sep=0.25cm, yshift=-0.4cm] {$d$};
    \end{tikzpicture}
\end{center}
otteniamo che $m \mid d$ e $n \mid d$, da cui la tesi.
\end{proof}

\begin{definition}
    Un'estensione $\faktor{L}{K}$ si dice \vocab{algebrica} se $\forall \alpha \in L$, $\alpha$ è algebrico su $K$.
\end{definition}

\begin{proposition}[Estensione finita$\implies$algebrica]
    \label{3.11}
    Ogni estensione di grado finito è algebrica.
\end{proposition}

\begin{proof}
    Vogliamo verificare che $\forall \alpha \in L$, $\alpha$ è algebrico su $K$, abbiamo che la torre:
    \[ K \subseteq K(\alpha) \subseteq L
        \]
    è finita per ipotesi ($[L : K] < +\infty$), pertanto la sottoestensione $K \subseteq K(\alpha)$ è a sua volta finita (volendo per il \hyperref[torri]{Teorema delle torri}),
    quindi per la \hyperref[3.7]{Proposizione 3.7} $\alpha$ è algebrico su $K$ (poiché siamo nel caso di un'estensione semplice), ed essendo vero per ogni $\alpha$, allora $L$ è un'estensione algebrica di $K$.
\end{proof}

\begin{proposition}[Campo delle estensioni algebriche]
    \label{3.12}
    Data un'estensione $\faktor{L}{K}$, sia $A = \{\alpha \in L | \, \text{$\alpha$ è algebrico su $K$}\}$, allora $A$ è un campo (ed ovviamente
    è un'estensione algebrica di $K$).
\end{proposition}

\begin{proof}
    Verifichiamo che $A$ sia un campo; siano $\alpha,\beta \in A$, allora $[K(\alpha):K] <+\infty$ e $[K(\beta):K]<+\infty$, consideriamo la torre:
    \[ K \subseteq K(\alpha) \subseteq K(\alpha)(\beta) = K(\alpha,\beta)
        \]
    la prima estensione è finita per ipotesi, mentre la seconda è finita per la \hyperref[3.9]{Proposizione 3.9} in quanto estensione composta da $K(\alpha)$ e $K(\beta)$ (entrambe semplici e quindi finite perché algebriche, per la \hyperref[3.7]{Proposizione 3.7}).
    Essendo dunque $K \subseteq K(\alpha,\beta)$ un'estensione finita, per la \hyperref[3.11]{Proposizione 3.11}
    è algebrica, quindi tutti gli elementi $\alpha\pm\beta$, $\alpha\beta$ e $\displaystyle\frac{1}{\beta}$ sono algebrici su $K$, pertanto $A$ è un campo.
\end{proof}

\begin{remark}
    Il viceversa della \hyperref[3.11]{Proposizione 3.11} in generale è falso; non lo è nel caso in cui l'estensione sia semplice, come visto nella \hyperref[3.7]{Proposizione 3.7}.
\end{remark}

\begin{example}
    [Estensione algebrica non finita]
    Consideriamo l'estensione $\QQ \subset \CC$, e l'insieme $\overline\QQ = \{\alpha \in \CC | \, \text{$\alpha$ è algebrico su $\QQ$}\}$, costruito come nella \hyperref[3.12]{Proposizione 3.12},
    che è un'estensione algebrica di $\QQ$. Verifichiamo che $[\overline\QQ:\QQ] = +\infty$; consideriamo la torre:
    \[ \QQ \subset \QQ(\sqrt[n]{2}) \subset \overline \QQ \qquad \forall n \geq 2
        \]
    l'estensione $\QQ \subset \QQ(\sqrt[n]{2})$ ha grado $n$, in quanto il suo polinomio minimo è il 2-Eisenstein:
    \[ \mu_{\sqrt[n]{2}}(x) = x^n - 2
        \]
    per il \hyperref[torri]{Teorema delle torri}:
    \[ [\overline \QQ: \QQ] = [\overline \QQ: \QQ(\sqrt[n]{2})] [\QQ(\sqrt[n]{2}) : \QQ] \geq n \qquad \forall n \geq 2
        \]
    dunque l'estensione $\QQ \subset \overline\QQ$ ha grado $+\infty$, in quanto ha grado maggiore di $n$ definitivamente.
\end{example}

\begin{remark}
    Ricordiamo che avevamo definito un'estensione \vocab{finitamente generata} una scrittura del tipo $L = K(\alpha_1,\ldots,\alpha_n)$ che può essere definita equivalentemente come:
    \[ K(\alpha_1,\ldots,\alpha_n) = K(\alpha_1)\ldots(\alpha_n) = \{p(\alpha_1,\ldots,\alpha_n) | p(x_1,\ldots,x_n) \in K[x_1,\ldots,x_n] \}
        \]
    con $\alpha_1,\ldots,\alpha_n$ algebrici su $K$, o anche come il più piccolo campo che contiene $K$ e gli elementi $\alpha_1,\ldots,\alpha_n$:
    \[ K(\alpha_1,\ldots,\alpha_n) = \bigcap_{\substack{K \subseteq M \subseteq L \\ \alpha_1,\ldots,\alpha_n \in M}}M
        \]
\end{remark}

\begin{proposition}[Estensione algebrica e finitamente generata$\implies$finita]
    \label{3.16}
    Sia $\faktor{L}{K}$ finitamente generata da elementi algebrici, $L = K(\alpha_1,\ldots,\alpha_n)$, allora $\faktor{L}{K}$ è finita.
\end{proposition}

\begin{proof}
    Possiamo dimostrare la tesi per induzione; per $n = 1$, la tesi è vera per la \hyperref[3.7]{Proposizione 3.7}, in quanto abbiamo un'estensione semplice ed algebrica. Supponiamo la tesi vera per $n-1$, abbiamo che:
    \[ K(\alpha_1,\ldots,\alpha_n) = K(\alpha_1,\ldots,\alpha_{n-1})(\alpha_n)
        \]
    consideriamo la torre:
    \[ K \subset K(\alpha_1,\ldots,\alpha_{n-1}) \subset K(\alpha_1,\ldots,\alpha_{n-1})(\alpha_n)
        \]
    dove la prima estensione è finita per ipotesi induttiva, mentre la seconda perché estensione semplice ed algebrica su un campo più piccolo, o alternativamente perché estensione composta di estensioni finite per quanto già detto, 
    quindi vale \hyperref[3.9]{Proposizione 3.9}; pertanto per $\faktor{L}{K}$ è finita per ogni $n \in \NN$.
\end{proof}

\pagebreak
\begin{remark}
    Della \hyperref[3.16]{Proposizione 3.16} vale anche il viceversa, infatti, presa un'estensione finita $[L:K] = n$, consideriamo $v_1,\ldots,v_n$ una $K$-base di $L$, allora 
    è vero che è finitamente generata:
    \[ L = \left<v_1,\ldots,v_n\right>_K = K(v_1,\ldots,v_n)
        \]
    dunque si può affermare che \textbf{un'estensione è finita se e solo se è finitamente generata da elementi algebrici}.
\end{remark}

\begin{proposition}
    [Proprietà delle estensioni algebriche rispetto a torri e composto]
    Valgono le seguenti proprietà per le estensioni algebriche:
    \begin{enumerate}[(1)]
        \item Data una torre di estensioni $K \subset L \subset F$, $\faktor{F}{K}$ è algebrica se e solo se $\faktor{F}{L}$ e $\faktor{L}{K}$ sono algebriche.
        \item Date due estensioni $\faktor{L}{K}$ e $\faktor{M}{K}$ esse sono algebriche se e solo se l'estensione composta $\faktor{LM}{K}$ è algebrica.
    \end{enumerate}
\end{proposition}

\begin{remark}[Sulla definizione di estensione composta]
    Dati $L,M \subset \Omega$ sottocampi dello stesso campo $\Omega$, abbiamo che il composto è:
    \[ LM = L(M) = M(L)
        \]
    ovvero il più piccolo sottocampo di $\Omega$ che contiene sia $L$ che $M$. Se $K$ ed $L$ sono finitamente generati:
    \[ L = K(\alpha_1,\ldots,\alpha_n) \qquad \text e \qquad M = K(\beta_1,\ldots,\beta_m)
        \]
    allora il loro composto è dato da:
    \[ LM = K(\alpha_1,\ldots,\alpha_n,\beta_1,\ldots,\beta_m)
        \]
\end{remark}

\begin{proof}
    Proviamo le due affermazioni:
    \begin{enumerate}[(1)]
        \item La prima implicazione segue quasi immediatamente, infatti, se $\faktor{F}{K}$ è algebrica, allora $\forall \alpha \in F$, $\alpha$ è algebrico su $K$,
            dunque tutti gli $\alpha \in L \subset F$ sono algebrici su $K$ perché elementi anche di $F$, pertanto $L$ è algebrico su $K$, inoltre, $F$ è algebrico su $L$ perché contiene $K$, dunque tutti gli elementi di $F$ sono già algebrici su $K$ e quindi lo saranno anche in un campo più grande.\\
            Viceversa sia $\alpha \in F$, per ipotesi sappiamo che $\faktor{F}{L}$ e $\faktor{L}{K}$ sono algebriche, quindi $\alpha$ è algebrico su $L$, dunque:
            \[ \exists f(x) \in L[x]\setminus\{0\} : f(\alpha) = 0 \qquad \text{con} \qquad f(x) = \sum_{i=0}^n a_ix^i
                \]
            consideriamo $L_0=K(a_0,\ldots,a_n)$ (estensione finitamente generata ed algebrica di $K$), si ha per costruzione che $f(x) \in L_0[x]$, dunque $\alpha$ è algebrico su $L_0$:
            \[ K \subset L_0 \subset L_0(\alpha)
                \]
            dove la prima estensione è finitamente generata ed algebrica (perché $\forall a_i \in L$, $a_i$ è algebrico su $K$), pertanto è finita per la \hyperref[3.16]{Proposizione 3.16}, inoltre la seconda 
            estensione è a sua volta finita perché algebrica (quindi vale la \hyperref[3.11]{Proposizione 3.11}), di conseguenza, per il \hyperref[torri]{Teorema delle torri}, la torre di estensione è finita,
            $[L_0(\alpha) : K] < +\infty$, dunque, sempre per la \hyperref[3.11]{Proposizione 3.11} $\faktor{L_0(\alpha)}{K}$ è algebrica. Abbiamo quindi dimostrato che $\alpha \in F$ è algebrico su $K$, $\forall \alpha \in F$,
            dunque $\faktor{F}{K}$ è algebrica.
         \item Sia $\faktor{LM}{K}$ algebrica e sia $\alpha \in M \subset LM$, allora è algebrico su $K$, in quanto tutti gli elementi di $LM$ già lo erano, dunque non stiamo facendo altro che prendere una sottoestensione di un'estensione algebrica, che quindi 
         sarà a sua volta algebrica, ovvero $\faktor{M}{K}$ algebrica; in maniera identica $\faktor{L}{K}$ è anch'essa algebrica.\\
         Supponiamo ora che $\faktor{L}{K}$ e $\faktor{M}{K}$ siano algebriche, consideriamo $\alpha \in LM = L(M)$, ovvero $\alpha = \sum_{i=1}^{n} \lambda_i m_i$, con $\lambda_i \in L$ e $m_i \in M$, dunque:
         \[ \alpha \in F = K(\lambda_1,\ldots,\lambda_n,m_1,\ldots,m_n)
            \]
        dove tale estensione è algebrica (perché tutti gli elementi che abbiamo aggiunto sono algebrici), quindi per la \hyperref[3.16]{Proposizione 3.16} è finita,
        e dunque $\alpha$ è algebrico su $K$, $\forall \alpha \in LM$, da cui la tesi.
    \end{enumerate}
\end{proof}

\newpage
\subsection{Chiusura algebrica di un campo}
\begin{definition}
    Un campo $\Omega$ si dice \vocab{algebricamente chiuso} se ogni polinomio $f(x) \in \Omega[x]$ non costante ha almeno una radice in $\Omega$.
\end{definition}

\begin{example}
    Per il Teorema Fondamentale dell'Algebra $\CC$ è algebricamente chiuso.
\end{example}

\begin{remark}
    Se $\Omega$ è algebricamente chiuso gli unici polinomi irriducibili di $\Omega[x]$ sono quelli di grado 1.
\end{remark}

\begin{definition}
    Data l'estensione $\faktor{\Omega}{K}$, $\Omega$ è una \vocab{chiusura algebrica} di $K$ se:
    \begin{itemize}
        \item $\Omega$ è algebricamente chiuso.
        \item $\faktor{\Omega}{K}$ è un'estensione algebrica.
    \end{itemize}
\end{definition}

\begin{example}
    Osserviamo che:
    \begin{enumerate}[(1)]
        \item $\CC$ è la chiusura algebrica di $\RR$.
        \item $\CC$ non è la chiusura algebrica di $\QQ$, poiché non tutti gli elementi di $\CC$ sono algebrici su $\QQ$.
    \end{enumerate}
\end{example}

\begin{theorem}
    [Esistenza e unicità della chiusura algebrica]
    Sia $K$ un campo, allora esiste sempre una sua chiusura algebrica. Inoltre due qualsiasi chiusure algebriche su $K$ sono isomorfe.\footnote{Nel senso che gli isomorfismi tra le varie chiusure algebriche fissano $K$.}
\end{theorem}

\begin{example}
    Consideriamo $\overline\QQ = \{\alpha \in \CC | \, \text{$\alpha$ è algebrico su $\QQ$}\}$, tale insieme è un campo, per quanto asserito nella
    \hyperref[3.12]{Proposizione 3.12}, ed è per una chiusura algebrica di $\QQ$, infatti $\faktor{\overline\QQ}{\QQ}$ è un'estensione algebrica per definizione.
    Verifichiamo che $\overline\QQ$ è algebricamente chiuso; sia $f(x) \in \overline\QQ[x]$ non costante, allora $f(x)$ ammette almeno una radice $\alpha \in \CC$, poiché
    $f(x) \in \overline\QQ[x] \subset \CC[x]$ e $\CC$ è algebricamente chiuso. Per mostrare che $\alpha \in \overline\QQ$, bisogna
    dimostrare che $\alpha$ è algebrico su $\QQ$, consideriamo la torre:
    \[ \QQ \subset \overline \QQ \subset \overline\QQ (\alpha)
        \]
    la prima estensione è algebrica per quanto già discusso, la seconda è algebrica perché semplice (\hyperref[3.7]{Proposizione 3.7}), dunque per \hyperref[torri]{torri} $\alpha$
    è algebrico su $\QQ$, inoltre $\alpha \in \CC$, per cui $\alpha \in \overline\QQ$ che quindi è algebricamente chiuso, perché quanto detto vale per qualunque polinomio in $\overline\QQ[x]$.
\end{example}

\begin{remark}
    L'argomento dell'esempio precedente può essere utilizzato per costruire $\overline K$ chiusura algebrica di $K$ ogni volta che si ha:
    \[ K \subset \Omega
        \]
    con $\Omega$ algebricamente chiuso, definendo $\overline K = \{\alpha \in \Omega |  \, \text{$\alpha$ è algebrico su $K$}\}$, da cui $\overline K$ chiusura algebrica si $K$ (infatti $\overline K$ è un estensione algebrica 
    di $K$ per definizione, ed iterando il ragionamento precedente si mostra come sia anche algebricamente chiuso).
\end{remark}

\begin{definition}
    Sia $f(x) \in K[x]$, con $\deg f(x) \geq 1$, e siano $\alpha_1,\ldots,\alpha_n \in \overline K$ le radici di $f(x)$, si definisce \vocab{campo di spezzamento} di $f(x)$ su $K$ il sottocampo di $\overline K$:
    \[ K(\alpha_1,\ldots,\alpha_n)
        \]
\end{definition}

\begin{remark}
    La definizione di campo di spezzamento può essere estesa ad una famiglia di polinomi $\mathcal{F} = \{f_i(x) | i \in I\} \subset K[x]$, infatti, dato l'insieme delle radici di un polinomio $f_i(x)$ della famiglia,
    $ \{\alpha_{ij}\}_{i \in I}^{j = 1,\ldots,n_i} $, il campo di spezzamento di $\mathcal{F}$ su $K$ è dato da:
    \[ K(\{\alpha_{ij}\}_{j = 1,\ldots,n_i} | i \in I)
        \]
    Tale estensione della definizione risulta necessaria soltanto nel caso di famiglie infinite di polinomi, infatti nel caso finito è sufficiente considerare il campo di spezzamento dell'unico polinomio prodotto di tutti gli altri 
    (prodotto che non è definito nel caso di una famiglia infinita).
\end{remark}

\begin{example}
    [Campo di spezzamento di $x^3 - 2$]
    Consideriamo $f(x) = x^3 - 2 \in \QQ[x]$, le radici di $f(x)$ in $\CC[x]$ sono $\sqrt[3]{2}$,$\sqrt[3]{2}\zeta_3$ e $\sqrt[3]{2}\zeta_3^2$, dunque il campo di spezzamento di $f(x)$ su $\QQ$ è dato da :
    \[ \QQ(\sqrt[3]{2},\sqrt[3]{2}\zeta_3,\sqrt[3]{2}\zeta_3^2) = \QQ(\sqrt[3]{2},\zeta_3)
        \]
    dove l'ultima uguaglianza andrebbe verificata mediante doppio contenimento.
\end{example}

\begin{example}
    [Campo di spezzamento di $x^n - 1$]
    Consideriamo $\Phi_n(x) = x^n - 1 \in \QQ[x]$, le radici di $\Phi_n(x)$ in $\CC[x]$ sono gli elementi del gruppo $\left<\zeta_n\right> = \{\alpha \in \CC | \alpha^n = 1\}$, con $\zeta_n$ radice $n$-esima primitiva di 1,
    pertanto $\QQ(\zeta_n)$ è il campo di spezzamento del polinomio ciclotomico $n$-esimo:
    \[ x^n - 1 = \prod_{i = 0}^{n-1} (x - \zeta_n^i)
        \]
\end{example}

Ci poniamo ora il seguente problema: dato un campo $K$, la sua chiusura algebrica $\overline K$ e $\alpha \in \overline K$ (algebrico su $K$), vogliamo sapere in quanti modi si può immerge $K(\alpha)$ in $\overline K$ con:
\[ \varphi : K(\alpha) \varlonghookrightarrow \overline K \qquad \text{con} \qquad \varphi_{|K} = id_K
    \]

\begin{remark}[Omomorfismi da campi]
    \label{3.32}
Osserviamo che in generale un omomorfismo definito su un campo può avere solo nucleo banale, infatti, esso è in particolare un omomorfismo tra anelli, dunque il nucleo è un ideale, ma per la (2) della \hyperref[2.26]{Proposizione 2.26} gli unici ideali di un campo 
sono se stesso e $\{0\}$, dunque il nucleo o è tutto o è banale, pertanto gli unici omomorfismi possibili da un campo sono o iniettivi o nulli.    
\end{remark}

Possiamo rispondere alla domanda iniziale per mezzo della seguente proposizione.

\begin{proposition}[Numero di estensioni  via identità di $K(\alpha)$ a $\overline K$]
    \label{3.33}
    Dato un campo $K$ ed $\alpha \in \overline K$, con $\overline K$ chiusura algebrica di $K$, detto $k$ il numero di radici distinte di $\mu_\alpha(x)$ in $\overline K$, allora:
    \[ \exists \varphi_1,\ldots,\varphi_k : K(\alpha) \varlonghookrightarrow \overline K \qquad \text{con} \qquad \varphi_{i|K} = id_K
        \]
    ovvero esistono esattamente $k$ estensioni distinte da $K(\alpha)$ a $\ol K$.
\end{proposition}

\begin{proof}
    Per quanto detto sull'omomorfismo di valutazione sappiamo che $\displaystyle K(\alpha) \cong \frac{K[x]}{(\mu_\alpha(x))}$, dunque, determinando un omomorfismo da $K[x]$ ad $\overline K$ possiamo applicare il 
    \hyperref[omo]{Primo Teorema di Omomorfismo}. Sia dunque:
    \[ \widetilde{\varphi} : K[x] \longrightarrow \overline K : x \longmapsto \beta , p(x) \longmapsto p(\beta) \qquad \forall \beta \in \overline K
        \]
    per tale omomorfismo abbiamo che $(\mu_\alpha(x)) \subset \ker \widetilde{\varphi} \iff \mu_\alpha(x) \in \ker \widetilde{\varphi}$ (per le proprietà degli ideali), quindi se e solo se $\mu_\alpha(\beta) = 0$,
    quindi per tutte le radici di $\mu_\alpha(x)$ in $\overline K$ si ha che:
    \[\begin{tikzcd}
		K[x] \arrow[d, "\pi_{(\mu_\alpha(x))}" left, twoheadrightarrow] \arrow[r, "\widetilde{\varphi}"] & \overline K \\	
		\displaystyle K(\alpha) \cong \frac{K[x]}{(\mu_\alpha(x))} \arrow[ur, "\varphi"', hookrightarrow] & 
	\end{tikzcd}
        \]
    dove $\varphi$ è un omomorfismo per il \hyperref[omo]{Primo Teorema di Omomorfismo}, ed è iniettivo (per quanto già visto nell'\hyperref[3.32]{Osservazione 3.32}) perché omomorfismo \textbf{non nullo} tra campi (ad esempio perché $1 \longmapsto 1$). Si osserva infine che 
    le $\varphi_i$ sono distinte perché danno diversa immagine quando calcolate su $x$.
\end{proof}

Dalla proposizione appena dimostrata ci poniamo un secondo problema, quello di contare il numero di radici di $\mu_\alpha (x)$ in $\overline K$, essendo $\overline K$ algebricamente chiuso, allora il numero delle radici del polinomio, ciascuna contata 
con la propria molteplicità, è dato da $\deg \mu_\alpha(x) = n$, a questo punto, per determinare se $\mu_\alpha(x)$ abbia o meno radici multiple possiamo fare uso del criterio che segue.

\begin{theorem}
	[Criterio della Derivata]
	\label{derivata}
	Sia $f(x) \in K[x]$, $f(x)$, allora $f(x)$ ha radici multiple in $\overline K$ se e solo se $(f(x),f^{\prime}(x)) \ne 1$. Inoltre se $f(x)$ è irriducibile in $K[x]$, allora $f(x)$ ha radici multiple se e solo se $f^{\prime}(x) = 0$.
\end{theorem}

\begin{proof}
    Il primo fatto è stato trattato in \href{https://github.com/diego-unipi/Appunti-Aritmetica}{\textcolor{purple}{Aritmetica}}, quindi non ne forniremo qui una dimostrazione. Per il secondo fatto si può osservare che $f(x) \in K[x] \implies f^{\prime}(x) \in K[x]$,
    da cui $(f(x),f^{\prime}(x)) \in K[x]$ (perché si ottiene mediante l'Algoritmo di Euclide), e se $f(x)$ è irriducibile in $K[x]$ l'M.C.D. fa 1 oppure $f(x)$; dove il secondo caso si realizza se e solo se $f^{\prime}(x) = 0$ (dato che $\deg f^{\prime}(x) < \deg f(x)$).
\end{proof}

Dunque se in $K[x]$ i polinomi irriducibili non hanno derivata nulla, allora il numero delle loro radici distinte coincide con il loro grado.

\begin{definition}
    Un campo $K$ tale per cui tutti i polinomi irriducibili in $K[x]$ hanno derivata non nulla prende il nome di \vocab{campo perfetto}.
\end{definition}

\begin{remark}
    [Campi perfetti]
    Osserviamo che:
    \begin{itemize}
        \item Se $\ch K = 0$, allora $K$ è un campo perfetto, infatti, detto $f(x) = \sum_{i=0}^n a_ix^i$, allora $f^{\prime}(x) = \sum_{i=0}^{n-1} ia_ix^i$, $\forall n \geq 1$, ovvero la derivata di un polinomio non 
        costante non può essere mai nulla.
        \item $\Fpn$ è perfetto $\forall p$ primo, $\forall n \geq 1$.
        \item Al contrario possiamo vedere che per campi con caratteristica diversa da 0 esistono polinomi irriducibili con derivata nulla, sia $K = \Fp(t)$ e $f(x) = x^p - t \in K[x]$, abbiamo che $f^{\prime}(x) = px^{p-1} = 0$,
        possiamo verificare che $f(x)$ è irriducibile in $K[x]$. Osserviamo che:
        \[ f(x) \in \Fp[t][x] := A[x]
            \]
        dove $A$ UFD (essendo un ED), dunque per il \hyperref[gauss]{Lemma di Gauss} verificare che $f(x)$ è irriducibile in $K[x] = \Fp(t)[x]$ è equivalente a verificare che sia irriducibile in $A[x] = \Fp[t][x]$.
        In $A[x]$, essendo UFD vale il \hyperref[eisenstein]{Criterio di Eisenstein}, dunque $f(x)$ è irriducibile rispetto all'ideale $P = (t)$, che è primo in quanto massimale, infatti $\displaystyle\frac{A}{(t)} \cong \Fp$.
        Inoltre se $\alpha \in \overline K$ è una radice di $f(x)$, abbiamo che $f(\alpha) = \alpha^p - t = \implies t = \alpha^p$, da cui:
        \[ f(x) = x^p - \alpha^p = (x - \alpha)^p
            \]
        dove nell'ultima uguaglianza abbiamo usato il lemma del Binomio Ingenuo, pertanto in realtà $f(x)$ ha un'unica radice in $\overline K$.
    \end{itemize}
\end{remark}

Ci limiteremo (assumendolo anche senza ripeterlo ogni volta) ai campi perfetti, pertanto un polinomio irriducibile di $K[x]$ di grado $n$ avrà esattamente $n$ radici distinte in $\overline K$.

\begin{proposition}[Numero di estensioni di $K(\alpha)$ a $\overline K$]
    \label{3.37}
    Dato $\alpha \in \overline K$, con $[K(\alpha) : K] = n$, si ha che $\forall \varphi : K \varlonghookrightarrow \overline K$ immersione, esistono esattamente $n$ estensioni a $K(\alpha)$, cioè:
    \[ \exists \varphi_1,\ldots,\varphi_n : K(\alpha) \varlonghookrightarrow \overline K \qquad \text{con} \qquad \varphi_{i|K} = \varphi
        \]
\end{proposition}

\begin{remark}
    Abbiamo già visto che ciò è vero se $\varphi = id_K$, nella \hyperref[3.33]{Proposizione 3.33},
    la nuova proposizione ci permette di contare il numero di estensioni di un omomorfismo da $K$ in $\ol K$ ad uno da $K(\alpha)$ in $\ol K$.
    Ad esempio, dato $K = \QQ(\sqrt[3]{2})$ e l'omomorfismo:
    \[ \varphi : K \varlonghookrightarrow \ol K : \sqrt[3]{2} \longmapsto \sqrt[3]{2}\zeta_3
        \]
    ci chiediamo quanti sono gli omomorfismi:
    \[ \varphi_i : K(\zeta_3) \varlonghookrightarrow \ol K \qquad \text{con} \qquad \varphi_{i|K} = \varphi
        \]
    cioè che quando ristretti a $K$ si comportano come $\varphi$.
\end{remark}

\begin{proof}
    In analogia con quanto fatto nella dimostrazione della \hyperref[3.33]{Proposizione 3.33}, consideriamo:
    \[ \widetilde{\varphi} : K[x] \longrightarrow \overline K : x \longmapsto \beta : p(x) \longmapsto p(\beta) \qquad \text{con} \qquad \varphi_{i|K} = \varphi
        \]
    per tale omomorfismo abbiamo che $(\mu_\alpha(x)) \subseteq \ker \widetilde{\varphi} \iff \widetilde{\varphi}(\mu_\alpha(x)) = 0 \iff \varphi(\mu_\alpha(x))(\beta) = 0$, 
    dunque applichiamo $\varphi$ ai coefficienti di $\mu_\alpha(x)$ (prima venivano lasciati fissi perché usavamo l'identità) e poi valutiamo il nuovo polinomio in $\beta$, pertanto $\beta$
    deve essere una radice di $\varphi(\mu_\alpha(x))$, dunque le estensioni di $\varphi$ a $K(\alpha)$ sono tante quante le radici distinte di $\varphi(\mu_\alpha(x))$ in $\overline K$. \\
    Poiché $\mu_\alpha(x)$ è irriducibile per definizione, allora $\varphi(\mu_\alpha(x))$ è irriducibile, inoltre, $\deg \mu_\alpha(x) = \deg \varphi(\mu_\alpha(x))$ (poiché l'omomorfismo è
    iniettivo), pertanto, essendo il campo perfetto il numero di radici distinte di $\varphi(\mu_\alpha(x))$ è uguale al suo grado, e quindi a quello di $\mu_\alpha(x)$.
\end{proof}

\begin{corollary}[Numero di estensioni a $\overline K$ di un'estensione qualsiasi]
    \label{3.39}
    Sia $\faktor{E}{K}$ un'estensione, con $[E:K] = n$, allora $\forall \varphi : K \varlonghookrightarrow \overline K$ immersione, esistono $n$ immersioni:
    \[ \varphi_1,\ldots,\varphi_n : E \varlonghookrightarrow \overline K \qquad \text{con} \qquad \varphi_{i|K} = \varphi
        \]
\end{corollary}

\begin{proof}
    La dimostrazione segue facilmente per induzione, infatti per $n = 1$ abbiamo che l'estensione è semplice e quindi vale la \hyperref[3.37]{Proposizione 3.37}; per $n>1$ consideriamo
    $\alpha \in E \setminus K$ per il quale si ha la torre di estensioni:
    \[ K \subset K(\alpha) \subset E
        \]
    con $[K(\alpha) : K] = m$ e $[E : K(\alpha)] = d$. Se $m = n$, allora $E = K(\alpha)$ e siamo ancora nel caso precedente; se $1 < m < n \implies d < n$,
    essendo $n = md$ (in pratica stiamo supponendo di aver già messo nell'estensione almeno un nuovo elemento), dunque per la \hyperref[3.37]{Proposzione 3.37} $\varphi$ si estende in $m$ modi a $K(\alpha)$:
    \[ \varphi_1,\ldots,\varphi_m : K(\alpha) \varlonghookrightarrow \overline K \qquad \text{con} \qquad \varphi_{i|K} = \varphi
        \]
    Ogni $\varphi_i : K(\alpha) \varlonghookrightarrow \overline K$ si estende a sua volta per ipotesi induttiva:
    \[ \varphi_{i1},\ldots,\varphi_{1d} : E \varlonghookrightarrow \overline K \qquad \text{con} \qquad \varphi_{ij|K(\alpha)} = \varphi_i
        \]
    dunque abbiamo $\{\varphi_{ij}\}_{i = 1,\ldots,m}^{j = 1,\ldots,d} : E \varlonghookrightarrow \overline K$, ovvero $md = n$ estensioni, con $\varphi_{ij | K(\alpha)} = \varphi_i $, per cui $\varphi_{ij|K} = \varphi_{i | K} = \varphi$, quindi tutte le $n$ estensioni funzionano.
\end{proof}

\newpage
\subsection{Estensioni normali}

\begin{definition}
    Dato $\alpha \in \overline K$, diciamo che i \vocab{coniugati} di $\alpha$ su $K$ sono le radici del polinomio minimo di $\alpha$ su $K$. 
\end{definition}

\begin{definition}
    Data un'estensione algebrica $K \subset L$, essa si dice \vocab{separabile} se il polinomio minimo di ogni elemento è un
    \vocab{polinomio separabile}, ovvero se ha radici tutte distinte in un suo campo di spezzamento.\footnote{Questa definizione
    è stata aggiunta per completezza al materiale della professoressa, infatti verrà citata successivamente qualche volta, pertanto, sebbene non vi sarà una trattazione ulteriore 
    a riguardo, ho ritenuto opportuno aggiungerla qui.}
\end{definition}

Nella trattazione di teoria di campi di queste dispense considereremo soltanto estensioni separabili.

\begin{example}
    Sia $f(x) = x^3 - 2$, tale polinomio coincide con $\mu_{\sqrt[3]{2}/\QQ}(x)$, ovvero il polinomio minimo di $\alpha = \sqrt[3]{2}$ su $\QQ$, vogliamo studiare le immersioni di $\QQ(\alpha)$ in $\overline \QQ$:
    \[ \varphi : \QQ(\alpha) \varlonghookrightarrow \overline \QQ \qquad \text{con} \qquad \varphi_{i|\QQ} = id_\QQ
        \]
    dati i coniugati di $\alpha$ (dunque le radici di $\mu_{\alpha/\QQ}$): $\alpha$, $\alpha\zeta_3$, $\alpha\zeta_3^2$, il nostro omomorfismo deve mandare ogni radice in un'altra, dunque abbiamo più possibilità:
    \[ \varphi(\alpha) = \begin{cases}
        \alpha \\
        \alpha\zeta_3 \\
         \alpha\zeta_3^2
        \end{cases}
        \]
    pertanto in base alla scelta di $\varphi(\alpha)$ abbiamo le possibilità:
    \[ \varphi(\QQ(\alpha)) = \QQ(\varphi(\alpha)) = \begin{cases}
        \QQ(\alpha)\\
        \QQ(\alpha\zeta_3)\\
        \QQ(\alpha\zeta_3^2)\\
    \end{cases}
        \]
    da cui $\QQ(\alpha)$ è isomorfo su $\QQ$ (o si dice anche i tre isomorfismi fissano $\QQ$ puntualmente) a questi tre campi (distinti), ciò è in accordo con la \hyperref[3.33]{Proposizione 3.33}.
\end{example}

\begin{example}
    Sia $p$ un primo e consideriamo il campo ciclotomico $p$-esimo $\QQ(\zeta_p)$; per tale campo abbiamo che:
    \[ \mu_{\zeta_p/\QQ} (x) = \frac{x^p - 1}{x -1} = x^{p-1} + \ldots + x + 1
        \] 
    il quale è irriducibile perché è traslato di un $p$-Eisenstein, pertanto i coniugati di $\zeta_p$ sono $\zeta_p^i$, con $1 \leq i < p$ (ovvero sono $p-1$), da cui:
    \[ [\QQ(\zeta_p) : \QQ] = \phi(p) = p - 1 
        \]
    Le immersioni di $\faktor{\QQ(\zeta_p)}{\QQ}$ sono del tipo:
    \[ \varphi_i : \QQ(\zeta_p) \longrightarrow \overline \QQ : \zeta_p \longmapsto \zeta_p^i \qquad \text{con} \qquad \varphi_{i|\QQ} = id_\QQ
        \]
    ed abbiamo quindi $p - 1$ possibili immersioni, ancora una volta in accordo con la \hyperref[3.33]{Proposizione 3.33}, per cui abbiamo:
    \[ \varphi_i(\QQ(\zeta_p)) = \QQ(\varphi_i(\zeta_p)) = \QQ(\zeta_p^i) = \QQ(\zeta_p) \qquad \forall i \in \{1,\ldots,p-1\}
        \]
    dove l'ultima uguaglianza deriva dal fatto che $\QQ(\zeta_p^i) \subseteq \QQ(\zeta_p)$ e $\mu_{\zeta_p^i}(x) = \mu_{\zeta_p}(x)$, dunque abbiamo un contenimento di estensioni
    che hanno lo stesso grado, quindi sono la stessa estensione.
\end{example}

\begin{definition}
    Un'estensione algebrica $\faktor{F}{K}$ si dice \vocab{normale} se:
    \[ \forall \varphi : F \varlonghookrightarrow \overline K \qquad \text{con} \qquad \varphi_{|K} = id_K
        \]
    si ha che $\varphi(F) = F$, ovvero l'estensione viene fissata da ogni immersione del campo di partenza nella sua chiusura algebrica.
\end{definition}

\begin{example}
    [Estensione normale]
    Alcuni esempi di estensioni normali possono essere:
    \begin{itemize}
        \item $\QQ(\zeta_p)$ è un'estensione normale su $\QQ$.
        \item Detto $F = \QQ(\sqrt[3]{2},\zeta_3)$, allora l'estensione $\faktor{F}{\QQ}$ è normale, infatti data:
        \[ \varphi : F \varlonghookrightarrow \overline \QQ \qquad \text{con} \qquad \varphi_{|\QQ} = id_\QQ
            \]
        con:
        \[ \varphi(\QQ(\sqrt[3]{2},\zeta_3)) = \QQ(\varphi(\sqrt[3]{2}),\varphi(\zeta_3)) =\footnote{Le immagini di entrambi gli elementi devono essere loro coniugati per quanto detto in precedenza.} \,\QQ(\sqrt[3]{2}\zeta_3^i,\zeta_3^j) = \QQ(\sqrt[3]{2},\zeta_3)
            \]
        per $i \in \{0,1,2\}$ e $j \in \{1,2\}$ (dunque abbiamo 6 immersioni, come previsto dall'\hyperref[3.39]{Corollario 3.39}), quindi abbiamo $F$ invariante rispetto a $\varphi$, per cui $\faktor{F}{\QQ}$ è un'estensione normale.
    \end{itemize}
\end{example}

\begin{proposition}
    [Caratterizzazione delle estensioni normali]
    \label{3.46}
    Sia $\faktor{F}{K}$ un'estensione algebrica (finita)\footnote{La proposizione è vera anche senza questa ipotesi, ma la dimostriamo solo in questo caso.}, sono fatti equivalenti:
    \begin{enumerate}[(1)]
        \item $\faktor{F}{K}$ normale.
        \item Ogni polinomio irriducibile $f(x) \in K[x]$ che ha una radice in $F$ ha tutte le sue radici in $F$.
        \item $F$ è il campo di spezzamento su $K$ di una famiglia di polinomi di $K[x]$.
    \end{enumerate}
\end{proposition}

\begin{proof}
    Verifichiamo le varie equivalenze:
    \begin{itemize}
        \item \underline{\textbf{$(1)\implies (2)$}}: sia $f(x) \in K[x]$ e siano $\alpha_1,\ldots,\alpha_n \in \overline K$ le radici di $F$, per ipotesi sappiamo che $f(x)$ ha almeno una radice in $F$, supponiamo sia $\alpha_1$, allora $K(\alpha_1) \subset F$, dunque
        $\forall i \in \{1,\ldots,n\}$ consideriamo le immersioni:
        \[ \varphi_i : K(\alpha_1) \varlonghookrightarrow K(\alpha_i) \subseteq \overline K : \alpha_1 \longmapsto \alpha_i \qquad \text{con} \qquad \varphi_{i|K} = id_K
            \]
        esse esistono sempre per la \hyperref[3.33]{Proposizione 3.33}, inoltre, $\forall i \in \{1,\ldots,n\}$ sia $\widetilde{\varphi_i}$ un'estensione di $\varphi_i$ a $F$, cioè ogni immersione di $K(\alpha_i)$ si estende ad $F$
        in tanti modi quanti il grado $[F : K(\alpha_1)]$, pertanto, fissata un'estensione di $\varphi_i$:
        \[ \widetilde{\varphi_i} : (K(\alpha_i) \subset)F \longrightarrow \overline K \qquad \text{con} \qquad \widetilde{\varphi}_{i|K(\alpha_1)} = \varphi_i \implies \widetilde{\varphi}_{i|K} = id_K
            \]
        cioè $\widetilde{\varphi}_i$ si restringe a $K$ proprio come $\varphi_i$, da cui, essendo $\faktor{F}{K}$ normale, si ha che $\widetilde{\varphi}_i(F) = F$, ma in particolare ciò significa che
        dalla radice $\alpha_1$ di $f(x)$, che va nei suoi coniugati mediante $\widetilde{\varphi}_i$, otteniamo tutte le altre radici dentro $F$:
        \[ \widetilde{\varphi}_i(\alpha_1) = \varphi_i(\alpha_1) = \alpha_i \in F \qquad \forall i \in \{1,\ldots,n\}
            \]
        che prova la tesi.
        \item \underline{\textbf{$(2)\implies (3)$}}: Consideriamo $F_0$ il campo di spezzamento su $K$ della famiglia di polinomi:
        \[ \mathcal{F} = \{\mu_\alpha(x) | \alpha \in F, \, \text{$\mu_\alpha(x)$ polinomio minimo di $\alpha$ su $K$}\}
            \]
        abbiamo che $F \subseteq F_0$, poiché abbiamo aggiunto tutte le radici di tutti i polinomi minimi di tutti gli elementi di $F$, dunque $F_0$ contiene almeno $F$.
        D'altra parte:
        \[ F_0 = K(\beta | \, \text{$\beta$ radice di $\mu_\alpha(x) \in \mathcal{F}$})
            \]
        dove $\mu_\alpha(x)$ è irriducibile su $K[x]$ e $\alpha$ è una sua radice in $F$, dunque per ipotesi $F$ contiene tutte le radici $\beta$ di $\mu_\alpha(x)$, $\forall \mu_\alpha(x) \in \mathcal{F}$, ovvero $F_0 \subseteq F$,
        quindi $F = F_0$, per cui $F$ è proprio il campo di spezzamento dei polinomi della famiglia $\mathcal{F}$.
        \item \underline{\textbf{$(3)\implies (1)$}}: Consideriamo:
        \[ \varphi : F \varlonghookrightarrow \overline K \qquad \text{con} \qquad \varphi_{|K} = id_K
            \]
        vogliamo dimostrare che $\varphi(F) = F$, avendo per ipotesi che $F$ è il campo di spezzamento di:
        \[ \mathcal{F} = \{f_1(x),\ldots,f_k(x)\}
            \]
        per cui $\{\alpha_{ij}\}_{j = 1,\ldots,n_i}$ sono le radici di $f_i(x)$, dunque possiamo riscrivere $F$ come:
        \[ F = K(\{\alpha_{ij}\} | i = 1,\ldots,k, j = 1, \ldots, n_i) \footnote{Potremmo anche considerare il campo di spezzamento del polinomio prodotto dei precedenti, ma le radici sarebbero sempre le stesse.}
            \]
        Sappiamo che per ogni $i$ e $j$, poiché $\varphi$ è un'immersione deve mandare le radici in loro coniugati, dunque $\varphi(\alpha_{ij}) = \alpha_{ij^{\prime}}$ (ovvero un'altra radice dello stesso polinomio $f_i(x) \in K[x]$, con $\mu_{\alpha_{ij}} (x) \mid f_i(x)$), da cui abbiamo che:
        \begin{multline*}
            \varphi(F) = \varphi(K(\{\alpha_{ij}\} | i = 1,\ldots,k, j = 1, \ldots, n_i)) = \\
             = K(\varphi(\alpha_{ij}) | i = 1,\ldots,k, j = 1, \ldots, n_i) \subset F
        \end{multline*}
        dove il contenimento segue dal fatto che abbiamo soltanto permutato gli elementi $\{\alpha_{ij}\}$, per cui $K \subset \varphi(F) \subset F$, ma poiché $F$ e $\varphi(F)$ hanno lo stesso grado finito su $K$,
        allora $\varphi(F) = F$.
    \end{itemize}
\end{proof}

\begin{example}
[Ogni estensione di grado 2 è normale]
Sia $\faktor{F}{K}$ un'estensione, supponiamo di essere in caratteristica diversa da 2\footnote{Per la precisione, l'osservazione funziona anche in questo caso,
infatti l'unico polinomio di grado 2 irriducibile in un campo di caratteristica 2 può essere solo $x^2+x+1$ (dato che tutti quei campi estendono $\F{2}$) e per questo si verifica che se $\alpha$ è una radice,
$\alpha+1$ è l'altra radice e quindi effettivamente l'estensione è sempre di grado 2.}, con $[F : K] = 2$ e sia $\alpha \in F \setminus K \implies F = K(\alpha)$ abbiamo che quindi il polinomio minimo è della forma:
\[ \mu_\alpha(x) = x^2 + bx + x \in K[x]
    \]
con:
\[ \alpha_{1,2} = \frac{-b \pm \sqrt{\Delta}}{2} \implies \alpha = \frac{-b + \sqrt{\Delta}}{2} \implies F = K(\alpha) = K(\sqrt{\Delta})
    \]
infatti si verifica facilmente che $\alpha_1,\alpha_2 \in K(\sqrt{\Delta}) = F$, ovvero $F$ è il campo di spezzamento di $\mu_{\alpha/K}(x)$, pertanto per la (3) della \hyperref[3.46]{Proposizione 3.46} $\faktor{F}{K}$ è normale.
\end{example}

\begin{example}
    [Estensione non normale]
    L'estensione $\faktor{\QQ(\sqrt[3]{2})}{\QQ}$ non è normale, infatti, preso $f(x) = x^3 - 2$ irriducibile su $\QQ$ abbiamo in $\QQ(\sqrt[3]{2})$ una sola radice di $f(x)$ e non tutte, pertanto tale estensione non può essere normale.
\end{example}

\begin{proposition}
[Proprietà delle estensioni normali rispetto a composto ed intersezione]
Date due estensioni $\faktor{F}{K}$ e $\faktor{L}{K}$ in una fissata chiusura algebrica $\overline K$ normali, allora $\faktor{FL}{K}$ e $\faktor{F \cap L}{K}$ sono normali.
\end{proposition}

\begin{proof}
    La proprietà del composto segue immediatamente dal fatto che l'immersione nella chiusura algebrica:
    \[ \varphi : FL \longrightarrow \overline K \qquad \text{con} \qquad \varphi_{|K} = id_K
        \]
    è un omomorfismo, per cui:
    \[ \varphi(FL) =\footnote{Ciò deriva dal fatto che gli elementi di $FL$ sono combinazioni polinomiali degli elementi di $F$ e di $L$.} \varphi(F)\varphi(L) = FL
        \]
    dunque $FL$ è un'estensione normale. Analogamente per l'intersezione abbiamo:
    \[ \varphi(F \cap L) =\footnote{Ciò andrebbe verificato, ma è abbastanza semplice.} \varphi(F) \cap \varphi(L) = F \cap L
        \]
    ed in questo caso stiamo estendendo $\varphi$ sia ad $L$ sia ad $F$.
\end{proof}

\begin{proposition}
[Proprietà delle estensione normali rispetto alle torri]
\label{3.50}
Data una torre di estensioni $K \subset F \subset L$ in una fissata chiusura algebrica $\overline K$, se $\faktor{L}{K}$ è normale, allora $\faktor{L}{F}$ è normale.\footnote{In generale $\faktor{F}{K}$ non è normale.}
\end{proposition}

\begin{proof}
Per ipotesi sappiamo che:
\[ \forall \varphi : F \varlonghookrightarrow \overline K \qquad \text{con} \qquad \varphi_{|F} = id_F
    \]
essendo $K \subset F$, allora $id_F$ include già $id_K$, quindi $\varphi_{|F} = id_F \implies \varphi_{|K} = id_K$, dato che $\faktor{L}{K}$ è normale,
abbiamo $\varphi(L)=L$ per tutte le immersioni $\varphi$ di $L$ che fissano $K$, ma abbiamo visti che le immersioni che fissano $F$ fissano $K$ dunque $\varphi(L) = L$ anche un questo caso,
pertanto $\faktor{L}{F}$ è normale. \\
\emph{Dimostrazione alternativa.}\footnote{Proposta da Francesco Sorce.}\; $\faktor{L}{K}$ normale significa che è campo di spezzamento di una famiglia di polinomi in $K[x]\subseteq F[x]$, quindi $L$ è il campo di spezzamento
della stessa famiglia vista come polinomi in $F$.
\end{proof}

\begin{remark}
    Prendendo $K = \QQ$, $F = \QQ(\sqrt[3]{2})$ e $L = \QQ(\sqrt[3]{2},\zeta_3)$, abbiamo che $\faktor{L}{K}$ è normale in quanto tutte le radici del polinomio minimo dei due elementi sono in $L$, ma per lo stesso motivo 
    $\faktor{F}{K}$ non è normale.
\end{remark}

\pagebreak

\begin{remark}
    Osserviamo che il viceversa della \hyperref[3.50]{Proposizione 3.50} è falso, presa ad esempio la torre:\
    \[ \begin{tikzpicture}
    \node (Q1) at (0,-1) {$\QQ$};
    \node (Q3) at (0,0) {$\QQ(\sqrt{2})$};
    \node (Q4) at (0,1) {$\QQ(\sqrt[4]{2})$};
    \draw (Q4)--(Q3) node [right, yshift=-0.5cm] {\small$2$};
    \draw (Q1)--(Q3) node [right, yshift=0.5cm] {\small$2$};
    \end{tikzpicture}
        \]
    entrambe le estensioni sono normali, poiché di grado 2, ma la torre non è normale perché $\faktor{\QQ(\sqrt[4]{2})}{\QQ}$ contiene solo due radici di $x^4-2$.
\end{remark}

\newpage
\subsection{Gruppo di Galois}
\begin{definition}
    Un'estensione $\faktor{E}{K}$ si dice \vocab{estensione di Galois} se è normale e separabile.
\end{definition}

In queste dispense tratteremo soltanto il caso di estensioni di Galois finite.
Poiché $\faktor{E}{K}$ è normale possiamo considerare l'insieme:
\[ \{\varphi : E \varlonghookrightarrow \overline K| \varphi_{|K} = id_K\}
    \]
considerando ora lo stesso insieme, poiché l'estensione è normale, dunque $\varphi(E) = E$ per ogni $\varphi$, possiamo restringere l'insieme di arrivo degli omomorfismi ottenendo:
\[ \Aut_K E = \{\varphi : E \underset{\sim}{\longrightarrow} E| \varphi_{|K} = id_K\}
    \]
i $K$-automorfismi di $E$, cioè gli automorfismi dell'estensione che fissano $K$, tale insieme con l'operazione di composizione forma il \vocab{gruppo di Galois}:
\[ \Gal\left(\faktor{E}{K}\right) := \Aut_K(E)
    \]
con:
\[ \Big|\Gal\left(\faktor{E}{K}\right)\Big| = [E : K]
    \]

\begin{proposition}
    Il gruppo di Galois dell'estensione di Galois $\faktor{E}{K}$, $\Gal\left(\faktor{E}{K}\right)$ è un gruppo.
\end{proposition}

\begin{proof}
    Essendo un sottoinsieme del gruppo degli automorfismi di un campo, è sufficiente mostrare che un sottogruppo, dunque date $\varphi,\psi \in \Gal\left(\faktor{E}{K}\right)$, allora 
    si verifica che $\varphi \circ \psi \in \Gal\left(\faktor{E}{K}\right)$, inoltre, essendo automorfismi sono invertibili e i loro inversi sono ancora automorfismi, dunque:
    \[ \forall \varphi \in \Gal\left(\faktor{E}{K}\right) \implies \varphi^{-1} \in \Gal\left(\faktor{E}{K}\right)
        \]
\end{proof}
   
\begin{proposition}
    \label{3.55}
    Dato $f(x) \in K[x]$ irriducibile di grado $n$ e detto $F$ il suo campo di spezzamento su $K$, allora:
    \[ n \mid [F : K] \mid n!
        \]
    e $\Gal\left(\faktor{F}{K}\right) \varlonghookrightarrow S_n$. 
\end{proposition}

\begin{proof}
    Dette $\alpha_1,\ldots,\alpha_n$ le radici di $f(x)$ in $\overline K$, allora $F = K(\alpha_1,\ldots,\alpha_n)$, da cui si ha la torre:
    \[ K \subseteq K(\alpha_1) \subseteq F \implies n = [K(\alpha_1) : K] \mid [F : K] = \Big|\Gal\left(\faktor{F}{K}\right)\Big|
        \]
    dove la divisibilità segue ovviamente dal \hyperref[torri]{Teorema delle torri}. Consideriamo ora:
    \[ \phi: \Gal\left(\faktor{F}{K}\right) \longrightarrow S(\{\alpha_1,\ldots,\alpha_n\}) \cong S_n : \varphi \longmapsto \varphi_{|\{\alpha_1,\ldots,\alpha_n\}}
        \]
    Osserviamo che:
    \begin{itemize}
        \item \underline{\textbf{$\phi$ è ben definita}}: poiché $\forall \varphi \in \Gal\left(\faktor{F}{K}\right)$, $\varphi$ permuta le radici di $f(x)$ poiché manda ciascuna in un suo coniugato.
        \item \underline{\textbf{$\phi$ è un omomorfismo}}: si verifica direttamente che:
        \[ \phi(\varphi \circ \psi) = (\varphi \circ \psi)_{|{\{\alpha_1,\ldots,\alpha_n\}}} = \varphi(\psi{|{\{\alpha_1,\ldots,\alpha_n\}}}) = \varphi_{|{\{\alpha_1,\ldots,\alpha_n\}}} \circ \psi_{|{\{\alpha_1,\ldots,\alpha_n\}}}
            \]
        $\forall \phi,\psi \in \Gal\left(\faktor{F}{K}\right)$, in questo caso stiamo restringendo $\varphi$ alle radici perché $\psi$ sulle radici ha immagine nelle radici, per quanto detto sopra.    
        \item \underline{\textbf{$\phi$ è iniettiva}}: poiché, avendo dimostrato che è un omomorfismo possiamo studiarne il nucleo:
        \[ \ker \phi = \left\{\varphi \in \Gal\left(\faktor{F}{K}\right) \m \varphi(\alpha_i) = id(\alpha_i) = \alpha_i, \forall i \in \{1,\ldots,n\}\right\} = \{id\}
            \]
        dove l'ultima uguaglianza è data dal fatto che se $\varphi$ fissa tutti i generatori di $\faktor{F}{K}$, l'unica possibilità è che sia l'identità (banalmente perché stiamo considerando funzioni che permutano delle radici,
        dunque ce n'è una sola che le lascia tutte fisse ed è l'identità). In alternativa si poteva anche osservare che $\varphi$ è in particolare $K$ lineare e le radici formano una base su $K$ di $F$, dunque dato che $\varphi$ coincide
        con l'identità su queste essa è l'identità.
    \end{itemize}
\end{proof}

\begin{remark}
Con la dimostrazione precedente abbiamo anche visto che il gruppo di Galois agisce fedelmente su $\{\alpha_1,\ldots,\alpha_n\}$. Inoltre tale azione è transitiva perché ha un'unica orbita:
\[ \Orb(\alpha_1) = \left\{\varphi(\alpha_1) \m \varphi \in \Gal\left(\faktor{F}{K}\right)\right\} = \{\alpha_1,\ldots,\alpha_n\}
    \]
infatti, essendo $[K(\alpha_1) : K] = n$, allora ho esattamente $n$ immersioni che permutano i coniugati (ognuna delle quali si può estendere ad $F$):
\[ \forall i \in \{1,\ldots,n\},  \exists \psi_i : K(\alpha_1) \longrightarrow K(\alpha_i) \subset \overline K : \alpha_1 \longmapsto \alpha_i
    \]
e per tali immersioni abbiamo appunto che l'unica orbita è quella vista sopra.
\end{remark}

\begin{example}
    [Gruppo di Galois]
    Sia $K$ un campo e $f(x) \in K[x]$ irriducibile, con $F$ campo di spezzamento di $f(x)$ su $K$, studiamo il gruppo di Galois di $\faktor{F}{K}$ al variare del grado di $f(x)$:
    \begin{itemize}
        \item Se $\deg f(x) = 2$, allora $[F : K] = 2$, per la \hyperref[3.55]{Proposizione 3.55}, dunque $\Gal\left(\faktor{F}{K}\right) \cong \Z{2}$, pertanto $\Gal\left(\faktor{F}{K}\right) = \{id,\varphi\}$, dove, essendo $F = K(\sqrt{\Delta})$ abbiamo che:
        \[ id : a+b\sqrt{\Delta} \longmapsto a+b\sqrt{\Delta} \qquad \text{e} \qquad \varphi : a+b\sqrt{\Delta} \longmapsto a-b\sqrt{\Delta}
            \]
        o analogamente, se fosse $f(x) = (x - \alpha_1)(x - \alpha_2 )$ le applicazioni sarebbero state tali che $id : \alpha_1 \longmapsto \alpha_1$ e $\varphi : \alpha_1 \longmapsto \alpha_2$.
        \item Se $\deg f(x) = 3$, allora $3 \mid [F : K] \mid 6$, per la \hyperref[3.55]{Proposizione 3.55} sappiamo che $\Gal\left(\faktor{F}{K}\right) \leqslant S_3$ pertanto:
        \[ \Gal\left(\faktor{F}{K}\right) \cong \begin{cases}
            \mathcal{A}_3 \cong \Z{3} \\
            S_3
        \end{cases}
            \]
        Discutiamo di seguito entrambi i casi con due esempi.
    \end{itemize}
\end{example}

\begin{example}
    Se fosse $f(x) = x^3 - 2$, con $K = \QQ$ e $F = \QQ(\sqrt[3]{2},\zeta_3) \implies [F : K] = 6$ e quindi $\Gal\left(\faktor{F}{K}\right) \cong S_3$; per quanto detto sulle immersioni sappiamo che qualsiasi automorfismo del gruppo di Galois 
    deve mandare un elemento nei suoi coniugati, quindi abbiamo in totale appunto 6 possibili immersioni al variare di $\varphi$:
    \[ \varphi(\sqrt[3]{2}) = \sqrt[3]{2}\zeta_3^i \qquad \text{e} \qquad \varphi(\zeta_3) = \zeta_3^j \qquad \text{con}\quad i \in \{0,1,2\},j \in \{0,1\}
        \]
    Necessariamente $\varphi$ deve verificare le relazioni sopra\footnote{E per la precisione, affinché il discorso fatto sopra funzioni gli elementi devono essere algebricamente indipendenti.} che
    danno al più 6 possibili $\varphi$. Poiché il grado è 6, tutte e 6 funzionano cioè si estendono ad $F$, ciò perché:
    \[ \begin{tikzpicture}
        \node (Q1) at (0,0) {$\QQ$};
        \node (Q2) at (1.5,1.5) {$\QQ(\zeta_3)$};
        \node (Q3) at (0,3) {$F$};
        \node (Q4) at (-1.5,1.5) {$\QQ(\sqrt[3]2)$};
        \draw (Q1)--(Q2) node [pos=0.7, below,inner sep=0.25cm] {2};
        \draw (Q1)--(Q4) node [pos=0.7, below,inner sep=0.25cm] {3};
        \draw (Q3)--(Q4) node [pos=0.7, above]{};
        \draw (Q2)--(Q3) node [pos=0.3, above]{};
        \draw (Q1)--(Q3) node [pos=0.7, below, right,inner sep=0.25cm, yshift=-0.4cm]{};
        \end{tikzpicture}
    \]
    ovvero $[F : \QQ] = [Q (\sqrt[3]{2}) : \QQ][\QQ(\zeta_3) : \QQ] = 6$.
\end{example}

\begin{example}[Vi sconsiglio caldamente di leggere quest'esempio prima che l'abbia riscritto]
\footnote{Questo è l'esercizio risolto dalla Del Corso con la spiegazione più contorta in assoluto, prometto di riscriverlo per bene nella prima revisione.}
Consideriamo $f(x) = x^3 + x^2 - 2x -1$, ovvero il polinomio minimo di $\zeta_7+\zeta_7^{-1}$ su $\QQ$ e $\faktor{\QQ(\zeta_7+\zeta^{-1})}{\QQ}$, che è normale, quindi 
di Galois\footnote{Ciò sempre in virtù del fatto che stiamo lavorando soltanto su estensioni separabili.}, abbiamo quindi che:
\[ \Gal\left(\faktor{\QQ(\zeta_7+\zeta_7^{-1})}{\QQ}\right) \cong \mathcal{A}_3
    \]
verifichiamo quanto appena detto. Abbiamo che $[\QQ(\zeta_7) : \QQ] = \phi(7) = 6$, pertanto:
\[ \Gal\left(\faktor{\QQ(\zeta_7)}{\QQ}\right) = \{\varphi : \zeta_7 \longmapsto \zeta_7^i | (i,7) = 1\}
    \]
Osserviamo che il polinomio scritto sopra è proprio il polinomio minimo di $\alpha = \zeta_7+\zeta_7^{-1} \in \QQ(\zeta_7)$, infatti per l'appartenenza appena citata sappiamo che tutte le radici
di $\mu_{\zeta_7+\zeta_7^{-1}}(x)$ stanno in $\QQ(\zeta_7)$, ed esse sono:
\[ \left\{\varphi(\alpha) \m \varphi \in \Gal\left(\faktor{\QQ(\zeta_7)}{\QQ}\right) \right\}
    \]
le radici sono queste poiché, data la torre:
\[ \QQ \subseteq \QQ(\alpha) \subseteq \QQ(\zeta_7)
    \]
le immersioni $\psi$ mandano $\alpha$ in una qualsiasi radice coniugata di $\mu_{\alpha}(x)$, per cui l'insieme:
\[ \{\psi(\alpha)\}
    \]
che, al variare delle immersioni $\psi$ di $\QQ(\alpha)$ in $\overline \QQ$, è l'insieme delle radici di $\mu_\alpha(x)$, dunque possiamo scrivere:
\[ \mu_\alpha(x) = \prod_{\psi}(x - \psi(\alpha))
    \]
D'altra parte ogni immersione $\psi$ si estende ad uno $\varphi \in \Gal\left(\faktor{\QQ(\zeta_7)}{\QQ}\right)$ e $\forall \varphi \in \Gal\left(\faktor{\QQ(\zeta_7)}{\QQ}\right)$, sappiamo che 
$\varphi(\alpha)$ sarà ancora una radice di $\mu_\alpha(x)$, pertanto abbiamo:
\[ \{\text{radici di $\mu_\alpha(x)$}\} = \{\psi(\alpha)\}_{\psi : \QQ(\alpha) \longrightarrow \QQ} = \{\varphi(\alpha)\}_{\varphi \in \Gal\left(\faktor{\QQ(\zeta_7)}{\QQ}\right)}
    \]
infine abbiamo che:
\[  \{\text{radici di $\mu_\alpha(x)$}\} = \{\varphi(\zeta_7+\zeta_7^{-1}) = \zeta_7^i+\zeta_7^{-i} | i = 1,\ldots,6\} = \{\zeta_7+\zeta_7^{-1}, \zeta_7^2+\zeta_7^{-2},\zeta_7^3+\zeta_7^{-3}\}
    \]
ciò dimostra l'assunto iniziale $[\QQ(\zeta_7+\zeta_7^{-1}) : \QQ] = \deg \mu_{\zeta_7+\zeta_7^{-1}}(x) = 3$. Per concludere osserviamo che l'estensione è di Galois perché è il campo di spezzamento di $\mu_{\zeta_7+\zeta_7^{-1}}(x)$ in quanto:
\[ \zeta_7^2+\zeta_7^{-2} = (\zeta_7+\zeta_7^{-1})^2 - 2 \in \QQ(\zeta_7+\zeta_7^{-1})
    \]
e analogamente:
\[ \zeta_7^3+\zeta_7^{-3} = (\zeta_7+\zeta_7^{-1})^3 - 3(\zeta_7+\zeta_7^{-1}) \in \QQ(\zeta_7+\zeta_7^{-1})
    \]
si può verificare che im tal modo, con qualche calcolo, è possibile trovare l'espressione di $\mu_\alpha(x)$.
\end{example}

\newpage
\subsection{Gruppo di Galois di $\F{q^d}/\F{q}$}

\begin{proposition}[L'estensione $\F{q^d}/\F{q}$]
    Data l'estensione $\F{q^d}/\F{q}$, con $q = p^r$, essa è normale.
\end{proposition}

\begin{proof}
    Osserviamo che:
    \[ \forall \varphi : \Fpn \varlonghookrightarrow \overline \Fp
        \]
    $\varphi(\Fpn)$ è un sottocampo di $\overline\Fp$ con $p^n$ elementi (essendo $\varphi$ iniettiva), e per l'unicità dei campi con $p^n$ elementi deve essere necessariamente che $\varphi(\Fpn) = \Fpn$,
    pertanto l'estensione è normale.\footnote{Alternativamente si ricorda che in \href{https://github.com/diego-unipi/Appunti-Aritmetica}{\textcolor{purple}{Aritmetica}}, avevamo costruito tutte le estensioni $\Fpn$ di campi finiti $\Fp$, come campi di spezzamento del polinomio $f(x) = x^{p^n} - x \in \Fp[x]$.}
\end{proof}

\begin{remark}
    Osserviamo che tutte le estensioni di campi finiti sono normali, infatti, considerando la torre:
    \[\begin{tikzpicture}
        \node (Q1) at (0,-1.5) {$\Fp$};
        \node (Q3) at (0,0) {$\F{q} = \F{p^r}$};
        \node (Q4) at (0,1.5) {$\Fpn = \F{q^d}$};
        \draw (Q4)--(Q3) node [right, yshift=-0.75cm] {};
        \draw (Q1)--(Q3) node [right, yshift=0.75cm] {NOR.};
        \draw(Q4) to [bend right=45] (Q1) node[left, yshift=1.20cm, xshift=-0.5cm] {NOR.};
    \end{tikzpicture}
        \]
    dove per quanto detto $\faktor{\Fpn}{\Fp}$ è normale, dunque per la \hyperref[3.50]{Proposizione 3.50} tutte le estensioni di campi finiti sono normali.
\end{remark}

\begin{definition}
    Si dice \vocab{automorfismo di Frobenius} l'automorfismo:
    \[ \phi : \F{q^d} \underset{\sim}{\longrightarrow} \F{q^d} : x \longmapsto x^q
        \]
\end{definition}

\begin{theorem}[Gruppo di Galois di estensioni di campi finiti]
    Il gruppo di Galois $\Gal\left(\faktor{\F{q^d}}{\F{q}}\right)$, con $q = p^r$, è generato dall'automorfismo di Frobenius $\phi$ di $\F{q^d}$:
    \[ \Gal\left(\faktor{\F{q^d}}{\F{q}}\right) = \left<\phi\right>
        \]
    con $\phi$ automorfismo di Frobenius del campo $\F{q^d}$:
    \[ \phi : \F{q^d} \underset{\sim}{\longrightarrow} \F{q^d} : x \longmapsto x^q
        \]
\end{theorem}

\begin{proof}
    Per prima cosa verifichiamo che $\phi \in \Gal\left(\faktor{\F{q^d}}{\F{q}}\right)$; $\phi$ è un omomorfismo di anelli in quanto:
    \[ \phi(\alpha+\beta) = (\alpha+\beta)^q = \alpha^q + \beta^q = \phi(\alpha) + \phi(\beta) \qquad \forall \alpha,\beta \in \F{q^d} 
        \]
    dove l'ultima uguaglianza segue dal fatto che $q = p^r$, dunque siamo in caratteristica $p$ e vale il Lemma del Binomio Ingenuo. Analogamente:
    \[ \phi(\alpha\beta) = (\alpha\beta)^q = \alpha^q \beta^q = \phi(\alpha)\phi(\beta) \qquad \forall \alpha,\beta \in \F{q^d} 
        \]
    si verifica facilmente inoltre che è iniettivo e surgettivo. $\forall \alpha \in \F{q}$ si ha che $\phi(\alpha) = \alpha^q = \alpha$ (perché stiamo considerando elementi di $\F{q}$), dunque $\phi$ è un automorfismo di $\F{q^d}$ che lascia fisso $\F{q}$, pertanto
    $\phi \in \Gal\left(\faktor{\F{q^d}}{\F{q}}\right)$, per definizione. Osserviamo che per ipotesi $\Gal\left(\faktor{\F{q^d}}{\F{q}}\right)$ è un gruppo di ordine $d$ e quindi 
    ovviamente $\ord \phi = k \mid d$; d'altra parte se $\phi^k = id$, $\forall \alpha \in \F{q^d}$ si ha che:
    \[ \phi^k(\alpha) = \alpha^{q^k} = \alpha
        \]
    cioè il polinomio $f(x) = x^{q^k} - x$ ha come radici tutti i $q^d$ elementi di $\F{q^d}$, pertanto abbiamo:
    \[ \deg f(x) = q^k \geq q^d = |\F{q^d}|\implies k \geq d
        \]
    da cui $k = d$ e quindi la tesi:
    \[ \Gal\left(\faktor{\F{q^d}}{\F{q}}\right) = \left<\phi\right>
        \]
\end{proof}

\begin{example}
    Consideriamo $E = \faktor{\QQ(\sqrt{2},\sqrt{3})}{\QQ}$, abbiamo che le estensioni alla chiusura algebrica via identità:
    \[ \varphi : E \varlonghookrightarrow \ol\QQ
        \]
    sono determinate dalle immagini di $\sqrt{2}$ e $\sqrt{3}$, dunque abbiamo:
    \[ \varphi = \begin{cases}
        \sqrt{2} \longmapsto \pm \sqrt{2} \\
        \sqrt{3} \longmapsto \pm \sqrt{3}
    \end{cases}
        \]
    da cui il diagramma:
    \[      \begin{tikzpicture}
		    \node (Q1) at (0,0) {$\QQ$};
		    \node (Q2) at (1.5,1.5) {$\QQ(\sqrt{2})$};
		    \node (Q3) at (0,3) {$E$};
		    \node (Q4) at (-1.5,1.5) {$\QQ(\sqrt{3})$};
		    \draw (Q1)--(Q2) node [pos=0.7, below,inner sep=0.25cm] {2};
		    \draw (Q1)--(Q4) node [pos=0.7, below,inner sep=0.25cm] {2};
		    \draw (Q3)--(Q4) node [pos=0.7, above] {};
		    \draw (Q2)--(Q3) node [pos=0.3, above] {};
		    \draw (Q1)--(Q3) node [pos=0.7, below, right,inner sep=0.25cm, yshift=-0.4cm] {$4$};
		    \end{tikzpicture}
        \]
    dove il composto e le due estensioni semplici sono normali (perché di grado 2); le quattro immersioni possibili sono:
    \[ id = \varphi_1 = \begin{cases}
        \sqrt{2} \longmapsto  \sqrt{2} \\
        \sqrt{3} \longmapsto  \sqrt{3}
    \end{cases}
    \quad
    \varphi_2 = \begin{cases}
        \sqrt{2} \longmapsto -\sqrt{2} \\
        \sqrt{3} \longmapsto  \sqrt{3}
    \end{cases}
    \quad
    \varphi_3 = \begin{cases}
        \sqrt{2} \longmapsto  \sqrt{2} \\
        \sqrt{3} \longmapsto  -\sqrt{3}
    \end{cases}
    \]\[
    \varphi_4 = \begin{cases}
        \sqrt{2} \longmapsto -\sqrt{2} \\
        \sqrt{3} \longmapsto -\sqrt{3}
    \end{cases}
        \]
    con $\varphi_i(E) = E$, $\forall i = 1,\ldots,4$. Il gruppo di Galois è dato da $\Z2\times\Z2$, poiché ha ordine 4, e tutti gli elementi, esclusa l'identità, hanno ordine 2.
    Si verifica facilmente via contenimenti che $\QQ(\sqrt{2},\sqrt{3}) = \QQ(\sqrt{2} + \sqrt{3})$, tuttavia possiamo anche osservare a questo punto che, data la torre:
    \[ \QQ \subseteq \QQ(\sqrt{2} + \sqrt{3}) \subseteq E
        \]
    di grado 4, abbiamo che:
    \[ \varphi_1(\sqrt{2}+\sqrt{3}) = \sqrt{2} + \sqrt{3} \qquad \varphi_2(\sqrt{2}+\sqrt{3}) = -\sqrt{2} + \sqrt{3}
        \]\[ \varphi_1(\sqrt{2}+\sqrt{3}) = \sqrt{2} - \sqrt{3} \qquad \varphi_1(\sqrt{2}+\sqrt{3}) = - \sqrt{2} - \sqrt{3}
            \]
    $\gamma = \sqrt{2}+\sqrt{3}$ ha quattro immagini distinte\footnote{Per essere precisi ciò significa che, data
    la base $\{1,\sqrt2,\sqrt3,\sqrt6\}$ di $E$, per mezzo di questa si verifica che tutte e quattro quelle immagini hanno scrittura unica e distinta.}
    attraverso le immersioni del gruppo di Galois, e mediante questo $\gamma$ viene mandato in suoi coniugati su $\QQ$;
    dunque il polinomio minimo di $\gamma$ su $\QQ$ ha almeno grado 4, ma per la torre precedente ciò significa che $\QQ(\gamma) = E$.
\end{example}

\begin{remark}
    In questo caso abbiamo assunto direttamente che $\deg \mu_\gamma (x) = 4$, perché $\deg \mu_\gamma = \#\{\psi : \QQ(\gamma) \varlonghookrightarrow \ol\QQ\}$ e ogni $\psi$ si 
    estende ad $E$, via:
    \[ \varphi_i : E \varlonghookrightarrow \ol \QQ
        \]
    pertanto $\{\varphi_i(\gamma)\}$ è l'insieme dei coniugati di $\gamma$ su $\QQ$.
\end{remark}

\newpage
\subsection{Teorema dell'elemento primitivo}
\begin{theorem}
    [Teorema dell'elemento primitivo]
    \label{prim}
    Sia $K$ un campo e sia $\faktor{E}{K}$ un'estensione finita (e separabile), allora $\faktor{E}{K}$ è semplice, cioè:
    \[ \exists\gamma \in E : E = K(\gamma)
        \] 
\end{theorem}

\begin{proof}
    Distinguiamo due casi:
    \begin{itemize}
        \item \underline{\textbf{$K$ campo infinito}}: Per ipotesi abbiamo che $\faktor{E}{K}$ è finita, dunque per la \hyperref[3.16]{Proposizione 3.16}, ciò è equivalente al dire che $E$ è finitamente generata da 
        elementi algebrici, $E = K(\alpha_1,\ldots,\alpha_n)$, dimostriamo per induzione che $\faktor{E}{K}$ è semplice. Per $n = 2$ (trattiamo solo di questo caso in quanto una volta dimostrata la tesi per $n = 2$, basterà
        come al solito aggiungere un altro alla volta e procedere per induzione usando il caso $n = 2$ come fatto che rende vero il passo induttivo\footnote{Andrebbe dimostrato anche il caso $n = 1$, ma è banale.}) abbiamo $E = K(\alpha,\beta)$, sia $[E : K] = n$, allora per il \hyperref[3.39]{Corollario 3.35}:
        \[ \exists \varphi_1,\ldots,\varphi_n : E \varlonghookrightarrow \ol K \qquad \text{con} \qquad \varphi_{i|K} = id_K
            \]
        sia $x$ un'indeterminata, consideriamo i polinomi $\alpha+\beta x$, possiamo definire il polinomio:
        \[ F(x) = \prod_{i<j} (\varphi_i(\alpha)+x\varphi_i(\beta)-\varphi_j(\alpha)-x\varphi_j(\beta)) \in \ol K[x] \footnote{È come se applicassimo $\varphi_i$ al polinomio $\alpha + \beta x$ e poi vi sottraessimo $\varphi_j$ applicata allo stesso.}
            \]
        con $\displaystyle\deg F(x) \leq \binom{n}{2}$ e $F(x) \ne 0$ in quanto se un fattore fosse 0, quindi $\varphi_i(\alpha)+x\varphi_i(\beta) = \varphi_j(\alpha)+x\varphi_j(\beta)$, da cui $x(\varphi_i(\beta) - \varphi_j(\beta)) +\varphi_i(\alpha) - \varphi_i(\alpha) = 0$, per il principio di identità dei polinomi avremmo:
        \[ \begin{cases}
            \varphi_i(\beta) - \varphi_j(\beta) = 0\\
            \varphi_i(\alpha) - \varphi_i(\alpha) = 0
        \end{cases}
        \qquad
        \iff
        \qquad
        \begin{cases}
            \varphi_i(\beta) = \varphi_j(\beta)\\
            \varphi_i(\alpha) = \varphi_i(\alpha)
        \end{cases}
            \]
        da cui $\varphi_i \equiv \varphi_j$ perché $E = K(\alpha,\beta)$ (infatti due immersioni in cui i generatori coincidono sono la stessa immersione), ma ciò è assurdo, in quanto avevamo assunto $i < j$. Dunque il polinomio è non nullo ed ha grado limitato, 
        sappiamo quindi che $F(x)$ ha al più $\deg F(x)$ radici in $\ol K$ e poiché $K$ è un campo infinito, allora $\exists t \in K$ tale che $F(t) \ne 0$, dunque:
        \[ F(t) = \prod_{i<j} (\underbrace{\varphi_i(\alpha)+t\varphi_i(\beta)}_{ = \varphi_i(\alpha+x\beta)}\underbrace{-\varphi_j(\alpha)-t\varphi_j(\beta)}_{ = -\varphi_j(\alpha+x\beta)}) \ne 0
            \]
        da ciò abbiamo che:
        \[ \varphi_i(\alpha+t\beta) \ne \varphi_j(\alpha+t\beta) \qquad \forall i \ne j
            \]
        quindi $\gamma = \alpha+t\beta$ ha $n$ coniugati, pertanto $[K(\gamma) : K] = n \implies E = K(\gamma)$ (ovvero le due estensioni dello stesso campo hanno lo stesso grado e quindi coincidono).
        \item \underline{\textbf{$K$ campo finito}}: Se $\faktor{E}{K}$ è finita, allora $E$ è finito, da cui, per la nota proprietà sui sottogruppi moltiplicativi di un campo finito, sia che $E^*$ è un 
        sottogruppo moltiplicativo finito di $E$ ed è ciclico, $E^* = \left<\gamma\right>$, ma per un altro teorema noto da \href{https://github.com/diego-unipi/Appunti-Aritmetica}{\textcolor{purple}{Aritmetica}}, si ha che $E = K(\gamma)$.
        \end{itemize}
\end{proof}

\newpage
\subsection{Teorema di corrispondenza di Galois}
Data l'estensione di Galois (finita) $\faktor{L}{K}$ e $H < \Gal\left(\faktor{L}{K}\right)$ sottogruppo, definiamo la scrittura $L^H$ come:
\[ L^H = \Fix(H) := \{ \alpha \in L | \varphi(\alpha) = \alpha, \forall \alpha \in H \} \subseteq L
    \]
ovvero il sottocampo\footnote{Andrebbe verificato che è un campo.} di $L$ di tutti gli elementi fissati da tutti i $K$-automorfismi del sottogruppo $H$. Si osserva che per tale sottocampo si ha che:
\[ K \subseteq L^H \subseteq L
    \]
la seconda inclusione segue immediatamente dalla definizione, la prima deriva dal fatto che essendo i $K$-automorfismi in particolare delle immersioni di $L$ in $\ol K$ via identità (per come è definito il gruppo di Galois), allora almeno tutti gli elementi del campo $K$ devono essere fissati 
per definizione.

\begin{lemma}[Il campo fissato è quello base$\iff$fissiamo rispetto a tutto il gruppo di Galois]
    \label{3.67}
    Sia $\faktor{L}{M}$ un'estensione di Galois e $H < \Gal\left(\faktor{L}{M}\right)$, allora:
    \[ M = L^H \iff H = \Gal\left(\faktor{L}{M}\right)
        \]
\end{lemma}

\begin{proof}
    Dimostriamo separatamente le due implicazioni:
    \begin{itemize}
        \item Supponiamo che $G = \Gal\left(\faktor{L}{M}\right)$ e dimostriamo che $L^G = M$. Se $M \subsetneq L^G$, allora $[L^G : M]>1$, quindi:
        \[ \exists\varphi : L^G \longrightarrow \ol M \qquad \text{con} \qquad \varphi_{|M} = id_M
            \]
        con $\varphi \ne id$ in quanto il grado è maggiore di 1, per cui non c'è solo l'identità tra le immersioni; a questo punto sappiamo dalla \hyperref[3.39]{Proposizione 3.39} che l'immersione si può estendere ad 
        un campo più grande, dunque sia:
        \[ \widetilde{\varphi} : L \longrightarrow \ol M \qquad \text{con} \qquad \widetilde{\varphi}_{|L^G} = \varphi
            \]
        dove $\widetilde{\varphi}_{|M} = \varphi_{|M} = id$. D'altra parte abbiamo per ipotesi che $\faktor{L}{M}$ è normale, dunque $\widetilde{\varphi}(L) = L$, quindi:
        $\widetilde{\varphi} \in \Gal\left(\faktor{L}{M}\right)$ (perché è un automorfismo che fissa puntualmente $L$ e si restringe all'identità su $M$), da ciò abbiamo che $\widetilde{\varphi}$ fissa puntualmente
        $L^G$, che è assurdo perché avevamo supposto il grado dell'estensione maggiore di 1 (quindi con un elemento non fissato da $\widetilde{\varphi}$).
        \item Essendo $\faktor{L}{M}$ un'estensione finita, per il \hyperref[prim]{Teorema dell'elemento primitivo} $L = M(\alpha)$, sia $H \leqslant G$ e consideriamo:
        \[ f(x) = \prod_{\sigma \in H}(x - \sigma(\alpha)) \qquad \text{con} \qquad \deg f(x) = |H|
            \]
        si ha $f(x) \in L^H[x]$, poiché, se consideriamo $\rho \in H$, allora si ha:
        \[ \rho f(x) = \prod_{\sigma \in H}(x - \rho(\sigma(\alpha))) = \prod_{\sigma \in H}(x - \sigma(\alpha)) = f(x)
            \]
        dove l'ultima uguagliando deriva dal fatto che anche $\rho$ è un elemento di $H$, e dunque l'unica cosa che fa è permutare gli elementi dell'insieme stesso. Se $H = G$
        abbiamo $|G| = [L : M] = [M(\alpha) : M] = \deg \mu_{\alpha/M}(x) \geq \deg f(x) = |H|$, in quanto $f(x) \in L^M[x] = M[x]$ per ipotesi e $f(\alpha) = 0$, da cui $\mu_\alpha(x) \mid f(x)$ e quindi la tesi $H = G$.
    \end{itemize}
\end{proof}

\begin{lemma}
    \label{3.68}
    Data l'estensione $\faktor{L}{K}$ di Galois e $H < \Gal\left(\faktor{L}{K}\right)$, sia $\sigma \in \Gal\left(\faktor{L}{K}\right)$, allora:
    \[ L^{\sigma H \sigma^{-1}} = \sigma(L^H)
        \]
\end{lemma}

\begin{proof}
    Ricordando che $L^H = \{\alpha \in L | \varphi(\alpha) = \alpha, \forall \varphi \in H\}$, abbiamo che:
\[ \sigma(L^H) = \{\sigma(\alpha) | \alpha \in L^H\} = \{\underbrace{\sigma(\alpha)}_{= \beta} | \varphi(\alpha) = \alpha, \forall \varphi \in H\}
        \]
da cui:
\begin{multline*}
    \sigma(L^H) = \{\beta \in L | (\varphi \circ \sigma^{-1})(\beta) = \sigma^{-1}(\beta), \forall \varphi \in H\} = \\ 
    = \{\beta \in L | (\sigma \circ \varphi \circ \sigma^{-1})(\beta) = \beta, \forall \varphi \in H\} = L^{\sigma H \sigma^{-1}}
\end{multline*}
\end{proof}

\begin{remark}
    Non è detto che $\sigma(L^H)$ faccia $L^H$, tuttavia sarà sempre una sottoestensione di $\faktor{L}{K}$, poiché $L$ è ancora fissato,
    quindi in generale abbiamo:
    \[\begin{tikzpicture}
        \node (Q1) at (0,0) {$K$};
        \node (Q2) at (1.5,1.5) {$\sigma(L^H)$};
        \node (Q3) at (0,3) {$L$};
        \node (Q4) at (-1.5,1.5) {$L^H$};
        \draw (Q1)--(Q2) node [pos=0.7, below,inner sep=0.25cm] {};
        \draw (Q1)--(Q4) node [pos=0.7, below,inner sep=0.25cm] {};
        \draw (Q3)--(Q4) node [pos=0.7, above] {};
        \draw (Q2)--(Q3) node [pos=0.3, above] {};
        \draw (Q1)--(Q3) node [pos=0.7, below, right,inner sep=0.25cm, yshift=-0.4cm] {};
        \end{tikzpicture}
        \] 
\end{remark}

\begin{theorem}
    [Teorema di Corrispondenza di Galois]
    \label{corrG}
    Data l'estensione di Galois (finita) $\faktor{L}{K}$ c'è una corrispondenza biunivoca tra l'insieme delle sottoestensioni di $\faktor{L}{K}$ e l'insieme dei sottogruppi di $\Gal\left(\faktor{L}{K}\right)$. Inoltre $H \triangleleft G$ se e solo se 
    $\faktor{L^H}{K}$ è normale, ed in tal caso abbiamo:
    \[ \Gal\left(\faktor{L^H}{K}\right) \cong \frac{\Gal\left(\faktor{L}{K}\right)}{\Gal\left(\faktor{L}{L^H}\right)} = \faktor{G}{H}
        \]
\end{theorem}

\begin{proof}
    Detto $\mathcal{E}_{L/K} = \{F | K \subseteq F \subseteq L\}$ l'insieme delle sottoestensioni di $\faktor{L}{K}$ e $\mathcal{G}_{L/K} = \left\{H < \Gal\left(\faktor{L}{K}\right)\right\}$ l'insieme dei sottogruppi del gruppo di Galois dell'estensione, allora essi sono in bigezione:
    \[ \mathcal{E}_{L/K} \longleftrightarrow \mathcal{G}_{L/K}
        \]
    mediante le applicazioni:
    \[ \alpha : \mathcal{E}_{L/K} \longrightarrow \mathcal{G}_{L/K} : F \longmapsto \Gal\left(\faktor{L}{F}\right) \footnote{Per la precisione, avendo assunto che estensione normale e di Galois siano la stessa cosa e che $\faktor{L}{K}$ sia di Galois, allora $\faktor{L}{F}$ è di Galois, mentre non è detto che $\faktor{F}{K}$ lo sia, per la \hyperref[3.50]{Proposizione 3.50}, ed essendo di Galois ciò ci permette di usare la corrispondenza scritta sopra.}
        \]
    e:
    \[ \beta : \mathcal{G}_{L/K} \longrightarrow \mathcal{E}_{L/K} : H \longmapsto L^H
        \]
    Si osserva che essendo $L^H$ un campo e $K \subseteq L^H \subseteq L$ per definizione, allora $\beta$ è ben definita; osserviamo anche che si ha:
    \[ \Gal\left(\faktor{L}{F}\right) < \Gal\left(\faktor{L}{K}\right)
        \]
    perché un automorfismo di $F$ che fissa puntualmente $F$, ovviamente fissa puntualmente anche il suo sottoinsieme $K$, dunque l'immagine via $\alpha di F$ sta in $\mathcal{G}$,
    pertanto $\alpha$ è ben definita. Verifichiamo ora che $\alpha$  e $\beta$ descrivono una bigezione tra $\mathcal{E}_{L/K}$ e $\mathcal{G}_{L/K}$, mostriamo che sono una l'inversa dell'altra:
    \[ \beta \circ \alpha (F) = \beta\left(\Gal\left(\faktor{L}{F}\right)\right) = L^{\Gal\left(\faktor{L}{F}\right)} = F \qquad \forall F \in \mathcal{E}_{F/K}
        \]
    dove l'ultima uguaglianza è garantita dal \hyperref[3.67]{Lemma 3.67}, in quanto stiamo considerando il campo fissato da tutto il gruppo di Galois $\Gal\left(\faktor{L}{F}\right)$. Viceversa:
    \[ \alpha \circ \beta (H) = \alpha(L^H) = \Gal\left(\faktor{L}{L^H}\right) = H  \qquad \forall H \in \mathcal{G}_{F/K}
        \]
    infatti, $H < \Gal\left(\faktor{L}{L^H}\right)$ in quanto tutti gli elementi di $L^H$ sono fissati dagli elementi di $H$ per definizione e quindi $H \subseteq \Gal\left(\faktor{L}{L^H}\right)$; d'altra parte 
    $\Gal\left(\faktor{L}{L^H}\right) \subseteq H$, perché, detto $L^H = M$, abbiamo $H \subseteq \Gal\left(\faktor{L}{M}\right)$ e $L^H = M$, dunque per il \hyperref[3.67]{Lemma 3.67} segue che $H = \Gal\left(\faktor{L}{L^H}\right)$.\\
    Verifichiamo ora la seconda parte del teorema; sappiamo che $H \triangleleft \Gal\left(\faktor{L}{K}\right) \iff gHg^{-1} = H$, $\forall \sigma \in \Gal\left(\faktor{L}{K}\right)$ e per il 
    \hyperref[3.68]{Lemma 3.68} abbiamo che:
    \[ \sigma(L^H) = L^{\sigma H \sigma^{-1}} = L^H \qquad \forall \sigma \in \Gal\left(\faktor{L}{K}\right)
        \]
    e ciò è equivalente al dire che $\faktor{L^H}{K}$ è normale, perché, $\forall \psi : L^H \varlonghookrightarrow \ol K$, con $\psi_{|K} = id_K$, si ha che $\psi(L^H) = L^H$, poiché ogni $\psi$ si estende a $L$ e quindi:
    \[ \forall \varphi \in \Gal\left(\faktor{L}{K}\right), \exists \sigma \in \Gal\left(\faktor{L}{K}\right) : \sigma_{|L^H} = \psi
        \]
    Infine, resta da verificare che $\Gal\left(\faktor{L^H}{K}\right) \cong \faktor{G}{H}$, consideriamo l'omomorfismo di restrizione:
    \[ \Gamma : \Gal\left(\faktor{L}{K}\right) \twoheadlongrightarrow \Gal\left(\faktor{L^H}{K}\right) : \varphi \longmapsto \varphi_{|L^H}
        \]
    esso è ovviamente surgettivo perché ogni $\psi \in \Gal\left(\faktor{L^H}{K}\right)$ si estende ad $L$, ed inoltre:
    \begin{multline*}
        \ker \Gamma = \left\{\varphi \in \Gal\left(\faktor{L}{K}\right) \m \varphi_{|L^H} = id \right\} = \left\{\varphi \in \Gal\left(\faktor{L}{K}\right) \m \varphi(\alpha) = \alpha, \forall \alpha \in L^H\right\} = \\ 
        = \Gal\left(\faktor{L}{L^H}\right) \cong H
    \end{multline*}
\end{proof}

\begin{remark}
    Il teorema ci dice che, data ad esempio la torre:
    \[\begin{tikzpicture}
        \node (Q1) at (0,-1.5) {$K$};
        \node (Q3) at (0,0) {$L^H$};
        \node (Q4) at (0,1.5) {$L$};
        \draw (Q4)--(Q3) node [right, yshift=-0.75cm, red] {$\faktor{G}{H}$};
        \draw (Q1)--(Q3) node [right, yshift=0.75cm, red] {$H$};
        \draw(Q4) to [bend right=45] (Q1) node[left, yshift=1.20cm, xshift=-0.5cm, red] {$G$};
    \end{tikzpicture}
        \]
    dove $G$ è il gruppo di Galois di $\faktor{L}{K}$. Essendo anche $\faktor{L}{L^H}$ di Galois 
    per ipotesi, e detto $H$ il suo gruppo di Galois, allora, per il teorema precedente, se $H \triangleleft G$,
    si ha che anche $\faktor{L^H}{K}$ è di Galois ed il suo gruppo di Galois è $\faktor{G}{H}$.
\end{remark}

\begin{proposition}[Proprietà della corrispondenza di Galois]
Dati $H,S < \Gal\left(\faktor{L}{K}\right)$, allora valgono le seguenti:
\begin{enumerate}[(1)]
    \item $H \leqslant  S \iff L^H \supset L^S$.
    \item $L^{H \cap S} = L^H L^S$.\footnote{Intendiamo il composto dei due campi.} 
    \item $L^{\left<S,H\right>} = L^H \cap L^S$.
\end{enumerate}
\end{proposition}

\begin{proof}
Dimostriamo le affermazione:
\begin{enumerate}[(1)]
    \item "Ovvio".
    \item Osserviamo che $H \cap S \subset H,S$, dunque $L^{H\cap S} \supset L^H,L^S$, per il punto (1); pertanto un campo che contiene entrambi i sottocampi 
    conterrà anche il composto, $L^{H \cap S} \supset L^H L^S$. D'altra parte $L^H L^S \subset L \implies \exists N \leqslant \Gal\left(\faktor{L}{K}\right) $
    tale che $L^H L^S = L^N$, per il \hyperref[corrG]{Teorema di corrispondenza di Galois}, inoltre:
    \[ \Gal\left(\faktor{L}{L^N}\right) = N \subset \left(\underbrace{\Gal\left(\faktor{L}{L^H}\right)}_{= H \cap S} \cap \Gal\left(\faktor{L}{L^S}\right)\right)
        \]
    da cui $N \subset H \cap S \iff L^N \supset L^{H \cap S}$.
    \item Abbiamo che $H \subseteq \left<H,S\right>$ e $S \subseteq \left<H,S\right>$, da cui $L^S,L^H \supseteq L^{\left<H,S\right>}$, da cui $L^{\left<H,S\right>} \subseteq L^H \cap L^S$.\\
    Viceversa sia $\alpha \in L^H \cap L^S$, allora $\varphi(\alpha) = \alpha$, $\forall \varphi \in H,S$, quindi $\alpha$ è fissato dai generatori
    del gruppo $\left<H,S\right>$, pertanto $\alpha$ è fissato da tutti gli elementi di $\left<H,S\right>$, da cui la tesi:
    \[ L^H \cap L^S \subseteq L^{\left<H,S\right>}
        \]
\end{enumerate}
\end{proof}

\end{document}